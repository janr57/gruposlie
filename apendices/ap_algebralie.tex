% ap_algebralie.tex
%
% Copyright (C) 2022--2025 José A. Navarro Ramón <janr.devel@gmail.com>
% Licencia del código GPLv2
% Licencia Creative Commons Recognition Non-Commercial Share-alike.
% (CC-BY-NC-SA)

\chapter{Álgebra de Lie}
\label{ap:algebralie}

De la \href{https://es.wikipedia.org/wiki/Álgebra_de_Lie}{Wikipedia}:

Un álgebra de Lie $\mathfrak{a}$ es un espacio vectorial sobre un cierto cuerpo
  $\mathbb{F}$ junto con una operación binaria 
  $[\cdot, \cdot]\!: \mathfrak{a}\times\mathfrak{a} \to \mathfrak{a}$, llamada
  \textbf{corchete de Lie}, que satisface las propiedades siguientes:
\begin{enumerate}
\item es bilineal, es decir, para todo $a,b\in\mathbb{F}$ y $x,y,z\in\mathfrak{a}$.
  \begin{itemize}
  \item $[ax+by,z] = a[x,z] + b[y,z]$
  \item$ [z,ax+by] = a[z,x] + b[z,y]$
  \end{itemize}
\item satisface la identidad de Jacobi, es decir, para todo
  $x,y,z\in\mathfrak{a}$:
  \begin{itemize}
  \item $[[x,y],z] + [[z,x],y] + [[y,z], x] = 0$
  \end{itemize}
\item Para todo $x\in\mathfrak{a}$:
  \begin{itemize}
  \item $[x,x] = 0$
  \end{itemize}
\end{enumerate}

Adviértase que la primera y tercera propiedad juntas, implican el carácter
anticonmutativo ($[x,y] = -[y,x]$), para todo $x,y\in\mathfrak{a}$, si el cuerpo
$\mathbb{F}$ es de característica diferente de dos. Téngase en cuenta también que la
multiplicación representada por el corchete de Lie no es, en general, asociativa,
es decir $[[x,y],z]$ no es necesariamente igual a $[x,[y,z]]$.

Ejemplos:
\begin{itemize}
\item Cada espacio vectorial se convierte en un álgebra de Lie abeliana trivial si
  definimos el corchete de Lie como idénticamente cero.
\item El espacio euclídeo $\mathbb{R}^3$ se convierte en un álgebra de Lie, si se
  considera el corchete de Lie que usa el producto vectorial.
\item Si se da un álgebra asociativa $\mathfrak{A}$ con la multiplicación $*$, se
  puede dar un álgebra de Lie definiendo $[x,y] = x*y - y*x$, esta expresión se llama
  conmutador de $x$ e $y$.
\item Inversamente, puede ser demostrado que cada álgebra de Lie se puede sumergir en
  otra que surja de un álgebra asociativa de esa manera.
\item Otro ejemplo importante viene de la topología diferencial: los campos vectoriales
  en una variedad diferenciable forman un álgebra de Lie de dimensión infinita. Estos
  campos vectoriales actúan como operadores diferenciales sobre las funciones sobre
  la variedad. Dados dos campos vectoriales $X$ e $Y$, el corchete de Lie
  $[X,Y]$ se define como $[X,Y]f = (XY - YX)f$ y puede comprobarse que este operador
  corresponde a un campo vectorial. Las generalizaciones adecuadas de la teoría de
  variedades al caso de dimensión infinita muestra que esta álgebra de Lie es la
  asociada (ver siguiente punto) al grupo de Lie de los difeomorfismos de la
  variedad.
\item En el caso de una variedad que sea un grupo de Lie $G$ a su vez, un subespacio
  de los campos vectoriales queda inalterado por las transformaciones dadas por el
  propio grupo, en el sentido de que en cada punto $g$ del mismo, el campo no es más
  que $X(g) = dl_g\left(X(e)\right)$. Este subespacio es de dimensión finita (e igual
  a la del grupo), dado que se corresponde con el espacio tangente en la identidad.
  Además hereda la estructura de álgebra de Lie definida en el punto anterior, y se le
  denomina el \textbf{álgebra de Lie asociada al grupo} $G$.
\item Como ejemplo concreto, consideremos el grupo de Lie $\mathbf{SL}(n,\mathbb{R})$
  de todas las matrices $n\times n$ con valores reales y determinante 1. El espacio
  tangente en la matriz identidad se puede identificar con el espacio de todas las
  matrices reales $n\times n$ con traza $0$ y la estructura de álgebra de Lie que
  viene del grupo de Lie coincide con el que surge del conmutador de la multiplicación
  de matrices.
\end{itemize}
%}







%%% Local Variables:
%%% coding: utf-8
%%% mode: latex
%%% TeX-engine: luatex
%%% TeX-master: "../gruposlie.tex"
%%% End:

