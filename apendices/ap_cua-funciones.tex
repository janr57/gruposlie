\chapter{Acerca de generador de rotación para funciones}
\label{chapt:apcua-acerca-generador-funciones}

En este apéndice presento ideas que pueden estar completamente equivocadas.
Se trata de situaciones que no controlo muy bien, pero que analizo de
la mejor forma que encuentro, cuando no he entendido muy bien algún detalle
en el desarrollo de las lecciones originales en vídeo.

\section{Problema con el desarrollo original de
  \mathinhead{ \tilde{\Psi}(\xtilde{r})}{ddpsir} en los vídeos}
En la sección~\ref{sect:cua-generador-rotacion-funciones}
(\emph{Generador de rotación para funciones}) del
capítulo~\ref{chapt:cua-momentoangular-cuantica}
(\emph{Momento angular en física cuántica})
se desarrolló en serie de potencias una función $\tilde{\Psi}$ compleja
de variable real definida en $\symbb{R}^3$ cuando se rota un ángulo
infinitesimal $\varepsilon$ alrededor de una dirección $\xhat{n}$.

El desarrollo original en los vídeos de YouTube se planteó de una forma
ligeramente diferente, cambiando el orden de las matrices
\begin{subequations}
\begin{equation}\label{eq:apcua-desarrollo-original}
  \tilde{\Psi}
  = \Psi(\mmm{R}^{-1}\vvv{r})
  = \Psi(\vvv{r} + i\varepsilon(\xhat{n}\cdot\mmm{J})\vvv{r})
    = \Psi(\vvv{r})
    + \left(\frac{\partial\Psi}{\partial\vvv{r}}\right)^{\dagger}
    i\varepsilon (\xhat{n}\cdot\mmm{J})\vvv{r}
\end{equation}

En el vídeo se sustituye el segundo sumando del desarrollo
\begin{equation}\label{eq:apcua-segundo-sumando}
  \left(\frac{\partial\Psi}{\partial\vvv{r}}\right)^{\dagger}
  i\varepsilon (\xhat{n}\cdot\mmm{J})\vvv{r}
\end{equation}
por su traspuesta conjugada
\begin{equation}\label{eq:apcua-z}
  \left[
    \left(\frac{\partial\Psi}{\partial\vvv{r}}\right)^{\dagger}
    i\varepsilon (\xhat{n}\cdot\mmm{J})\vvv{r}
  \right]^\dagger
  =
  -i\varepsilon\vvv{r}^\dagger (\xhat{n}\cdot\mmm{J})
  \frac{\partial\Psi}{\partial\vvv{r}}
\end{equation}

Quedando
\begin{equation}\label{eq:apcua-desarrollo-original-sust}
  \tilde{\Psi}
  = \Psi(\mmm{R}^{-1}\vvv{r})
  = \Psi(\vvv{r} + i\varepsilon(\xhat{n}\cdot\mmm{J})\vvv{r})
    = \Psi(\vvv{r})
    - i\varepsilon\vvv{r}^\dagger (\xhat{n}\cdot\mmm{J})
    \frac{\partial\Psi}{\partial\vvv{r}}
\end{equation}

Bueno, creo que este cambio sería correcto si $\Psi(x,y,z)$ fuera una función
real definida en el espacio euclídeo, pero que si se trata de una función
compleja, habría que sustituirla por
\begin{equation}\label{eq:apcua-zconj}
  -i\varepsilon\vvv{r}^\dagger (\xhat{n}\cdot\mmm{J})
  \frac{\partial\Psi^*}{\partial\vvv{r}}
\end{equation}
que es el opuesto del complejo conjugado de~\eqref{eq:apcua-z}.
Y si esto último fuera correcto, el primer sumando contiene la
función original $\Psi$ y el segundo sumando tendría el complejo
conjugado de la función $\Psi^*$ y no se podría sacar factor común
posteriormente como se puede estudiar en la sección correspondiente.

Voy a intentar explicar la situación.
Obsérvese que el segundo sumando~\eqref{eq:apcua-segundo-sumando} es un
número complejo porque el producto de tres matrices
$(1\times 3) \cdot (3\times 3) \cdot (3\times 1)$
es una matriz de una fila y una columna, $1\times 1$, es decir, un escalar.
Este escalar debe ser un número complejo porque tratamos con funciones
complejas de variable real.

Analizamos Con más detalle las matrices de este sumando y llamaremos $z$ al
complejo que se obtiene
\begin{align*}
  i\varepsilon \left(\frac{\partial\Psi}{\partial\vvv{r}}\right)^{\dagger}
  (\xhat{n}\cdot\mmm{J})\vvv{r}
  &=
    i\varepsilon
    \begin{pmatrix}
      \left(\dfrac{\partial\Psi}{\partial x}\right)^*
      & \left(\dfrac{\partial\Psi}{\partial y}\right)^*
      & \left(\dfrac{\partial\Psi}{\partial z}\right)^*
    \end{pmatrix}
    \begin{pmatrix}
      0 & -in_z & in_y\\
      in_z & 0 & -in_x\\
      -in_y & in_x & 0
    \end{pmatrix}
    \begin{pmatrix}
      x \\ y \\ z
    \end{pmatrix}\\
  &=
    i^2\varepsilon
    \begin{pmatrix}
      \dfrac{\partial\Psi^*}{\partial x}
      & \dfrac{\partial\Psi^*}{\partial y}
      & \dfrac{\partial\Psi^*}{\partial z}
    \end{pmatrix}
    \begin{pmatrix}
      -yn_z+zn_y \\ xn_z-zn_x\\-xn_y+yn_x
    \end{pmatrix}\\
  &=
    \varepsilon
    \left[
    (yn_z-zn_y)\frac{\partial\Psi^*}{\partial x}
    + (zn_x-xn_z)\frac{\partial\Psi^*}{\partial y}
    + (xn_y-yn_x)\frac{\partial\Psi^*}{\partial z}
    \right]\\
  &=
    z\in\symbb{C}
\end{align*}
En la expresión anterior sólo tenemos con valor complejo $\Psi$; los
demás valores son reales.

Ahora vamos a desarrollar de la misma manera la expresión por la que se
sustituyó~\eqref{eq:apcua-z}. Observemos que es justo la traspuesta
conjugada o adjunta de la expresión original. Encontramos que el resultado
es el opuesto del complejo conjugado de la expresión original.
Y sólo cuando la función es real $\Psi^* = \Psi$, se pueden considerar
equivalentes las dos expresiones (porque serían números reales sin componente
imaginaria)
\begin{align*}
  \left[
  i\varepsilon\left(\frac{\partial\Psi}{\partial\vvv{r}}\right)^{\dagger}
  (\xhat{n}\cdot\vvv{J})\vvv{r}
  \right]^{\dagger}
  &=
    i^*\varepsilon
    \vvv{r}^{\dagger} (\xhat{n}\cdot\mmm{J}^{\dagger})
    \frac{\partial\Psi}{\partial\vvv{r}}
  =
    -i\varepsilon
    \vvv{r}^{\dagger} (\xhat{n}\cdot\mmm{J})
    \frac{\partial\Psi}{\partial\vvv{r}}\\
  &=
    i\varepsilon
    \begin{pmatrix}
      x & y & z\\
    \end{pmatrix}
    \begin{pmatrix}
      0 & -in_z & in_y\\
      in_z & 0 & -in_x\\
      -in_y & in_x & 0
    \end{pmatrix}
    \begin{pmatrix}
      \frac{\partial\Psi}{\partial x}\\[1.2ex]
      \frac{\partial\Psi}{\partial y}\\[1.2ex]
      \frac{\partial\Psi}{\partial z}
    \end{pmatrix}\\
  &=
    i^2\varepsilon
    \begin{pmatrix}
      x & y & z
    \end{pmatrix}
    \begin{pmatrix}
      -n_z\frac{\partial\Psi}{\partial y}
      +n_y\frac{\partial\Psi}{\partial z}\\[1.2ex]
      n_z\frac{\partial\Psi}{\partial x}
      -n_x\frac{\partial\Psi}{\partial z}\\[1.2ex]
      -n_y\frac{\partial\Psi}{\partial x}
      +n_x\frac{\partial\Psi}{\partial y}
    \end{pmatrix}\\
%  &=
%    -\varepsilon\left[%
%    -xn_z\frac{\partial\Psi}{\partial y}
%    + xn_y\frac{\partial\Psi}{\partial z}
%    + yn_z\frac{\partial\Psi}{\partial x}
%    - yn_x\frac{\partial\Psi}{\partial z}
%    - zn_y\frac{\partial\Psi}{\partial x}
%    + zn_x\frac{\partial\Psi}{\partial y}
%    \right]\\
  &= -\varepsilon \left[
    (yn_z-zn_y)\frac{\partial\Psi}{\partial x} +
    (zn_x-xn_z)\frac{\partial\Psi}{\partial y} +
    (xn_y-yn_x)\frac{\partial\Psi}{\partial z} \right]\\
  &=
    -z^*\in\symbb{C}
\end{align*}
Por tanto, \eqref{eq:apcua-segundo-sumando}
es distinto de \eqref{eq:apcua-z} y no se puede sustituir directamente,
a menos que la función tenga sólo valores reales.
\end{subequations}







%%% Local Variables:
%%% coding: utf-8
%%% mode: latex
%%% TeX-engine: luatex
%%% TeX-master: "../gruposlie"
%%% End:

