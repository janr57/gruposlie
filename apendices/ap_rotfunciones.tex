% ap_rotfunciones.tex
%
% Copyright (C) 2022--2025 José A. Navarro Ramón <janr.devel@gmail.com>
% Licencia del código GPLv2
% Licencia Creative Commons Recognition Non-Commercial Share-alike.
% (CC-BY-NC-SA)

\chapter{Rotación de funciones}
\label{ap:rotfunciones}

En este apéndice se aprovecha la sección \ref{sect:invariancia-bajo-rotaciones} para ...
desarrollan unos razonamientos acerca del significado relacionado con la rotación de
funciones. No se espere una gran profundidad.

\section{Rotación de puntos de espacios euclídeos}
En los capítulos \ref{ch:gruposlie}, \ref{ch:SO3} y \ref{ch:algebralie}
se estudiaron los grupos de Lie SO(2), SO(3) y, en general SO(N), que describen las
posibles rotaciones en espacios euclídeos.

\subsection{Funciones definidas en \mathinhead{\symbb{R}}{rdulskfR}}
Supongamos una función $y=f(x)$, que a cada punto del eje $x$ le asigna uno y solo
un punto $y$. Los ejes $x$ e $y$ son 



\section{Rotación de una función real definida en
\mathinhead{\symbb{R}^3}{rdufrdeR3}}

En esta sección se analizará la rotación de una función real $f(x, y, z)$, definida en el
espacio euclídeo, alrededor de una dirección dada por el versor $\xhat{n}$.

\subsubsection{¿Qué se entiende por girar una función?}
Girar una función implica lo siguiente:
\begin{enumerate}
\item Deberíamos rotar todos los puntos del espacio $\vvv{r} = (x,y,z)$, donde esté
  definida la función, transformándolos en otros
  $\tilde{r}=(\tilde{x},\tilde{y},\tilde{z})$
  \[
    \vvv{r} = (x,y,z)\hspace{1em}
    \longrightarrow
    \hspace{1em}
    \vvv{\tilde{r}} = (\tilde{x},\tilde{y},\tilde{z})
  \]
  
  Al tratarse de una rotación activa (rotación de vectores), la matriz de rotación es
  \begin{subequations}
  \begin{equation}\label{eq:cla-matrizrotacion-funcion}
    \mmm{R}^{-1} = \mmm{R}^{-1}\!\left(\xhat{n},\theta\right) = e^{-\theta \xhat{n}\cdot\mmm{G}}
  \end{equation}
  y las coordenadas del punto rotado se hallan aplicando esta matriz a las originales
  \begin{equation}\label{eq:cla-rotacion-aciva-funcion-coordenadasrotadas}
    \vvv{\tilde{r}} = \mmm{R}^{-1}\,\vvv{r} = e^{-\theta\xhat{n}\cdot\mmm{G}}\vvv{r}
  \end{equation}
\end{subequations}
Sabemos rotar segmentos orientados mediante matrices, como
en~\eqref{eq:cla-rotacion-aciva-funcion-coordenadasrotadas}, pero una función no es un
segmento orientado; de hecho, no tendría ningún sentido la expresión equivalente para
funciones, si $\mmm{R}^{-1}$ se representara mediante una matriz
\[
  \cancelout{\tilde{f} = \mmm{R}^{-1}\, f}
\]

Es como si una función no tuviera \emph{elementos físicos} que permitieran girarla
mediante la \emph{herramienta} matriz de rotación.
Habrá que idear \emph{otra herramienta} o mecanismo para su giro\footnotemark{}.
\footnotetext{Más adelante se verá que la expresión $\tilde{f} = \mmm{R} f$
  tendrá sentido cuando $\mmm{R}$ se represente como operador diferencial.}

\item ¿Cómo podremos dar con una expresión matemática para la función rotada $\tilde{f}$?
  
  Cada punto, al rotar, debe \emph{llevarse consigo} de alguna manera el valor que tenía
  en la función original $f$\, hasta la rotada $\tilde{f}$; en otras palabras,
  \emph{la nueva función se debe obtener a partir de la original y tener el mismo valor
    para los puntos originales y sus correspondientes rotados}.
  En otras palabras, se debe cumplir $f$ y $\tilde{f}$ tengan el mismo valor
  \begin{equation}\label{eq:cla-funcion-relacion-funcionrotada}
    f(x,y,z) = \tilde{f}(\tilde{x},\tilde{y},\tilde{z})
  \end{equation}
  siempre que $\tilde{x}$, $\tilde{y}$, $\tilde{z}$ sean las coordenadas rotadas del
  punto original $x$, $y$, $z$.

  Obsérvese que una función se define a través de sus argumentos y de la expresión
  matemática que especifica cómo se opera con estos.
  Por ejemplo, en la siguiente función los argumentos son las coordenadas $x$, $y$, $z$ y
  la expresión $x^2 - y + 2xz$ detalla cómo se opera con ellos para obtener el valor de
  $f$
  \[
    f(x,y,z) = x^2 - y + 2xz
  \]
  
  Así para girar la función debemos recurrir a modificar los argumentos y obtener una
  expresión matemática para la función rotada.
  
  Para obtener la expresión matemática de la nueva función
  $\tilde{f}(\tilde{x},\tilde{y},\tilde{z})$, partimos del valor de la función original
  $f(x,y,z)$ y sustituimos las coordenadas originales por sus equivalentes rotadas. Esto
  se realiza mediante la inversa de la
  matriz de rotación~\eqref{eq:cla-matrizrotacion-funcion}
  \begin{subequations}
    \begin{equation}\label{eq:cla-matrizrotacioninversa-funcion}
      \mmm{R}^{-1} = \mmm{R}^{-1}(\xhat{n}, \theta) = e^{\theta \xhat{n}\cdot\mmm{G}}
    \end{equation}
    Se aplica esta matriz a las coordenadas rotadas
    \begin{equation}\label{eq:cla-rotacion-aciva-funcion-coordenadasoriginales}
      \vvv{r}
      = \mmm{R}^{-1} \vvv{\tilde{r}}
      = e^{\theta\xhat{n}\cdot\mmm{G}}\vvv{\tilde{r}}
    \end{equation}
    Se obliga a que se cumpla la
    igualdad~\eqref{eq:cla-funcion-relacion-funcionrotada}
    \begin{equation}\label{eq:cla-rotacion-de-funcion}
      f(\vvv{r}) = f(\mmm{R}^{-1}\vvv{r}) = \tilde{f}(\vvv{\tilde{r}})
    \end{equation}
  \end{subequations}

  Se podría vislumbrar la necesidad de la rotación inversa mediante el siguiente
  argumento, que se intenta ilustrar a través de la
  figura~\eqref{fig:cla-calculo-f-girada}:

  Para calcular el valor de la función girada en el punto $\tilde{P}$,
  $\tilde{f}(\vvv{\tilde{r}})$, se le aplica la rotación inversa
  $\mmm{R}^{-1}$ al punto rotado, para obtener el punto original $P$ y
  entonces calcular el valor de la función original en éste, $f(\vvv{r})$
  \begin{figure}[ht]
    \def\scl{1}
    % Eje x
    \pgfmathsetmacro{\XMLONG}{0}
    \pgfmathsetmacro{\XPLONG}{3}
    % Eje y
    \pgfmathsetmacro{\YMLONG}{0}
    \pgfmathsetmacro{\YPLONG}{3}
    % Ángulo rotado
    \pgfmathsetmacro{\ANGROT}{20}
    % Vector P'
    \pgfmathsetmacro{\PPRIMAMOD}{2.5}
    \pgfmathsetmacro{\PPRIMAANG}{60}
    % Vector P
    \pgfmathsetmacro{\PMOD}{\PPRIMAMOD}
    \pgfmathsetmacro{\PANG}{\PPRIMAANG - \ANGROT}
    % Fondo
    \pgfmathsetmacro{\HORZ}{0.5}
    \pgfmathsetmacro{\VERT}{0.5}
    % 
    \centering
    \begin{tikzpicture}[%
      scale=\scl,
      every node/.style={black,font=\small},
      eje/.style={->},
      vector/.style={-{Latex}, shorten >=1.2pt, line width=.8pt},
      vectorrotado/.style={vector, draw=green!50!black},
      pcirculo/.style={fill=red, draw=black},
      pprimacirculo/.style={green!90!black, draw=black},      
      background/.style={
        line width=\bgborderwidth,
        draw=\bgbordercolor,
        fill=\bgcolor,
      },      
      ]
      % Coordenadas
      \coordinate (O) at (0,0);
      \coordinate (under_origin) at (0,-3mm);
      \coordinate (left_origin) at (-3mm,0);
      \coordinate (xini) at (-\XMLONG cm,0);
      \coordinate (xfin) at (\XPLONG cm,0);
      \coordinate (yini) at (0,-\YMLONG cm);
      \coordinate (yfin) at (0,\YPLONG cm);
      \coordinate (P') at (\PPRIMAANG:\PPRIMAMOD cm);
      \coordinate (P) at  (\PANG:\PMOD);
      \path (O) -- coordinate (OPmidway) (P);
      \path (O) -- coordinate (OP'midway) (P');
      \path (O) -- coordinate[pos=1.2] (parrow) (P);
      \path (O) -- coordinate[pos=1.2] (p'arrow) (P');
      % Ángulo \varepsilon
      \path (P) -- (O) -- (P') pic
      [draw=black!50!,fill=green!20,"\footnotesize $\varepsilon$",angle
      radius=6mm, angle eccentricity=1.5] {angle = P--O--P'};
      % Ejes
      \draw[eje] (xini) -- (xfin);
      \node[right, name=letraejex] at (xfin) {$x$};
      \draw[eje] (yini) -- (yfin);
      \node[above, name=letraejey] at (yfin) {$y$};
      % Punto P'
      \draw[vectorrotado] (O) -- (P');
      \node[above=5pt] at (OP'midway) {$\vvv{\tilde{r}}$};
      \fill[pprimacirculo] (P') circle [radius=1.4pt];
      \node[above] at (P') {$\tilde{P}$};

      % Punto P
      \draw[vector] (O) -- (P);
      \node[right=0pt] at (OPmidway) {$\vvv{r}$};
      \fill[pcirculo] (P) circle [radius=1.4pt];
      \node[right] at (P) {$P$};

      % Función rotada y original
      \node[above] at (p'arrow) {$\tilde{f}(\vvv{\tilde{r}})$};
      \node[right] at (parrow) {$f(\vvv{r})$};
      
      % Sentido de giro del vector
      \draw [-{Latex},green!40!black,shorten <= 3pt]
      (p'arrow) to[bend left=30] coordinate (minversa) (parrow);

      % Matriz inversa
      \node[above right] at (minversa)
      {\footnotesize $\mmm{R}^{-1}\vvv{\tilde{r}}$};

      % Fondo amarillo
      \coordinate (SW) at ($(current bounding box.south west) + (-\HORZ cm,-\VERT cm)$);
      \coordinate (NE) at ($(current bounding box.north east) + (\HORZ cm,\VERT cm)$);    
      \begin{scope}[on background layer]
        \draw[background] (SW) rectangle (NE);
      \end{scope}
    \end{tikzpicture}
    \caption{Para calcular el valor de la función rotada $\tilde{f}$ en el punto
      $\tilde{P}$, esto es, ($\tilde{f}(\vvv{\tilde{r}}$), giramos el punto
      \emph{en sentido contrario} al giro que se le dio a la función para localizar el
      punto original $P$ y se calcula el valor de la función original en ese punto $P$
      sin girar $f(\vvv{r}=$. Finalmente se igualan los valores
      $\tilde{f}(\vvv{\tilde{r}}) = f(\vvv{r})$.}
    \label{fig:cla-calculo-f-girada}
  \end{figure}

\end{enumerate}

Es importante que se comprenda que
\emph{al transformar las coordenadas originales $x$, $y$, $z$ de la función mediante la
  transformación inversa $\mmm{R}^{-1}$, estamos rotando la función según la
  transformación directa $\mmm{R}$}.

Para terminar esta explicación podríamos dar una \emph{receta} para rotar una función:
\begin{itemize}
\item Identifíquese una matriz de rotación $\mmm{R}$ para la función.
\item Aplique la inversa de la matriz anterior $\mmm{R}^{-1}$, para expresar las
  coordenadas originales en función de las nuevas
  $\vvv{r} = \mmm{R}^{-1} \vvv{\tilde{r}}$.
\item Sustituya las coordenadas originales en la función original por sus equivalentes
  obtenidas en el punto anterior, para obtener la función rotada,
  $\tilde{f}(\vvv{\tilde{r}})=f(\mmm{R}^{-1}\vvv{\tilde{r}})$
\item Opcionalmente, reescriba la función rotada quitando las tildes a las coordenadas,
  para obtener $\tilde{f}(x,y,z)$.
\end{itemize}

\subsubsection{Primer ejemplo}
Rotaremos la función $f(x,y,z) = 2x+3y-z$ un ángulo de \ang{90} alrededor del eje $z$.

Identificamos la matriz de rotación
\begin{equation}\label{eq:cla-matriz-ej1}
  \mmm{R}_z
  =
  \mmm{R}_z(\pi/2)
  = e^{-\frac{\pi}{2}\/\mmm{G}_z}
  = \begin{pmatrix}
    \cos(\pi/2) & -\sin(\pi/2) & 0\\
    \sin(\pi/2) & \cos(\pi/2) & 0\\
    0 & 0 & 1
  \end{pmatrix}
  = \begin{pmatrix}
    0 & -1 & 0\\
    1 &  0 & 0\\
    0 &  0 & 1
  \end{pmatrix}                
\end{equation}

La inversa de la matriz, la traspuesta porque es una matriz ortogonal, es
\begin{equation}\label{eq:cla-matrizinversa-ej1}
  \mmm{R}_z^{-1}
  = \mmm{R}_z^\trasp
  = \begin{pmatrix}
    0 &  1 & 0\\
    -1 &  0 & 0\\
    0 &  0 & 1
  \end{pmatrix}    
\end{equation}

Aplicamos~\eqref{eq:cla-matrizinversa-ej1} para obtener las coordenadas originales en
función de las rotadas
\[
  \vvv{r} = \mmm{R}_z^{-1} \vvv{\tilde{r}}
\]
\[
  \begin{pmatrix}
    x\\
    y\\
    z
  \end{pmatrix}
  = \begin{pmatrix}
    0 &  1 & 0\\
    -1 &  0 & 0\\
    0 &  0 & 1      
  \end{pmatrix}
  \begin{pmatrix}
    \tilde{x}\\
    \tilde{y}\\
    \tilde{z}
  \end{pmatrix}
\]
produciendo las siguientes relaciones
\begin{align*}
  x &= \tilde{y}\\
  y &= -\tilde{x}\\
  z &= \tilde{z}
\end{align*}

La función transformada que buscamos se obtiene sustituyendo las coordenadas originales
en función de las rotadas, obligando a que tenga el mismo valor que la original en los
puntos originales y rotados
\[
  \tilde{f}(\tilde{x},\tilde{y},\tilde{z})
  = f(x,y,z)
\]
\[
  \tilde{f}(\tilde{x}, \tilde{y}, \tilde{z})
  = f(x,y,z)
  = 2x + 3y - z
  = 2\tilde{y} + 3 (-\tilde{x}) - \tilde{z}
  = 2\tilde{y} + 3\tilde{x} - \tilde{z}
\]

Los argumentos de la función (coordenadas) las podemos reescribir en la
función rotada $\tilde{f}$
\[
  \tilde{x}\rightarrow x;\hspace{1em}
  \tilde{y}\rightarrow y;\hspace{1em}
  \tilde{z}\rightarrow z
\]
\[
  \tilde{f}(x, y, z) = 2y + 3x - z
\]

Para terminar, vamos a aseguranos de que hemos obtenido la función rotada correcta.
Para comprobarlo, calcularemos el valor de la función en un punto, por ejemplo
$\vvv{r} = (1,0,0)$
\[
  f(1,0,0) = 2\cdot 1 + 3\cdot 0 - 0 = 2
\]

Ahora giramos el punto (vector) usando la matriz de rotación activa $R_z(\pi/2)$
\[
  \begin{pmatrix}
    \tilde{x}\\
    \tilde{y}\\
    \tilde{z}
  \end{pmatrix}
  = \begin{pmatrix}
    0 & -1 & 0\\
    1 &  0 & 0\\
    0 &  0 & 1      
  \end{pmatrix}
  \begin{pmatrix}
    1\\
    0\\
    0
  \end{pmatrix}
  = \begin{pmatrix}
    0\\
    1\\
    0
  \end{pmatrix}
\]

El punto rotado es $\vvv{\tilde{r}} = (0,1,0)$
\[
  \vvv{r}(1,0,0)
  \longrightarrow
  \vvv{r}'(0,1,0)
\]
Finalmente, calculamos el valor de la función rotada
$\tilde{f}(x,y,z) = 2y + 3x - z$ y debería dar el mismo resultado que la función original
aplicada al punto sin rotar $f(1,0,0)=2$, véase la figura~\ref{fig:cla-rot-funcion-ej-1}
\[
  \tilde{f}(0,1,0) = 2\cdot 1 + 3\cdot 0 - 0 = 2
\]
%% .......................................................................
\begin{figure}[ht]
  \def\scl{1}
  \def\longeje{3}
  % 
  \tdplotsetmaincoords{60}{110}
  % 
  \pgfmathsetmacro{\rvec}{1.9}
  \pgfmathsetmacro{\thetavec}{90}
  \pgfmathsetmacro{\phivec}{90}
  % Fondo
  \pgfmathsetmacro{\HORZ}{0.25}
  \pgfmathsetmacro{\VERT}{0.25}
  % 
  \tikzfading[%
  name=fade out, inner color=transparent!0, outer color=transparent!100
  ]
  
  \centering
  \begin{tikzpicture}[%
    scale=\scl,tdplot_main_coords,
    axisrotation/.style={%
      line width=1.6pt,
      -{Latex[round,length=12pt,width=7.0pt,bend]},
      color=green!85!black},
    angulogiro/.style={%
      line width=1.4pt,-{Latex[round,length=7pt,width=4.5pt,bend]},green!85!black},
    background/.style={%
      line width=\bgborderwidth,
      draw=\bgbordercolor,
      fill=\bgcolor,
    },
    ]
    % SISTEMA ORIGINAL (A LA IZQUIERDA)
    % Para 'tikz-3dplot' son los ejes principales
    \draw[very thick,->] (0,0,0) -- (\longeje,0,0) coordinate (ejex);
    \node[below left,name=letraejex] at (ejex) {$x$};
    \draw[very thick,->] (0,0,0) -- (0,\longeje,0) coordinate (ejey);
    \node[below right,name=letraejey] at (ejey) {$y$};
    \draw[very thick,->] (0,0,0) -- (0,0,\longeje) coordinate (ejez);
    \node[above,name=letraejez] at (ejez){$z$};
    
    % REPRESENTACIÓN DEL GIRO ALREDEDOR DEL EJE Z
    % \tdplotsetthetaplanecoords{-90}
    % Notice you have to tell tiks-3dplot you are now in rotated coords
    % Since tikz-3dplot swaps the planes in tdplotsetthetaplanecoords,
    % the former y axis is now the z axis.
    \tdplotdrawarc[axisrotation] {(0,0,2.2)}{0.35}{70}{410}{}

    % Reponer parte final del eje z para simular perspectiva
    \draw[very thick,->] (0,0,2.2) -- (0,0,\longeje);  
    
    % Vector de posición del punto P en gris
    \draw[-{Latex[width=7pt]},black!50,line width=2pt]
    (0,0,0) -- node[above left=0pt and -3pt] {\small $\vvv{r}$}
    (\rvec,0,0) coordinate (P);
    \filldraw[fill=gray, line width=0.4pt] (P) circle[radius=1.3pt];
    \node[above right=-8pt and 0pt,black!50] at (P) {\small $P$};
    \node[above left=0pt and 5pt] at (P)
    {\footnotesize $f(\vvv{r}) = 2$};
    
    % Vector de posición del punto (P') rotado en rojo
    \draw[-{Latex[width=7pt]},red!90!black,line width=2pt]
    (0,0,0) -- node[above left=0pt and -3pt] {\small $\vvv{\tilde{r}}$}
    (0,\rvec,0) coordinate (P');
    \filldraw[fill=red, line width=0.4pt] (P') circle[radius=1.3pt];
    \node[below left] at (P') {\small $\tilde{P}$};
    \node[above right=4pt and 0pt] at (P')
    {\footnotesize $\tilde{f}(\vvv{\tilde{r}}) = 2$};
    
    % ÁNGULO PHI 
    \tdplotsetthetaplanecoords{\thetavec}
    % Dibuja el ángulo \phi y etiquétalo
    % sintaxis:
    % \tdplotdrawarc[coordinate frame, draw options]
    % {center point}{r}{angle}{label options}{label}
    \tdplotdrawarc[angulogiro]
    {(0,0,0)}{0.8}{8}{\phivec-10}{anchor=north}
    {\textcolor{green!50!black}{\footnotesize $\pi/2$}};
    % Texto de rotación
    \node[right=1.5em] at (0,0,2.4) {\small Rotación activa de};
    \node[right=1.5em] at (0,0,2.0) {\small $f= 2x+3y-z$};

    % Fondo amarillo
    \coordinate (SW) at ($(current bounding box.south west) + (-\HORZ cm,-\VERT cm)$);
    \coordinate (NE) at ($(current bounding box.north east) + (\HORZ cm,\VERT cm)$);    
    \begin{scope}[on background layer]
      \draw[background] (SW) rectangle (NE);
    \end{scope}
  \end{tikzpicture}
  \caption{Rotación activa de la función $f(x,y,z) = 2x + 3y -z$, del ejemplo 1,
    alrededor del eje $z$, un ángulo de \ang{90}, que se representa mediante la rotación
    de un punto $P(1,0,0)$ hasta $\tilde{P}(0,1,0)$ del espacio tridimensional.}
  \label{fig:cla-rot-funcion-ej-1}
\end{figure}

\subsubsection{Segundo ejemplo}
Rotar la función $f(x,y,z) = 2x+3y-z$ un ángulo $\theta$ alrededor del eje $z$.
  
La matriz de rotación y su inversa en función del ángulo son
\[
  \mmm{R}_z
  = \begin{pmatrix}
    \cos\theta & -\sin\theta & 0\\
    \sin\theta &  \cos\theta & 0\\
    0 &  0 & 1
  \end{pmatrix}
  \hspace*{3em}
  \mmm{R}_z^{-1}
  = \begin{pmatrix}
    \cos\theta & \sin\theta & 0\\
    -\sin\theta &  \cos\theta & 0\\
    0 &  0 & 1
  \end{pmatrix}    
\]

Las coordenadas originales en función de las transformadas se obtendrían mediante la
matriz inversa
\[
  \begin{pmatrix}
    x\\
    y\\
    z
  \end{pmatrix}
  = \begin{pmatrix}
    \cos\theta & \sin\theta & 0\\
    -\sin\theta &  \cos\theta & 0\\
    0 &  0 & 1
  \end{pmatrix}
  \begin{pmatrix}
    \tilde{x}\\
    \tilde{y}\\
    \tilde{z}
  \end{pmatrix}
  = \begin{pmatrix}
    \tilde{x}\cos\theta + \tilde{y}\sin\theta\\
    -\tilde{x}\sin\theta + \tilde{y}\cos\theta\\
    z
  \end{pmatrix}
\]

La función rotada queda
\begin{align*}
  \tilde{f}(\tilde{x}, \tilde{y}, \tilde{z})
  &=
    2x + 3y - z
    =
    2(\tilde{x}\cos\theta + \tilde{y}\sin\theta)
    + 3(-\tilde{x}\sin\theta + \tilde{y}\cos\theta)
    - \tilde{z}\\
  &=
    (2\cos\theta-3\sin\theta)\,\tilde{x}+(2\sin\theta+3\cos\theta)\,\tilde{y}
    - \tilde{z}
\end{align*}

Ahora comprobaremos que la anterior es la función rotada correcta, calculando el valor de
la función original en un punto genérico $\vvv{r}_0 = (x_0, y_0, z_0)$
\[
  f(\vvv{r}_0) = f(x_0, y_0, z_0) = 2x_0 + 3y_0 -z_0
\]

Ahora rotamos el punto $\vvv{r}_0 = (x_0, y_0, z_0)$ utilizando la matriz de rotación
activa $\mmm{R}_z$, obteniendo
\[
  \vvv{r}_0 = (x_0, y_0, z_0)
  \longrightarrow
  \vvv{\tilde{r}}_0 = (\cos\theta\,x_0 - \sin\theta\,y_0,
  \sin\theta\,x_0 + \cos\theta\,y_0,
  z_0)
\]

La función rotada en el nuevo punto da
{\footnotesize
  \begin{align*}
    \tilde{f}
    &=
      \tilde{f}(\tilde{x}_0, \tilde{y}_0, \tilde{z}_0)
    =
      f(\cos\theta\,x_0-\sin\theta\,y_0,\sin\theta\,x_0+\cos\theta\,y_0,z_0)\\
    &=
      (2\cos\theta-3\sin\theta)(\cos\theta\,x_0-\sin\theta\,y_0)
      + (2\sin\theta+3\cos\theta)(\sin\theta\,x_0+\cos\theta\,y_0)
      - z_0\\
    &=
      (2\cos^2\theta
      - \cancelout{3\sin\theta\cos\theta}
      + 2\sin^2\theta+\cancelout{3\sin\theta\cos\theta})
      \,x_0\\
    &\hspace*{1.0em}
      + (\cancelout{-2\sin\theta\cos\theta}
      + 3\sin^2\theta
      + \cancelout{2\sin\theta\cos\theta}+3\cos^2\theta)
      \,y_0
      -z_0\\
    &=
      2x_0 + 3y_0 - z_0
  \end{align*}
}
que coincide con el valor de la función $f(x_0,y_0,z_0) = 2x_0+3y_0-z_0$.

\subsubsection{Tercer ejemplo}
Nos proponemos rotar la función $f(x,y,z) = 5x-4y+z$ un ángulo de
$\ang{60}=\pi/3\,\si{\radian}$ alrededor del eje $z$.

El valor de la función un punto cualquiera, como $\vvv{r} = P(3,2,1)$ es
\begin{equation}\label{eq:cla-valor-funcion-ej3}
  f(\vvv{r}) = f(3,2,1) = 5\cdot 3 - 4\cdot 2 + 1 = 15 - 8 + 1 = 8
\end{equation}

Rotamos este punto mediante la matriz
\begin{equation}\label{eq:cla-matriz-ej3}
  \mmm{R}
  = \mmm{R}((0,0,1), \pi/3)
  = e^{-\frac{\pi}{3}\,\mmm{G}_z}
  = \begin{pmatrix}
    \cos(\pi/3) & -\sin(\pi/3) & 0\\
    \sin(\pi/3) & \cos(\pi/3) & 0\\
    0 & 0 & 1
  \end{pmatrix}
  = \begin{pmatrix}
    1/2 & -\sqrt{3}/2 & 0\\
    \sqrt{3}/2 & 1/2 & 0\\
    0 & 0 & 1
  \end{pmatrix}
\end{equation}
convirtiéndose en
\[
  \begin{pmatrix}
    \tilde{x}\\
    \tilde{y}\\
    \tilde{z}
  \end{pmatrix}
  = \begin{pmatrix}
    1/2 & -\sqrt{3}/2 & 0\\
    \sqrt{3}/2 & 1/2 & 0\\
    0 & 0 & 1
  \end{pmatrix}
  \begin{pmatrix}
    3\\
    2\\
    1
  \end{pmatrix}
  = \begin{pmatrix}
    (3-2\sqrt{3})/2\\
    (3\sqrt{3}+2)/2\\
    1
  \end{pmatrix}
\]
\[
  \vvv{r} = P(3,2,1)
  \longrightarrow
  \vvv{\tilde{r}}
  = \tilde{P}\left(\frac{3-2\sqrt{3}}{2}, \frac{3\sqrt{3}+2}{2}, 1\right)
\]

Tenemos que encontrar la función rotada
$\tilde{f}(\tilde{x},\tilde{y},\tilde{z})$ de manera que se cumpla
\[
  \tilde{f}(\vvv{\tilde{r}}) = f(\vvv{r}) = 8
\]

Para encontrar la función para cualquier valor de los argumentos, aplicamos la matriz
inversa de~(\ref{eq:cla-matriz-ej3}) para encontrar las coordenadas originales en función
de las rotadas
\[
  \begin{pmatrix}
    x\\
    y\\
    z
  \end{pmatrix}
  = \begin{pmatrix}
    1/2 & \sqrt{3}/2 & 0\\
    -\sqrt{3}/2 & 1/2 & 0\\
    0 & 0 & 1
  \end{pmatrix}
  \begin{pmatrix}
    \tilde{x}\\
    \tilde{y}\\
    \tilde{z}
  \end{pmatrix}
  = \begin{pmatrix}
    \frac{1}{2}\tilde{x} + \frac{\sqrt{3}}{2}\tilde{y}\\
    \frac{-\sqrt{3}}{2}\tilde{x} + \frac{1}{2}\tilde{y}\\
    \tilde{z}
  \end{pmatrix}
\]

Sustituimos estas coordenadas en la función original, para obtener la función rotada
\begin{align*}
  \tilde{f}(\tilde{x},\tilde{y},\tilde{z})
  &=
    5x - 4y + z
    = 5\left(\frac{1}{2}\tilde{x} + \frac{\sqrt{3}}{2}\tilde{y}\right)
    - 4\left(\frac{-\sqrt{3}}{2}\tilde{x} + \frac{1}{2}\tilde{y}\right)
    + \tilde{z}\\
  &=
    \frac{5}{2}\tilde{x} + \frac{5\sqrt{3}}{2}\tilde{y}
    + \frac{4\sqrt{3}}{2}\tilde{x} - \frac{4}{2}\tilde{y}
    + \tilde{z}
\end{align*}

Operando la expresión anterior, la función rotada queda
\[
  \tilde{f}(\tilde{x},\tilde{y},\tilde{z})
  = \frac{5+4\sqrt{3}}{2}\tilde{x}
  + \frac{5\sqrt{3}-4}{2}\tilde{y}
  + \tilde{z}
\]

Los argumentos de la  función son mudos y en realidad son coordenadas cartesianas, se
pueden reescribir 
$\tilde{x}\rightarrow x$, $\tilde{y}\rightarrow y$ y $\tilde{z}\rightarrow z$
quedando la función rotada en su forma final
\begin{equation}\label{eq:cla-functionrotada-ej3}
  \tilde{f}(x,y,z)
  = \frac{5+4\sqrt{3}}{2} x
  + \frac{5\sqrt{3}-4}{2} y
  + z
\end{equation}

Por último, comprobamos que el valor de la función rotada en el punto rotado $\tilde{P}$
es el mismo que el de la función original en el punto $P$,
ver~\eqref{eq:cla-valor-funcion-ej3}
\begin{align*}
  \tilde{f}(\vvv{\tilde{r}})
  &=
  \tilde{f}\left(\frac{3-2\sqrt{3}}{2},\frac{3\sqrt{3}+2}{2},1\right)\\
  &=
    \frac{5+4\sqrt{3}}{2}\cdot \frac{3-2\sqrt{3}}{2}
    + \frac{5\sqrt{3}-4}{2}\cdot \frac{3\sqrt{3}+2}{2}
    + 1\\
  &=
    \frac{15-\cancelout{10\sqrt{3}}+\cancelout{12\sqrt{3}}-24}{4}
    + \frac{45+\cancelout{10\sqrt{3}}-\cancelout{12\sqrt{3}-8}}{4}
    + 1\\
  &=
    \frac{15-24+45-8}{4} + 1 = \frac{28}{4} + 1 = 7 + 1 = 8
\end{align*}

\subsubsection{Cuarto ejemplo}
Rotar la función un ángulo $\theta$ alrededor del eje $z$
\[
  f(x,y,z) = 2xz^2 - x^2y
\]

La matriz de rotación activa que querríamos aplicar a la función es
\[
  \mmm{R}_z(\theta) = e^{-\theta \mmm{G}_z}
\]

Necesitamos la matriz inversa para obtener la expresión rotada de la función
\[
  \mmm{R}_z^{-1}(\theta)
  =
  e^{\theta \mmm{G}_z}
  = \begin{pmatrix}
    \cos\theta & \sin\theta & 0\\
    -\sin\theta & \cos\theta & 0\\
    0 & 0 & 1
  \end{pmatrix}
\]

Las coordenadas originales se pueden poner en función de las rotadas mediante la matriz
inversa
\[
  \vvv{r}
  = \mmm{R}_z^{-1}(\theta)\,\vvv{\tilde{r}}
  = e^{\theta\mmm{G}_z} \,\vvv{\tilde{r}}
\]
\[
  \begin{pmatrix}
    x \\
    y \\
    z
  \end{pmatrix}
  = \begin{pmatrix}
    \cos\theta & \sin\theta & 0\\
    -\sin\theta & \cos\theta & 0\\
    0 & 0 & 1
  \end{pmatrix}
  \begin{pmatrix}
    \tilde{x} \\
    \tilde{y} \\
    \tilde{z}
  \end{pmatrix}
  = \begin{pmatrix}
    \tilde{x}\cos\theta + \tilde{y}\sin\theta\\
    -\tilde{x}\sin\theta\,\tilde{x} + \tilde{y}\cos\theta\\
    \tilde{z}
    \end{pmatrix}
\]

Produciendo las siguientes transformaciones de coordenadas
\begin{align*}
  x &= \tilde{x}\cos\theta + \tilde{y}\sin\theta\\
  y &= -\tilde{x}\sin\theta + \tilde{y}\cos\theta\\
  z &= \tilde{z}
\end{align*}

La función rotada debe tener el mismo valor que la original con las coordenadas sin rotar
\begin{align*}
  \tilde{f}(\tilde{x}, \tilde{y}, \tilde{z})
  &= f(x,y,z) =
    f(
    \tilde{x}\cos\theta+\tilde{y}\sin\theta
    ,-\tilde{x}\sin\theta+\tilde{y}\cos\theta
    ,\tilde{z}
    )\\
  &=
    2\kern1pt\tilde{z}^2 \kern1pt(\tilde{x}\cos\theta+\tilde{y}\sin\theta)
    - (\tilde{x}\cos\theta+\tilde{y}\sin\theta)^2
    (-\tilde{x}\sin\theta+\tilde{y}\cos\theta)
\end{align*}

Y se puede escribir como
\[
  \tilde{f}(x,y,z)
  = 
    2z^2 (x\cos\theta+y\sin\theta)
    - (x\cos\theta+y\sin\theta)^2
    (-x\sin\theta+y\cos\theta)
\]


Si $\theta=\pi/2\,\si{\radian} = \ang{90}$
{\small
\begin{align*}
  \tilde{f}(x, y, z)
  &=
    2z^2 \left(x\cos\frac{\pi}{2}+y\sin\frac{\pi}{2}\right)
    - \left(x\cos\frac{\pi}{2}+y\sin\frac{\pi}{2}\right)^2
    \left(-x\sin\frac{\pi}{2}+y\cos\frac{\pi}{2}\right)\\
  &=
    2z^2y
    -y^2
    (-x)
    = 2yz^2 + xy^2
\end{align*}
}






%%% Local Variables:
%%% coding: utf-8
%%% mode: latex
%%% TeX-engine: luatex
%%% TeX-master: "../gruposlie.tex"
%%% End:

