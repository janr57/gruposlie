% prefacio.tex
%
% Copyright (C) 2022--2025 José A. Navarro Ramón <janr.devel@gmail.com>
% Licencia Creative Commons Recognition Share-alike.
% (CC-BY-SA)

\chapter{Prefacio}
Este texto consiste en unos apuntes personales tomados de unas lecciones en vídeo
publicadas en
\href{%
  https://www.youtube.com/watch?v=4NE-KNwHKSI\&list=PLAnA8FVrBl8DTFTMP8kXbDnRJHQKqfjaw
}
{Youtube}
por \href{https://www.patreon.com/JavierGarcia/}{Javier García}.

En principio he escrito el texto para que me fuera útil, aunque también lo podría ser
para otras personas.

Confecciono estos apuntes por un motivo doble: el primero es para poder repasar las
lecciones o alguna parte de ellas de forma ágil; el segundo, porque su redacción me
ayuda a asimilar mejor el contenido ---como efecto adverso , al redactarlo a mi
conveniencia se podrían generar errores e inexactitudes---.

Si alguien diera con este texto y le interesara, recomiendo que estudie primero las
lecciones en vídeo, pues tienen valor pedagógico que no se puede transcribir en unos
apuntes.

Cuando se intenta aprender esta materia de forma autodidacta, cuesta mucho avanzar y se
suele terminar perdido. El autor de las lecciones introduce los grupos de Lie partiendo
de conocimientos básicos de bachillerato o de primero de carrera y desde mi punto de
vista consigue su objetivo. En todo caso, sirven su propósito para iniciarse en los
grupos de Lie de forma efectiva, de manera que posteriormente podría ampliar estos
conocimientos.

En algunos puntos del texto he añadido algo, y en otros he eliminado alguna explicación
o desarrollo, para adaptar el texto a mi proceso de aprendizaje.

Los apuntes están confeccionados utilizando Lua\LaTeX\ y están siendo continuamente
corregidos con cada relectura del material. Por ahora no considero que estén listos para
ser publicados y los marco como un borrador.


 
%%% Local Variables:
%%% mode: latex
%%% TeX-engine: luatex
%%% TeX-master: "../gruposlie.tex"
%%% End:
