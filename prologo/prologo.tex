\chapter{Prólogo}
Este texto consiste en unos apuntes personales tomados
de unas lecciones en vídeo publicadas en
\href{%
  https://www.youtube.com/watch?v=4NE-KNwHKSI\&list=PLAnA8FVrBl8DTFTMP8kXbDnRJHQKqfjaw
}
{Youtube}
por \href{https://www.patreon.com/JavierGarcia/}{Javier García}.

Estos apuntes me proporcionan dos objetivos, el primero es poder repasar las
lecciones o alguna parte de ellas de forma ágil; por otro lado, su redacción
me ayuda a asimilar mejor el contenido. Un efecto adverso sería que, al
redactarlo a mi conveniencia se generen errores e inexactitudes indeseables.

En principio he escrito el texto para que me fuera útil, aunque también lo
podría ser para otras personas.

Si alguien diera con este texto le sugiero encarecidamente que estudie
primero las lecciones en vídeo, pues tienen valor pedagógico que no se puede
transcribir en unos apuntes.

Quien haya querido aprender esta materia de forma autodidacta se habrá
dado cuenta de que casi siempre termina perdido y le cuesta mucho avanzar.
El autor de las lecciones introduce los grupos de Lie partiendo de
conocimientos básicos de bachillerato o de primero de carrera y desde mi punto
de vista consigue su objetivo.
De hecho, me ha ocurrido que en las sucesivas relecturas de estos apuntes, para
poderlos corregir, me he sentido tentado a añadir más material, aunque no
lo he hecho casi nunca. Lo que quiero decir con esto es que las lecciones
permiten al autodidacta introducirse en los grupos de Lie de forma efectiva
y por esta razón se encontrará,  posteriormente,  en disposición de ampliar
conocimientos.

Los apuntes están confeccionados utilizando Lua\LaTeX y, como ya he observado,
están siendo continuamente corregidos con cada relectura del material.
Por ahora no considero que estén listos para ser publicados, y el contenido
lo considero temporalmente privado. Cuando lo considere oportuno lo pondré
bajo la licencia libre de Creative Commons, aunque aún no he decidido por
cuál de las variantes posibles optaré.

Además, creo que antes de hacerlo debería consultar con el autor original
de las lecciones por si tiene algún inconveniente en que fueran publicadas.
En caso de que no lo considerase oportuno, mantendría los apuntes con una
licencia privada que sólo me permite a mí utilizarlos. Si se publicaran
y se considerase útil, esperaría comentarios y correcciones que una
vez analizados pudieran mejorar el texto.


 
%%% Local Variables:
%%% mode: latex
%%% TeX-engine: luatex
%%% TeX-master: "../gruposlie.tex"
%%% End:
