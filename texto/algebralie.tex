% algebralie.tex
%
% Copyright (C) 2022--2025 José A. Navarro Ramón <janr.devel@gmail.com>
% Licencia del código GPLv2
% Licencia Creative Commons Recognition Non-Commercial Share-alike.
% (CC-BY-NC-SA)

\chapter{Álgebra de Lie}
Cada grupo de Lie lleva asociado una cierta estructura algebraica conocida como su \emph{álgebra de Lie}.
Un álgebra de Lie es un espacio vectorial que tiene una estructura adicional: una
operación binaria llamada corchete de Lie.

Por ejemplo, los elementos del álgebra de Lie del grupo de rotaciones SO(3) se pueden
asimilar a velocidades angulares.
El corchete de Lie para este grupo resultaría ser el producto vectorial.
Para ampliar, consúltese la
\href{https://es.wikipedia.org/wiki/Álgebra_de_Lie}{Wikipedia} o algún texto apropiado.
En el apéndice~\ref{ap:algebralie} se puede encontrar un extracto de la Wikipedia.

\section{Conmutadores}
El grupo de rotaciones no es conmutativo; en general no da el mismo resultado rotar
primero con respecto a un eje y luego con respecto a otro que al revés, a menos que los
dos ejes sean el mismo o algún ángulo de rotación sea cero. Esta observación nos conduce
al concepto de \emph{conmutador}.

Supongamos dos rotaciones, $\mmm{R}_1$ y $\mmm{R}_2$, con respecto a dos ejes diferentes
\begin{align*}
  \mmm{R}_1 &= e^{\mmm{A}}\\
  \mmm{R}_2 &= e^{\mmm{B}}
\end{align*}
donde $\mmm{A}$ y $\mmm{B}$ son matrices de rotación alrededor de un eje y ángulo
arbitrarios.

En general no dará lo mismo si rotamos en distinto orden
\[
  \mmm{R}_2 \mmm{R}_1 \neq \mmm{R}_1 \mmm{R}_2
\]

A la diferencia entre estas operaciones le llamamos \emph{conmutador} de $\mmm{R}_1$ y
$\mmm{R}_2$, y lo escribimos así
\[
  [\mmm{R}_1, \mmm{R}_2] = \mmm{R}_1\mmm{R}_2 - \mmm{R}_2\mmm{R}_1
\]

Si el conmutador de dos elementos de un grupo fuera cero, diríamos que esos dos
elementos y las operaciones que representan, conmutan y daría igual el orden en el que
se aplicaran (en caso contrario no conmutan).
El conocimiento del valor del conmutador es lo que nos va a conducir al
\emph{álgebra de Lie}.

Como ya hicimos anteriormente, la forma más sencilla de estudiar estas operaciones es
utilizando rotaciones infinitesimales porque la expresión matricial es más simple y aun
así no perdemos generalidad; las conclusiones generales seguirían siendo válidas para
rotaciones finitas. Así, ahora consideraremos que $\mmm{A}$ y $\mmm{B}$ son matrices
de rotación infinitesimales. Por tanto
\begin{align*}
  \mmm{R}_1 &= \mmm{I} + \mmm{A}
              + \mathcal{O}\left(\mmm{A}^2\right)
              + \mathcal{O}\left(\mmm{A}^3\right) + \cdots\\
  \mmm{R}_2 &= \mmm{I} + \mmm{B}
              + \mathcal{O}\left(\mmm{B}^2\right)
              + \mathcal{O}\left(\mmm{B}^3\right) + \cdots\\
\end{align*}
Como los giros los consideramos infinitesimales, despreciamos los términos de orden
superior
\begin{align*}
  \mmm{R}_1 &\approx \mmm{I} + \mmm{A}\\
  \mmm{R}_2 &\approx \mmm{I} + \mmm{B}
\end{align*}

Rotamos mediante la segunda y después con respecto a la primera
\[
  \mmm{R}_1 \mmm{R}_2
  \approx (\mmm{I} + \mmm{A})\, (\mmm{I} + \mmm{B})
  = \mmm{I} + \mmm{A} + \mmm{B} + \mmm{A}\mmm{B}
\]

Invertimos el orden de las rotaciones
\[
  \mmm{R}_2 \mmm{R}_1
  \approx (\mmm{I} + \mmm{B})\, (\mmm{I} + \mmm{A})
  = \mmm{I} + \mmm{B} + \mmm{A} + \mmm{B}\mmm{A}
\]

Calculamos el conmutador de ambas rotaciones, es decir, la diferencia entre las dos
operaciones
\begin{equation}\label{eq:alg-conmutador_matrices_infinitesimales}
  [\mmm{R}_1, \mmm{R}_2]
  = \mmm{R}_1 \mmm{R}_2 - \mmm{R}_2 \mmm{R}_1
  \approx \cancelout{\mmm{I}} + \cancelout{\mmm{A}} + \cancelout{\mmm{B}}
  + \mmm{A}\mmm{B} - \cancelout{\mmm{I}} - \cancelout{\mmm{B}}
  - \cancelout{\mmm{A}} - \mmm{BA}
  = \mmm{A}\mmm{B} - \mmm{B}\mmm{A}
\end{equation}

De la expresión anterior y de la definición de conmutador deducimos que
\[
  [\mmm{R}_1,\mmm{R}_2]
  = - [\mmm{R}_2,\mmm{R}_1]
\]

Ahora supongamos que las matrices infinitesimales antisimétricas $\mmm{A}$ y $\mmm{B}$,
que definen a dos operadores de rotación, conmutan
\[
  [\mmm{A},\mmm{B}] = \mmm{A}\mmm{B}-\mmm{B}\mmm{A}=0
\]
entonces, según la ecuación~(\ref{eq:alg-conmutador_matrices_infinitesimales}), las dos
matrices, $\mmm{R}_1$ y $\mmm{R}_2$ también conmutarían.

Se nos podría ocurrir una objección a la afirmación anterior: sabemos que los operadores
infinitesimales tenían términos de orden dos y superiores que se despreciaban
\begin{align*}
  \mmm{R}_1 &= \mmm{I} + \mmm{A} + \mathcal{O}(A^2) + \cdots\\
  \mmm{R}_2 &= \mmm{I} + \mmm{B} + \mathcal{O}(B^2) + \cdots
\end{align*}

Si construimos el conmutador teniendo en cuenta estos términos, nos encontraremos, junto
con lo visto en la ecuación~(\ref{eq:alg-conmutador_matrices_infinitesimales}), términos
infinitesimales adicionales de orden tres y superiores
\[
  [\mmm{R}_1,\mmm{R}_2]
  = \cdots
  = [\mmm{A},\mmm{B}]
  + \mathcal{O}(A^2B) + \mathcal{O}(AB^2) + \cdots
\]

Supongamos que las matrices antisimétricas que definen la rotación conmutan, entonces
\[
  [\mmm{R}_1,\mmm{R}_2] = \mathcal{O}(A^2B) + \mathcal{O}(AB^2) + \cdots
\]

¿No podría ser que esta suma infinita de términos infinitesimales que hemos despreciado
afectara al resultado, haciendo que el conmutador no fuera cero?
La respuesta es que no. La identidad de Baker--Campbell--Hausdorff, es una expresión del
valor de un conmutador como suma infinita de términos
\[
  [\mmm{R}_1,\mmm{R}_2] = [\mmm{A},\mmm{B}] +
  \frac{1}{6}\,\left([\mmm{A},[\mmm{A},\mmm{B}]] + [\mmm{B},
    [\mmm{B},\mmm{A}]]\right) + \cdots
\]

Lo importante de esta expresión es que cada sumando contiene el factor
$[\mmm{A},\mmm{B}]$ de las matrices infinitesimales o su opuesto, de manera que si estas
matrices conmutan, todos los sumandos valdrán cero y los dos operadores también
conmutarán.

Así
\begin{equation}
  [\mmm{R}_1,\mmm{R}_2] = [\mmm{A},\mmm{B}]
\end{equation}


\section{Conmutador de matrices antisimétricas}
En esta sección veremos una forma de expresar el conmutador de dos matrices
antisimétricas, como las $\mmm{A}$ y $\mmm{B}$ del punto anterior, en función de los
generadores de la rotación $\mmm{G}_i$, lo que nos acercará al concepto de
\emph{constantes de estructura de grupo}, que se define y desarrolla en la siguiente
sección.

Por comodidad, y para poder generalizar a grupos de rotación de dimensionalidad más
elevada, denotaremos a los ejes $x$, $y$ y $z$ como $1$, $2$ y $3$, respectivamente.

Supongamos dos operadores de rotación en función de los generadores de SO(3)
\begin{align*}
  \mmm{R}_1 (\alpha)
  &= e^{\mmm{A}} = e^{\alpha\xhat{n}\cdot \mmm{G}}
    = \exp\left\{\alpha (n_1\mmm{G}_1 + n_2\mmm{G}_2 + n_3\mmm{G}_3)\right\}
    = \exp\left\{\alpha\,\sum_{i=1}^3 n_i\mmm{G}_i\right\}\\
  \mmm{R}_2 (\alpha{'})
  &= e^{\mmm{B}} = e^{\alpha{'}\!\xhat{n}{'}\!\cdot \mmm{G}}
    = \exp\left\{
    \alpha{'} (n_1^{'}\mmm{G}_1 + n_2^{'}\mmm{G}_2 + n_3^{'}\mmm{G}_3)
    \right\}
    = \exp\left\{\alpha{'}\sum_{i=1}^3 n_i^{'}\mmm{G}_i\right\}
\end{align*}

El conmutador de las matrices antisimétricas, $\mmm{A}$ y $\mmm{B}$, es
\begin{align*}
  [\mmm{A},\mmm{B}]
  &= \left[
    \alpha\sum_{i=1}^{3} n_i\mmm{G}_i,\alpha^{'}\sum_{j=1}^{3} n^{'}_j\mmm{G}_j
    \right]
    = \alpha \alpha^{'} \sum_{i=1}^{3}\sum_{j=1}^{3}n_in^{'}_j
    \left[\mmm{G}_i,\mmm{G}_j\right]
\end{align*}

Analizaremos qué propiedades tendrá el resultado de $[\mmm{G}_i,\mmm{G}_j]$, sabiendo
que, $\mmm{G}_i$ y $\mmm{G}_j$ son matrices antisimétricas; esto significa que sus
transpuestas son iguales a las respectivas inversas
\[
  \mmm{G}_i^{\trasp} = -\mmm{G}_i
  \hspace{1em}
  \text{y}
  \hspace{1em}
  \mmm{G}_j^{\trasp} = -\mmm{G}_j
\]

Desarrollamos el conmutador
\[
  [\mmm{G}_i,\mmm{G}_j]
  = \mmm{G}_i\mmm{G}_j - \mmm{G}_j\mmm{G}_i
\]

Tomamos la transpuesta de la expresión anterior
\[
  [\mmm{G}_i,\mmm{G}_j]^\trasp
  = (\mmm{G}_i\mmm{G}_j - \mmm{G}_j\mmm{G}_i)^\trasp
\]

Desarrollamos la expresión, teniendo en cuenta que son matrices antisimétricas
\begin{align*}
  [\mmm{G}_i,\mmm{G}_j]^\trasp
  &= (\mmm{G}_i\mmm{G}_j - \mmm{G}_j\mmm{G}_i)^\trasp
    =
    (\mmm{G}_i\mmm{G}_j)^\trasp - (\mmm{G}_j\mmm{G}_i)^\trasp
    =
    \mmm{G}_j^{\trasp} \mmm{G}_i^{\trasp} - \mmm{G}_i^{\trasp} \mmm{G}_j^{\trasp}\\
  &=
    (-\mmm{G}_j) (-\mmm{G}_i) - (-\mmm{G}_i) (-\mmm{G}_j)
    =
    \mmm{G}_j \mmm{G}_i - \mmm{G}_i \mmm{G}_j
    =
    [\mmm{G}_j,\mmm{G}_i]\\
  &= -[\mmm{G}_i,\mmm{G}_j]
\end{align*}

Este resultado es muy importante, pues resulta que el conmutador de dos matrices
antisimétricas, como $[\mmm{G}_i,\mmm{G}_j]$, da como resultado otra matriz
antisimétrica.

Es importante porque las $\mmm{G}_i$ son matrices generadoras que forman una base de
todas las matrices antisimétricas de la dimensión considerada. Esto significa que el resultado del conmutador de matrices generadoras se puede escribir como una combinación
lineal de las matrices generadoras
\begin{equation}\label{eq:alg-conmutador_generadores_ij}
  [\mmm{G}_i,\mmm{G}_j] = \sum_{k=1}^{3}c_{ijk}\mmm{G}_k = c_{ijk}\mmm{G}_k
\end{equation}
en el último término se ha seguido el convenio de la suma de Einstein.


\section{Constantes de estructura de grupo}
Lie se dio cuenta de que la manera en que conmutan los generadores determina
completamente la estructura del grupo.

Las constantes $c_{ijk}$ de la ecuación~(\ref{eq:alg-conmutador_generadores_ij})
determinan la forma en que conmutan los generadores, aunque para que se les llame
constantes de estructura del grupo les falta la unidad imaginaria $i$, donde $i^2=-1$,
(no confundir con el subíndice $i$).
Esto se hace para poder considerar otros grupos en los que aparecen números complejos,
como los grupos unitarios especiales que son fundamentales en mecánica cuántica.
Al final de la sección se dará una explicación un poco más completa.

Para encontrar los coeficientes $c_{ijk}$, proporcionales a las constantes de estructura
se deben calcular todos los conmutadores posibles.
Vamos a utilizar como ejemplo el grupo SO(3).

Los conmutadores que contienen el mismo generador dan cero, pues
toda matriz conmuta consigo misma
\[
  [\mmm{G}_1,\mmm{G}_1]
  =
  [\mmm{G}_2,\mmm{G}_2]
  =
  [\mmm{G}_3,\mmm{G}_3]
  = 0
\]

Realizamos los demás cálculos
\begin{align*}
  [\mmm{G}_1,\mmm{G}_2]
  &=
    \begin{pmatrix}
      0 & 0 & 0\\
      0 & 0 & 1\\
      0 & -1 & 0
    \end{pmatrix}
    \begin{pmatrix}
      0 & 0 & -1\\
      0 & 0 & 0\\
      1 & 0 & 0
    \end{pmatrix}
    -
    \begin{pmatrix}
      0 & 0 & -1\\
      0 & 0 & 0\\
      1 & 0 & 0
    \end{pmatrix}
    \begin{pmatrix}
      0 & 0 & 0\\
      0 & 0 & 1\\
      0 & -1 & 0
    \end{pmatrix}\\
  &=
    \begin{pmatrix}
      0 & 0 & 0\\
      1 & 0 & 0\\
      0 & 0 & 0
    \end{pmatrix}
    -
    \begin{pmatrix}
      0 & 1 & 0\\
      0 & 0 & 0\\
      0 & 0 & 0
    \end{pmatrix}
    =
    \begin{pmatrix}
      0 & -1 & 0\\
      1 & 0 & 0\\
      0 & 0 & 0
    \end{pmatrix}
    = -\mmm{G}_3
\end{align*}
\begin{align*}
  [\mmm{G}_3,\mmm{G}_1]
  &=
    \begin{pmatrix}
      0 & 1 & 0\\
      -1 & 0 & 0\\
      0 & 0 & 0
    \end{pmatrix}
    \begin{pmatrix}
      0 & 0 & 0\\
      0 & 0 & 1\\
      0 & -1 & 0
    \end{pmatrix}
    -
    \begin{pmatrix}
      0 & 0 & 0\\
      0 & 0 & 1\\
      0 & -1 & 0
    \end{pmatrix}
    \begin{pmatrix}
      0 & 1 & 0\\
      -1 & 0 & 0\\
      0 & 0 & 0
    \end{pmatrix}\\
  &=
    \begin{pmatrix}
      0 & 0 & 1\\
      0 & 0 & 0\\
      0 & 0 & 0
    \end{pmatrix}
    -
    \begin{pmatrix}
      0 & 0 & 0\\
      0 & 0 & 0\\
      1 & 0 & 0
    \end{pmatrix}
    =
    \begin{pmatrix}
      0 & 0 & 1\\
      0 & 0 & 0\\
      -1 & 0 & 0
    \end{pmatrix}
    = -\mmm{G}_2
\end{align*}
\begin{align*}
  [\mmm{G}_2,\mmm{G}_3]
  &=
    \begin{pmatrix}
      0 & 0 & -1\\
      0 & 0 & 0\\
      1 & 0 & 0
    \end{pmatrix}
    \begin{pmatrix}
      0 & 1 & 0\\
      -1 & 0 & 0\\
      0 & 0 & 0
    \end{pmatrix}
    -
    \begin{pmatrix}
      0 & 1 & 0\\
      -1 & 0 & 0\\
      0 & 0 & 0
    \end{pmatrix}
    \begin{pmatrix}
      0 & 0 & -1\\
      0 & 0 & 0\\
      1 & 0 & 0
    \end{pmatrix}\\
  &=
    \begin{pmatrix}
      0 & 0 & 0\\
      0 & 0 & 0\\
      0 & 1 & 0
    \end{pmatrix}
    -
    \begin{pmatrix}
      0 & 0 & 0\\
      0 & 0 & 1\\
      0 & 0 & 0
    \end{pmatrix}
    =
    \begin{pmatrix}
      0 & 0 & 0\\
      0 & 0 & -1\\
      0 & 1 & 0
    \end{pmatrix}
    = -\mmm{G}_1
\end{align*}

Escribimos el valor de los conmutadores
\vspace*{-6ex}
\begin{multicols}{2}
  \begin{align*}
    [\mmm{G}_1,\mmm{G}_2] &= -\mmm{G}_3\\
    [\mmm{G}_3,\mmm{G}_1] &= -\mmm{G}_2\\
    [\mmm{G}_2,\mmm{G}_3] &= -\mmm{G}_1
  \end{align*}
  
  \columnbreak
  
  \begin{align*}
    [\mmm{G}_2,\mmm{G}_1] &= \mmm{G}_3\\
    [\mmm{G}_1,\mmm{G}_3] &= \mmm{G}_2\\
    [\mmm{G}_3,\mmm{G}_2] &= \mmm{G}_1
  \end{align*}
\end{multicols}
 
Para que aparezca el número imaginario $i$ en las expresiones y poder obtener los
coeficientes de estructura, llamamos a $\mmm{G}_k = i\kern1pt\mmm{J}_k$
\begin{table}[ht]
  \centering
  \begin{tabular}{lclcl}
    $[\mmm{G}_1,\mmm{G}_2] = -\mmm{G}_3$
    &$\longrightarrow$&
    $[i\kern1pt\mmm{J}_1,i\kern1pt\mmm{J}_2] = -i\kern1pt\mmm{J}_3$
    &$\longrightarrow$&
       $[\mmm{J}_1,\mmm{J}_2] = \phantom{+}i\kern1pt\mmm{J}_3$\\
       $[\mmm{G}_3,\mmm{G}_1] = -\mmm{G}_2$
       &$\longrightarrow$&
       $[i\kern1pt\mmm{J}_3,i\kern1pt\mmm{J}_1] = -i\kern1pt\mmm{J}_2$
       &$\longrightarrow$&
       $[\mmm{J}_3,\mmm{J}_1] = \phantom{+}i\kern1pt\mmm{J}_2$\\
       $[\mmm{G}_2,\mmm{G}_3] = -\mmm{G}_1$
       &$\longrightarrow$&
       $[i\kern1pt\mmm{J}_2,i\kern1pt\kern1pt\mmm{J}_3] = -i\kern1pt\mmm{J}_1$
       &$\longrightarrow$&
       $[\mmm{J}_2,\mmm{J}_3] = \phantom{+}i\kern1pt\mmm{J}_1$\\[1ex]
       $[\mmm{G}_2,\mmm{G}_1] = \phantom{+}\mmm{G}_3$
       &$\longrightarrow$&
       $[i\kern1pt\mmm{J}_2,i\kern1pt\mmm{J}_1] = \phantom{+}i\kern1pt\mmm{J}_3$
       &$\longrightarrow$&
       $[\mmm{J}_2,\mmm{J}_1] = -i\kern1pt\mmm{J}_3$\\
       $[\mmm{G}_1,\mmm{G}_3] = \phantom{+}\mmm{G}_2$
       &$\longrightarrow$&
       $[i\kern1pt\mmm{J}_1,i\kern1pt\mmm{J}_3] = \phantom{+}i\kern1pt\mmm{J}_2$
       &$\longrightarrow$&
       $[\mmm{J}_1,\mmm{J}_3] = -i\kern1pt\mmm{J}_2$\\
       $[\mmm{G}_3,\mmm{G}_2] = \phantom{+}\mmm{G}_1$
       &$\longrightarrow$&
       $[i\kern1pt\mmm{J}_3,i\kern1pt\mmm{J}_2] = \phantom{+}i\kern1pt\mmm{J}_1$
       &$\longrightarrow$&
       $[\mmm{J}_3,\mmm{J}_2] = -i\kern1pt\mmm{J}_1$
  \end{tabular}
\end{table}

Podemos resumir los resultados anteriores mediante la siguiente fórmula, utilizando los
\emph{coeficientes de estructura}, $c_{ijk}$
\begin{equation}\label{eq:alg-j1-j2-coef-estructura}
  [\mmm{J}_i,\mmm{J}_j] = i c_{ijk}\kern1pt\mmm{J}_k
\end{equation}

En SO(3) los $c_{ijk}$ pueden valer $-1$, $0$, $1$.
Resulta que $c_{ijk}$ es el símbolo de Levi-Civita, $c_{ijk} = \epsilon_{ijk}$
\begin{equation}\label{eq:alg-j1-j2-Levi-Civita}
  [\mmm{J}_i,\mmm{J}_j] = i \epsilon_{ijk}\kern1pt\mmm{J}_k
\end{equation}
Es muy fácil recordarlo si consideramos la estructura cíclica $123123\cdots$.
Si en la fórmula anterior, los índices $ijk$ siguen la estructura cíclica, el coeficiente
será el número $+1$; en caso contrario su valor será el opuesto, $-1$.
Si se repiten dos índices su valor es cero.

Ahora las rotaciones pasivas las podemos escribir en función de $\mmm{J}$
\begin{equation}\label{eq:alg-rot-pasiva}
  \mmm{R}(\xhat{n},\theta)
  = \exp\left\{\theta\,\xhat{n}\cdot \mmm{G}\right\}
  = \exp\left\{i\theta\,\/\xhat{n}\cdot \mmm{J}\right\}
\end{equation}

Cuando lo que rotan son los objetos (vectores, puntos, funciones, ...), la matriz es la
inversa de la anterior (el signo negativo se debe a que los ejes giran en sentido
negativo)
\begin{equation}\label{eq:alg-rot-activa}
  \mmm{R}(\xhat{n},\theta)
  = \exp\left\{-\theta\,\xhat{n}\cdot \mmm{G}\right\}
  = \exp\left\{-i\theta\,\/\xhat{n}\cdot \mmm{J}\right\}
\end{equation}

¿Cómo podemos distinguir entre ellas? Si el signo del exponente de la matriz de rotación
es positivo, son los ejes los que rotan en torno a $\xhat{n}$.
Si fuera negativo, son los objetos (vectores, puntos, funciones, ...) los que giran.

Las tres matrices generadoras describen rotaciones alrededor de cada uno de los ejex $x$,
$y$ y $z$, que llamamos $1$, $2$ y $3$. Teniendo en cuenta que
\[
  \mmm{G}_k = i\kern1pt\mmm{J}_k
  ;\hspace{1em}
  \mmm{J}_k = \dfrac{\mmm{G}_k}{i} = -i\/\kern1pt\mmm{G}_k
\]
se pueden calcular los generadores $\mmm{J}_i$
{\small
  \begin{equation}
    \label{eq:alg-j1_j2_j3}
    \mmm{J}_1 = \mmm{J}_x =
    \begin{pmatrix} 0 & 0 & 0
      \\ 0 & 0 & -i\\ 0 & i & 0
    \end{pmatrix}
    ;\hspace*{1em}
    \mmm{J}_2 = \mmm{J}_y = 
    \begin{pmatrix} 0 & 0 & i
      \\ 0 & 0 & 0\\ -i & 0 & 0
    \end{pmatrix}
    ;\hspace*{1em}
    \mmm{J}_3 = \mmm{J}_z = 
    \begin{pmatrix} 0 & -i & 0
      \\ i & 0 & 0\\ 0 & 0 & 0
    \end{pmatrix}
  \end{equation}
}

En mecánica cuántica hay dos tipos de matrices u operadores importantes. 
Por un lado tenemos las matrices hermíticas o autoadjuntas (operadores hermíticos o
autoadjuntos), que son las que se corresponden con observables de la naturaleza.
Por otro lado, tenemos las matrices unitarias (operadores unitarios) que nos dan la
evolución temporal, o la dinámica del sistema, y también las simetrías.

Una matriz, $\mmm{A}$, es hermítica si
$\mmm{A}^\dagger = \left(\mmm{A}^\trasp\right)^* = \mmm{A}$.
Una matriz unitaria es aquella que cumple
$\mmm{U}^\dagger \mmm{U} = \mmm{I}$.
   
Las matrices $\mmm{G}_1$, $\mmm{G}_2$ y $\mmm{G}_3$ no son hermíticas,
pero las $\mmm{J}_1$, $\mmm{J}_2$ y $\mmm{J}_3$, sí lo son; lo
que es crucial para la mecánica cuántica.  Además, el operador
$\mmm{R}(\xhat{n},\theta)$, escrito con $\vvv{J}$, es unitario.

Como resumen, hemos definido los nuevos generadores, $\mmm{J}$, para que sean hermíticos
y las rotaciones pasivas
$\mmm{R}(\xhat{n},\theta)=\exp\{i\theta\kern2pt\xhat{n}\cdot\mmm{J}\}$,
o sus inversas, que representan rotaciones de objetos
$\mmm{R}(\xhat{n},\theta)=\exp\{-i\theta\kern2pt\xhat{n}\cdot\mmm{J}\}$,
sean unitarias. En mecánica clásica no serían necesarias las $\mmm{J}$.


\section{Generalización de SO(2) y SO(3) a SO(4) y
  \mathinhead{\cdots}{cdots} SO(N)}
Ahora completaremos alguna información sobre los grupos, SO(2) y SO(3).
Después trataremos con SO(4) y por último intentaremos generalizar a SO(N).

\subsection{Grupo SO(2)}
Recordemos en que en el espacio euclídeo de dos dimensiones sólo había un posible giro
pasivo antihorario. Así, este grupo contiene un único generador, que llamábamos
$\mmm{G}$, ecuación~(\ref{eq:alg-g-so2})
\begin{equation}\label{eq:alg-g-so2}
  \mmm{G}
  =
  \begin{pmatrix}
    0 & 1\\
    -1 & 0
  \end{pmatrix}
\end{equation}

Pero ese giro no era alrededor de ningún eje ---el eje perpendicular a $x$ e $y$ no
existe en dos dimensiones--. El único sentido que le podemos dar a este giro es el
de giro en el plano xy.


Recordemos también que cuando se generalizaba a tres dimensiones, al considerar que el
plano era el $xy$ o $12$, llamábamos $z$ a ese eje de giro o poníamos el subíndice $3$.

Así, como el tercer eje no existe en dos dimensiones, es más sensato etiquetar a este
generador como el del único giro independiente en el plano $xy$, o bien el formado por
los ejes $12$. Para ello, vamos a fijarnos dónde está el elemento $1$ en la matriz del
generador: se encuentra en la primera fila, segunda columna. De manera que etiquetamos
el generador como $\mmm{G}_{12}$
\begin{equation}\label{eq:alg-g12-so2}
  \mmm{G} = \mmm{G}_{12}
  =
  \begin{pmatrix}
    0 & 1\\
    -1 & 0
  \end{pmatrix}
\end{equation}

La razón de este cambio de nomenclatura, es que nos facilitará la generalización a
dimensiones más elevadas, de SO(4) en adelante\footnotemark{}.
\footnotetext{En dos y en tres dimensiones se puede seguir utilizando la nomenclatura
  anterior sin problemas.}

El operador de rotación pasiva es
\[
  \mmm{R}(\theta) = \exp\left\{\theta\/\mmm{G}\right\}
\]

El de rotación de objetos
\[
  \mmm{R}(\theta) = \exp\left\{-\theta\/\mmm{G}\right\}
\]

En el caso de que nos interesaran matrices $2\times 2$ en mecánica cuántica, la matriz
hermítica generadora del grupo SU(2) sería
\begin{equation}\label{eq:alg-j12-su2}
  \mmm{J} = \mmm{J}_{12}
  = -i\/\mmm{G}_{12}
  =
  \begin{pmatrix}
    0 & -i\\
    i & 0
  \end{pmatrix}
\end{equation}
Obsérvese que para etiquetar la matriz $\mmm{J}$, hay que fijarse en la fila y columna
del elemento $-i$.

El operador unitario de una rotación pasiva del grupo $SU(2)$ sería
\[
  \mmm{R}(\theta) = \exp\left\{i\theta\/\mmm{J}\right\}
\]

\subsection{Grupo SO(3)}
El espacio euclídeo es de tres dimensiones y está formado por tres generadores
independientes, $\mmm{G}_1$, $\mmm{G}_2$ y $\mmm{G}_3$.
Al coincidir el número de generadores y el de dimensiones, se tiende a asignar un eje a
cada generador, $\mmm{G}_x$, $\mmm{G}_y$ y $\mmm{G}_z$, entendiéndose por ejemplo, que el
generador $\mmm{G}_3 = \mmm{G}_z$ es el que genera una rotación alrdedor del eje $3$ o $z$. Pero esta coincidencia no se va a dar en ningún grupo de rotaciones de más ejes
$SO(4), \cdots, SO(N)$
\[
  \mmm{G}_1 = \mmm{G}_x = 
    \begin{pmatrix} 0 & 0 & 0
      \\ 0 & 0 & 1\\ 0 & -1 & 0
    \end{pmatrix}
    \hspace*{1em}
  \mmm{G}_2 = \mmm{G}_y = 
    \begin{pmatrix} 0 & 0 & -1
      \\ 0 & 0 & 0\\ 1 & 0 & 0
    \end{pmatrix}
    \hspace*{1em}
  \mmm{G}_3 = \mmm{G}_z =
    \begin{pmatrix} 0 & 1 & 0
      \\ -1 & 0 & 0\\ 0 & 0 & 0
    \end{pmatrix}
\]

El operador de rotación pasiva es
\[
  \mmm{R}(\xhat{n}, \alpha)
  = e^{\alpha (n_x\mmm{G}_x + n_y\mmm{G}_y + n_z\mmm{G}_z)}
  = \exp\left\{\alpha\sum_{i=1}^{3} n_i\/\mmm{G}_i\right\}
  = \exp\{\alpha\/\xhat{n} \cdot \mmm{G}\}
\]

El de rotación de objetos
\[ 
  \mmm{R}(\xhat{n}, \alpha)
  = e^{-\alpha (n_x\mmm{G}_x + n_y\mmm{G}_y + n_z\mmm{G}_z)}
  = \exp\left\{-\alpha\sum_{i=1}^{3} n_i\/\mmm{G}_i\right\}
  = \exp\{-\alpha\/\xhat{n} \cdot \mmm{G}\}
\]

Cuando cambiemos a otras dimensiones no existirá esa relación entre ejes y generadores.
En esas situaciones debemos cambiar la nomenclatura, pasando a describir, no los ejes de
giro, sino los planos que giran.
En SO(3), el generador $\mmm{G}_x$ tiene el elemento $1$ en la segunda fila y tercera
columna, luego se etiquetará como $\mmm{G}_{23}$, porque el giro se produce en el plano
$yz$.
$\mmm{G}_y$ tiene el elemento $1$ en la tercera fila y primera columna, y pasará a ser
$\mmm{G}_{31}$, pues el giro tiene lugar en el plano $zx$.
Por último, el $\mmm{G}_z$ se etiquetará como $\mmm{G}_{12}$ con giro en el plano $xy$.

A continuación presentamos este conjunto de matrices linealmente independientes entre sí,
con la nueva nomenclatura
\begin{align*}
  &\mmm{G}_{12}
  =
  \begin{pmatrix}
    0 & 1 & 0 \\
    -1& 0 & 0 \\
    0 & 0 & 0 
  \end{pmatrix}
  \hspace*{2em}
  \hfil
  \mmm{G}_{31}
  =
  \begin{pmatrix}
    0 & 0 & -1\\
    0 & 0 & 0\\
    1& 0 & 0\\
  \end{pmatrix}
  \hspace*{2em}
  \mmm{G}_{23}
  =
  \begin{pmatrix}
    0 & 0 & 0\\
    0 & 0 & 1\\
    0 & -1& 0\\
  \end{pmatrix}
\end{align*}

Y obtenemos las matrices generadoras $\mmm{J}$, haciendo $\mmm{J} = -i\kern1pt\mmm{G}$,
siendo las $\mmm{G}$ los generadores de SO(3). Para etiquetar las matrices $\mmm{J}$, hemos de fijarnos en la fila y columna del $-i$
\begin{equation}
  \label{eq:alg-j12-j13-j23-su3}
  \mmm{J}_{12} = 
  \begin{pmatrix}
    0 & -i & 0 \\
    i & 0 & 0 \\
    0 & 0 & 0
  \end{pmatrix}
  \hspace*{2em}
  \mmm{J}_{31} =
  \begin{pmatrix}
    0 & 0 & i \\
    0 & 0 & 0 \\
    -i & 0 & 0
  \end{pmatrix}
  \hspace*{2em}
  \mmm{J}_{23} =
  \begin{pmatrix}
    0 & 0 & 0 \\
    0 & 0 & -i \\
    0 & i & 0
  \end{pmatrix}
\end{equation}

El operador de rotación pasiva es
\[ 
  \mmm{R}(\xhat{n}, \alpha)
  = e^{\alpha (n_{12}\mmm{J}_{12} + n_{31}\mmm{J}_{31} + n_{23}\mmm{J}_{23})}
  = \exp\{i\kern1pt\alpha\, \xhat{n} \cdot \mmm{J}\}
\]

Y el de objetos
\[ 
  \mmm{R}(\xhat{n}, \alpha)
  = e^{-\alpha (n_{12}\mmm{J}_{12} + n_{31}\mmm{J}_{31} + n_{23}\mmm{J}_{23})}
  = \exp\{-i\kern1pt\alpha\, \xhat{n} \cdot \mmm{J}\}
\]

\subsection{Generalización a grupos SO(4), $\cdots$, SO(N)}
Nos preguntamos cuántos y cuales serán los generadores de este grupo. Sabemos que forman
una base para cualquier matriz antisimétrica $4\times 4$. Cualquier matriz antisimétrica
tiene elementos diagonales nulos; además, los elementos no diagonales correspondientes
deben ser opuestos
\[
  \begin{pmatrix}
    0 & a & b & c\\
    -a& 0 & d & e\\
    -b& -d& 0 & f\\
    -c& -e& -f& 0
  \end{pmatrix}
\]
Como vemos, hay seis parámetros independientes para formar una matriz antisimétrica de
esta dimensión: $a$, $b$, $c$, $d$, $e$ y $f$, lo que se traduce en que hay seis matrices
generadoras en este grupo. Estas son
\begin{align*}
  &\mmm{G}_{12}
  =
  \begin{pmatrix}
    0 & 1 & 0 & 0\\
    -1& 0 & 0 & 0\\
    0 & 0 & 0 & 0\\
    0 & 0 & 0 &0
  \end{pmatrix}
  \hspace{2.5em}
  \mmm{G}_{13}
  =
  \begin{pmatrix}
    0 & 0 & 1 & 0\\
    0 & 0 & 0 & 0\\
    -1& 0 & 0 & 0\\
    0 & 0 & 0 &0
  \end{pmatrix}
  \hspace{2.5em}
  \mmm{G}_{14}
  =
  \begin{pmatrix}
    0 & 0 & 0 & 1\\
    0 & 0 & 0 & 0\\
    0 & 0 & 0 & 0\\
    -1& 0 & 0 &0
  \end{pmatrix}\\
  &\mmm{G}_{23}
  =
  \begin{pmatrix}
    0 & 0 & 0 & 0\\
    0 & 0 & 1 & 0\\
    0 & -1& 0 & 0\\
    0 & 0 & 0 &0
  \end{pmatrix}
  \hspace{2.5em}
  \mmm{G}_{24}
  =
  \begin{pmatrix}
    0 & 0 & 0 & 0\\
    0 & 0 & 0 & 1\\
    0 & 0 & 0 & 0\\
    0 & -1& 0 &0
  \end{pmatrix}
  \hspace{2.5em}
  \mmm{G}_{34}
  =
  \begin{pmatrix}
    0 & 0 & 0 & 0\\
    0 & 0 & 0 & 0\\
    0 & 0 & 0 & 1\\
    0 & 0 & -1& 0
  \end{pmatrix}
\end{align*}
y sus opuestas
\begin{align*}
  &\mmm{G}_{21}
  =
  \begin{pmatrix}
    0 & -1& 0 & 0\\
    1 & 0 & 0 & 0\\
    0 & 0 & 0 & 0\\
    0 & 0 & 0 &0
  \end{pmatrix}
  \hspace{2.5em}
  \mmm{G}_{31}
  =
  \begin{pmatrix}
    0 & 0 & -1& 0\\
    0 & 0 & 0 & 0\\
    1 & 0 & 0 & 0\\
    0 & 0 & 0 &0
  \end{pmatrix}
  \hspace{2.5em}
  \mmm{G}_{41}
  =
  \begin{pmatrix}
    0 & 0 & 0 & -1\\
    0 & 0 & 0 & 0\\
    0 & 0 & 0 & 0\\
    1 & 0 & 0 &0
  \end{pmatrix}\\
  &\mmm{G}_{32}
  =
  \begin{pmatrix}
    0 & 0 & 0 & 0\\
    0 & 0 & -1& 0\\
    0 & 1 & 0 & 0\\
    0 & 0 & 0 &0
  \end{pmatrix}
  \hspace{2.5em}
  \mmm{G}_{42}
  =
  \begin{pmatrix}
    0 & 0 & 0 & 0\\
    0 & 0 & 0 & -1\\
    0 & 0 & 0 & 0\\
    0 & 1 & 0 &0
  \end{pmatrix}
  \hspace{2.5em}
  \mmm{G}_{43}
  =
  \begin{pmatrix}
    0 & 0 & 0 & 0\\
    0 & 0 & 0 & 0\\
    0 & 0 & 0 & -1\\
    0 & 0 & 1 & 0
  \end{pmatrix}
\end{align*}

Obsérvese que, al contrario de lo que ocurría en SO(3), ahora el número de generadores
6, no coincide con el de ejes 4. Por esta razón, ya no podemos aplicar la "expresión"
$\xhat{n}\cdot G = n_xG_x+n_yG_y+n_zG_z$, tal y como hacíamos en SO(3).

Vamos a estudiar las relaciones de conmutación. En estos conmutadores aparecen cuatro
índices, $\text{a}$, $\text{b}$, $\text{c}$ y $\text{d}$
\[
  [\mmm{G}_{ab},\mmm{G}_{cd}]
\]

Si calculáramos todos los conmutadores, nos daríamos cuenta de que cuando sólo se repite
el primer índice de cada generador en el conmutador, el resultado es otro generador que
no contiene el índice repetido y los demás índices están en orden inverso al que aparecen
en el conmutador. Ejemplos
\[
  [\mmm{G}_{\mathbf{\textcolor{red!80!black}{1}}2},\mmm{G}_{\textcolor{red!80!black}{\mathbf{1}}3}]
  = \mmm{G}_{32}
  ;\hspace{0.5em}
  [\mmm{G}_{\textcolor{red!80!black}{\mathbf{1}}2},\mmm{G}_{\textcolor{red!80!black}{\mathbf{1}}4}]
  = \mmm{G}_{42}
  ;\hspace{0.5em}\cdots;\hspace{.5em}
  [\mmm{G}_{\textcolor{red!80!black}{\mathbf{3}}2},\mmm{G}_{\textcolor{red!80!black}{\mathbf{3}}4}]
  = \mmm{G}_{42}
  ;\hspace{0.5em}\cdots
\]
Por tanto, podemos crear una regla para tener en cuenta este
comportamiento
\[
  [\mmm{G}_{ab},\mmm{G}_{cd}] = \mmm{G}_{db}\kern1pt\delta_{ac}
\]

Otro tanto ocurre cuando se repiten sólo los segundos índices
\[
  [\mmm{G}_{1\textcolor{red!80!black}{\mathbf{2}}},\mmm{G}_{3\textcolor{red!80!black}{\mathbf{2}}}]
  = \mmm{G}_{31} ;\hspace{0.5em}
  [\mmm{G}_{3\textcolor{red!80!black}{\mathbf{2}}},\mmm{G}_{1\textcolor{red!80!black}{\mathbf{2}}}]
  = \mmm{G}_{13} ;\hspace{0.5em}
  [\mmm{G}_{3\textcolor{red!80!black}{\mathbf{4}}},\mmm{G}_{2\textcolor{red!80!black}{\mathbf{4}}}]
  = \mmm{G}_{23} ;\hspace{0.5em}\cdots
\]

En este caso la regla es
\[
  [\mmm{G}_{ab},\mmm{G}_{cd}] = \mmm{G}_{ca}\delta_{bd}
\]

Cuando se repiten los índices externos, el resultado contiene los otros dos índices en el
mismo orden en el que se encuentran en el conmutador
\[
  [\mmm{G}_{\textcolor{red!80!black}{\mathbf{2}}4},\mmm{G}_{3\textcolor{red!80!black}{\mathbf{2}}}]
  = \mmm{G}_{43} ;\hspace{0.5em}
  [\mmm{G}_{\textcolor{red!80!black}{\mathbf{1}}2},\mmm{G}_{4\textcolor{red!80!black}{\mathbf{1}}}]
  = \mmm{G}_{24} ;\hspace{0.5em}
  [\mmm{G}_{\textcolor{red!80!black}{\mathbf{2}}4},\mmm{G}_{1\textcolor{red!80!black}{\mathbf{2}}}]
  = \mmm{G}_{41} ;\hspace{0.5em}\cdots
\]
La regla es
\[
  [\mmm{G}_{ab},\mmm{G}_{cd}] = \mmm{G}_{bc}\delta_{ad}
\]

Cuando son los índices internos los que se repiten, de nuevo el resultado queda con los
otros dos índices en el orden en el que aparecen en el conmutador
\[
  [\mmm{G}_{3\textcolor{red!80!black}{\mathbf{1}}},\mmm{G}_{\textcolor{red!80!black}{\mathbf{1}}2}]
  = \mmm{G}_{32} ;\hspace{0.5em}
  [\mmm{G}_{3\textcolor{red!80!black}{\mathbf{4}}},\mmm{G}_{\textcolor{red!80!black}{\mathbf{4}}1}]
  = \mmm{G}_{31} ;\hspace{0.5em}
  [\mmm{G}_{4\textcolor{red!80!black}{\mathbf{1}}},\mmm{G}_{\textcolor{red!80!black}{\mathbf{1}}2}]
  = \mmm{G}_{42} ;\hspace{0.5em}\cdots
\]

Y la regla correspondiente es
\[
  [\mmm{G}_{ab},\mmm{G}_{cd}] = \mmm{G}_{ad}\delta_{bc}
\]

Cuando se repiten los dos índices, sin importar el orden, o cuando no se repite ningún
índice, el conmutador es cero.

Lo anterior se puede resumir en una fórmula
\[
  [\mmm{G}_{ab},\mmm{G}_{cd}]
  =
  \mmm{G}_{db}\kern1pt\delta_{ac}
  + \mmm{G}_{\text{ca}}\kern1pt\delta_{\text{bd}}
  + \mmm{G}_{bc}\kern1pt\delta_{ad}
  + \mmm{G}_{ad}\kern1pt\delta_{bc}
\]

Ahora nos interesa que los subíndices en las $\mmm{G}_{ij}$ sigan el mismo orden --por
ejemplo, el orden con que aparecen en el conmutador $a,b,c,d$--.
A tal fin utilizaremos la igualdad $\mmm{G}_{ij} = -\mmm{G}_{ji}$ allá donde interese.
La delta de Kronecker no cambia de signo $\delta_{ij} = \delta_{ji}$
\begin{align*}
  [\mmm{G}_{ab},\mmm{G}_{cd}]
  =
  -\mmm{G}_{bd}\kern1pt\delta_{ac}
  -\mmm{G}_{ac}\kern1pt\delta_{bd}
  + \mmm{G}_{bc}\kern1pt\delta_{ad}
  + \mmm{G}_{ad}\kern1pt\delta_{bc}
\end{align*}

Todas estas observaciones se pueden resumir en una expresión
\begin{equation}
  \label{eq:alg-reglas_conmutacion-g}
  [\mmm{G}_{ab},\mmm{G}_{cd}]
  =
  \mmm{G}_{bc}\kern1pt\delta_{ad}
  + \mmm{G}_{ad}\kern1pt\delta_{bc}
  - \mmm{G}_{bd}\kern1pt\delta_{ac}
  - \mmm{G}_{ac}\kern1pt\delta_{bd}
\end{equation}

Lo importante de la expresión anterior es que sirve, no sólo para SO(4), sino también
para cualquier grupo de rotación SO(N).

Si tratamos con grupos unitarios, obtenemos las $\mmm{J}$, haciendo
$\mmm{G}_{jk} = i\kern1pt\mmm{J}_{jk}$
\[
  [i\mmm{J}_{ab},i\mmm{J}_{cd}]
  =
  i\mmm{J}_{bc}\kern1pt\delta_{ad}
  + i\mmm{J}_{ad}\kern1pt\delta_{bc}
  - i\mmm{J}_{bd}\kern1pt\delta_{ac}
  - i\mmm{J}_{ac}\kern1pt\delta_{bd}\\
\]
\[
  i^{2} [\mmm{J}_{ab},\mmm{J}_{cd}]
  =
  i\left(\mmm{J}_{bc}\kern1pt\delta_{ad}
    + \mmm{J}_{ad}\kern1pt\delta_{bc}
    - \mmm{J}_{bd}\kern1pt\delta_{ac}
    - \mmm{J}_{ac}\kern1pt\delta_{bd}\right)
\]
\[
  -[\mmm{J}_{ab},\mmm{J}_{cd}]
  =
  i\left(\mmm{J}_{bc}\kern1pt\delta_{ad}
    + \mmm{J}_{ad}\kern1pt\delta_{bc}
    - \mmm{J}_{bd}\kern1pt\delta_{ac}
    - \mmm{J}_{ac}\kern1pt\delta_{bd}\right)
\]

Resultando
\begin{equation}
  \label{eq:ali-reglas_conmutacion-j}
  [\mmm{J}_{ab},\kern1pt\mmm{J}_{cd}]
  =
  i\left(\mmm{J}_{bd}\kern1pt\delta_{ac}
    + \kern1pt\mmm{J}_{ac}\kern1pt\delta_{bd}
    - \kern1pt\mmm{J}_{bc}\kern1pt\delta_{ad}
    - \kern1pt\mmm{J}_{ad}\kern1pt\delta_{bc}\right)
\end{equation}
Tenemos las reglas de conmutación de los generadores SO(N), lo que aparece entre
paréntesis nos permite calcular las constantes de estructura, $c_{abcd}$.

\subsubsection{Ejercicios en SO(4)}
Realicemos algunos cálculos con SO(4) utilizando la fórmula anterior.

\begin{itemize}
\item Primer ejercicio
  \begin{align*}
    [\mmm{J}_{12},\mmm{J}_{23}]
    &=
      [\mmm{J}_{ab},\mmm{J}_{cd}]
      =
      i\kern1pt(\mmm{J}_{bd}\kern1pt\delta_{ac}
      + \mmm{J}_{ac}\kern1pt\delta_{bd}
      - \mmm{J}_{bc}\kern1pt\delta_{ad}
      - \mmm{J}_{ad}\kern1pt\delta_{bc})\\
    &= 
      i\kern1pt(\mmm{J}_{23}\kern1pt\delta_{12}
      + \mmm{J}_{12}\kern1pt\delta_{23}
      - \mmm{J}_{22}\kern1pt\delta_{13}
      - \mmm{J}_{13}\kern1pt\delta_{22})
      = i\kern1pt(0 + 0 - 0 - \mmm{J}_{13}\cdot 1)\\
    &= i (-1)\kern1pt\mmm{J}_{13}
      = i c_{122313}\kern1pt\mmm{J}_{13}
  \end{align*}
  
  y como el conmutador es una combinación lineal de los seis generadores de SO(4)
  {\small
    \[
      [\mmm{J}_{12},\mmm{J}_{23}]
      =
      i(c_{122312}\kern1pt\mmm{J}_{12}
      + c_{122313}\kern1pt\mmm{J}_{13}
      + c_{122314}\kern1pt\mmm{J}_{14}
      + c_{122323}\kern1pt\mmm{J}_{23}
      + c_{122324}\kern1pt\mmm{J}_{24}
      + c_{122334}\kern1pt\mmm{J}_{34})
    \]
  }
  
  El coeficiente de estructura $c_{122313}$ vale $-1$
  \[
    c_{122313} = -1
  \]
  y los cinco restantes, cero
  \[
    c_{122312} = c_{122314} = c_{122323} = c_{122324} = c_{122334} = 0
  \]
  
  El resultado es
  \[
    [\mmm{J}_{12},\mmm{J}_{23}] = -i \mmm{J}_{13}
  \]
  
\item Segundo ejercicio
  \begin{align*}
    [\mmm{J}_{12},\mmm{J}_{43}]
    &=
      [\mmm{J}_{ab},\mmm{J}_{cd}]
      =
      i\kern1pt(\mmm{J}_{bd}\kern1pt\delta_{ac}
      + \mmm{J}_{ac}\kern1pt\delta_{bd}
      - \mmm{J}_{bc}\kern1pt\delta_{ad}
      - \mmm{J}_{ad}\kern1pt\delta_{bc})\\
    &= 
      i\kern1pt(\mmm{J}_{23}\kern1pt\delta_{14}
      + \mmm{J}_{14}\kern1pt\delta_{23}
      - \mmm{J}_{24}\kern1pt\delta_{13}
      - \mmm{J}_{13}\kern1pt\delta_{24})\\
    &= i\kern1pt
      (0\kern1pt\mmm{J}_{23} + 0\kern1pt\mmm{J}_{14} - 0\kern1pt\mmm{J}_{24}
      - 0\kern1pt\mmm{J}_{13})\\
    &= 0
  \end{align*}
  
  y como el conmutador es una combinación lineal de los seis generadores de SO(3)
  {\small
    \[
      [\mmm{J}_{12},\mmm{J}_{43}]
      =
      i(c_{124312}\kern1pt\mmm{J}_{12}
      + c_{124313}\kern1pt\mmm{J}_{13}
      + c_{124314}\kern1pt\mmm{J}_{14}
      + c_{124323}\kern1pt\mmm{J}_{23}
      + c_{124324}\kern1pt\mmm{J}_{24}
      + c_{124334}\kern1pt\mmm{J}_{34})
    \]
  }
  resulta que los seis coeficientes de estructura relacionados con este conmutador valen
  cero
  \[
    c_{124312} = c_{124313} = c_{124314} = c_{124323} = c_{124324} = c_{124334} = 0
  \]
  
\item Último ejercicio
  \begin{align*}
    [\mmm{J}_{12},\mmm{J}_{21}]
    &=
      [\mmm{J}_{ab},\mmm{J}_{cd}]
      =
      i\kern1pt(\mmm{J}_{bd}\kern1pt\delta_{ac}
      + \mmm{J}_{ac}\kern1pt\delta_{bd}
      - \mmm{J}_{bc}\kern1pt\delta_{ad}
      - \mmm{J}_{ad}\kern1pt\delta_{bc})\\
    &= 
      i\kern1pt(\mmm{J}_{21}\kern1pt\delta_{12}
      + \mmm{J}_{12}\kern1pt\delta_{21}
      - \mmm{J}_{22}\kern1pt\delta_{11}
      - \mmm{J}_{11}\kern1pt\delta_{22})\\
    &= i\kern1pt
      (\mmm{J}_{21}\cdot 0 + \mmm{J}_{12}\cdot 0 - 0 \cdot 1 - 0 \cdot 1)\\
    = 0
  \end{align*}
  Los términos, $\mmm{J}_{11}$ y $\mmm{J}_{22}$ se anulan porque son los elementos
  diagonales de una matriz antisimétrica.
  
  Como el conmutador es una combinación lineal de los seis generadores de SO(3)
  {\small
    \[
      [\mmm{J}_{12},\mmm{J}_{21}]
      =
      i(c_{122112}\kern1pt\mmm{J}_{12}
      + c_{122113}\kern1pt\mmm{J}_{13}
      + c_{122114}\kern1pt\mmm{J}_{14}
      + c_{122123}\kern1pt\mmm{J}_{23}
      + c_{122124}\kern1pt\mmm{J}_{24}
      + c_{122134}\kern1pt\mmm{J}_{34})
    \]
  }
  
  Los seis coeficientes de estructura valen cero
  \[
    c_{122112} = c_{122113} = c_{122114} = c_{122123} = c_{122124} = c_{122134} = 0
  \]
\end{itemize}

\section{Dimensión del álgebra de Lie}
Los generadores de SO(N) son matrices antisimétricas, por tanto sus elementos diagonales
valen cero y los no diagonales son opuestos.

El número de generadores es igual al de elementos independientes de una matriz
antisimétrica, que son la mitad de los elementos no diagonales.

El número total de elementos de una matriz $N\times N$ es $N^2$, mientras que el de
elementos diagonales es $N$, de manera que el número de elementos no diagonales
independientes es la mitad de los no diagonales.
Los elementos no diagonales son todos, $N^2$, menos los diagonales, $N$.
\[
  \text{Número de generadores } = \dfrac{N^2-N}{2} = \dfrac{N(N-1)}{2}
  =
  \begin{pmatrix}
    N \\ 2
  \end{pmatrix}
\]

El número de generadores es el número de combinaciones de $N$ elementos tomados de dos en
dos.  El número de generadores se conoce también como la \emph{dimensión del álgebra de
Lie}.
 













 
%%% Local Variables:
%%% mode: latex
%%% TeX-engine: luatex
%%% TeX-master: "../gruposlie.tex"
%%% End:
