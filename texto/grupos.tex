% grupos.tex
%
% Copyright (C) 2022 José A. Navarro Ramón <janr.devel@gmail.com>
% Licencia Creative Commons Recognition Share-alike.
% (CC-BY-SA)

\chapter{Grupos}

\section{Definición de grupo}
Un grupo $\{G,\cdot\,\}$ es una estructura algebraica formada por un
conjunto $G$, asociado a una operación '$\cdot\,$', llamada
genéricamente \emph{producto interno}, que asocia a cada par ordenado
$(x,y)$ de elementos de $G$, un elemento $x\cdot y$ de $G$, con las
tres siguientes propiedades:
\begin{enumerate}
\item \textbf{Propiedad asociativa}. Para tres elementos cualesquiera de $G$:
  \begin{equation}\label{eq:gru-asociativa}
    \forall x, y, z \in G, \quad (x \cdot y) \cdot z = x \cdot (y \cdot z)
  \end{equation}
\item \textbf{Elemento identidad}. Hay un único elemento, $e$, de $G$, llamado
  elemento identidad o neutro que, combinado con cualquier elemento del grupo,
  lo deja inalterado:
  \begin{equation}\label{eq:gru-identidad}
    \exists !\, e \in G \ \text{tal que}\ \forall x \in G, \quad e \cdot x
    = x \cdot e = x
  \end{equation}
\item \textbf{Elemento inverso}. Todo elemento $x$ de $G$ tiene un
  inverso\footnotemark, $x^{-1}$, de manera que la composición de ambos
  produce el elemento neutro: \footnotetext{Se podría utilizar el término
    ``opuesto'' ($-x$) si la operación se pudiera relacionar con algún tipo de
    adición; en caso contrario se suele denominar ``inverso'': $x^{-1}$.}
  \begin{equation}\label{eq:gru-inverso}
    \forall x \in G, \exists\, !\, x^{-1} \
    \text{tal que}\ x^{-1} \cdot x = x \cdot  x^{-1} = e
  \end{equation}
\end{enumerate}

Si además, la operación tuviera la propiedad conmutativa, se diría que
el grupo es \emph{conmutativo} o \emph{abeliano}:
\begin{enumerate}
\item[4.] \textbf{Propiedad conmutativa}. Para cualesquiera dos
  elementos $x$ e $y$ de $G$ se cumple que no importa el orden en el
  que se realice el producto:
  \begin{equation}\label{eq:gru-conmutativa}
    \forall x,y \in G, \quad x \cdot y = y \cdot x
  \end{equation}
\end{enumerate}



 
%%% Local Variables:
%%% mode: latex
%%% TeX-engine: luatex
%%% TeX-master: "../gruposlie.tex"
%%% End:
