% momangclasico.tex
%
% Copyright (C) 2022--2025 José A. Navarro Ramón <janr.devel@gmail.com>
% Licencia del código GPLv2
% Licencia Creative Commons Recognition Non-Commercial Share-alike.
% (CC-BY-NC-SA)

\chapter{El momento angular en física clásica}
El momento angular se comporta algo así como un vector bajo rotaciones. A continuación
vamos a estudiarlo por encima desde el punto de vista de la física clásica.

\section{Definición}
El momento angular de una partícula se define como el producto vectorial de su posición
por la cantidad de movimiento o momento lineal
\[
  \vvv{L} \equiv \vvv{r} \prodvec \vvv{p}
\]

Tanto la posición como el momento lineal se definen, en general, en función de las
coordenadas generalizadas $q$; aunque en los casos más sencillos, esta última magnitud
será el familiar producto de la masa de la partícula por su velocidad,
$\vvv{p} = m\vvv{v}$ en coordenadas cartesianas.

El producto vectorial sólo está definido en $\symbb{R}^3$ y, aunque se puede generalizar
a dimensiones más elevadas, no lo está en $\symbb{R}^2$, porque se necesita un mínimo de
tres dimensiones.

Ambas magnitudes, posición $\vvv{r} = \vvv{r}(t)$ y momento lineal
$\vvv{p} = \vvv{p}(t)$, dependen en general del tiempo, de modo que el momento angular
también suele depender de este $\vvv{L} = \vvv{L}(t)$.

En ciertas ocasiones el momento angular es constante, por ejemplo, cuando una partícula
está sujeta únicamente a una fuerza central (por ejemplo la Luna rotando alrededor de la
Tierra y despreciando el resto de interacciones, como las debidas a otros astros).
Cuando ocurre esto, se pueden resolver las ecuaciones del movimiento de una forma más
sencilla.

Calculamos el momento angular de una partícula en coordenadas cartesianas rectangulares
\[
  \vvv{L}
  =
  \vvv{r} \prodvec \vvv{p}
  =
  \begin{vmatrix}
    \uvec{\i} & \uvec{\j} & \xhat{k}\\
    x & y & z\\
    p_x & p_y & p_z
  \end{vmatrix}
  =
  (yp_z-zp_y)\,\uvec{i}
  + (zp_x-xp_z)\,\uvec{j}
  + (xp_y-yp_z)\,\xhat{k}
\]

De la expresión anterior, deducimos sus componentes cartesianas:
\begin{align*}
  L_x &= yp_z - zp_y\\
  L_y &= zp_x - xp_z\\
  L_z &= xp_y - yp_x
\end{align*}
  
Hay una forma más general para calcular la componente del momento angular (o de cualquier
vector) con respecto a cualquier dirección, y es multiplicar escalarmente el vector
unitario que tenga esa dirección por el vector. Así, si queremos calcular, por ejemplo
las componentes cartesianas del momento angular, podemos hacer
\begin{align*}
  L_x &= \uvec{\i}\cdot\vvv{L}\\
  L_y &= \uvec{\j}\cdot\vvv{L}\\
  L_z &= \xhat{k}\cdot\vvv{L}
\end{align*}


\section{Componente del momento angular con respecto a un eje}
En general, si $\xhat{n} = (n_x, n_y, n_z)$ es el versor que define una dirección
cualquiera del espacio, $\vvv{r} = (x, y, z)$ y $\vvv{p} = (p_x, p_y, p_z)$, la
componente del momento angular según esa dirección es
\[
  L_n = \xhat{n}\cdot\vvv{L}
  = n_x(yp_z-zp_y) + n_y(zp_x-xp_z) + n_z(xp_y-yp_x)
\]

Esta componente también se puede representar de forma matricial
\[
  L_n
  = \xhat{n}\cdot\vvv{L}
    =
    \begin{pmatrix}
      x & y & z
    \end{pmatrix}
    \begin{pmatrix}
      0 & n_z & -n_y\\
      -n_z & 0 & n_x\\
      n_y & -n_x & 0
    \end{pmatrix}
    \begin{pmatrix}
      p_x \\ p_y \\ p_z
    \end{pmatrix}
\]
\[
  L_n
   =
    \begin{pmatrix}
      x & y & z
    \end{pmatrix}
    \left[
    n_x
    \begin{pmatrix}
      0 & 0 & 0\\
      0 & 0 & 1\\
      0 & -1 & 0
    \end{pmatrix}
    + n_y
    \begin{pmatrix}
      0 & 0 & -1\\
      0 & 0 & 0\\
      1 & 0 & 0
    \end{pmatrix}
    + n_z
    \begin{pmatrix}
      0 & 1 & 0\\
      -1 & 0 & 0\\
      0 & 0 & 0
    \end{pmatrix}
  \right]
    \begin{pmatrix}
      p_x \\ p_y \\ p_z
    \end{pmatrix}
\]

En esta última expresión reconocemos los generadores del grupo SO(3)
\begin{align*}
    \xhat{n}\cdot\vvv{L}
   &=
    \begin{pmatrix}
      x & y & z
    \end{pmatrix}
    \left(
      n_x \mmm{G_1}
      + n_y \mmm{G_2}
      + n_z \mmm{G_3}
    \right)
    \begin{pmatrix}
      p_x \\ p_y \\ p_z
    \end{pmatrix}\\
&=
  \begin{pmatrix}
      x & y & z
    \end{pmatrix}
    (\xhat{n} \cdot \mmm{G})
    \begin{pmatrix}
      p_x \\ p_y \\ p_z
    \end{pmatrix}\\
\end{align*}

Finalmente obtenemos una expresión que nos relaciona la componente del momento angular
con una rotación en el espacio --grupo SO(3)--
\begin{equation}\label{eq:cla-Ln}
  L_n
  =
  \xhat{n}\cdot\vvv{L}
  =
    \vvv{r}^\trasp (\xhat{n}\cdot \mmm{G}) \,\vvv{p}
\end{equation}

\subsection{Rotación de vectores en el espacio euclídeo}
Ya hemos visto que al girar los ejes de coordenadas alrededor de un cierto eje (rotación
pasiva) se produce un cambio en las coordenadas de un vector
\[
  \vvv{\tilde{r}}
  = \mmm{R}\!\left(\xhat{n},\theta\right)\, \vvv{r}
  = e^{\theta\/\xhat{n}\cdot\mmm{G}} \vvv{r}
\]
y que matematicamente equivale a dejar los ejes inmóviles y a girar en sentido contrario
(horario) ese objeto con respecto al mismo eje (rotación activa).
Si quisiéramos rotar el vector en sentido positivo (antihorario), deberíamos realizar la
operación inversa
\[
  \vvv{\tilde{r}}
  = \mmm{R}^{-1}\!\left(\xhat{n},\theta\right)\, \vvv{r}
  = e^{-\theta\/\xhat{n}\cdot\mmm{G}} \vvv{r}
\]

En Física asignamos un significado distinto a los dos tipos de transformaciones:
%Discutiremos el significado que podemos asignar al hecho de que los vectores
%se transformen debido a una rotación alrededor de un eje. Nos interesa un
%punto de vista físico más que matemático.
\begin{itemize}
\item Rotación pasiva

Recurrimos a una transformación pasiva (traslación o rotación) cuando queremos presentar
el punto de vista de un observador situado en otro sistema de referencia, que puede estar
desplazado o rotado con respecto al sistema de referencia original.
En este caso estamos considerando \emph{el mismo fenómeno u objeto físico} desde otro
punto de vista.% y los dos puntos de vista se pueden considerar
%en el mismo o en distintos instantes de tiempo.

\item Rotación activa

Cuando consideramos una rotación activa (en general, cualquier transformación activa, como
traslación o rotación), estamos considerando que
\emph{el objeto físico está cambiando con respecto a un observador}, esto es, estamos
considerando una situación dinámica que cambia al transcurrir el tiempo.
\end{itemize}

%\section{Rotación de una función real definida en
%\mathinhead{\symbb{R}^3}{rdufrdeR3}}
%\label{sect:cla-rotacionfunciones}
%En esta sección se analizará la rotación de una función real $f(x, y, z)$, definida en el
%espacio euclídeo, alrededor de una dirección dada por el versor $\xhat{n}$.
%
%\subsubsection{¿Qué se entiende por girar una función?}
%Girar una función implica lo siguiente:
%\begin{enumerate}
%\item Deberíamos rotar todos los puntos del espacio $\vvv{r} = (x,y,z)$, donde esté
%  definida la función, transformándolos en otros
%  $\tilde{r}=(\tilde{x},\tilde{y},\tilde{z})$
%  \[
%    \vvv{r} = (x,y,z)\hspace{1em}
%    \longrightarrow
%    \hspace{1em}
%    \vvv{\tilde{r}} = (\tilde{x},\tilde{y},\tilde{z})
%  \]
%  
%  Al tratarse de una rotación activa (rotación de vectores), la matriz de rotación es
%  \begin{subequations}
%  \begin{equation}\label{eq:cla-matrizrotacion-funcion}
%    \mmm{R}^{-1} = \mmm{R}^{-1}\!\left(\xhat{n},\theta\right) = e^{-\theta \xhat{n}\cdot\mmm{G}}
%  \end{equation}
%  y las coordenadas del punto rotado se hallan aplicando esta matriz a las originales
%  \begin{equation}\label{eq:cla-rotacion-aciva-funcion-coordenadasrotadas}
%    \vvv{\tilde{r}} = \mmm{R}^{-1}\,\vvv{r} = e^{-\theta\xhat{n}\cdot\mmm{G}}\vvv{r}
%  \end{equation}
%\end{subequations}
%Sabemos rotar segmentos orientados mediante matrices, como
%en~\eqref{eq:cla-rotacion-aciva-funcion-coordenadasrotadas}, pero una función no es un
%segmento orientado; de hecho, no tendría ningún sentido la expresión equivalente para
%funciones, si $\mmm{R}^{-1}$ se representara mediante una matriz
%\[
%  \cancelout{\tilde{f} = \mmm{R}^{-1}\, f}
%\]
%
%Es como si una función no tuviera \emph{elementos físicos} que permitieran girarla
%mediante la \emph{herramienta} matriz de rotación.
%Habrá que idear \emph{otra herramienta} o mecanismo para su giro\footnotemark{}.
%\footnotetext{Más adelante se verá que la expresión $\tilde{f} = \mmm{R} f$
%  tendrá sentido cuando $\mmm{R}$ se represente como operador diferencial.}
%
%\item ¿Cómo podremos dar con una expresión matemática para la función rotada $\tilde{f}$?
%  
%  Cada punto, al rotar, debe \emph{llevarse consigo} de alguna manera el valor que tenía
%  en la función original $f$\, hasta la rotada $\tilde{f}$; en otras palabras,
%  \emph{la nueva función se debe obtener a partir de la original y tener el mismo valor
%    para los puntos originales y sus correspondientes rotados}.
%  En otras palabras, se debe cumplir $f$ y $\tilde{f}$ tengan el mismo valor
%  \begin{equation}\label{eq:cla-funcion-relacion-funcionrotada}
%    f(x,y,z) = \tilde{f}(\tilde{x},\tilde{y},\tilde{z})
%  \end{equation}
%  siempre que $\tilde{x}$, $\tilde{y}$, $\tilde{z}$ sean las coordenadas rotadas del
%  punto original $x$, $y$, $z$.
%
%  Obsérvese que una función se define a través de sus argumentos y de la expresión
%  matemática que especifica cómo se opera con estos.
%  Por ejemplo, en la siguiente función los argumentos son las coordenadas $x$, $y$, $z$ y
%  la expresión $x^2 - y + 2xz$ detalla cómo se opera con ellos para obtener el valor de
%  $f$
%  \[
%    f(x,y,z) = x^2 - y + 2xz
%  \]
%  
%  Así para girar la función debemos recurrir a modificar los argumentos y obtener una
%  expresión matemática para la función rotada.
%  
%  Para obtener la expresión matemática de la nueva función
%  $\tilde{f}(\tilde{x},\tilde{y},\tilde{z})$, partimos del valor de la función original
%  $f(x,y,z)$ y sustituimos las coordenadas originales por sus equivalentes rotadas. Esto
%  se realiza mediante la inversa de la
%  matriz de rotación~\eqref{eq:cla-matrizrotacion-funcion}
%  \begin{subequations}
%    \begin{equation}\label{eq:cla-matrizrotacioninversa-funcion}
%      \mmm{R}^{-1} = \mmm{R}^{-1}(\xhat{n}, \theta) = e^{\theta \xhat{n}\cdot\mmm{G}}
%    \end{equation}
%    Se aplica esta matriz a las coordenadas rotadas
%    \begin{equation}\label{eq:cla-rotacion-aciva-funcion-coordenadasoriginales}
%      \vvv{r}
%      = \mmm{R}^{-1} \vvv{\tilde{r}}
%      = e^{\theta\xhat{n}\cdot\mmm{G}}\vvv{\tilde{r}}
%    \end{equation}
%    Se obliga a que se cumpla la
%    igualdad~\eqref{eq:cla-funcion-relacion-funcionrotada}
%    \begin{equation}\label{eq:cla-rotacion-de-funcion}
%      f(\vvv{r}) = f(\mmm{R}^{-1}\vvv{r}) = \tilde{f}(\vvv{\tilde{r}})
%    \end{equation}
%  \end{subequations}
%
%  Se podría vislumbrar la necesidad de la rotación inversa mediante el siguiente
%  argumento, que se intenta ilustrar a través de la
%  figura~\eqref{fig:cla-calculo-f-girada}:
%
%  Para calcular el valor de la función girada en el punto $\tilde{P}$,
%  $\tilde{f}(\vvv{\tilde{r}})$, se le aplica la rotación inversa
%  $\mmm{R}^{-1}$ al punto rotado, para obtener el punto original $P$ y
%  entonces calcular el valor de la función original en éste, $f(\vvv{r})$
%  \begin{figure}[ht]
%    \def\scl{1}
%    % Eje x
%    \pgfmathsetmacro{\XMLONG}{0}
%    \pgfmathsetmacro{\XPLONG}{3}
%    % Eje y
%    \pgfmathsetmacro{\YMLONG}{0}
%    \pgfmathsetmacro{\YPLONG}{3}
%    % Ángulo rotado
%    \pgfmathsetmacro{\ANGROT}{20}
%    % Vector P'
%    \pgfmathsetmacro{\PPRIMAMOD}{2.5}
%    \pgfmathsetmacro{\PPRIMAANG}{60}
%    % Vector P
%    \pgfmathsetmacro{\PMOD}{\PPRIMAMOD}
%    \pgfmathsetmacro{\PANG}{\PPRIMAANG - \ANGROT}
%    % Fondo
%    \pgfmathsetmacro{\HORZ}{0.5}
%    \pgfmathsetmacro{\VERT}{0.5}
%    % 
%    \centering
%    \begin{tikzpicture}[%
%      scale=\scl,
%      every node/.style={black,font=\small},
%      eje/.style={->},
%      vector/.style={-{Latex}, shorten >=1.2pt, line width=.8pt},
%      vectorrotado/.style={vector, draw=green!50!black},
%      pcirculo/.style={fill=red, draw=black},
%      pprimacirculo/.style={green!90!black, draw=black},      
%      background/.style={
%        line width=\bgborderwidth,
%        draw=\bgbordercolor,
%        fill=\bgcolor,
%      },      
%      ]
%      % Coordenadas
%      \coordinate (O) at (0,0);
%      \coordinate (under_origin) at (0,-3mm);
%      \coordinate (left_origin) at (-3mm,0);
%      \coordinate (xini) at (-\XMLONG cm,0);
%      \coordinate (xfin) at (\XPLONG cm,0);
%      \coordinate (yini) at (0,-\YMLONG cm);
%      \coordinate (yfin) at (0,\YPLONG cm);
%      \coordinate (P') at (\PPRIMAANG:\PPRIMAMOD cm);
%      \coordinate (P) at  (\PANG:\PMOD);
%      \path (O) -- coordinate (OPmidway) (P);
%      \path (O) -- coordinate (OP'midway) (P');
%      \path (O) -- coordinate[pos=1.2] (parrow) (P);
%      \path (O) -- coordinate[pos=1.2] (p'arrow) (P');
%      % Ángulo \varepsilon
%      \path (P) -- (O) -- (P') pic
%      [draw=black!50!,fill=green!20,"\footnotesize $\varepsilon$",angle
%      radius=6mm, angle eccentricity=1.5] {angle = P--O--P'};
%      % Ejes
%      \draw[eje] (xini) -- (xfin);
%      \node[right, name=letraejex] at (xfin) {$x$};
%      \draw[eje] (yini) -- (yfin);
%      \node[above, name=letraejey] at (yfin) {$y$};
%      % Punto P'
%      \draw[vectorrotado] (O) -- (P');
%      \node[above=5pt] at (OP'midway) {$\vvv{\tilde{r}}$};
%      \fill[pprimacirculo] (P') circle [radius=1.4pt];
%      \node[above] at (P') {$\tilde{P}$};
%
%      % Punto P
%      \draw[vector] (O) -- (P);
%      \node[right=0pt] at (OPmidway) {$\vvv{r}$};
%      \fill[pcirculo] (P) circle [radius=1.4pt];
%      \node[right] at (P) {$P$};
%
%      % Función rotada y original
%      \node[above] at (p'arrow) {$\tilde{f}(\vvv{\tilde{r}})$};
%      \node[right] at (parrow) {$f(\vvv{r})$};
%      
%      % Sentido de giro del vector
%      \draw [-{Latex},green!40!black,shorten <= 3pt]
%      (p'arrow) to[bend left=30] coordinate (minversa) (parrow);
%
%      % Matriz inversa
%      \node[above right] at (minversa)
%      {\footnotesize $\mmm{R}^{-1}\vvv{\tilde{r}}$};
%
%      % Fondo amarillo
%      \coordinate (SW) at ($(current bounding box.south west) + (-\HORZ cm,-\VERT cm)$);
%      \coordinate (NE) at ($(current bounding box.north east) + (\HORZ cm,\VERT cm)$);    
%      \begin{scope}[on background layer]
%        \draw[background] (SW) rectangle (NE);
%      \end{scope}
%    \end{tikzpicture}
%    \caption{Para calcular el valor de la función rotada $\tilde{f}$ en el punto
%      $\tilde{P}$, esto es, ($\tilde{f}(\vvv{\tilde{r}}$), giramos el punto
%      \emph{en sentido contrario} al giro que se le dio a la función para localizar el
%      punto original $P$ y se calcula el valor de la función original en ese punto $P$
%      sin girar $f(\vvv{r}=$. Finalmente se igualan los valores
%      $\tilde{f}(\vvv{\tilde{r}}) = f(\vvv{r})$.}
%    \label{fig:cla-calculo-f-girada}
%  \end{figure}
%
%\end{enumerate}
%
%Es importante que se comprenda que
%\emph{al transformar las coordenadas originales $x$, $y$, $z$ de la función mediante la
%  transformación inversa $\mmm{R}^{-1}$, estamos rotando la función según la
%  transformación directa $\mmm{R}$}.
%
%Para terminar esta explicación podríamos dar una \emph{receta} para rotar una función:
%\begin{itemize}
%\item Identifíquese una matriz de rotación $\mmm{R}$ para la función.
%\item Aplique la inversa de la matriz anterior $\mmm{R}^{-1}$, para expresar las
%  coordenadas originales en función de las nuevas
%  $\vvv{r} = \mmm{R}^{-1} \vvv{\tilde{r}}$.
%\item Sustituya las coordenadas originales en la función original por sus equivalentes
%  obtenidas en el punto anterior, para obtener la función rotada,
%  $\tilde{f}(\vvv{\tilde{r}})=f(\mmm{R}^{-1}\vvv{\tilde{r}})$
%\item Opcionalmente, reescriba la función rotada quitando las tildes a las coordenadas,
%  para obtener $\tilde{f}(x,y,z)$.
%\end{itemize}
%
%\subsubsection{Primer ejemplo}
%Rotaremos la función $f(x,y,z) = 2x+3y-z$ un ángulo de \ang{90} alrededor del eje $z$.
%
%Identificamos la matriz de rotación
%\begin{equation}\label{eq:cla-matriz-ej1}
%  \mmm{R}_z
%  =
%  \mmm{R}_z(\pi/2)
%  = e^{-\frac{\pi}{2}\/\mmm{G}_z}
%  = \begin{pmatrix}
%    \cos(\pi/2) & -\sin(\pi/2) & 0\\
%    \sin(\pi/2) & \cos(\pi/2) & 0\\
%    0 & 0 & 1
%  \end{pmatrix}
%  = \begin{pmatrix}
%    0 & -1 & 0\\
%    1 &  0 & 0\\
%    0 &  0 & 1
%  \end{pmatrix}                
%\end{equation}
%
%La inversa de la matriz, la traspuesta porque es una matriz ortogonal, es
%\begin{equation}\label{eq:cla-matrizinversa-ej1}
%  \mmm{R}_z^{-1}
%  = \mmm{R}_z^\trasp
%  = \begin{pmatrix}
%    0 &  1 & 0\\
%    -1 &  0 & 0\\
%    0 &  0 & 1
%  \end{pmatrix}    
%\end{equation}
%
%Aplicamos~\eqref{eq:cla-matrizinversa-ej1} para obtener las coordenadas originales en
%función de las rotadas
%\[
%  \vvv{r} = \mmm{R}_z^{-1} \vvv{\tilde{r}}
%\]
%\[
%  \begin{pmatrix}
%    x\\
%    y\\
%    z
%  \end{pmatrix}
%  = \begin{pmatrix}
%    0 &  1 & 0\\
%    -1 &  0 & 0\\
%    0 &  0 & 1      
%  \end{pmatrix}
%  \begin{pmatrix}
%    \tilde{x}\\
%    \tilde{y}\\
%    \tilde{z}
%  \end{pmatrix}
%\]
%produciendo las siguientes relaciones
%\begin{align*}
%  x &= \tilde{y}\\
%  y &= -\tilde{x}\\
%  z &= \tilde{z}
%\end{align*}
%
%La función transformada que buscamos se obtiene sustituyendo las coordenadas originales
%en función de las rotadas, obligando a que tenga el mismo valor que la original en los
%puntos originales y rotados
%\[
%  \tilde{f}(\tilde{x},\tilde{y},\tilde{z})
%  = f(x,y,z)
%\]
%\[
%  \tilde{f}(\tilde{x}, \tilde{y}, \tilde{z})
%  = f(x,y,z)
%  = 2x + 3y - z
%  = 2\tilde{y} + 3 (-\tilde{x}) - \tilde{z}
%  = 2\tilde{y} + 3\tilde{x} - \tilde{z}
%\]
%
%Los argumentos de la función (coordenadas) las podemos reescribir en la
%función rotada $\tilde{f}$
%\[
%  \tilde{x}\rightarrow x;\hspace{1em}
%  \tilde{y}\rightarrow y;\hspace{1em}
%  \tilde{z}\rightarrow z
%\]
%\[
%  \tilde{f}(x, y, z) = 2y + 3x - z
%\]
%
%Para terminar, vamos a aseguranos de que hemos obtenido la función rotada correcta.
%Para comprobarlo, calcularemos el valor de la función en un punto, por ejemplo
%$\vvv{r} = (1,0,0)$
%\[
%  f(1,0,0) = 2\cdot 1 + 3\cdot 0 - 0 = 2
%\]
%
%Ahora giramos el punto (vector) usando la matriz de rotación activa $R_z(\pi/2)$
%\[
%  \begin{pmatrix}
%    \tilde{x}\\
%    \tilde{y}\\
%    \tilde{z}
%  \end{pmatrix}
%  = \begin{pmatrix}
%    0 & -1 & 0\\
%    1 &  0 & 0\\
%    0 &  0 & 1      
%  \end{pmatrix}
%  \begin{pmatrix}
%    1\\
%    0\\
%    0
%  \end{pmatrix}
%  = \begin{pmatrix}
%    0\\
%    1\\
%    0
%  \end{pmatrix}
%\]
%
%El punto rotado es $\vvv{\tilde{r}} = (0,1,0)$
%\[
%  \vvv{r}(1,0,0)
%  \longrightarrow
%  \vvv{r}'(0,1,0)
%\]
%Finalmente, calculamos el valor de la función rotada
%$\tilde{f}(x,y,z) = 2y + 3x - z$ y debería dar el mismo resultado que la función original
%aplicada al punto sin rotar $f(1,0,0)=2$, véase la figura~\ref{fig:cla-rot-funcion-ej-1}
%\[
%  \tilde{f}(0,1,0) = 2\cdot 1 + 3\cdot 0 - 0 = 2
%\]
%%% .......................................................................
%\begin{figure}[ht]
%  \def\scl{1}
%  \def\longeje{3}
%  % 
%  \tdplotsetmaincoords{60}{110}
%  % 
%  \pgfmathsetmacro{\rvec}{1.9}
%  \pgfmathsetmacro{\thetavec}{90}
%  \pgfmathsetmacro{\phivec}{90}
%  % Fondo
%  \pgfmathsetmacro{\HORZ}{0.25}
%  \pgfmathsetmacro{\VERT}{0.25}
%  % 
%  \tikzfading[%
%  name=fade out, inner color=transparent!0, outer color=transparent!100
%  ]
%  
%  \centering
%  \begin{tikzpicture}[%
%    scale=\scl,tdplot_main_coords,
%    axisrotation/.style={%
%      line width=1.6pt,
%      -{Latex[round,length=12pt,width=7.0pt,bend]},
%      color=green!85!black},
%    angulogiro/.style={%
%      line width=1.4pt,-{Latex[round,length=7pt,width=4.5pt,bend]},green!85!black},
%    background/.style={%
%      line width=\bgborderwidth,
%      draw=\bgbordercolor,
%      fill=\bgcolor,
%    },
%    ]
%    % SISTEMA ORIGINAL (A LA IZQUIERDA)
%    % Para 'tikz-3dplot' son los ejes principales
%    \draw[very thick,->] (0,0,0) -- (\longeje,0,0) coordinate (ejex);
%    \node[below left,name=letraejex] at (ejex) {$x$};
%    \draw[very thick,->] (0,0,0) -- (0,\longeje,0) coordinate (ejey);
%    \node[below right,name=letraejey] at (ejey) {$y$};
%    \draw[very thick,->] (0,0,0) -- (0,0,\longeje) coordinate (ejez);
%    \node[above,name=letraejez] at (ejez){$z$};
%    
%    % REPRESENTACIÓN DEL GIRO ALREDEDOR DEL EJE Z
%    % \tdplotsetthetaplanecoords{-90}
%    % Notice you have to tell tiks-3dplot you are now in rotated coords
%    % Since tikz-3dplot swaps the planes in tdplotsetthetaplanecoords,
%    % the former y axis is now the z axis.
%    \tdplotdrawarc[axisrotation] {(0,0,2.2)}{0.35}{70}{410}{}
%
%    % Reponer parte final del eje z para simular perspectiva
%    \draw[very thick,->] (0,0,2.2) -- (0,0,\longeje);  
%    
%    % Vector de posición del punto P en gris
%    \draw[-{Latex[width=7pt]},black!50,line width=2pt]
%    (0,0,0) -- node[above left=0pt and -3pt] {\small $\vvv{r}$}
%    (\rvec,0,0) coordinate (P);
%    \filldraw[fill=gray, line width=0.4pt] (P) circle[radius=1.3pt];
%    \node[above right=-8pt and 0pt,black!50] at (P) {\small $P$};
%    \node[above left=0pt and 5pt] at (P)
%    {\footnotesize $f(\vvv{r}) = 2$};
%    
%    % Vector de posición del punto (P') rotado en rojo
%    \draw[-{Latex[width=7pt]},red!90!black,line width=2pt]
%    (0,0,0) -- node[above left=0pt and -3pt] {\small $\vvv{\tilde{r}}$}
%    (0,\rvec,0) coordinate (P');
%    \filldraw[fill=red, line width=0.4pt] (P') circle[radius=1.3pt];
%    \node[below left] at (P') {\small $\tilde{P}$};
%    \node[above right=4pt and 0pt] at (P')
%    {\footnotesize $\tilde{f}(\vvv{\tilde{r}}) = 2$};
%    
%    % ÁNGULO PHI 
%    \tdplotsetthetaplanecoords{\thetavec}
%    % Dibuja el ángulo \phi y etiquétalo
%    % sintaxis:
%    % \tdplotdrawarc[coordinate frame, draw options]
%    % {center point}{r}{angle}{label options}{label}
%    \tdplotdrawarc[angulogiro]
%    {(0,0,0)}{0.8}{8}{\phivec-10}{anchor=north}
%    {\textcolor{green!50!black}{\footnotesize $\pi/2$}};
%    % Texto de rotación
%    \node[right=1.5em] at (0,0,2.4) {\small Rotación activa de};
%    \node[right=1.5em] at (0,0,2.0) {\small $f= 2x+3y-z$};
%
%    % Fondo amarillo
%    \coordinate (SW) at ($(current bounding box.south west) + (-\HORZ cm,-\VERT cm)$);
%    \coordinate (NE) at ($(current bounding box.north east) + (\HORZ cm,\VERT cm)$);    
%    \begin{scope}[on background layer]
%      \draw[background] (SW) rectangle (NE);
%    \end{scope}
%  \end{tikzpicture}
%  \caption{Rotación activa de la función $f(x,y,z) = 2x + 3y -z$, del ejemplo 1,
%    alrededor del eje $z$, un ángulo de \ang{90}, que se representa mediante la rotación
%    de un punto $P(1,0,0)$ hasta $\tilde{P}(0,1,0)$ del espacio tridimensional.}
%  \label{fig:cla-rot-funcion-ej-1}
%\end{figure}
%
%\subsubsection{Segundo ejemplo}
%Rotar la función $f(x,y,z) = 2x+3y-z$ un ángulo $\theta$ alrededor del eje $z$.
%  
%La matriz de rotación y su inversa en función del ángulo son
%\[
%  \mmm{R}_z
%  = \begin{pmatrix}
%    \cos\theta & -\sin\theta & 0\\
%    \sin\theta &  \cos\theta & 0\\
%    0 &  0 & 1
%  \end{pmatrix}
%  \hspace*{3em}
%  \mmm{R}_z^{-1}
%  = \begin{pmatrix}
%    \cos\theta & \sin\theta & 0\\
%    -\sin\theta &  \cos\theta & 0\\
%    0 &  0 & 1
%  \end{pmatrix}    
%\]
%
%Las coordenadas originales en función de las transformadas se obtendrían mediante la
%matriz inversa
%\[
%  \begin{pmatrix}
%    x\\
%    y\\
%    z
%  \end{pmatrix}
%  = \begin{pmatrix}
%    \cos\theta & \sin\theta & 0\\
%    -\sin\theta &  \cos\theta & 0\\
%    0 &  0 & 1
%  \end{pmatrix}
%  \begin{pmatrix}
%    \tilde{x}\\
%    \tilde{y}\\
%    \tilde{z}
%  \end{pmatrix}
%  = \begin{pmatrix}
%    \tilde{x}\cos\theta + \tilde{y}\sin\theta\\
%    -\tilde{x}\sin\theta + \tilde{y}\cos\theta\\
%    z
%  \end{pmatrix}
%\]
%
%La función rotada queda
%\begin{align*}
%  \tilde{f}(\tilde{x}, \tilde{y}, \tilde{z})
%  &=
%    2x + 3y - z
%    =
%    2(\tilde{x}\cos\theta + \tilde{y}\sin\theta)
%    + 3(-\tilde{x}\sin\theta + \tilde{y}\cos\theta)
%    - \tilde{z}\\
%  &=
%    (2\cos\theta-3\sin\theta)\,\tilde{x}+(2\sin\theta+3\cos\theta)\,\tilde{y}
%    - \tilde{z}
%\end{align*}
%
%Ahora comprobaremos que la anterior es la función rotada correcta, calculando el valor de
%la función original en un punto genérico $\vvv{r}_0 = (x_0, y_0, z_0)$
%\[
%  f(\vvv{r}_0) = f(x_0, y_0, z_0) = 2x_0 + 3y_0 -z_0
%\]
%
%Ahora rotamos el punto $\vvv{r}_0 = (x_0, y_0, z_0)$ utilizando la matriz de rotación
%activa $\mmm{R}_z$, obteniendo
%\[
%  \vvv{r}_0 = (x_0, y_0, z_0)
%  \longrightarrow
%  \vvv{\tilde{r}}_0 = (\cos\theta\,x_0 - \sin\theta\,y_0,
%  \sin\theta\,x_0 + \cos\theta\,y_0,
%  z_0)
%\]
%
%La función rotada en el nuevo punto da
%{\footnotesize
%  \begin{align*}
%    \tilde{f}
%    &=
%      \tilde{f}(\tilde{x}_0, \tilde{y}_0, \tilde{z}_0)
%    =
%      f(\cos\theta\,x_0-\sin\theta\,y_0,\sin\theta\,x_0+\cos\theta\,y_0,z_0)\\
%    &=
%      (2\cos\theta-3\sin\theta)(\cos\theta\,x_0-\sin\theta\,y_0)
%      + (2\sin\theta+3\cos\theta)(\sin\theta\,x_0+\cos\theta\,y_0)
%      - z_0\\
%    &=
%      (2\cos^2\theta
%      - \cancelout{3\sin\theta\cos\theta}
%      + 2\sin^2\theta+\cancelout{3\sin\theta\cos\theta})
%      \,x_0\\
%    &\hspace*{1.0em}
%      + (\cancelout{-2\sin\theta\cos\theta}
%      + 3\sin^2\theta
%      + \cancelout{2\sin\theta\cos\theta}+3\cos^2\theta)
%      \,y_0
%      -z_0\\
%    &=
%      2x_0 + 3y_0 - z_0
%  \end{align*}
%}
%que coincide con el valor de la función $f(x_0,y_0,z_0) = 2x_0+3y_0-z_0$.
%
%\subsubsection{Tercer ejemplo}
%Nos proponemos rotar la función $f(x,y,z) = 5x-4y+z$ un ángulo de
%$\ang{60}=\pi/3\,\si{\radian}$ alrededor del eje $z$.
%
%El valor de la función un punto cualquiera, como $\vvv{r} = P(3,2,1)$ es
%\begin{equation}\label{eq:cla-valor-funcion-ej3}
%  f(\vvv{r}) = f(3,2,1) = 5\cdot 3 - 4\cdot 2 + 1 = 15 - 8 + 1 = 8
%\end{equation}
%
%Rotamos este punto mediante la matriz
%\begin{equation}\label{eq:cla-matriz-ej3}
%  \mmm{R}
%  = \mmm{R}((0,0,1), \pi/3)
%  = e^{-\frac{\pi}{3}\,\mmm{G}_z}
%  = \begin{pmatrix}
%    \cos(\pi/3) & -\sin(\pi/3) & 0\\
%    \sin(\pi/3) & \cos(\pi/3) & 0\\
%    0 & 0 & 1
%  \end{pmatrix}
%  = \begin{pmatrix}
%    1/2 & -\sqrt{3}/2 & 0\\
%    \sqrt{3}/2 & 1/2 & 0\\
%    0 & 0 & 1
%  \end{pmatrix}
%\end{equation}
%convirtiéndose en
%\[
%  \begin{pmatrix}
%    \tilde{x}\\
%    \tilde{y}\\
%    \tilde{z}
%  \end{pmatrix}
%  = \begin{pmatrix}
%    1/2 & -\sqrt{3}/2 & 0\\
%    \sqrt{3}/2 & 1/2 & 0\\
%    0 & 0 & 1
%  \end{pmatrix}
%  \begin{pmatrix}
%    3\\
%    2\\
%    1
%  \end{pmatrix}
%  = \begin{pmatrix}
%    (3-2\sqrt{3})/2\\
%    (3\sqrt{3}+2)/2\\
%    1
%  \end{pmatrix}
%\]
%\[
%  \vvv{r} = P(3,2,1)
%  \longrightarrow
%  \vvv{\tilde{r}}
%  = \tilde{P}\left(\frac{3-2\sqrt{3}}{2}, \frac{3\sqrt{3}+2}{2}, 1\right)
%\]
%
%Tenemos que encontrar la función rotada
%$\tilde{f}(\tilde{x},\tilde{y},\tilde{z})$ de manera que se cumpla
%\[
%  \tilde{f}(\vvv{\tilde{r}}) = f(\vvv{r}) = 8
%\]
%
%Para encontrar la función para cualquier valor de los argumentos, aplicamos la matriz
%inversa de~(\ref{eq:cla-matriz-ej3}) para encontrar las coordenadas originales en función
%de las rotadas
%\[
%  \begin{pmatrix}
%    x\\
%    y\\
%    z
%  \end{pmatrix}
%  = \begin{pmatrix}
%    1/2 & \sqrt{3}/2 & 0\\
%    -\sqrt{3}/2 & 1/2 & 0\\
%    0 & 0 & 1
%  \end{pmatrix}
%  \begin{pmatrix}
%    \tilde{x}\\
%    \tilde{y}\\
%    \tilde{z}
%  \end{pmatrix}
%  = \begin{pmatrix}
%    \frac{1}{2}\tilde{x} + \frac{\sqrt{3}}{2}\tilde{y}\\
%    \frac{-\sqrt{3}}{2}\tilde{x} + \frac{1}{2}\tilde{y}\\
%    \tilde{z}
%  \end{pmatrix}
%\]
%
%Sustituimos estas coordenadas en la función original, para obtener la función rotada
%\begin{align*}
%  \tilde{f}(\tilde{x},\tilde{y},\tilde{z})
%  &=
%    5x - 4y + z
%    = 5\left(\frac{1}{2}\tilde{x} + \frac{\sqrt{3}}{2}\tilde{y}\right)
%    - 4\left(\frac{-\sqrt{3}}{2}\tilde{x} + \frac{1}{2}\tilde{y}\right)
%    + \tilde{z}\\
%  &=
%    \frac{5}{2}\tilde{x} + \frac{5\sqrt{3}}{2}\tilde{y}
%    + \frac{4\sqrt{3}}{2}\tilde{x} - \frac{4}{2}\tilde{y}
%    + \tilde{z}
%\end{align*}
%
%Operando la expresión anterior, la función rotada queda
%\[
%  \tilde{f}(\tilde{x},\tilde{y},\tilde{z})
%  = \frac{5+4\sqrt{3}}{2}\tilde{x}
%  + \frac{5\sqrt{3}-4}{2}\tilde{y}
%  + \tilde{z}
%\]
%
%Los argumentos de la  función son mudos y en realidad son coordenadas cartesianas, se
%pueden reescribir 
%$\tilde{x}\rightarrow x$, $\tilde{y}\rightarrow y$ y $\tilde{z}\rightarrow z$
%quedando la función rotada en su forma final
%\begin{equation}\label{eq:cla-functionrotada-ej3}
%  \tilde{f}(x,y,z)
%  = \frac{5+4\sqrt{3}}{2} x
%  + \frac{5\sqrt{3}-4}{2} y
%  + z
%\end{equation}
%
%Por último, comprobamos que el valor de la función rotada en el punto rotado $\tilde{P}$
%es el mismo que el de la función original en el punto $P$,
%ver~\eqref{eq:cla-valor-funcion-ej3}
%\begin{align*}
%  \tilde{f}(\vvv{\tilde{r}})
%  &=
%  \tilde{f}\left(\frac{3-2\sqrt{3}}{2},\frac{3\sqrt{3}+2}{2},1\right)\\
%  &=
%    \frac{5+4\sqrt{3}}{2}\cdot \frac{3-2\sqrt{3}}{2}
%    + \frac{5\sqrt{3}-4}{2}\cdot \frac{3\sqrt{3}+2}{2}
%    + 1\\
%  &=
%    \frac{15-\cancelout{10\sqrt{3}}+\cancelout{12\sqrt{3}}-24}{4}
%    + \frac{45+\cancelout{10\sqrt{3}}-\cancelout{12\sqrt{3}-8}}{4}
%    + 1\\
%  &=
%    \frac{15-24+45-8}{4} + 1 = \frac{28}{4} + 1 = 7 + 1 = 8
%\end{align*}
%
%\subsubsection{Cuarto ejemplo}
%Rotar la función un ángulo $\theta$ alrededor del eje $z$
%\[
%  f(x,y,z) = 2xz^2 - x^2y
%\]
%
%La matriz de rotación activa que querríamos aplicar a la función es
%\[
%  \mmm{R}_z(\theta) = e^{-\theta \mmm{G}_z}
%\]
%
%Necesitamos la matriz inversa para obtener la expresión rotada de la función
%\[
%  \mmm{R}_z^{-1}(\theta)
%  =
%  e^{\theta \mmm{G}_z}
%  = \begin{pmatrix}
%    \cos\theta & \sin\theta & 0\\
%    -\sin\theta & \cos\theta & 0\\
%    0 & 0 & 1
%  \end{pmatrix}
%\]
%
%Las coordenadas originales se pueden poner en función de las rotadas mediante la matriz
%inversa
%\[
%  \vvv{r}
%  = \mmm{R}_z^{-1}(\theta)\,\vvv{\tilde{r}}
%  = e^{\theta\mmm{G}_z} \,\vvv{\tilde{r}}
%\]
%\[
%  \begin{pmatrix}
%    x \\
%    y \\
%    z
%  \end{pmatrix}
%  = \begin{pmatrix}
%    \cos\theta & \sin\theta & 0\\
%    -\sin\theta & \cos\theta & 0\\
%    0 & 0 & 1
%  \end{pmatrix}
%  \begin{pmatrix}
%    \tilde{x} \\
%    \tilde{y} \\
%    \tilde{z}
%  \end{pmatrix}
%  = \begin{pmatrix}
%    \tilde{x}\cos\theta + \tilde{y}\sin\theta\\
%    -\tilde{x}\sin\theta\,\tilde{x} + \tilde{y}\cos\theta\\
%    \tilde{z}
%    \end{pmatrix}
%\]
%
%Produciendo las siguientes transformaciones de coordenadas
%\begin{align*}
%  x &= \tilde{x}\cos\theta + \tilde{y}\sin\theta\\
%  y &= -\tilde{x}\sin\theta + \tilde{y}\cos\theta\\
%  z &= \tilde{z}
%\end{align*}
%
%La función rotada debe tener el mismo valor que la original con las coordenadas sin rotar
%\begin{align*}
%  \tilde{f}(\tilde{x}, \tilde{y}, \tilde{z})
%  &= f(x,y,z) =
%    f(
%    \tilde{x}\cos\theta+\tilde{y}\sin\theta
%    ,-\tilde{x}\sin\theta+\tilde{y}\cos\theta
%    ,\tilde{z}
%    )\\
%  &=
%    2\kern1pt\tilde{z}^2 \kern1pt(\tilde{x}\cos\theta+\tilde{y}\sin\theta)
%    - (\tilde{x}\cos\theta+\tilde{y}\sin\theta)^2
%    (-\tilde{x}\sin\theta+\tilde{y}\cos\theta)
%\end{align*}
%
%Y se puede escribir como
%\[
%  \tilde{f}(x,y,z)
%  = 
%    2z^2 (x\cos\theta+y\sin\theta)
%    - (x\cos\theta+y\sin\theta)^2
%    (-x\sin\theta+y\cos\theta)
%\]
%
%
%Si $\theta=\pi/2\,\si{\radian} = \ang{90}$
%{\small
%\begin{align*}
%  \tilde{f}(x, y, z)
%  &=
%    2z^2 \left(x\cos\frac{\pi}{2}+y\sin\frac{\pi}{2}\right)
%    - \left(x\cos\frac{\pi}{2}+y\sin\frac{\pi}{2}\right)^2
%    \left(-x\sin\frac{\pi}{2}+y\cos\frac{\pi}{2}\right)\\
%  &=
%    2z^2y
%    -y^2
%    (-x)
%    = 2yz^2 + xy^2
%\end{align*}
%}

\section{Invariancia de una función bajo rotaciones}
Para el propósito que nos ocupa utilizaremos una función definida en $\symbb{R}^2$, como
\[
  f(x,y) = x^2 + y^2
\]

\subsection{Rotación finita}
La vamos a rotar un ángulo finito $\theta$ los ejes $x$ e $y$.
Las coordenadas se transforman como
\[
  \begin{pmatrix}
    \tilde{x}\\
    \tilde{y}
  \end{pmatrix}
  =
  \begin{pmatrix}
    \cos\theta & \sin\theta\\
    -\sin\theta & \cos\theta
  \end{pmatrix}
  \,
  \begin{pmatrix}
    x\\
    y
  \end{pmatrix}
\]
Operando la expresión matricial anterior, obtenemos las nuevas coordenadas (con tilde)
en función de las originales
\begin{align*}
  \tilde{x}
  &= x\cos\theta + y\sin\theta\\
  \tilde{y}
  &= -x\sin\theta + y\cos\theta
\end{align*}

La función rotada un ángulo finito $\theta$ es
{\small
\begin{align*}
  \tilde{f}(\tilde{x},\tilde{y})
  &=
    \tilde{x}^2 + \tilde{y}^2
  = (\tilde{x}\cos\theta + \tilde{y}\sin\theta)^2
    + (-\tilde{x}\sin\theta + \tilde{y}\cos\theta)^2\\
  &=
    \tilde{x}^2\cos^2\theta + \tilde{y}^2\sin^2\theta
    + \cancelout{2\tilde{x}\tilde{y}\sin\theta\cos\theta}
    + \tilde{x}^2\sin^2\theta + \tilde{y}^2\cos^2\theta
    - \cancelout{2\tilde{x}\tilde{y}\sin\theta\cos\theta}\\
  &= \tilde{x}^2(\sin^2\theta + \cos^2\theta)
    + \tilde{y}^2(\sin^2\theta + \cos^2\theta)
    = x^2 + y^2
\end{align*}
}

Así, la función rotada es idéntica a la original%, ver figura~\ref{fig:cla-x2y2}
\[
  \tilde{f}(\tilde{x},\tilde{y}) = \tilde{x}^2 + \tilde{y}^2 = x^2 + y^2 = f(x,y)
\]

Cuando la función tiene la misma forma antes que después de rotar los ejes, como en este
caso, se dice que $f$ es invariante bajo la rotación.

\subsection{Rotación infinitesimal}
Hemos probado la invariancia con una rotación finita. Ahora llevaremos a cabo una
rotación infinitesimal de un ángulo $\varepsilon$, porque a menudo es más comodo realizar
transformaciones infinitesimales.

Primero escribimos el desarrollo de Taylor de las funciones seno y coseno
\begin{align*}
  \cos\theta &= 1 - \dfrac{\theta^2}{2!} + \dfrac{\theta^4}{4!} + \cdots\\
  \sin\theta &= \theta - \dfrac{\theta^3}{3!} + \dfrac{\theta^5}{5!} + \cdots
\end{align*}

Recordemos que cuando consideramos una cantidad $\varepsilon$ como infinitesimal, estamos
despreciando los términos en $\varepsilon^2$ y superiores. Las funciones anteriores, en
función de $\varepsilon$ quedan
\begin{align*}
  \cos\varepsilon
  &=
    1-\dfrac{\varepsilon^2}{2!} + \dfrac{\varepsilon^4}{4!} + \cdots \approx 1\\
  \sin\varepsilon
  &=
    \varepsilon - \dfrac{\varepsilon^3}{3!}
    + \dfrac{\varepsilon^5}{5!} + \cdots \approx \varepsilon
\end{align*}

La rotación en el plano $xy$ modifica las coordenadas
\begin{align*}
  x \longrightarrow \,&\tilde{x} = \cos\varepsilon\, x + \sin\varepsilon\, y
                        = x + \varepsilon y\\
  y \longrightarrow \,&\tilde{y} = -\sin\varepsilon\,x + \cos\varepsilon\,y
                        = -\varepsilon x + y = y -\varepsilon x
\end{align*}

El cambio en las coordenadas es
\begin{align*}
  \Delta x &= \tilde{x} - x = x + \varepsilon y - x = \varepsilon y\\
  \Delta y &= \tilde{y} - y = -\varepsilon x + y -y = -\varepsilon x
\end{align*}

El desarrollo en serie de la función $\tilde{f}(\tilde{x},\tilde{y})$, de dos variables
después de la rotación infinitesimal es
\[
  \tilde{f}(\tilde{x},\tilde{y})
  =
  f(x,y)
  + \dfrac{\partial f}{\partial x} \Delta x
  + \dfrac{\partial f}{\partial y} \Delta y
  = x^2 + y^2
  + \cancelout{2x\,(y\varepsilon)}
  + \cancelout{2y\,(-x\varepsilon)}
  = x^2 + y^2
\]

Resulta que
\[
  \tilde{f} = f
\]








\section{El momento angular y el teorema de Noether}
El teorema de Noether relaciona la invariancia del lagrangiano bajo ciertas
transformaciones con la conservación de algunas magnitudes físicas.
Por ejemplo, cuando el lagrangiano es invariante bajo una rotación, entonces se conserva
el momento angular.
En este contexto también se puede afirmar que hay una relación entre una rotación y el
momento angular.

\subsection{Invariancia de una función bajo rotaciones}
Supongamos que tenemos una función de dos variables\footnotemark{}, como
\footnotetext{Elegimos una función de dos variables, en lugar de tres, para que se pueda
  comprobar gráficamente la invariancia; una función de tres variables no se podría
  representar gráficamente.}
\begin{equation}\label{eq:cla-funcion-invariante-ejemplo}
  f(x,y) = x^2 + y^2
\end{equation}
La función rotada un ángulo finito $\theta$ es
{\small
\begin{align*}
  \tilde{f}(\tilde{x},\tilde{y})
  &=
    x^2 + y^2
  = (\tilde{x}\cos\theta + \tilde{y}\sin\theta)^2
    + (-\tilde{x}\sin\theta + \tilde{y}\cos\theta)^2\\
  &=
    \tilde{x}^2\cos^2\theta + \tilde{y}^2\sin^2\theta
    + \cancelout{2\tilde{x}\tilde{y}\sin\theta\cos\theta}
    + \tilde{x}^2\sin^2\theta + \tilde{y}^2\cos^2\theta
    - \cancelout{2\tilde{x}\tilde{y}\sin\theta\cos\theta}\\
  &= \tilde{x}^2(\sin^2\theta + \cos^2\theta)
    + \tilde{y}^2(\sin^2\theta + \cos^2\theta)
    = \tilde{x}^2 + \tilde{y}^2
\end{align*}
}

Así, la función rotada es idéntica a la original, ver figura~\ref{fig:cla-x2y2}
\[
  \tilde{f}(x,y) = x^2 + y^2 = f(x,y)
\]

Cuando la función tiene la misma forma antes que después de rotar los ejes, como en este
caso, se dice que $f$ es invariante bajo la rotación.

Hemos probado la invariancia con una rotación finita. Ahora llevaremos a cabo una
rotación infinitesimal de un ángulo $\varepsilon$, porque a menudo es más comodo realizar
transformaciones infinitesimales.

Primero escribimos el desarrollo de Taylor de las funciones seno y coseno
\begin{align*}
  \cos\theta &= 1 - \dfrac{\theta^2}{2!} + \dfrac{\theta^4}{4!} + \cdots\\
  \sin\theta &= \theta - \dfrac{\theta^3}{3!} + \dfrac{\theta^5}{5!} + \cdots
\end{align*}

Recordemos que cuando consideramos una cantidad $\varepsilon$ como infinitesimal, estamos
despreciando los términos en $\varepsilon^2$ y superiores. Las funciones anteriores, en
función de $\varepsilon$ quedan
\begin{align*}
  \cos\varepsilon
  &=
    1-\dfrac{\varepsilon^2}{2!} + \dfrac{\varepsilon^4}{4!} + \cdots \approx 1\\
  \sin\varepsilon
  &=
    \varepsilon - \dfrac{\varepsilon^3}{3!}
    + \dfrac{\varepsilon^5}{5!} + \cdots \approx \varepsilon
\end{align*}

La rotación en el plano $xy$ modifica las coordenadas
\begin{align*}
  x \longrightarrow \,&\tilde{x} = \cos\varepsilon\, x + \sin\varepsilon\, y
                        = x + \varepsilon y\\
  y \longrightarrow \,&\tilde{y} = -\sin\varepsilon\,x + \cos\varepsilon\,y
                        = -\varepsilon x + y = y -\varepsilon x
\end{align*}

El cambio en las coordenadas es
\begin{align*}
  \Delta x &= \tilde{x} - x = x + \varepsilon y - x = \varepsilon y\\
  \Delta y &= \tilde{y} - y = -\varepsilon x + y -y = -\varepsilon x
\end{align*}

El desarrollo en serie de la función $\tilde{f}(\tilde{x},\tilde{y})$, de dos variables
después de la rotación infinitesimal es
\[
  \tilde{f}(\tilde{x},\tilde{y})
  =
  f(x,y)
  + \dfrac{\partial f}{\partial x} \Delta x
  + \dfrac{\partial f}{\partial y} \Delta y
  = x^2 + y^2
  + \cancelout{2x\,(y\varepsilon)}
  + \cancelout{2y\,(-x\varepsilon)}
  = x^2 + y^2
\]

Resulta que
\[
  \tilde{f} = f
\]

Por tanto, la función $f(x,y) = x^2 + y^2$ es invariante bajo una rotación infinitesimal.
\begin{figure}[ht]
  \def\scl{1}
  % Fondo
  \pgfmathsetmacro{\HORZIZDA}{1.0}
  \pgfmathsetmacro{\HORZDCHA}{0.25}
  \pgfmathsetmacro{\VERT}{0.25}
  % 
  \centering
  \begin{tikzpicture}[%
    scale=\scl,
    baseline,
    background/.style={
      line width=\bgborderwidth,
      draw=\bgbordercolor,
      fill=\bgcolor,
    },    
    ]
    \begin{axis}[%
      width=7cm,
      xlabel=$x$,ylabel=$y$,zlabel=$z$,
      %enlargelimits=false,
      mesh/interior colormap name=greenyellow,
      %mesh/interior colormap name=hot,
      colormap/blackwhite,
      %data cs=polar,
      %domain=0:360,
      %y domain=0:5,
      %zmin=0,
      %zmax=25,
      variable = \u,
      variable y = \r,
      domain=0:359.99999,
      y domain = 0:5,
      enlargelimits=true,
      zmax=32,
      plot box ratio=1 1 4,
      hide axis,
      ]
      \addplot3[surf,samples=60] ({r*cos(u)},{r*sin(u)},{r^2});
      % Eje z
      \draw (0,0,22.8) -- (0,0,32);
      %node[above,name=letraejez] {\small $z$};
      % Eje y
      \draw[-{>}] (0,0,0) -- (7,0,0)
      node[right,name=letraejey] {\small $y$};
      % Eje x
      \draw[-{>}] (0,0,0) -- (0,-9,0)
      node[left,name=letraejex] {\small $x$};
      % Parte de la función que tapa la parte inferior del eje z
      %\addplot3[surf,samples=50,domain=180:190] ({r*cos(u)},{r*sin(u)},{r^2});
      % Inversa del eje y para que el recuadro del gráfico esté centrado
      % en el eje z
      %\path[-{>}] (0,0,0) -- (-7,0,0)
      %node[left,name=letraejeyinv] {\small $\phantom{y}$};

      % Rotación
      \draw[-{Stealth[round,width=4.5pt]}, green!80!black, line width=1.2pt]
      (0,1.25,30) arc[%
      start angle=-35,end angle=310,x radius=1,y radius=1
      ];
      %\draw[-{Stealth[round,width=4.5pt]}, black!40, line width=1.2pt]
      %(0,1.25,25) arc[%
      %start angle=-35,end angle=310,x radius=1,y radius=1
      %];
      % Reponer eje z
      \draw[-{>}] (0,0,32) -- (0,0,35)
      node[above,name=letraejez] {\small $z$};
      %\filldraw[fill=red,draw=black,ultra thin] (0,0,22.7) circle [radius=.2pt];

      % Fondo amarillo
      \coordinate (SW)
      at ($(current bounding box.south west) + (-\HORZIZDA cm,-\VERT cm)$);
      \coordinate (NE)
      at ($(current bounding box.north east) + (\HORZDCHA cm,\VERT cm)$);    
      \begin{scope}[on background layer]
        \draw[background] (SW) rectangle (NE);
      \end{scope}
    \end{axis}
  \end{tikzpicture}%
  \caption{La función $f(x,y) = x^2+y^2$ no varía al rotar alrededor del
    eje $z$.}
  \label{fig:cla-x2y2}
\end{figure}

¿Bastaría con saber que se mantiene invariable la función bajo una transformación
infinitesimal para colegir que ocurrirá lo mismo para una transformación finita?
La respuesta es sí, con la condición de que la transformación forme un grupo continuo en
el que se obtenga la transformación identidad cuando el parámetro sea cero, $\theta=0$,
osea que la transformación forme un grupo de Lie.

\subsection{Invariancia del lagrangiano bajo rotaciones}
El lagrangiano de un sistema es una función de las coordenadas generalizadas,
$q_1, q_2, \cdots$ y las velocidades respectivas, $\dot{q}_1, \dot{q}_2, \cdots$.
Se define como la energía cinética menos la energía potencial del sistema
\[
  \symcal{L}(\vvv{q},\vvv{\dot{q}})
  = T - U
\]

A partir de él se pueden deducir las ecuaciones de Euler-Lagrange, que en mecánica son el
equivalente de las ecuaciones del movimiento
\begin{equation}\label{eq:cla-EulerLagrange}
  \dfrac{\partial \symcal{L}}{\partial q_i}
  =
  \dfrac{d}{dt}\,\left(\dfrac{\partial\symcal{L}}{\partial \dot{q_i}}\right)
  \hspace*{2em}
  (i = 1, 2, \cdots, N)
\end{equation}

Se suele llamar \emph{momento generalizado} asociado a la coordenada $q_i$ a la cantidad
\[
  p_i = p_{q{_i}} = \dfrac{\partial\symcal{L}}{\partial\dot{q}_i}
\]

Nos preguntamos en qué tipo de situaciones físicas es invariante el lagrangiano bajo una
rotación.
Como las rotaciones forman un grupo de Lie, si el lagrangiano fuera invariante bajo una
rotación infinitesimal, querría decir que también lo sería bajo cualquier rotación
finita. 

La matriz de rotación infinitesimal activa (antihoraria) del lagrangiano es
\[
  \mmm{R}(\xhat{n},\varepsilon)
  =
  e^{-\varepsilon\xhat{n}\cdot\mmm{G}}
  \mmm
  \approx{I} - \varepsilon\xhat{n}\cdot\mmm{G}
\]

Consideraremos el caso más simple de un sistema formado por una partícula\footnotemark{}.
\footnotetext{El resultado se podría generalizar fácilmente para un sistema de $N$
  partículas.}
El lagrangiano $\symcal{L}$ de una partícula es una función de su posición y su velocidad.
Cuando lo rotamos, el nuevo lagrangiano $\tilde{\symcal{L}}$ dependerá de la posición y
velocidad rotadas
\[
  \symcal{L}(\vvv{r}, \vvv{\dot{r}})
  \longrightarrow
  \tilde{\symcal{L}}(\vvv{\tilde{r}}, \vvv{\dot{\tilde{r}}})
\]
y queremos averiguar bajo qué condiciones se cumple la invariancia
$\tilde{\symcal{L}} = \symcal{L}$.

Como $\symcal{L}$ es una función, para rotarlo mediante $\mmm{R}$, aplicamos la rotación
inversa a las coordenadas de la función
\[
  \vvv{\tilde{r}}
  =
  \mmm{R}^{-1}(\varepsilon)\vvv{r}
  =
  e^{\varepsilon \xhat{n}\cdot\mmm{G}} \vvv{r}
  =
  (\mmm{I} + \varepsilon\xhat{n}\cdot\mmm{G})\vvv{r}
  =
  \vvv{r} + \varepsilon(\xhat{n}\cdot\mmm{G})\vvv{r}
\]

Se ha producido un cambio en la posición
\begin{equation}\label{eq:cla-delta-posicion}
  \Delta\vvv{r}
  =
  \vvv{\tilde{r}} - \vvv{r}
  =
  \cancelout{\vvv{r}}
  + \varepsilon(\xhat{n}\cdot\mmm{G})\vvv{r}
  - \cancelout{\vvv{r}}
  =
  \varepsilon(\xhat{n}\cdot\mmm{G})\vvv{r}
\end{equation}

De forma similar podríamos haber calculado el cambio en la velocidad como consecuencia de
la rotación infinitesimal
\begin{equation}\label{eq:cla-delta-velocidad}
  \Delta\dot{\vvv{r}}
  =
  \vvv{\dot{\tilde{r}}} - \dot{\vvv{r}}
  =
  \cancelout{\vvv{\dot{r}}}
  + \varepsilon(\xhat{n}\cdot\mmm{G})\vvv{\dot{r}}
  - \cancelout{\vvv{\dot{r}}}
  =
  \varepsilon(\xhat{n}\cdot\mmm{G})\vvv{\dot{r}}
\end{equation}
donde hemos tenido en cuenta que $\xhat{n}$ y $\mmm{G}$ son independientes del tiempo.

Desarrollamos en serie el lagrangiano transformado por la rotación infinitesimal,
despreciando los términos de orden $\varepsilon^2$ o superior y sustiyendo los
incrementos~\eqref{eq:cla-delta-posicion} y~\eqref{eq:cla-delta-velocidad}
\begin{align*}
  \tilde{\symcal{L}}(\tilde{\vvv{r}},\dot{\tilde{\vvv{r}}})
  &=
  \symcal{L}
  (\vvv{r}+\varepsilon\xhat{n}\cdot\mmm{G}\vvv{r},
    \vvv{\dot{r}}+\varepsilon\xhat{n}\cdot\mmm{G}\vvv{\dot{r}})\\
  &\approx \symcal{L}(\vvv{r},\vvv{\dot{r}})
    + \left(\dfrac{\partial\symcal{L}}{\partial \vvv{r}}\right)^\trasp
    \Delta\vvv{r}
    + \left(\dfrac{\partial\symcal{L}}{\partial \vvv{\dot{r}}}\right)^\trasp
    \Delta\vvv{\dot{r}}\\
  &= \symcal{L}(\vvv{r},\vvv{\dot{r}})
    + \left(\dfrac{\partial\symcal{L}}{\partial \vvv{r}}\right)^\trasp
    \varepsilon(\xhat{n}\cdot\mmm{G})\vvv{r}
    + \left(\dfrac{\partial\symcal{L}}{\partial \vvv{\dot{r}}}\right)^\trasp
    \varepsilon(\xhat{n}\cdot\mmm{G})\vvv{\dot{r}}
\end{align*}

La traspuesta de la parcial del lagrangiano con respecto a las coordenadas es
\[
  \left(\dfrac{\partial\symcal{L}}{\partial\vvv{r}}\right)^\trasp
  =
\begin{pmatrix}
    \dfrac{\partial\symcal{L}}{\partial x}
    & \dfrac{\partial\symcal{L}}{\partial y}
    & \dfrac{\partial\symcal{L}}{\partial z}
    \end{pmatrix}
\]
y una fórmula similar para la traspuesta de la parcial con respecto a la velocidad.

Abreviamos ligeramente la notación
\[
  \tilde{\symcal{L}}
  =
    \symcal{L}
    + \left(\dfrac{\partial\symcal{L}}{\partial\vvv{r}}\right)^\trasp
    \varepsilon\,(\xhat{n}\cdot\mmm{G})\kern1pt\vvv{r}
    + \vvv{p}^\trasp
    \varepsilon\,(\xhat{n}\cdot\mmm{G})\kern1pt\vvv{\dot{r}}\\
\]
    
A continuación sacamos factor común $\varepsilon$ y sustituimos
$\partial\symcal{L}/\partial\vvv{r}$
por la expresión de las ecuaciones de Euler-Lagrange~(\ref{eq:cla-EulerLagrange})
\begin{align*}
  \tilde{\symcal{L}}
  &=
    \symcal{L}
    + \varepsilon\,\left[
    \dfrac{d}{dt}\left(
    \frac{\symcal{L}}{\partial\vvv{\dot{r}}}
    \right)^\trasp
    (\xhat{n}\cdot\mmm{G})\kern1pt\vvv{r}
    + \vvv{p}^\trasp
    (\xhat{n}\cdot\mmm{G})\kern1pt\vvv{\dot{r}}
    \right]\\
  &=
    \symcal{L}
    + \varepsilon\,\left[
    \left(\dfrac{d\vvv{p}}{dt}\right)^\trasp
    (\xhat{n}\cdot\mmm{G})\kern1pt\vvv{r}
    + \vvv{p}^\trasp (\xhat{n}\cdot\mmm{G})\kern1pt\vvv{\dot{r}}
    \right]\\
  &=
  \symcal{L}
    + \varepsilon\,\left[\vvv{\dot{p}}^\trasp
    (\xhat{n}\cdot\mmm{G})\kern1pt\vvv{r}
    + \vvv{p}^\trasp
    (\xhat{n}\cdot\mmm{G})\kern1pt\vvv{\dot{r}}\right]
\end{align*}

Reconocemos la derivada temporal de un producto en la ecuación anterior
\begin{equation}\label{eq:cla-lagrangiana_rotada_1}
  \tilde{\symcal{L}}
  =
    \symcal{L}
    + \varepsilon\,\dfrac{d}{dt}\left[\vvv{p}^\trasp
      (\xhat{n}\cdot\mmm{G})\,\vvv{r} \right]
\end{equation}
Hemos tenido en cuenta que $\xhat{n}\cdot\mmm{G}$ no depende del tiempo
\[
  \xhat{n}\cdot\mmm{G}
  =
  \begin{pmatrix}
    0 & n_3 & -n_2\\
    -n_3 & 0 & n_1\\
    n_2 & -n_1 & 0
  \end{pmatrix}
\]
y que es una matriz antisimética
\[
  (\xhat{n}\cdot\mmm{G})^\trasp = -\xhat{n}\cdot\mmm{G}
\]

Fijémomos en el término
$\vvv{p}^\trasp (\xhat{n}\cdot\mmm{G}) \vvv{r}$ de la
ecuación~(\ref{eq:cla-lagrangiana_rotada_1}). Como es un número, su traspuesta es el
mismo número
\[
  \vvv{p}^\trasp (\xhat{n}\cdot\mmm{G}) \vvv{r}
  =
  [\vvv{p}^\trasp (\xhat{n}\cdot\mmm{G}) \vvv{r}]^\trasp
\]
Desarrollamos el segundo miembro de la igualdad
\[
  \vvv{p}^\trasp (\xhat{n}\cdot\mmm{G}) \vvv{r}
  =
  [\vvv{r}^\trasp (\xhat{n}\cdot\mmm{G})^\trasp \vvv{p}]
\]
Como $\xhat{n}\cdot\mmm{G}$ es antisimétrica
\[
  \vvv{p}^\trasp (\xhat{n}\cdot\mmm{G}) \vvv{r}
  =
  -\vvv{r}^\trasp (\xhat{n}\cdot\mmm{G}) \vvv{p}
\]

Susituimos la expresión anterior en~(\ref{eq:cla-lagrangiana_rotada_1})
\[
  \tilde{\symcal{L}}
  = \symcal{L} -
  \varepsilon\,\dfrac{d}{dt}\left[\vvv{r}^\trasp
    (\xhat{n}\cdot\mmm{G})\,\vvv{p} \right]
\]

Teniendo en cuenta~\eqref{eq:cla-Ln}, vemos que el interior del corchete es la componente
del momento angular del sistema con respecto de la dirección $\xhat{n}$ de rotación del
lagrangiano.
Llegamos a la expresión del lagrangiano cuando se rota infinitesimalmente
\begin{equation}
  \tilde{\symcal{L}}
  = \symcal{L} -
  \varepsilon\,\dfrac{d}{dt}\left(\xhat{n}\cdot\vvv{L}\right)
\end{equation}

Si resultase que la componente del momento angular con respecto de cualquier eje no
dependiera del tiempo
\[
  \dfrac{d}{dt}(\xhat{n}\cdot\vvv{L}) = 0
\]
entonces el lagrangiano sería un invariante con respecto a las rotaciones continuas
SO(3), y $\xhat{n}\cdot\vvv{L}$ sería constante, lo que quiere decir que la componente
del momento angular con respecto a cualquier eje es constante, por lo que el momento angular $\vvv{L}$ no cambia con el tiempo.

Así, según el teorema de Noether, si el lagrangiano de un sistema es invariante bajo
rotaciones SO(3), se conserva el momento angular.
En general, si se presentara otro tipo de simetría (traslaciones, etc.) se conservaría
otra magnitud física.

El momento angular está pues relacionado con las rotaciones.
Para terminar adelantamos que como el spin es un momento angular, también estará
relacionado con rotaciones, pero ¿qué tipo de rotaciones?
La respuesta la descubriremos más adelante.



 
%%% Local Variables:
%%% mode: latex
%%% TeX-engine: luatex
%%% TeX-master: "../gruposlie.tex"
%%% End:
