% momangclasico.tex
%
% Copyright (C) 2022--2025 José A. Navarro Ramón <janr.devel@gmail.com>
% Licencia del código GPLv2
% Licencia Creative Commons Recognition Non-Commercial Share-alike.
% (CC-BY-NC-SA)

\chapter{El momento angular en física clásica}
El momento angular se comporta algo así como un vector bajo rotaciones. A continuación
vamos a estudiarlo por encima desde el punto de vista de la física clásica.

\section{Definición}
El momento angular de una partícula se define como el producto vectorial de su posición
por la cantidad de movimiento o momento lineal
\[
  \vvv{L} \equiv \vvv{r} \prodvec \vvv{p}
\]

Tanto la posición como el momento lineal se definen, en general, en función de las
coordenadas generalizadas $q$; aunque en los casos más sencillos, esta última magnitud
será el familiar producto de la masa de la partícula por su velocidad,
$\vvv{p} = m\vvv{v}$ en coordenadas cartesianas.

El producto vectorial sólo está definido en $\symbb{R}^3$ y, aunque se puede generalizar
a dimensiones más elevadas, no lo está en $\symbb{R}^2$, porque se necesita un mínimo de
tres dimensiones.

Ambas magnitudes, posición $\vvv{r} = \vvv{r}(t)$ y momento lineal
$\vvv{p} = \vvv{p}(t)$, dependen en general del tiempo, de modo que el momento angular
también suele depender de este $\vvv{L} = \vvv{L}(t)$.

En ciertas ocasiones el momento angular es constante, por ejemplo, cuando una partícula
está sujeta únicamente a una fuerza central (por ejemplo la Luna rotando alrededor de la
Tierra y despreciando el resto de interacciones, como las debidas a otros astros).
Cuando ocurre esto, se pueden resolver las ecuaciones del movimiento de una forma más
sencilla.

Calculamos el momento angular de una partícula en coordenadas cartesianas rectangulares
\[
  \vvv{L}
  =
  \vvv{r} \prodvec \vvv{p}
  =
  \begin{vmatrix}
    \uvec{\i} & \uvec{\j} & \xhat{k}\\
    x & y & z\\
    p_x & p_y & p_z
  \end{vmatrix}
  =
  (yp_z-zp_y)\,\uvec{i}
  + (zp_x-xp_z)\,\uvec{j}
  + (xp_y-yp_z)\,\xhat{k}
\]

De la expresión anterior, deducimos sus componentes cartesianas:
\begin{align*}
  L_x &= yp_z - zp_y\\
  L_y &= zp_x - xp_z\\
  L_z &= xp_y - yp_x
\end{align*}
  
Hay una forma más general para calcular la componente del momento angular (o de cualquier
vector) con respecto a cualquier dirección, y es multiplicar escalarmente el vector
unitario que tenga esa dirección por el vector. Así, si queremos calcular, por ejemplo
las componentes cartesianas del momento angular, podemos hacer
\begin{align*}
  L_x &= \uvec{\i}\cdot\vvv{L}\\
  L_y &= \uvec{\j}\cdot\vvv{L}\\
  L_z &= \xhat{k}\cdot\vvv{L}
\end{align*}


\section{Componente del momento angular con respecto a un eje}
En general, si $\xhat{n} = (n_x, n_y, n_z)$ es el versor que define una dirección
cualquiera del espacio, $\vvv{r} = (x, y, z)$ y $\vvv{p} = (p_x, p_y, p_z)$, la
componente del momento angular según esa dirección es
\[
  L_n = \xhat{n}\cdot\vvv{L}
  = n_x(yp_z-zp_y) + n_y(zp_x-xp_z) + n_z(xp_y-yp_x)
\]

Esta componente también se puede representar de forma matricial
\[
  L_n
  = \xhat{n}\cdot\vvv{L}
    =
    \begin{pmatrix}
      x & y & z
    \end{pmatrix}
    \begin{pmatrix}
      0 & n_z & -n_y\\
      -n_z & 0 & n_x\\
      n_y & -n_x & 0
    \end{pmatrix}
    \begin{pmatrix}
      p_x \\ p_y \\ p_z
    \end{pmatrix}
\]
\[
  L_n
   =
    \begin{pmatrix}
      x & y & z
    \end{pmatrix}
    \left[
    n_x
    \begin{pmatrix}
      0 & 0 & 0\\
      0 & 0 & 1\\
      0 & -1 & 0
    \end{pmatrix}
    + n_y
    \begin{pmatrix}
      0 & 0 & -1\\
      0 & 0 & 0\\
      1 & 0 & 0
    \end{pmatrix}
    + n_z
    \begin{pmatrix}
      0 & 1 & 0\\
      -1 & 0 & 0\\
      0 & 0 & 0
    \end{pmatrix}
  \right]
    \begin{pmatrix}
      p_x \\ p_y \\ p_z
    \end{pmatrix}
\]

En esta última expresión reconocemos los generadores del grupo SO(3)
\begin{align*}
    \xhat{n}\cdot\vvv{L}
   &=
    \begin{pmatrix}
      x & y & z
    \end{pmatrix}
    \left(
      n_x \mmm{G_1}
      + n_y \mmm{G_2}
      + n_z \mmm{G_3}
    \right)
    \begin{pmatrix}
      p_x \\ p_y \\ p_z
    \end{pmatrix}\\
&=
  \begin{pmatrix}
      x & y & z
    \end{pmatrix}
    (\xhat{n} \cdot \mmm{G})
    \begin{pmatrix}
      p_x \\ p_y \\ p_z
    \end{pmatrix}\\
\end{align*}

Finalmente obtenemos una expresión que nos relaciona la componente del momento angular
con una rotación en el espacio --grupo SO(3)--
\begin{equation}\label{eq:cla-Ln}
  L_n
  =
  \xhat{n}\cdot\vvv{L}
  =
    \vvv{r}^\trasp (\xhat{n}\cdot \mmm{G}) \,\vvv{p}
\end{equation}

\subsection{Rotación de vectores en el espacio euclídeo
\mathinhead{\symbb{R}^3}{dlkjfjsR3}}
Ya hemos visto que al girar los ejes de coordenadas alrededor de un cierto eje (rotación
pasiva) se produce un cambio en las coordenadas de un vector
\[
  \xtilde{r}
  = \mmm{R}\!\left(\xhat{n},\theta\right)\, \vvv{r}
  = e^{\theta\/\xhat{n}\cdot\mmm{G}} \vvv{r}
\]
y que matematicamente equivale a dejar los ejes inmóviles y a girar en sentido contrario
(horario) ese objeto con respecto al mismo eje (rotación activa).
Si quisiéramos rotar el vector en sentido positivo (antihorario), deberíamos realizar la
operación inversa
\[
  \xtilde{r}
  = \mmm{R}^{-1}\!\left(\xhat{n},\theta\right)\, \vvv{r}
  = e^{-\theta\/\xhat{n}\cdot\mmm{G}} \vvv{r}
\]

En Física asignamos un significado distinto a los dos tipos de transformaciones:
%Discutiremos el significado que podemos asignar al hecho de que los vectores
%se transformen debido a una rotación alrededor de un eje. Nos interesa un
%punto de vista físico más que matemático.
\begin{itemize}
\item Rotación pasiva

Recurrimos a una transformación pasiva (traslación o rotación) cuando queremos presentar
el punto de vista de un observador situado en otro sistema de referencia, que puede estar
desplazado o rotado con respecto al sistema de referencia original.
En este caso estamos considerando \emph{el mismo fenómeno u objeto físico} desde otro
punto de vista.% y los dos puntos de vista se pueden considerar
%en el mismo o en distintos instantes de tiempo.

\item Rotación activa

Cuando consideramos una rotación activa (en general, cualquier transformación activa, como
traslación o rotación), estamos considerando que
\emph{el objeto físico está cambiando con respecto a un observador}, esto es, estamos
considerando una situación dinámica que cambia al transcurrir el tiempo.
\end{itemize}


\section{Invariancia de una función bajo rotaciones}
\label{sect:cla-invariancia-bajo-rotaciones}
Para el propósito que nos ocupa utilizaremos una función definida en $\symbb{R}^2$, como
\[
  f(x,y) = x^2 + y^2
\]

\subsection{Rotación finita}
La vamos a rotar un ángulo finito $\theta$ los ejes $x$ e $y$.
Las coordenadas se transforman como
\[
  \begin{pmatrix}
    \tilde{x}\\
    \tilde{y}
  \end{pmatrix}
  =
  \begin{pmatrix}
    \cos\theta & \sin\theta\\
    -\sin\theta & \cos\theta
  \end{pmatrix}
  \,
  \begin{pmatrix}
    x\\
    y
  \end{pmatrix}
\]
Operando la expresión matricial anterior, obtenemos las nuevas coordenadas (con tilde)
en función de las originales
\begin{subequations}
  \begin{align}\label{eq:mangclas-xtilde}
    \tilde{x}
    &= x\cos\theta + y\sin\theta\\
    \label{eq:mangclas-ytilde}
    \tilde{y}
    &= -x\sin\theta + y\cos\theta
  \end{align}
\end{subequations}

Para comprobar si esta función es invariante bajo rotaciones, podemos comprobar que en las
nuevas coordenadas tiene la misma forma. Por tanto, suponemos que la función rotada tiene
la misma forma (con las coordenadas rotadas),
$f(\tilde{x},\tilde{y}) = \tilde{x}^2 + \tilde{y}^2$. Sustituimos estas coordenadas
rotadas por las originales y comprobamos que $\tilde{x}^2+\tilde{y}^2 = x^2 + y^2$.
Comprobémoslo.

Sustituimos las ecuaciones anteriores \ref{eq:mangclas-xtilde} y \ref{eq:mangclas-ytilde}
en $\tilde{f}(\tilde{x},\tilde{y}) =  \tilde{x}^2 + \tilde{y}^2$
{\small
\begin{align*}
  \tilde{f}(\tilde{x},\tilde{y})
  &=
    \tilde{x}^2 + \tilde{y}^2
  = (\tilde{x}\cos\theta + \tilde{y}\sin\theta)^2
    + (-\tilde{x}\sin\theta + \tilde{y}\cos\theta)^2\\
  &=
    \tilde{x}^2\cos^2\theta + \tilde{y}^2\sin^2\theta
    + \cancelout{2\tilde{x}\tilde{y}\sin\theta\cos\theta}
    + \tilde{x}^2\sin^2\theta + \tilde{y}^2\cos^2\theta
    - \cancelout{2\tilde{x}\tilde{y}\sin\theta\cos\theta}\\
  &= \tilde{x}^2(\sin^2\theta + \cos^2\theta)
    + \tilde{y}^2(\sin^2\theta + \cos^2\theta)
    = x^2 + y^2
\end{align*}
}

Así, la función rotada es idéntica a la original, ver figura~\ref{fig:cla-x2y2}, donde
se aprecia la simetría con respecto a un giro de los ejes $x$ e $y$
\[
  \tilde{f}(\tilde{x},\tilde{y}) = \tilde{x}^2 + \tilde{y}^2 = x^2 + y^2 = f(x,y)
\]

Cuando la función tiene la misma forma antes que después de rotar los ejes, como en este
caso, se dice que $f$ es invariante bajo la rotación.

\vspace{1ex}
\begin{figure}[ht]
  \def\scl{1}
  % Fondo
  \pgfmathsetmacro{\HORZIZDA}{1.0}
  \pgfmathsetmacro{\HORZDCHA}{0.25}
  \pgfmathsetmacro{\VERT}{0.25}
  % 
  \centering
  \begin{tikzpicture}[%
    scale=\scl,
    baseline,
    background/.style={
      line width=\bgborderwidth,
      draw=\bgbordercolor,
      fill=\bgcolor,
    },    
    ]
    \begin{axis}[%
      width=7cm,
      xlabel=$x$,ylabel=$y$,zlabel=$z$,
      %enlargelimits=false,
      mesh/interior colormap name=greenyellow,
      %mesh/interior colormap name=hot,
      colormap/blackwhite,
      %data cs=polar,
      %domain=0:360,
      %y domain=0:5,
      %zmin=0,
      %zmax=25,
      variable = \u,
      variable y = \r,
      domain=0:359.99999,
      y domain = 0:5,
      enlargelimits=true,
      zmax=32,
      plot box ratio=1 1 4,
      hide axis,
      ]
      \addplot3[surf,samples=60] ({r*cos(u)},{r*sin(u)},{r^2});
      % Eje z
      \draw (0,0,22.8) -- (0,0,32);
      %node[above,name=letraejez] {\small $z$};
      % Eje y
      \draw[-{>}] (0,0,0) -- (7,0,0)
      node[right,name=letraejey] {\small $y$};
      % Eje x
      \draw[-{>}] (0,0,0) -- (0,-9,0)
      node[left,name=letraejex] {\small $x$};
      % Parte de la función que tapa la parte inferior del eje z
      %\addplot3[surf,samples=50,domain=180:190] ({r*cos(u)},{r*sin(u)},{r^2});
      % Inversa del eje y para que el recuadro del gráfico esté centrado
      % en el eje z
      %\path[-{>}] (0,0,0) -- (-7,0,0)
      %node[left,name=letraejeyinv] {\small $\phantom{y}$};

      % Rotación
      \draw[-{Stealth[round,width=4.5pt]}, green!80!black, line width=1.2pt]
      (0,1.25,30) arc[%
      start angle=-35,end angle=310,x radius=1,y radius=1
      ];
      %\draw[-{Stealth[round,width=4.5pt]}, black!40, line width=1.2pt]
      %(0,1.25,25) arc[%
      %start angle=-35,end angle=310,x radius=1,y radius=1
      %];
      % Reponer eje z
      \draw[-{>}] (0,0,32) -- (0,0,35)
      node[above,name=letraejez] {\small $z$};
      %\filldraw[fill=red,draw=black,ultra thin] (0,0,22.7) circle [radius=.2pt];

      % Fondo amarillo
      \coordinate (SW)
      at ($(current bounding box.south west) + (-\HORZIZDA cm,-\VERT cm)$);
      \coordinate (NE)
      at ($(current bounding box.north east) + (\HORZDCHA cm,\VERT cm)$);    
      \begin{scope}[on background layer]
        \draw[background] (SW) rectangle (NE);
      \end{scope}
    \end{axis}
  \end{tikzpicture}%
  \caption{La función $f(x,y) = x^2+y^2$ no varía al rotar alrededor del
    eje $z$.}
  \label{fig:cla-x2y2}
\end{figure}

Si $\tilde{f}(\tilde{x},\tilde{y}) \neq f(x,y)$, entonces la función no hubiera sido
invariante frente a rotaciones. En este caso nos podríamos preguntar cuál sería la forma
de la función rotada. La respuesta a esta pregunta se verá en la sección
\ref{sect:cua-rotacion-funciones} del capítulo siguiente, específicamente en la
subsección \ref{ssect:cua-generador-rotacion-funciones}.

\subsection{Rotación infinitesimal}
Hemos probado la invariancia con una rotación finita. Ahora llevaremos a cabo una
rotación infinitesimal de un ángulo $\varepsilon$, porque a menudo es más cómodo realizar
transformaciones infinitesimales.

Primero escribimos el desarrollo de Taylor de las funciones seno y coseno
\begin{align*}
  \cos\theta &= 1 - \dfrac{\theta^2}{2!} + \dfrac{\theta^4}{4!} + \cdots\\
  \sin\theta &= \theta - \dfrac{\theta^3}{3!} + \dfrac{\theta^5}{5!} + \cdots
\end{align*}

Recordemos que cuando consideramos una cantidad $\varepsilon$ como infinitesimal, estamos
despreciando los términos en $\varepsilon^2$ y superiores. Las funciones anteriores, en
función de $\varepsilon$ quedan
\begin{align*}
  \cos\varepsilon
  &=
    1-\dfrac{\varepsilon^2}{2!} + \dfrac{\varepsilon^4}{4!} + \cdots \approx 1\\
  \sin\varepsilon
  &=
    \varepsilon - \dfrac{\varepsilon^3}{3!}
    + \dfrac{\varepsilon^5}{5!} + \cdots \approx \varepsilon
\end{align*}

La rotación en el plano $xy$ modifica las coordenadas
\begin{align*}
  x \longrightarrow \,&\tilde{x} = \cos\varepsilon\, x + \sin\varepsilon\, y
                        = x + \varepsilon y\\
  y \longrightarrow \,&\tilde{y} = -\sin\varepsilon\,x + \cos\varepsilon\,y
                        = -\varepsilon x + y = y -\varepsilon x
\end{align*}

El cambio en las coordenadas es
\begin{align*}
  \Delta x &= \tilde{x} - x = x + \varepsilon y - x = \varepsilon y\\
  \Delta y &= \tilde{y} - y = -\varepsilon x + y -y = -\varepsilon x
\end{align*}

El desarrollo en serie de la función $\tilde{f}(\tilde{x},\tilde{y})$, de dos variables
después de la rotación infinitesimal es
\[
  \tilde{f}(\tilde{x},\tilde{y})
  =
  f(x,y)
  + \dfrac{\partial f}{\partial x} \Delta x
  + \dfrac{\partial f}{\partial y} \Delta y
  = x^2 + y^2
  + \cancelout{2x\,(y\varepsilon)}
  + \cancelout{2y\,(-x\varepsilon)}
  = x^2 + y^2
\]

Resulta que
\[
  \tilde{f} = f
\]
Por tanto, la función $f(x,y) = x^2 + y^2$ es invariante bajo una rotación infinitesimal.

¿Bastaría con saber que se mantiene invariable la función bajo una transformación
infinitesimal, para colegir que ocurrirá lo mismo para una transformación finita?
La respuesta es sí, con la condición de que la transformación forme un grupo continuo en
el que se obtenga la transformación identidad cuando el parámetro sea cero, $\theta=0$,
osea que la transformación forme un grupo de Lie.

\section{Rotación del vector posición de una partícula}
Empezamos rotando un cierto ángulo $\theta$, el vector de posición $\vvv{r}$ de una
partícula
\[
  \vvv{r} \longrightarrow \mmm{R}\vvv{r}
\]

Si derivamos el vector rotado $\mmm{R}\vvv{r}$ con respecto del tiempo, tendremos que
tener en cuenta que el ángulo es constante y que el vector posición depende del tiempo
\[
  \dfrac{d}{dt}(\mmm{R}\vvv{r})
  = \dfrac{d}{dt}(\mmm{R}\left(\theta)\vvv{r}(t)\right)
  = \mmm{R}(\theta)\dfrac{d\vvv{r}(t)}{dt}
  = \mmm{R}(\theta)\vvv{v}(t)
  = \mmm{R}\vvv{v}
\]
Multiplicamos la expresión anterior por la masa
\[
  m\dfrac{d}{dt}\left(\mmm{R}\vvv{r}\right)
  = m\mmm{R}\vvv{v}
  = \mmm{R}\left(m\vvv{v}\right)
  = \mmm{R}\vvv{p}
\]
donde la masa $m$ es una constante.
La conclusión que sacamos es que si la posición $\vvv{r}$ de una partícula rota un
ángulo $\theta$ determinado, su momento lineal rota de la misma manera, ver
figura~\ref{fig:cla-giro-pos-momento}

\[
  \vvv{p} &\longrightarrow \mmm{R}\vvv{p}
\]

\begin{figure}[ht]
  % Escala
  \def\scl{1}
  % Longitudes ejes
  \pgfmathsetmacro{\XLONG}{3}
  \pgfmathsetmacro{\YLONG}{3}
  % Ángulo rotado
  \pgfmathsetmacro{\ANGROT}{35}
  % Vector R
  \pgfmathsetmacro{\RMOD}{2.5}
  \pgfmathsetmacro{\RANG}{20}
  % Momento lineal p
  \pgfmathsetmacro{\PMOD}{1.2}
  \pgfmathsetmacro{\PANG}{90}
  % Vector R'
  \pgfmathsetmacro{\RPRIMAMOD}{\RMOD}
  \pgfmathsetmacro{\RPRIMAANG}{\RANG + \ANGROT}
  % Momento lineal p'
  \pgfmathsetmacro{\PPRIMAMOD}{\PMOD}
  \pgfmathsetmacro{\PPRIMAANG}{\PANG + \ANGROT}
  % Línea vertical gris
  \pgfmathsetmacro{\LINEAGRISMOD}{\PMOD}
  \pgfmathsetmacro{\LINEAGRISANG}{\PANG}
  % Fondo
  \pgfmathsetmacro{\HORZ}{0.23}
  \pgfmathsetmacro{\VERT}{0.25}
  % 
  \centering
  \begin{tikzpicture}[%
    scale=\scl,
    every node/.style={black,font=\small},
    eje/.style={->},
    posicion/.style={%
      -{Latex}, shorten >=1.2pt, line width=.8pt, draw=green!50!black},
    textopos/.style={%
      below right=-2.2pt and 0pt,green!50!black},
    momento/.style={%
      -{Latex}, line width=0.5pt, draw=green!50!black},
    textomomento/.style={color=green!50!black, right=-1.5pt},
    posicionrotado/.style={posicion, draw=red!75!black},
    textoposrotado/.style={%
      above=2pt, red!75!black},
    momentorotado/.style={%
      momento, draw=red!75!black},
    textomomrotado/.style={%
      below left=-4pt and -1pt, red!75!black},
    rcirculo/.style={fill=green!90!black, draw=black},
    rprimacirculo/.style={red, draw=black},
    background/.style={
      line width=\bgborderwidth,
      draw=\bgbordercolor,
      fill=\bgcolor,
    },
    ]
    % COORDENADAS
    % ORIGEN Y RELACIONADOS
    \coordinate (O) at (0,0);
    \coordinate (under_origin) at (0,-3mm);
    \coordinate (left_origin) at (-3mm,0);
    % EJES x E y
    \coordinate (xini) at (O);
    \coordinate (xfin) at (\XLONG cm,0);
    \coordinate (yini) at (0, 0 cm);
    \coordinate (yfin) at (0,\YLONG cm);
    % POSICIÓN VECTORES r Y p
    \coordinate (r) at (\RANG:\RMOD cm);
    \path (r) -- ++(\PANG:\PMOD cm) coordinate (p);
    % POSICIÓN VECTORES r' y p'
    \coordinate (r') at  (\RPRIMAANG:\RPRIMAMOD cm);
    \path (r') -- ++(\PPRIMAANG:\PPRIMAMOD cm) coordinate (p');
    % POSICIONES PUNTO MEDIO DE VECTORES r Y p PARA NODOS NOMBRE
    \path (r) -- coordinate (ORmidway) (O);
    \path (r) -- coordinate (RPmidway) (p);
    % POSICIONES PUNTO MEDIO DE VECTORES r' y p' PARA NODOS NOMBRE
    \path (r') -- coordinate (OR'midway) (O);
    \path (r') -- coordinate (R'P'midway) (p');
    % POSICIÓN FINAL LÍNEA VERTICAL GRIS
    \draw[lightgray] (r') -- ++(\LINEAGRISANG:\LINEAGRISMOD) coordinate (lineagris);
    % --- DIBUJO
    % ÁNGULO THETA EN VECTORES POSICIÓN
    \path (r) -- (O) -- (r') pic
    [-{Latex[length=3.6pt]},draw=black!70!,fill=green!20,"\footnotesize $\theta$",angle
    radius=6mm, angle eccentricity=1.4] {angle = r--O--r'};
    % ÁNGULO THETA EN VECTORES MOMENTO
    \path (lineagris) -- (r') -- (p') pic
    [-{Latex[length=3.6pt]},draw=black!70!,fill=green!20,"\footnotesize $\theta$",angle
    radius=5mm, angle eccentricity=1.5] {angle = lineagris--r'--p'};
    % EJES
    \draw[eje] (xini) -- (xfin);
    \node[right, name=letraejex] at (xfin) {$x$};
    \draw[eje] (yini) -- (yfin);
    \node[above, name=letraejey] at (yfin) {$y$};
    % VECTOR r
    \draw[posicion] (O) -- (r);
    \node[textopos] at (ORmidway) {$\vvv{r}$};
    % VECTOR p
    \draw[momento] (r) -- (p);
    \node[textomomento] at (RPmidway) {$\vvv{p}$};
    % PUNTO INICIAL PARTÍCULA
    \fill[rcirculo] (r) circle [radius=1.4pt];
    % Línea vertical en la posición final
    \draw[black!15] (r') -- (lineagris);
    % VECTOR r'
    \draw[posicionrotado] (O) -- (r');
    \node[textoposrotado] at (OR'midway) {$\vvv{r}'$};
    % VECTOR p'
    \draw[momentorotado] (r') -- (p');
    \node[textomomrotado] at (R'P'midway) {$\vvv{p}'$};
    % PUNTO FINAL PARTÍCULA
    \fill[rprimacirculo] (r') circle [radius=1.4pt];
%    % Origen
    \filldraw (O) circle [radius=.3pt];
    % Fondo amarillo
    \coordinate (SW) at ($(current bounding box.south west) + (-\HORZ cm,-\VERT cm)$);
    \coordinate (NE) at ($(current bounding box.north east) + (\HORZ cm,\VERT cm)$);    
    \begin{scope}[on background layer]
      \draw[background] (SW) rectangle (NE);
    \end{scope}
  \end{tikzpicture}
  \caption{El vector momento lineal de una partícula gira el mismo ángulo que el
    vector posición.}
  \label{fig:cla-giro-pos-momento}
\end{figure}


\section{El momento angular y el teorema de Noether}
\label{sect:cla-momento-angular-y-Noether}
El teorema de Noether relaciona la invariancia del lagrangiano, bajo ciertas
transformaciones con la conservación de algunas magnitudes físicas.
Por ejemplo, cuando el lagrangiano es invariante bajo una rotación, entonces se conserva
el momento angular.
En este contexto también se puede afirmar que hay una relación entre una rotación y el
momento angular.

\subsection{El lagrangiano de un sistema}
El lagrangiano de un sistema es una función de las coordenadas generalizadas$q_i$ y las
velocidades respectivas $\dot{q}_i$.
Se define como la energía cinética menos la energía potencial del sistema
\[
  \symcal{L}(\vvv{q},\vvv{\dot{q}})
  = T - U
\]

A partir de él se pueden deducir las ecuaciones de Euler-Lagrange, que en mecánica son el
equivalente de las ecuaciones del movimiento
\begin{equation}\label{eq:cla-EulerLagrange}
  \dfrac{\partial \symcal{L}}{\partial q_i}
  =
  \dfrac{d}{dt}\,\left(\dfrac{\partial\symcal{L}}{\partial \dot{q_i}}\right)
  \hspace*{2em}
  (i = 1, 2, \cdots, N)
\end{equation}

Se suele llamar \emph{momento generalizado} asociado a la coordenada $q_i$ a la cantidad
\[
  p_{q{_i}} = \dfrac{\partial\symcal{L}}{\partial\dot{q}_i}
\]

Cuando se utilizan coordenadas cartesianas, las ecuaciones de Euler-Lagrange para una
partícula tienen la forma
\[
  \dfrac{\partial \symcal{L}}{\partial x_i}
  =
  \dfrac{d}{dt}\,\left(\dfrac{\partial\symcal{L}}{\partial \dot{x_i}}\right)
  \hspace*{2em}
  (i = 1, 2, 3)
\]
donde $x_1=x$, $x_2=y$, $x_3=z$.
En este caso, tendríamos el momento lineal cartesiano de toda la vida
\[
  p_i = \dfrac{\partial\symcal{L}}{\partial\dot{x}_i}
  \hspace{2em}
  (i = 1, 2, 3)
\]

\subsection{Invariancia del lagrangiano bajo rotaciones}
Nos preguntamos en qué tipo de situaciones físicas es invariante el lagrangiano bajo una
rotación.
Como las rotaciones forman un grupo de Lie y, según la sección anterior
\ref{sect:cla-invariancia-bajo-rotaciones}, si el lagrangiano fuera invariante bajo una
rotación infinitesimal, querría decir que también lo sería bajo cualquier rotación
finita, vamos a aplicar al lagrangiano una rotación infinitesimal.

%La matriz de rotación infinitesimal pasiva del lagrangiano (antihoraria) es
%\[
%  \mmm{R}^{-1}(\xhat{n},\varepsilon)
%  =
%  e^{-\varepsilon\xhat{n}\cdot\mmm{G}}
%  \mmm
%  \approx{I} - \varepsilon\xhat{n}\cdot\mmm{G}
%\]

Consideraremos el caso más simple de un sistema formado por una partícula\footnotemark{}.
\footnotetext{El resultado se podría generalizar fácilmente para un sistema de $N$
  partículas.}
El lagrangiano $\symcal{L}$ de una partícula es una función de su posición y su velocidad.
Cuando lo rotamos, el nuevo lagrangiano $\tilde{\symcal{L}}$ dependerá de la posición y
velocidad rotadas\footnotemark{}
\footnotetext{Esta notación con el lagrangiano dependiendo de $\vvv{r}$ y $\vvv{\dot{r}}$
  está simplificada. En realidad debe haber una ecuación por cada coordenada linealmente
  independiente.}
\[
  \symcal{L}(\vvv{r}, \vvv{\dot{r}})
  \longrightarrow
  \tilde{\symcal{L}}(\xtilde{r}, \xdottilde{r})
\]
y queremos averiguar bajo qué condiciones se cumple la invariancia
$\tilde{\symcal{L}} = \symcal{L}$.

Para nuestro propósito en este apartado, nos da igual aplicar una rotación pasiva o
activa, pues llegaríamos a la misma conclusión.
Rotaremos la función lagrangiana $\symcal{L}$ utilizando la rotación pasiva, $\mmm{R}$ de
las coordenadas y, como se vio en la introducción de la sección
\ref{sect:cla-momento-angular-y-Noether}, también cambiarań de forma similar las
velocidades
\begin{align*}
  \xtilde{r}
  &=
  \mmm{R}(\varepsilon)\vvv{r}
  =
  e^{\varepsilon \xhat{n}\cdot\mmm{G}} \vvv{r}
  \approx
  (\mmm{I} + \varepsilon\xhat{n}\cdot\mmm{G})\vvv{r}
  =
    \vvv{r} + \varepsilon(\xhat{n}\cdot\mmm{G})\vvv{r}\\
  \xdottilde{r}
  &=
  \mmm{R}(\varepsilon)\vvv{\dot{r}}
  =
  e^{\varepsilon \xhat{n}\cdot\mmm{G}} \vvv{\dot{r}}
  \approx
  (\mmm{I} + \varepsilon\xhat{n}\cdot\mmm{G})\vvv{\dot{r}}
  =
  \vvv{\dot{r}} + \varepsilon(\xhat{n}\cdot\mmm{G})\vvv{\dot{r}}
\end{align*}

Se han producido cambios en la posición y en la velocidad
\begin{subequations}
\begin{align}\label{eq:cla-delta-posicion}
  \Delta\vvv{r}
  &=
  \xtilde{r} - \vvv{r}
  \approx
  \cancelout{\vvv{r}}
  + \varepsilon(\xhat{n}\cdot\mmm{G})\vvv{r}
  - \cancelout{\vvv{r}}
  =
    \varepsilon(\xhat{n}\cdot\mmm{G})\vvv{r}\\
  \label{eq:cla-delta-velocidad}
  \Delta\vvv{\dot{r}}
  &=
  \xdottilde{r} - \vvv{\dot{r}}
  \approx
  \cancelout{\vvv{\dot{r}}}
  + \varepsilon(\xhat{n}\cdot\mmm{G})\vvv{\dot{r}}
  - \cancelout{\vvv{\dot{r}}}
  =
  \varepsilon(\xhat{n}\cdot\mmm{G})\vvv{\dot{r}}
\end{align}
\end{subequations}
donde hemos tenido en cuenta que $\xhat{n}$ y $\mmm{G}$ son independientes del tiempo.

Desarrollamos en serie el lagrangiano transformado por la rotación infinitesimal,
despreciando los términos de orden $\varepsilon^2$ o superior y sustituyendo los
incrementos~\eqref{eq:cla-delta-posicion} y~\eqref{eq:cla-delta-velocidad}
\begin{align*}
  \tilde{\symcal{L}}(\tilde{\vvv{r}},\dot{\tilde{\vvv{r}}})
  &\approx
  \symcal{L}
  (\vvv{r}+\varepsilon\xhat{n}\cdot\mmm{G}\vvv{r},
    \vvv{\dot{r}}+\varepsilon\xhat{n}\cdot\mmm{G}\vvv{\dot{r}})\\
  &\approx \symcal{L}(\vvv{r},\vvv{\dot{r}})
    + \left(\dfrac{\partial\symcal{L}}{\partial \vvv{r}}\right)^\trasp
    \Delta\vvv{r}
    + \left(\dfrac{\partial\symcal{L}}{\partial \vvv{\dot{r}}}\right)^\trasp
    \Delta\vvv{\dot{r}}\\
  &= \symcal{L}(\vvv{r},\vvv{\dot{r}})
    + \left(\dfrac{\partial\symcal{L}}{\partial \vvv{r}}\right)^\trasp
    \varepsilon(\xhat{n}\cdot\mmm{G})\vvv{r}
    + \left(\dfrac{\partial\symcal{L}}{\partial \vvv{\dot{r}}}\right)^\trasp
    \varepsilon(\xhat{n}\cdot\mmm{G})\vvv{\dot{r}}
\end{align*}
donde se ha tenido en cuenta que $\Delta\vvv{r}$ y $\Delta\vvv{\dot{r}}$ son vectores
y, por tanto, las derivada parciales del lagrangiano con respecto de las coordenadas de
$\vvv{r}$ y $\vvv{\dot{r}}$ deben ser vectores traspuestos para que el resultado final
sea un escalar.

La traspuesta de la parcial del lagrangiano con respecto a la posición es
\[
  \left(\dfrac{\partial\symcal{L}}{\partial\vvv{r}}\right)^\trasp
  =
  \begin{pmatrix}
    \dfrac{\partial\symcal{L}}{\partial x}
    & \dfrac{\partial\symcal{L}}{\partial y}
    & \dfrac{\partial\symcal{L}}{\partial z}
  \end{pmatrix}
\]
Omitimos una fórmula similar para la traspuesta de la parcial con respecto a la velocidad.

Abreviamos ligeramente la notación
\[
  \tilde{\symcal{L}}
  =
    \symcal{L}
    + \left(\dfrac{\partial\symcal{L}}{\partial\vvv{r}}\right)^\trasp
    \varepsilon\,(\xhat{n}\cdot\mmm{G})\kern1pt\vvv{r}
    + \vvv{p}^\trasp
    \varepsilon\,(\xhat{n}\cdot\mmm{G})\kern1pt\vvv{\dot{r}}\\
\]
    
A continuación sacamos factor común $\varepsilon$ y sustituimos
$\partial\symcal{L}/\partial\vvv{r}$
por la expresión de las ecuaciones de Euler-Lagrange~(\ref{eq:cla-EulerLagrange})
\begin{align*}
  \tilde{\symcal{L}}
  &=
    \symcal{L}
    + \varepsilon\,\left[
    \dfrac{d}{dt}\left(
    \frac{\symcal{L}}{\partial\vvv{\dot{r}}}
    \right)^\trasp
    (\xhat{n}\cdot\mmm{G})\kern1pt\vvv{r}
    + \vvv{p}^\trasp
    (\xhat{n}\cdot\mmm{G})\kern1pt\vvv{\dot{r}}
    \right]\\
  &=
    \symcal{L}
    + \varepsilon\,\left[
    \left(\dfrac{d\vvv{p}}{dt}\right)^\trasp
    (\xhat{n}\cdot\mmm{G})\kern1pt\vvv{r}
    + \vvv{p}^\trasp (\xhat{n}\cdot\mmm{G})\kern1pt\vvv{\dot{r}}
    \right]\\
  &=
  \symcal{L}
    + \varepsilon\,\left[\vvv{\dot{p}}^\trasp
    (\xhat{n}\cdot\mmm{G})\kern1pt\vvv{r}
    + \vvv{p}^\trasp
    (\xhat{n}\cdot\mmm{G})\kern1pt\vvv{\dot{r}}\right]
\end{align*}

Reconocemos la derivada temporal de un producto en la ecuación anterior
\begin{equation}\label{eq:cla-lagrangiana_rotada_1}
  \tilde{\symcal{L}}
  =
    \symcal{L}
    + \varepsilon\,\dfrac{d}{dt}\left[\vvv{p}^\trasp
      (\xhat{n}\cdot\mmm{G})\,\vvv{r} \right]
\end{equation}
Hemos tenido en cuenta que $\xhat{n}\cdot\mmm{G}$ no depende del tiempo
\[
  \xhat{n}\cdot\mmm{G}
  =
  \begin{pmatrix}
    0 & n_3 & -n_2\\
    -n_3 & 0 & n_1\\
    n_2 & -n_1 & 0
  \end{pmatrix}
\]
y que es una matriz antisimética
\[
  (\xhat{n}\cdot\mmm{G})^\trasp = -\xhat{n}\cdot\mmm{G}
\]

Fijémomos en el término
$\vvv{p}^\trasp (\xhat{n}\cdot\mmm{G}) \vvv{r}$ de la
ecuación~(\ref{eq:cla-lagrangiana_rotada_1}). Como es un número, su traspuesta es el
mismo número
\[
  \vvv{p}^\trasp (\xhat{n}\cdot\mmm{G}) \vvv{r}
  =
  [\vvv{p}^\trasp (\xhat{n}\cdot\mmm{G}) \vvv{r}]^\trasp
\]
Desarrollamos el segundo miembro de la igualdad
\[
  \vvv{p}^\trasp (\xhat{n}\cdot\mmm{G}) \vvv{r}
  =
  [\vvv{r}^\trasp (\xhat{n}\cdot\mmm{G})^\trasp \vvv{p}]
\]
Como $\xhat{n}\cdot\mmm{G}$ es antisimétrica
\[
  \vvv{p}^\trasp (\xhat{n}\cdot\mmm{G}) \vvv{r}
  =
  -\vvv{r}^\trasp (\xhat{n}\cdot\mmm{G}) \vvv{p}
\]

Susituimos la expresión anterior en~(\ref{eq:cla-lagrangiana_rotada_1})
\[
  \tilde{\symcal{L}}
  = \symcal{L} -
  \varepsilon\,\dfrac{d}{dt}\left[\vvv{r}^\trasp
    (\xhat{n}\cdot\mmm{G})\,\vvv{p} \right]
\]

Teniendo en cuenta~\eqref{eq:cla-Ln}, vemos que el interior del corchete es la componente
del momento angular del sistema con respecto de la dirección $\xhat{n}$ de rotación del
lagrangiano.
Llegamos a la expresión del lagrangiano cuando se rota infinitesimalmente
\begin{equation}
  \tilde{\symcal{L}}
  = \symcal{L} -
  \varepsilon\,\dfrac{d}{dt}\left(\xhat{n}\cdot\vvv{L}\right)
\end{equation}

Si resultase que la componente del momento angular con respecto de cualquier eje no
dependiera del tiempo
\[
  \dfrac{d}{dt}(\xhat{n}\cdot\vvv{L}) = 0
\]
entonces el lagrangiano sería un invariante con respecto a las rotaciones continuas
SO(3), y $\xhat{n}\cdot\vvv{L}$ sería constante, lo que quiere decir que la componente
del momento angular con respecto a cualquier eje es constante, por lo que el momento angular $\vvv{L}$ no cambia con el tiempo.

Así, según el teorema de Noether, si el lagrangiano de un sistema es invariante bajo
rotaciones SO(3), se conserva el momento angular.
En general, si se presentara otro tipo de simetría (traslaciones, etc.) se conservaría
otra magnitud física.

El momento angular está pues relacionado con las rotaciones.
Para terminar, adelantamos que como el spin es un momento angular, también estará
relacionado con rotaciones, pero ¿qué tipo de rotaciones?
La respuesta la descubriremos más adelante.



 
%%% Local Variables:
%%% mode: latex
%%% TeX-engine: luatex
%%% TeX-master: "../gruposlie.tex"
%%% End:
