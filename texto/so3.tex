% so3.tex
%
% Copyright (C) 2022--2025 José A. Navarro Ramón <janr.devel@gmail.com>
% Licencia Creative Commons Recognition Share-alike.
% (CC-BY-SA)

\chapter{El grupo de rotaciones SO(3)}
En este capítulo estudiaremos el grupo de rotaciones en el espacio tridimensional.
Partimos de la expresión exponencial de la matriz de  rotación de los ejes $x$ e $y$ en
dos dimensiones en sentido antihorario
 \[
   \mmm{R} (\theta)
   = e^{\theta\mmm{G}}
   = \exp\left\{\theta\,\begin{pmatrix} 0 & 1 \\ -1 & 0 \end{pmatrix}\right\}
 \]
 siendo $\mmm{G}$ el único generador de rotación en el grupo SO(2).
 
\subsection{Número de generadores del grupo SO(3)}\label{subsec:num_gen_so3}
Para saber cuántos generadores encontraremos en el grupo SO(3), nos debemos fijar en la
expresión exponencial de una rotación en el espacio tridimensional
\[
  \mmm{R} = e^{\mmm{A}} = \exp\left\{
    \begin{pmatrix}
      A_{11} & A_{12} & A_{13}\\ A_{21} & A_{22} & A_{23}\\ A_{31} &
      A_{32} & A_{33}
    \end{pmatrix}\right\}
\]

Sabemos que la matriz $\mmm{A}$ debe ser antisimétrica:
\begin{itemize}
\item Los elementos diagonales deben ser cero, $A_{ii} = 0$
\[
    \mmm{A} =
    \begin{pmatrix}
      0 & A_{12} & A_{13}\\
      A_{21} & 0 & A_{23}\\
      A_{31} & A_{32} & 0
    \end{pmatrix}
\]

\item Los elementos que están fuera de la diagonal deben cumplir
  $A_{ij} = -A_{ji}$
  \[
    \mmm{A} =
    \begin{pmatrix}
      0 & A_{12} & A_{13}\\
      -A_{12} & 0 & A_{23}\\
      -A_{13} & -A_{23} & 0
    \end{pmatrix}
\]
\end{itemize}

Vemos que sólo tenemos tres grados de libertad: $A_{12}$, $A_{13}$ y
$A_{23}$, los demás elementos quedan completamente especificados.
Este grupo tendrá tres generadores.

Dejamos, como ejercicio, comprobar que una rotación en el plano --SO(2)-- tiene un único
generador, mientras que una rotación en el espacio de cuatro dimensiones --SO(4)--, tiene
seis.

\subsection{Generador para la rotación alrededor del eje
  \mathinhead{z}{z}, \mathinhead{\mmm{G}_z}{Gz}}
Esta rotación pasiva hace girar los ejes $x$ e $y$ de forma similar a como hemos visto
en la SO(2), así que la podemos relacionar con una rotación alrededor del eje $z$ en
tres dimensiones
\[
  \mmm{R}_z (\theta) = \exp\left\{\theta\,\begin{pmatrix} 0 & 1 \\ -1
      & 0 \end{pmatrix}\right\}
\]

Pero el generador debe ser una matriz $3\times 3$, por lo que se debe completar.
Como la matriz debe ser antisimétrica, los elementos diagonales deben ser cero
\[
  \mmm{R}_z (\theta) = \exp\left\{ \theta\,
    \begin{pmatrix}
      0 & 1 \\ -1
      & 0 \\ & & 0
    \end{pmatrix} \right\}
\]

Completamos con ceros el resto de elementos
\[
  \mmm{R}_z (\theta) = \exp\left\{ \theta\,
    \begin{pmatrix}
      0 & 1 & 0 \\ -1 & 0 & 0\\ 0 & 0 & 0
    \end{pmatrix}\right\}
\]
y el generador para la rotación pasiva alrededor del eje $z$ sería
\begin{equation}\label{eq:so3-generador_z}
  \mmm{G}_z =
  \begin{pmatrix} 0 & 1 & 0
    \\ -1 & 0 & 0\\ 0 & 0 & 0
  \end{pmatrix}
\end{equation}

Debido a la forma un poco arbitraria con que hemos obtenido este generador, tenemos que
comprobar que describe efectivamente una rotación alrededor del eje $z$.
Lo más sencillo es comprobarlo mediante una rotación infinitesimal, porque la matriz de
rotación es más sencilla, dado que los términos que contienen $\varepsilon^n$ con $n>=2$
son despreciables
\begin{equation}
  \label{eq:so3-rotacion_infinitesimal_z}
  \mmm{R}_z (\varepsilon) \approx
  \mmm{I} + \varepsilon\, \mmm{G}_z
  = \begin{pmatrix}
    1 & 0 & 0\\
    0 & 1 & 0\\
    0 & 0 & 1
  \end{pmatrix}
  + \varepsilon
  \begin{pmatrix}
    0 & 1 & 0 \\ -1 & 0 & 0\\ 0 & 0 & 0
  \end{pmatrix}
\end{equation}

Aplicamos esta pequeña rotación pasiva alrededor del eje $z$ a un
vector cualquiera del espacio de tres dimensiones
{
  \small
  \begin{align*}
   \begin{pmatrix}
     x' \\ y' \\ z'
   \end{pmatrix}
   &= \mmm{R}_z (\varepsilon)
   \begin{pmatrix}
     x \\ y \\ z
   \end{pmatrix}
   \approx
   \left[
   \begin{pmatrix}
     1 & 0 & 0\\
     0 & 1 & 0\\
     0 & 0 & 1
   \end{pmatrix}
   + \varepsilon
   \begin{pmatrix}
     0 & 1 & 0 \\ -1 & 0 & 0\\ 0 & 0 & 0
   \end{pmatrix}
  \right]\,
  \begin{pmatrix}
    x \\ y \\ z
  \end{pmatrix}\\
   &=
  \begin{pmatrix}
    x \\ y \\ z
  \end{pmatrix}
  + \varepsilon
  \begin{pmatrix}
    y \\ -x \\ 0
  \end{pmatrix}
  =
     \begin{pmatrix}
    x+\varepsilon y \\ y-\varepsilon x \\ z
  \end{pmatrix}
  \end{align*}
}

Calculamos las componentes después del giro de los ejes alrededor del eje $z$
\begin{subequations}\label{eq:so3-giro_z_incxyz}
\begin{align}
  \label{eq:so3-giro_z_incx}
  x' &= x + \varepsilon y\\
  \label{eq:so3-giro_z_incy}
  y' &= y - \varepsilon x\\
  \label{eq:so3-giro_z_incz}
  z' &= z
\end{align}
\end{subequations}

¿Cómo podemos saber que estas transformaciones de las coordenadas son compatibles con un
giro pasivo infinitesimal alrededor del eje $z$?
Aunque no es una demostración definitiva, podemos comprobarlo fijándonos en que un giro
antihorario de los ejes $x$ e $y$ alrededor del ejx $z$, es equivalente a un giro horario
del espacio (en este caso el punto $P$), dejando los ejes inalterados, como se puede
observar a la izquierda de la figura \ref{fig:so3-giros_zxy}.

Tal y como se aprecia en las ecuaciones \eqref{eq:so3-giro_z_incx},
\eqref{eq:so3-giro_z_incy} y \eqref{eq:so3-giro_z_incz},
a la izquierda de la figura \ref{fig:so3-giros_zxy},
la coordenada $x'$ del punto girado debe aumentar ligeramente, la coordenada $y'$
disminuir, también ligeramente, mientras que la coordenada $z'$ se mantiene constante.
\begin{figure}[ht]
  % Escala
  \def\scl{.76}
  \centering
  \begin{minipage}{0.31\linewidth}
    % Eje x
    \pgfmathsetmacro{\XMLONG}{0}
    \pgfmathsetmacro{\XPLONG}{3}
    % Eje y
    \pgfmathsetmacro{\YMLONG}{0}
    \pgfmathsetmacro{\YPLONG}{3}
    % Ángulo rotado
    \pgfmathsetmacro{\ANGROT}{20}
    % Vector P
    \pgfmathsetmacro{\PMOD}{2.5}
    \pgfmathsetmacro{\PANG}{60}
    % Vector P'
    \pgfmathsetmacro{\PPRIMAMOD}{\PMOD}
    \pgfmathsetmacro{\PPRIMAANG}{\PANG - \ANGROT}
    % Fondo
    \pgfmathsetmacro{\HORZ}{0.20}
    \pgfmathsetmacro{\VERT}{0.20}
    \pgfmathsetmacro{\MOD}{sqrt(\HORZ^2 + \VERT^2)}
    \pgfmathsetmacro{\ANGSD}{atan(\VERT / \HORZ)}
    \pgfmathsetmacro{\ANGII}{\ANGSD + 180.0}    
    % 
    \centering
    \begin{tikzpicture}[%
      scale=\scl,
      every node/.style={black,font=\small},
      eje/.style={->},
      vector/.style={-{Latex}, shorten >=1.2pt, line width=.8pt},
      vectorrotado/.style={vector, draw=green!50!black},
      pcirculo/.style={fill=red, draw=black},
      pprimacirculo/.style={green!90!black, draw=black},
      background/.style={
        line width=\bgborderwidth,
        draw=\bgbordercolor,
        fill=\bgcolor,
      },      
      ]
      % Coordenadas
      \coordinate (O) at (0,0);
      \coordinate (under_origin) at (0,-3mm);
      \coordinate (left_origin) at (-3mm,0);
      \coordinate (xini) at (-\XMLONG cm,0);
      \coordinate (xfin) at (\XPLONG cm,0);
      \coordinate (yini) at (0,-\YMLONG cm);
      \coordinate (yfin) at (0,\YPLONG cm);
      \coordinate (P) at (\PANG:\PMOD cm);
      \coordinate (P') at  (\PPRIMAANG:\PPRIMAMOD);
      \path (O) -- coordinate (OPmidway) (P);
      \path (O) -- coordinate (OP'midway) (P');
      \path (O) -- coordinate[pos=1.2] (parrow) (P);
      \path (O) -- coordinate[pos=1.2] (p'arrow) (P');
      % Ángulo \varepsilon
      \path (P') -- (O) -- (P) pic
      [draw=black!50!,fill=green!20,"\footnotesize $\varepsilon$",angle
      radius=6mm, angle eccentricity=1.5] {angle = P'--O--P};
      % Ejes
      \draw[eje] (xini) -- (xfin);
      \node[right] at (xfin) {$x$};
      \draw[eje] (yini) -- (yfin);
      \node[above] at (yfin) {$y$};
      % Punto P
      \draw[vector] (O) -- (P);
      \node[above=5pt] at (OPmidway) {$\vvv{r}$};
      \fill[pcirculo] (P) circle [radius=1.4pt];
      \node[above] at (P) {$P$};
      % Punto P'
      \draw[vectorrotado] (O) -- (P');
      \node[right=0pt] at (OP'midway) {$\vvv{r}'$};
      \fill[pprimacirculo] (P') circle [radius=1.4pt];
      \node[right] at (P') {$P'$};
      % Sentido de giro del vector
      \draw [-{Latex},green!40!black,shorten <= 3pt]
      (parrow) to[bend left=30] (p'arrow);
      % Incremento x
      \draw[fill=red,draw=black] (P |- O) circle[radius=1.5pt];
      \draw[fill=green!80!black,draw=black] (P' |- O)
      circle[radius=1.5pt];
      \draw[-{Latex}] (P |- under_origin) --
      node[below] {\scriptsize $\Delta x > 0$} (P' |- under_origin);
      % Incremento y
      \draw[fill=red,draw=black] (P -| O) circle[radius=1.5pt];
      \draw[fill=green!80!black,draw=black] (P' -| O)
      circle[radius=1.5pt];
      \draw[-{Latex}] (P -| left_origin) --
      node[above=-1pt,sloped,rotate=180]
      {\scriptsize $\Delta y < 0$} (P' -| left_origin);
      % Eje z
      \node[below left=1.8ex and -0.5em] at (O) {\scriptsize $\Delta z = 0$};
      \fill[fill=white,draw=black] (O) circle[radius=2.5pt]
      node[below left] {$z$}; \fill[fill=black] (O) circle[radius=.7pt];
      % Fondo amarillo
      \path (current bounding box.south west) ++(\ANGII:\MOD) coordinate (SW);
      \path (current bounding box.north east) ++(\ANGSD:\MOD) coordinate (NE);
      % \filldraw[draw=black, fill=red] (NE) circle[radius=1pt];
      \begin{scope}[on background layer]
        \draw[background] (SW) rectangle (NE);
      \end{scope}
    \end{tikzpicture}
  \end{minipage}
  \hspace{.4em}
  \begin{minipage}{0.31\linewidth}
    \centering
    % Eje y
    \pgfmathsetmacro{\YMLONG}{0}
    \pgfmathsetmacro{\YPLONG}{3}
    % Eje z
    \pgfmathsetmacro{\ZMLONG}{0}
    \pgfmathsetmacro{\ZPLONG}{3}
    % Ángulo rotado
    \pgfmathsetmacro{\ANGROT}{20}
    % Vector P
    \pgfmathsetmacro{\PMOD}{2.5}
    \pgfmathsetmacro{\PANG}{60}
    % Vector P'
    \pgfmathsetmacro{\PPRIMAMOD}{\PMOD}
    \pgfmathsetmacro{\PPRIMAANG}{\PANG - \ANGROT}
    % Fondo
    \pgfmathsetmacro{\HORZ}{0.20}
    \pgfmathsetmacro{\VERT}{0.20}
    \pgfmathsetmacro{\MOD}{sqrt(\HORZ^2 + \VERT^2)}
    \pgfmathsetmacro{\ANGSD}{atan(\VERT / \HORZ)}
    \pgfmathsetmacro{\ANGII}{\ANGSD + 180.0}        
    % 
    \centering
    \begin{tikzpicture}[%
      scale=\scl,
      every node/.style={black,font=\small},
      eje/.style={->},
      vector/.style={-{Latex}, shorten >=1.2pt, line width=.8pt},
      vectorrotado/.style={vector, draw=green!50!black},
      pcirculo/.style={fill=red, draw=black},
      pprimacirculo/.style={green!90!black, draw=black},
      background/.style={
        line width=\bgborderwidth,
        draw=\bgbordercolor,
        fill=\bgcolor,
      },      
      ]
      % Coordenadas
      \coordinate (O) at (0,0);
      \coordinate (under_origin) at (0,-3mm);
      \coordinate (left_origin) at (-3mm,0);
      \coordinate (yini) at (-\YMLONG cm,0);
      \coordinate (yfin) at (\YPLONG cm,0);
      \coordinate (zini) at (0,-\ZMLONG cm);
      \coordinate (zfin) at (0,\ZPLONG cm);
      \coordinate (P) at (\PANG:\PMOD cm);
      \coordinate (P') at  (\PPRIMAANG:\PPRIMAMOD);
      \path (O) -- coordinate (OPmidway) (P);
      \path (O) -- coordinate (OP'midway) (P');
      \path (O) -- coordinate[pos=1.2] (parrow) (P);
      \path (O) -- coordinate[pos=1.2] (p'arrow) (P');
      % Ángulo \varepsilon
      \path (P') -- (O) -- (P) pic
      [draw=black!50!,fill=green!20,"\footnotesize $\varepsilon$",angle
      radius=6mm, angle eccentricity=1.5] {angle = P'--O--P};
      % Ejes
      \draw[eje] (yini) -- (yfin);
      \node[right] at (yfin) {$y$};
      \draw[eje] (zini) -- (zfin);
      \node[above] at (zfin) {$z$};
      % Punto P
      \draw[vector] (O) -- (P);
      \node[above=5pt] at (OPmidway) {$\vvv{r}$};
      \fill[pcirculo] (P) circle [radius=1.4pt];
      \node[above] at (P) {$P$};
      % Punto P'
      \draw[vectorrotado] (O) -- (P');
      \node[right=0pt] at (OP'midway) {$\vvv{r}'$};
      \fill[pprimacirculo] (P') circle [radius=1.4pt];
      \node[right] at (P') {$P'$};
      % Sentido de giro del vector
      \draw [-{Latex},green!40!black,shorten <= 3pt]
      (parrow) to[bend left=30] (p'arrow);
      % Incremento y
      \draw[fill=red,draw=black] (P |- O) circle[radius=1.5pt];
      \draw[fill=green!80!black,draw=black] (P' |- O)
      circle[radius=1.5pt];
      \draw[-{Latex}] (P |- under_origin) --
      node[below] {\scriptsize $\Delta y > 0$} (P' |- under_origin);
      % Incremento z
      \draw[fill=red,draw=black] (P -| O) circle[radius=1.5pt];
      \draw[fill=green!80!black,draw=black] (P' -| O)
      circle[radius=1.5pt];
      \draw[-{Latex}] (P -| left_origin) --
      node[above=-1pt,sloped,rotate=180]
      {\scriptsize $\Delta z < 0$} (P' -| left_origin);
      % Eje x
      \node[below left=1.8ex and -0.5em] at (O) {\scriptsize $\Delta x = 0$};
      \fill[fill=white,draw=black] (O) circle[radius=2.5pt]
      node[below left] {$x$}; \fill[fill=black] (O) circle[radius=.7pt];
      % Fondo amarillo
      \path (current bounding box.south west) ++(\ANGII:\MOD) coordinate (SW);
      \path (current bounding box.north east) ++(\ANGSD:\MOD) coordinate (NE);
      % \filldraw[draw=black, fill=red] (NE) circle[radius=1pt];
      \begin{scope}[on background layer]
        \draw[background] (SW) rectangle (NE);
      \end{scope}
    \end{tikzpicture}
  \end{minipage}
  \hspace{.4em}
  \begin{minipage}{0.31\linewidth}
    \centering
    % Eje z
    \pgfmathsetmacro{\ZMLONG}{0}
    \pgfmathsetmacro{\ZPLONG}{3}
    % Eje x
    \pgfmathsetmacro{\XMLONG}{0}
    \pgfmathsetmacro{\XPLONG}{3}
    % Ángulo rotado
    \pgfmathsetmacro{\ANGROT}{20}
    % Vector P
    \pgfmathsetmacro{\PMOD}{2.5}
    \pgfmathsetmacro{\PANG}{60}
    % Vector P'
    \pgfmathsetmacro{\PPRIMAMOD}{\PMOD}
    \pgfmathsetmacro{\PPRIMAANG}{\PANG - \ANGROT}
    % Fondo
    \pgfmathsetmacro{\HORZ}{0.20}
    \pgfmathsetmacro{\VERT}{0.20}
    \pgfmathsetmacro{\MOD}{sqrt(\HORZ^2 + \VERT^2)}    
    \pgfmathsetmacro{\ANGSD}{atan(\VERT / \HORZ)}
    \pgfmathsetmacro{\ANGII}{\ANGSD + 180.0}            
    % 
    \centering
    \begin{tikzpicture}[%
      scale=\scl,
      every node/.style={black,font=\small},
      eje/.style={->},
      vector/.style={-{Latex}, shorten >=1.2pt, line width=.8pt},
      vectorrotado/.style={vector, draw=green!50!black},
      pcirculo/.style={fill=red, draw=black},
      pprimacirculo/.style={green!90!black, draw=black},
      background/.style={
        line width=\bgborderwidth,
        draw=\bgbordercolor,
        fill=\bgcolor,
      },      
      ]
      % Coordenadas
      \coordinate (O) at (0,0);
      \coordinate (under_origin) at (0,-3mm);
      \coordinate (left_origin) at (-3mm,0);
      \coordinate (zini) at (-\ZMLONG cm,0);
      \coordinate (zfin) at (\ZPLONG cm,0);
      \coordinate (xini) at (0,-\XMLONG cm);
      \coordinate (xfin) at (0,\XPLONG cm);
      \coordinate (P) at (\PANG:\PMOD cm);
      \coordinate (P') at  (\PPRIMAANG:\PPRIMAMOD);
      \path (O) -- coordinate (OPmidway) (P);
      \path (O) -- coordinate (OP'midway) (P');
      \path (O) -- coordinate[pos=1.2] (parrow) (P);
      \path (O) -- coordinate[pos=1.2] (p'arrow) (P');
      % Ángulo \varepsilon
      \path (P') -- (O) -- (P) pic
      [draw=black!50!,fill=green!20,"\footnotesize $\varepsilon$",angle
      radius=6mm, angle eccentricity=1.5] {angle = P'--O--P};
      % Ejes
      \draw[eje] (zini) -- (zfin);
      \node[right] at (zfin) {$z$};
      \draw[eje] (xini) -- (xfin);
      \node[above] at (xfin) {$x$};
      % Punto P
      \draw[vector] (O) -- (P);
      \node[above=5pt] at (OPmidway) {$\vvv{r}$};
      \fill[pcirculo] (P) circle [radius=1.4pt];
      \node[above] at (P) {$P$};
      % Punto P'
      \draw[vectorrotado] (O) -- (P');
      \node[right=0pt] at (OP'midway) {$\vvv{r}'$};
      \fill[pprimacirculo] (P') circle [radius=1.4pt];
      \node[right] at (P') {$P'$};
      % Sentido de giro del vector
      \draw [-{Latex},green!40!black,shorten <= 3pt]
      (parrow) to[bend left=30] (p'arrow);
      % Incremento z
      \draw[fill=red,draw=black] (P |- O) circle[radius=1.5pt];
      \draw[fill=green!80!black,draw=black] (P' |- O)
      circle[radius=1.5pt];
      \draw[-{Latex}] (P |- under_origin) --
      node[below] {\scriptsize $\Delta z > 0$} (P' |- under_origin);
      % Incremento x
      \draw[fill=red,draw=black] (P -| O) circle[radius=1.5pt];
      \draw[fill=green!80!black,draw=black] (P' -| O)
      circle[radius=1.5pt];
      \draw[-{Latex}] (P -| left_origin) --
      node[above=-1pt,sloped,rotate=180]
      {\scriptsize $\Delta x < 0$} (P' -| left_origin);
      % Eje y
      \node[below left=1.8ex and -0.5em] at (O) {\scriptsize $\Delta y = 0$};
      \fill[fill=white,draw=black] (O) circle[radius=2.5pt]
      node[below left] {$y$}; \fill[fill=black] (O) circle[radius=.7pt];
      % Fondo amarillo
      \path (current bounding box.south west) ++(\ANGII:\MOD) coordinate (SW);
      \path (current bounding box.north east) ++(\ANGSD:\MOD) coordinate (NE);
      % \filldraw[draw=black, fill=red] (NE) circle[radius=1pt];
      \begin{scope}[on background layer]
        \draw[background] (SW) rectangle (NE);
      \end{scope}
    \end{tikzpicture}
  \end{minipage}
  \caption{Rotación infinitesimal de un ángulo $\varepsilon$,
    alrededor de los ejes $z$, $x$ e $y$, respectivamente.
    Tal y como se indica, el eje perpendicular en cada diagrama, ejes z, x e y,
    respectivamente, sale hacia fuera del texto.}
  \label{fig:so3-giros_zxy}
\end{figure}

\subsection{Generador para la rotación alrededor del eje
   \mathinhead{x}{x}, \mathinhead{\mmm{G}_x}{Gx}} 
 A continuación, deseamos obtener el generador del giro en el eje $x$.
 Podemos hacernos una idea de cómo construirlo mediante una representación gráfica como
 la anterior, pero adaptada al giro alrededor del eje $x$.

 Fijémonos en el centro de la figura \ref{fig:so3-giros_zxy} y construyamos las
 transformaciones que se obtienen.
 La coordenada $x$ no cambia, porque el giro en este eje no la modifica; esto se refleja
 en la ecuación (\ref{eq:so3-giro_x_incx}).
 La coordenada $y$ aumenta ligeramente, de ahí la suma en la
 expresión~(\ref{eq:so3-giro_x_incy}).
 Por último, la coordenada $z$ disminuye ligeramente, y la ecuación
 (\ref{eq:so3-giro_x_incz}) representa
 esta transformación.
\begin{subequations}\label{eq:so3-giro_x_incxyz}
\begin{align}
  \label{eq:so3-giro_x_incx}
  x' &= x \\
  \label{eq:so3-giro_x_incy}
  y' &= y + \varepsilon z\\
  \label{eq:so3-giro_x_incz}
  z' &= z - \varepsilon y
\end{align}
\end{subequations}

{\small
  \begin{align*}
    \vvv{x'}
    &=
      \begin{pmatrix} x' \\ y' \\ z'\end{pmatrix}
    =
    \begin{pmatrix}
      x \\ y + \varepsilon z \\ z - \varepsilon y
    \end{pmatrix}
    = \begin{pmatrix}
      x \\ y \\ z
    \end{pmatrix}
    + \varepsilon
    \begin{pmatrix}
      0 \\ z \\ -y
    \end{pmatrix}\\
    &= \left[
      \begin{pmatrix}
        1 & 0 & 0\\
        0 & 1 & 0\\
        0 & 0 & 1
      \end{pmatrix}
    + \varepsilon
    \begin{pmatrix}
      0 & 0 & 0 \\
      0 & 0 & 1 \\
      0 & -1 & 0
    \end{pmatrix}
    \right]
    \begin{pmatrix}
      x \\ y \\ z
    \end{pmatrix}
    =
    [\mmm{I} + \varepsilon\kern1pt\mmm{G}_x] \vvv{x}
    \approx
    \mmm{R}_x(\varepsilon)\kern1pt\vvv{x}
  \end{align*}
}

El generador para la rotación pasiva alrededor del eje $x$ resulta
\begin{equation}
  \label{eq:so3-generador_x}
  \mmm{G}_x = 
  \begin{pmatrix} 0 & 0 & 0
    \\ 0 & 0 & 1\\ 0 & -1 & 0
  \end{pmatrix}
\end{equation}

\subsection{Generador para la rotación alrededor del eje
   \mathinhead{y}{y}, \mathinhead{\mmm{G}_y}{Gy}}
 Razonando de forma similar podemos obtener el generador para el giro alrededor del eje
 $y$, omitimos el razonamiento (ver parte derecha de la figura \ref{fig:so3-giros_zxy})
\begin{subequations}\label{eq:so3-giro_y_incxyz}
\begin{align}
  \label{eq:so3-giro_y_incx}
  x' &= x - \varepsilon z\\
  \label{giro_y_incy}
  y' &= y\\
  \label{eq:so3-giro_y_incz}
  z' &= z + \varepsilon x
\end{align}
\end{subequations}

{\small
  \begin{align*}
    \vvv{x'}
    &=
      \begin{pmatrix}
        x' \\ y' \\ z'
      \end{pmatrix}
    =
    \begin{pmatrix}
      x - \varepsilon z \\ y \\ z + \varepsilon x
    \end{pmatrix}
    = \begin{pmatrix}
      x \\ y \\ z
    \end{pmatrix}
    + \varepsilon
    \begin{pmatrix}
      -z \\ 0 \\ x
    \end{pmatrix}\\
    &= \left[
    \begin{pmatrix}
      1 & 0 & 0\\
      0 & 1 & 0\\
      0 & 0 & 1
      \end{pmatrix}
    + \varepsilon
    \begin{pmatrix}
      0 & 0 & -1 \\
      0 & 0 & 0 \\
      1 & 0 & 0
    \end{pmatrix}
    \right]
    \begin{pmatrix}
      x \\ y \\ z
    \end{pmatrix}
    =
    [\mmm{I} + \varepsilon\kern1pt\mmm{G}_y] \vvv{x}
    \approx
    \mmm{R}_y(\varepsilon)\kern1pt\vvv{x}
  \end{align*}
}

El generador para la rotación pasiva alrededor del eje $y$ resulta
\begin{equation}
  \label{eq:so3-generador_y}
  \mmm{G}_y = 
  \begin{pmatrix} 0 & 0 & -1
    \\ 0 & 0 & 0\\ 1 & 0 & 0
  \end{pmatrix}
\end{equation}

\subsection{Rotación genérica en tres dimensiones}
Hemos obtenido las tres matrices generadoras del grupo SO(3) que nos permiten obtener
rotaciones alrededor de cada uno de los ejex $x$, $y$ y $z$.
\begin{equation}
  \label{eq:so3-gx_gy_gz}
  \mmm{G}_x = 
  \begin{pmatrix}
    0 & 0 & 0\\
    0 & 0 & 1\\
    0 & -1 & 0
  \end{pmatrix}
  \hspace*{1em}
  \mmm{G}_y = 
  \begin{pmatrix}
    0 & 0 & -1\\
    0 & 0 & 0\\
    1 & 0 & 0
  \end{pmatrix}
  \hspace*{1em}
  \mmm{G}_z = 
  \begin{pmatrix}
    0 & 1 & 0\\
    -1 & 0 & 0\\
    0 & 0 & 0
  \end{pmatrix}
\end{equation}

La matriz de rotación genérica alrededor de cualquier eje, tiene la forma
\[
  \mmm{R}(\alpha) = e^{\alpha\mmm{G}}
\]
donde $\alpha$ es el ángulo girado alrededor de un eje y $\mmm{G}$ es una matriz
antisimétrica como en toda rotación en el espacio euclídeo.
Si sustituimos estas matrices generadoras en $\mmm{G}$, tendríamos las matrices de
rotación para cada uno de los ejes, $x$, $y$ y $z$.

¿Cómo se representaría una matriz de rotación en tres dimensiones alrededor de cualquier
eje, aunque no coincidiera con ningún eje de coordenadas?

Incluso antes de conocer la respuesta tenemos la seguridad de que la matriz $\mmm{G}$
seguirá siendo antisimétrica.
Pero cualquier matriz antisimétrica $3\times 3$, se construye como una combinación lineal
de las $G_x$, $G_y$ y $G_z$; esto es, las matrices generadoras~(\ref{eq:so3-gx_gy_gz})
forman una base, y cualquier otra matriz que genere una rotación dentro del grupo SO(3)
debe ser una combinación lineal de las anteriores.
\begin{equation}
  \label{eq:so3-combinacion_lineal}
  \mmm{G} = n_x\mmm{G}_x + n_y\mmm{G}_y + n_z\mmm{G}_z
\end{equation}
más adelante descubriremos el significado de los coeficientes, $n_x$, $n_y$ y $n_z$ de la
combinación lineal.

Así, la rotación genérica queda
\begin{equation}
  \label{eq:so3-rotacion_general_3d}
  \mmm{R}_n(\alpha) = e^{\alpha\,(n_x\mmm{G}_x + n_y\mmm{G}_y + n_z\mmm{G}_z)}
\end{equation}

\subsubsection{Rotación infinitesimal}
Una de las ventajas de los grupos de Lie se basa en que como son continuos, podemos
analizar su comportamiento utilizando cualquier rotación y la más sencilla es la
infinitesimal, representada por el ángulo $\varepsilon$
\begin{equation}
  \label{eq:so3-rotacion_infinitesimal_n}
  \mmm{R}_n(\varepsilon)
  = \exp\left\{\varepsilon (n_x\mmm{G}_x + n_y\mmm{G}_y + n_z\mmm{G}_z)\right\}
  \approx \mmm{I} + \varepsilon (n_x\mmm{G}_x + n_y\mmm{G}_y + n_z\mmm{G}_z)
\end{equation}

Seguimos desarrollando la matriz de rotación general, alrededor de un eje genérico
\begin{align*}
  \mmm{R}_n(\varepsilon)
  &\approx \mmm{I} + \varepsilon
    \left[
    n_x\begin{pmatrix}0 & 0 & 0\\ 0 & 0 & 1\\ 0 & -1 & 0\end{pmatrix}
    + n_y\begin{pmatrix}0 & 0 & -1\\ 0 & 0 & 0\\ 1 & 0 & 0\end{pmatrix}
    + n_z\begin{pmatrix}0 & 1 & 0\\ -1 & 0 & 0\\ 0 & 0 & 0\end{pmatrix}
    \right]\\
  &= \mmm{I} + \varepsilon
    \left[
    \begin{pmatrix}0 & 0 & 0\\ 0 & 0 & n_x\\ 0 & -n_x & 0\end{pmatrix}
    + \begin{pmatrix}0 & 0 & -n_y\\ 0 & 0 & 0\\ n_y & 0 & 0\end{pmatrix}
    + \begin{pmatrix}0 & n_z & 0\\ -n_z & 0 & 0\\ 0 & 0 & 0\end{pmatrix}
   \right]\\
   &= \mmm{I} + \varepsilon
     \begin{pmatrix} 0 & n_z & -n_y\\ -n_z & 0 & n_x\\ n_y & -n_x & 0\end{pmatrix}
\end{align*}

Aplicamos este giro genérico infinitesimal a un punto
\begin{align*}
  \begin{pmatrix}
    x' \\ y' \\ z'
  \end{pmatrix}
  &=
    \mmm{R}_n(\varepsilon)
    \begin{pmatrix}
      x \\ y \\ z
    \end{pmatrix}
  \approx \left[
  \mmm{I} + \varepsilon
  \begin{pmatrix}
    0 & n_z & -n_y\\
    -n_z & 0 & n_x\\
    n_y & -n_x & 0
  \end{pmatrix}
  \right]
  \begin{pmatrix}
    x \\ y \\ z
  \end{pmatrix}\\
  &=
    \begin{pmatrix}
      x \\ y \\ z
    \end{pmatrix}
  + \varepsilon
  \begin{pmatrix}
    0 & n_z & -n_y\\
    -n_z & 0 & n_x\\
    n_y & -n_x & 0
  \end{pmatrix}
  \begin{pmatrix}
    x \\ y \\ z
  \end{pmatrix}
  =
  \begin{pmatrix}
    x \\ y \\ z
  \end{pmatrix}
  + \varepsilon
  \begin{pmatrix}
    n_zy-n_yz \\ -n_zx+n_xz \\ n_yx - n_xy
  \end{pmatrix}
\end{align*}

Los elementos del vector columna que multiplica a $\varepsilon$ son las componentes del
producto vectorial
\[
  \vvv{x} \prodvec \vvv{n} =
  \begin{vmatrix}
    \hat\imath & \hat\jmath & \hat k\\
    x          & y          & z     \\
    n_x & n_y & n_z
  \end{vmatrix}
  = (n_zy-n_yz)\,\hat\imath + (n_xz-n_zx)\,\hat\jmath + (n_yx-n_xy)\,\hat k
\]

El resultado de operar la matriz de rotación infinitesimal con un vector cualquiera
quedaría
\begin{equation}
  \label{eq:so3-expr_con_n}
  \vvv{x'}
  = \mmm{R}_n(\varepsilon)\kern1pt\vvv{x}
  = \vvv{x} + \varepsilon\vvv{x}\prodvec \vvv{n}
\end{equation}

\subsubsection{Significado físico del vector \vvv{n}}
Los coeficientes $n_x$, $n_y$ y $n_z$ de la combinación lineal
(\ref{eq:so3-combinacion_lineal}) deben estar normalizados, esto es, su módulo debe ser
la unidad
\[
  n_x^2 + n_y^2 + n_z^2 = 1
\]
 
Aunque no es una demostración, se puede comprobar fácilmente cuando el eje es uno de los
de coordenadas.
Por ejemplo, rotemos infinitesimalmente alrededor del eje $z$ un punto $\vvv{x}$
utilizando la expresión (\ref{eq:so3-rotacion_infinitesimal_z})

\[
  \vvv{x'} = \mmm{R}_z(\varepsilon) \kern1pt\vvv{x} \approx (\mmm{I} +
  \varepsilon\mmm{G}_z)\kern1pt \vvv{x}
\]

Realizamos el mismo cálculo utilizando la expresión
general~(\ref{eq:so3-rotacion_infinitesimal_n}), considerando una rotación alrededor del
eje $z$; en este caso, $n_x = n_y = 0$
\[
  \mmm{R}_z(\varepsilon)\kern1pt \vvv{x} \approx (\mmm{I} +
  \varepsilon n_z \kern1pt\mmm{G}_z)\kern1pt \vvv{x}
\]
Comparando las dos últimas expresiones deducimos que $n_z$ debe valer uno.

Lo observado es compatible con que el vector formado por las componentes $n_x$, $n_y$ y
$n_z$ sea unitario (módulo unidad), que se suele escribir $\xhat{n}$.
La expresión~(\ref{eq:so3-expr_con_n}) quedaría
\begin{equation}
  \label{eq:so3-giro_infinitesimal_x}
  \vvv{x'}
  = \mmm{R}(\xhat{n}, \varepsilon)\kern1pt \vvv{x}
  = \vvv{x} + \varepsilon\kern1pt\vvv{x}\prodvec \xhat{n}
\end{equation}

%Teniendo en cuenta lo anterior,
La matriz de rotación general infinitesimal en tres
dimensiones~(\ref{eq:so3-rotacion_general_3d}) se puede abreviar sustituyendo,
---incorrectamente--- $n_xG_x+n_yG_y+n_zG_z$ por $\xhat{n}\cdot\mmm{G}$, como si fuera un
producto escalar\footnotemark{}.
\footnotetext{Téngase en cuenta que $\xhat{n}\cdot\mmm{G}$ no es un producto escalar,
  aunque lo parezca. Los factores del producto escalar o producto   interno tienen que
  ser \emph{elementos del mismo espacio vectorial}, pero $\xhat{n}$ y $\mmm{G}$ no lo
  son, porque $\xhat{n}$ es un vector del espacio   ordinario $\symbb{R}^3$ y $\mmm{G}$
  es un vector del espacio de   matrices cuadradas generadoras de las rotaciones en
  $\symbb{R}^3$.
  Obsérvese que un producto escalar produce un escalar y, en nuestro caso,
  $\xhat{n}\cdot\mmm{G}$ resulta ser una matriz.}
\begin{equation}\label{eq:so3-ali_rot_infinitesimal_SO3}
  \mmm{R}(\xhat{n},\varepsilon)
  =
  \mmm{I} + \mmm{A}(\xhat{n},\varepsilon)
  =
  \mmm{I}
  +
  \varepsilon\xhat{n}\cdot\mmm{G}
  = \mmm{I} + \varepsilon
  \begin{pmatrix}
    0 & n_z & -n_y\\ -n_z & 0 & n_x\\ n_y & -n_x & 0
  \end{pmatrix}
\end{equation}

La matriz general de una transformacion finita se podrá escribir como
\begin{equation}\label{eq:so3-ali_rot_SO3}
  \mmm{R}(\xhat{n}, \alpha)
  = e^{\alpha (n_x\mmm{G}_x + n_y\mmm{G}_y +
    n_z\mmm{G}_z)}
  = \exp(\alpha\, \xhat{n} \cdot \mmm{G})
\end{equation}
siendo $\xhat{n}$ el versor que indica el eje de rotación, $\alpha$ el ángulo que gira el
sistema de coordenadas alrededor del eje de rotación (rotación pasiva) y $\mmm{G}$ el
vector formado por los generadores de la rotación en tres dimensiones.

Si $\alpha$ representara el giro de los vectores alrededor del eje de rotación, entonces
la matriz de rotación sería la inversa
\begin{equation}\label{eq:so3-ali_rotobj_SO3}
  \mmm{R}(\xhat{n}, \alpha)
  = e^{-\alpha (n_x\mmm{G}_x + n_y\mmm{G}_y +
    n_z\mmm{G}_z)}
  = \exp(-\alpha\, \xhat{n} \cdot \mmm{G})
\end{equation}

\subsubsection{Rotación general de un ángulo finito por medios
  geométricos}
En (\ref{eq:so3-giro_infinitesimal_x}) obtuvimos las nuevas coordenadas de un vector
$\vvv{x}$, después de una rotación pasiva infinitesimal.
Ahora estamos interesados en obtener una expresión para una rotación pasiva de un ángulo
discreto $\alpha$ cualquiera, $0 \leq \alpha < 2\pi$.
Se podría conseguir desarrollando la exponencial~(\ref{eq:so3-rotacion_general_3d}),
pero sería muy laborioso; en su lugar seguiremos un razonaminento geométrico más breve.

En la figura \ref{fig:so3-giro_pasivo} se representa un giro finito positivo (en sentido
antihorario) alrededor de un eje de giro genérico.
En ella se observa un cambio aparente en las coordenadas $\vvv{x}$ de un punto a otras
$\vvv{x'}$, como si el punto hubiera girado en sentido horario el ángulo $\alpha$.
Es importante recalcar que $\vvv{x}$ y $\vvv{x'}$ son el mismo vector, y solo ha
cambiado aparentemente de posición\footnotemark{} debido a que el sistema de referencia
ha girado en sentido contrario a las agujas del reloj.
\footnotetext{En la figura el vector $\vvv{x}$ ha girado \emph{aparentemente} en sentido
  negativo (horario).}
Además, se aprecia que el extremo de $\vvv{x}$ describe un arco de circunferencia de
radio $R$.
Obsérvese que el eje de giro definido por el vector unitario $\xhat{n}$, pasa por el
centro de la circunferencia.

\pagebreak
Con ayuda de la figura \ref{fig:so3-giro_pasivo_vectores_referencia} obtendremos unas
igualdades que se necesitarán posteriormente:
\begin{itemize}
\item Vemos que $\vvv{x}$ se puede descomponer en una suma de dos vectores;
  el primero, $\vvv{x}_{\scriptstyle\parallel}$ es paralelo al  eje de rotación
  y el segundo, $\vvv{x}_{\scriptstyle\perp}$ es perpendicular al anterior y se encuentra
  en el plano del arco descrito por el giro del extremo de $\vvv{x}$
  \[
    \vvv{x} = \vvv{x}_{\scriptstyle\,\parallel}
    + \vvv{x}_{\scriptstyle\perp}
  \]
 
  Despejamos $\vvv{x}_{\scriptstyle\perp}$ y ya tenemos la primera expresión
  \begin{equation}
    \label{eq:so3-x_perpendicular}
    \vvv{x}_{\scriptstyle\perp} = \vvv{x} - \vvv{x}_{\scriptstyle\,\parallel}
  \end{equation}

\item Además, se aprecia que la proyección de $\vvv{x}$ sobre el eje de giro es el módulo
  del vector $\vvv{x}_{\scriptstyle\,\parallel}$
  (altura del cono), que se calcula mediante el producto escalar de $\vvv{x}$ por el
  versor $\xhat{n}$, que representa al eje de giro
  \[
    |\vvv{x}_{\scriptstyle\,\parallel}| = \vvv{x}\cdot \xhat{n} 
  \]
   
  Por tanto
  \begin{equation}
    \label{eq:so3-x_paralela}
    \vvv{x}_{\scriptstyle\,\parallel}
    = |\vvv{x}_{\scriptstyle\,\parallel}|\,\xhat{n}
    = (\vvv{x}\cdot \xhat{n})\,\xhat{n}
  \end{equation}

\item El vector unitario asociado a $\vvv{x}_{\scriptstyle\,\perp}$ es
  \begin{equation}
    \label{eq:so3-u_perp}
    \xhat{u}_{\scriptstyle\,\perp}
    = \frac{\vvv{x}_{\scriptstyle\,\perp}}{|\vvv{x}_{\scriptstyle\,\perp}|}
    = \frac{\vvv{x}_{\scriptstyle\,\perp}}{R}
  \end{equation}
  
\item Según la figura utilizaremos el producto vectorial para definir un vector unitario
  $\xhat{u}_{\scriptstyle\,\prodvec}$, perpendicular a $\vvv{x}_{\scriptstyle\,\perp}$, situado
  en el plano de la circunferencia.
  Además, representaremos por $\beta$ el ángulo que forman $\vvv{x}$ y
  $\xhat{n}$
  \begin{equation}
    \label{eq:so3-u_prodvect}
    \hat{\vvv{u}}_{\scriptstyle\prodvec}
    =
    \frac{\vvv{x}_\prodvec}{|\vvv{x}_\prodvec|}
    =
    \frac{\vvv{x}\prodvec \xhat{n}}{|\vvv{x}\prodvec \xhat{n}|}
    =
    \frac{\vvv{x}\prodvec \xhat{n}}{|\vvv{x}|\,\sin\beta}
    =
    \frac{\vvv{x}\prodvec \xhat{n}}{R} 
  \end{equation}

\end{itemize}
  
% #########################################################
% PRIMERA PAREJA DE FIGURAS
% #########################################################
\begin{figure}[ht]
  \centering
  \begin{minipage}{0.46\linewidth}
    \pgfmathsetmacro{\ALTURAEJEGIRO}{5.5}
    \pgfmathsetmacro{\GENERATRIZ}{4.0}
    \pgfmathsetmacro{\ANCHOELIPSE}{1.8}
    \pgfmathsetmacro{\ALTOELIPSE}{0.6}
    % Factor de reducción de la elipse para representar el ángulo de giro alfa
    \pgfmathsetmacro{\ARCOELIPSE}{0.4}
    \pgfmathsetmacro{\ANCHOARCO}{\ARCOELIPSE * \ANCHOELIPSE}
    \pgfmathsetmacro{\ALTOARCO}{\ARCOELIPSE * \ALTOELIPSE}
    \pgfmathsetmacro{\VERSORN}{1.6}
    % Fondo
    \pgfmathsetmacro{\HORZ}{0.60}
    \pgfmathsetmacro{\VERT}{0.30}
    \pgfmathsetmacro{\MOD}{sqrt(\HORZ^2 + \VERT^2)}
    \pgfmathsetmacro{\ANGSD}{atan(\VERT / \HORZ)}
    \pgfmathsetmacro{\ANGII}{\ANGSD + 180.0}          
    \begin{tikzpicture}[%
      scale=1.0,
      baseline,
      anguloarea/.style={fill=orange!60},
      anguloarco/.style={%
        draw=black!80,line width=0.8pt,-{Latex[width'=0.8pt 0.25]},shorten >= -2pt},
      elipse/.style={%
        draw=black!25,left color=black!15, middle color=white, right color=white,
        rotate=-30},
      cono/.style={%
        draw=black!30, left color=black!10, middle color=black!20,right color=black!40,
        opacity=0.85, line width=0.1pt, shading angle=60},
      vector/.style={green!50!black,-{Latex},line width=1.2pt},
      vectorgirado/.style={red!80!black,-{Latex},line width=1.2pt},
      ejegirooculto/.style={black, line width=0.9pt},
      ejegiro/.style={ejegirooculto, -{Latex[width'=0pt 0.6pt]}},
      generatrizoculta/.style={green!50!black, line width=0.9pt},
      vectorcomponente/.style={%
        generatrizoculta,-{Latex[width'=0pt 0.6]}},
      vectorcomponentegirado/.style={%
        red!80!black,-{Latex[width'=0pt 0.6]},line width=0.9pt},  
      background/.style={
        line width=\bgborderwidth,
        draw=\bgbordercolor,
        fill=\bgcolor,
      },
      ]
      %%% COORDENADAS
      % Vértice del cono
      \coordinate (vertice) at (0,0);
      \begin{scope}[rotate around={-30:(vertice)}]      
        % Centro de la base del cono
        \coordinate (centrobase) at (0,\GENERATRIZ);
        % Extremo del versor n
        \coordinate (extremoversorn) at (0,\VERSORN);
        % Extremo del eje de giro
        \coordinate (extremoejegiro) at (0,\ALTURAEJEGIRO);

        %% COORDENADAS DEFINIDAS A TRAVÉS DE PATHS
        % Extremos de los vectores xperp y su opuesto
        \path (centrobase) -- +(right:\ANCHOELIPSE) coordinate (endxperp);
        \path (centrobase) -- +(left:\ANCHOELIPSE) coordinate (endmenosxperp);
        
        %% PATHS
        % Path elipse (name=elipse)
        \path[name path=elipse]
        (centrobase) ellipse (\ANCHOELIPSE cm and \ALTOELIPSE cm);
        % Path altura cono (name=generatriz)
        \path[save path=\generatriz, name path=generatriz] (vertice) -- (centrobase);
        % Path Línea x perpendicular (name=xperp)
        \path[save path=\xperp, name path=xperp] (centrobase) -- +(right:\ANCHOELIPSE);
        % Path Línea (name=xprimaperp)
        \path[rotate=-30,name path=linexprimaperp] (centrobase) -- +(right:\ANCHOELIPSE);

        %%% INTERSECCIONES DE PATHS
        \path[name intersections={of=generatriz and elipse}]
        (intersection-1) coordinate (ptogeneratriz);
        \path[name intersections={of=elipse and linexprimaperp}]
        (intersection-1) coordinate (endxprimaperp);

        %%% OTROS PATHS
        % Eje de giro oculto
        \path[save path=\generatrizoculta] (vertice) -- (ptogeneratriz);
        % generatriz visible
        \path[save path=\generatrizvisible] (ptogeneratriz) -- (centrobase);
        % Vector eje de giro
        \path[save path=\ejedegiro] (ptogeneratriz) -- (extremoejegiro);
        % Segmentos en la base del cono
        \path[save path=\xperp] (centrobase) -- (endxperp);
        \path[save path=\xprimaperp] (centrobase) -- (endxprimaperp);
        % Vector original y girado
        \path[save path=\x] (vertice) -- (endxperp);
        \path[save path=\xprima] (vertice) -- (endxprimaperp);
        % Versor n del eje de giro
        \path[save path=\versorn] (vertice) -- (extremoversorn);
        % Ángulo de giro
        % -Área del ángulo
        \coordinate (anguloini) at ($(centrobase) + (0:\ANCHOARCO)$);
        \path[save path=\anguloarea] (anguloini)
        arc (0:-58:\ANCHOARCO cm and \ALTOARCO cm) -- (centrobase) -- cycle;
        % -Arco del ángulo
        \path[save path=\anguloarco] ($(centrobase) + (0:\ANCHOARCO)$)
        arc (0:-58:\ANCHOARCO cm and \ALTOARCO cm);
        \coordinate (textoangulo) at ($(centrobase) + (-13:0.8)$);
      \end{scope}

      %%% DIBUJO
      % ZONA OCULTA POR EL CONO
      % Parte de la generatriz semioculta por el cono
      \draw[ejegirooculto] [use path=\generatrizoculta];

      % Como el cono será translúcido, hay que situar los elementos que queramos
      % que se aprecien semiocultos, antes de esta línea.
      
      % ZONA CONO Y ELIPSE
      % Cono (en realidad es un triángulo, donde el lado opuesto al vértice será
      % tapado por la elipse).
      \shade[cono] (endmenosxperp) -- (vertice) -- (endxperp) -- cycle;
      % Elipse
      \shade[elipse] (centrobase) ellipse (\ANCHOELIPSE cm and \ALTOELIPSE cm);
      
      % ZONA VISIBLE O VISIBILIZADA (como el vector unitario del eje de giro)

      % Ángulo de giro
      % Área del ángulo
      \fill[anguloarea] [use path=\anguloarea];
      % Arco del ángulo
      \draw[anguloarco] [use path=\anguloarco];
      \node at (textoangulo) {$\scriptstyle\alpha$};
      % Eje de giro
      \draw[ejegiro] (ptogeneratriz) -- (extremoejegiro);
      \draw (extremoejegiro) node[right,black]{\small Eje de giro};
      
      % Líneas en la base del cono
      \draw [use path=\xperp];
      \path (centrobase) -- node[pos=0.6,above]
      {\small $R$} (endxperp);
      % Línea xprimaperp
      \draw [use path=\xprimaperp];

      % Vector x original
      \draw[vector] [use path=\x];
      \path (vertice) -- node[pos=0.5,right=4pt,green!50!black]
      {\small $\vvv{x}$} (endxperp);
      
      % Vector x girado
      \draw[vectorgirado] [use path=\xprima];
      \path (vertice) -- node[pos=0.6,above=2pt,red!80!black]
      {\small $\vvv{x'}$} (endxprimaperp);

      % Versor n del eje de giro (está oculto, pero le damos algo más de visibilidad)
      \draw[vectorcomponente, black!50, line width=1pt] [use path=\versorn];
      \node[black!60,above left=-2pt and -1pt of extremoversorn] {\small $\xhat{n}$};

      % Centro de la base y vértice
      \fill (centrobase) circle[radius=1pt];
      \fill (vertice) circle[radius=0.5pt];

      % Fondo amarillo
      \path (current bounding box.south west) ++(\ANGII:\MOD) coordinate (SW);
      \path (current bounding box.north east) ++(\ANGSD:\MOD) coordinate (NE);
      \begin{scope}[on background layer]
        \draw[background] (SW) rectangle (NE);
      \end{scope}            
    \end{tikzpicture}
    \caption{Rotación pasiva un ángulo $\alpha$ sobre un vector $\vvv{x}$ alrededor de
      un eje de giro genérico.}
    \label{fig:so3-giro_pasivo}
  \end{minipage}
  \hspace{1.0em}
  \begin{minipage}{0.4\linewidth} 
    \pgfmathsetmacro{\ALTURAEJEGIRO}{5.5}
    \pgfmathsetmacro{\GENERATRIZ}{4.0}
    \pgfmathsetmacro{\ANCHOELIPSE}{1.8}
    \pgfmathsetmacro{\ALTOELIPSE}{0.6}
    \pgfmathsetmacro{\VERSORN}{1.6}
    % Fondo
    \pgfmathsetmacro{\HORZ}{0.55}
    \pgfmathsetmacro{\VERT}{0.45}
    \pgfmathsetmacro{\MOD}{sqrt(\HORZ^2 + \VERT^2)}
    \pgfmathsetmacro{\ANGSD}{atan(\VERT / \HORZ)}
    \pgfmathsetmacro{\ANGII}{\ANGSD + 180.0}          
    \begin{tikzpicture}[%
      scale=1.0,
      baseline,
      elipse/.style={%
        draw=black!25,left color=black!15, middle color=white, right color=white,
        rotate=-30},
      cono/.style={%
        draw=black!30, left color=black!10, middle color=black!20,right color=black!40,
        opacity=0.85, line width=0.1pt, shading angle=60},
      vector/.style={green!50!black,-{Latex},line width=1.2pt},
      generatrizoculta/.style={green!50!black, line width=0.9pt},
      vectorcomponente/.style={%
        generatrizoculta,-{Latex[width'=0pt 0.6]}},
      vectorcomponentegirado/.style={%
        red!80!black,-{Latex[width'=0pt 0.6]},line width=0.9pt},  
      background/.style={
        line width=\bgborderwidth,
        draw=\bgbordercolor,
        fill=\bgcolor,
      },
      ]
      %%% COORDENADAS
      % Vértice del cono
      \coordinate (vertice) at (0,0);
      \begin{scope}[rotate around={-30:(vertice)}]      
        % Centro de la base del cono
        \coordinate (centrobase) at (0,\GENERATRIZ);
        % Extremo del versor n
        \coordinate (extremoversorn) at (0,\VERSORN);
        % Extremo del eje de giro
        \coordinate (extremoejegiro) at (0,\ALTURAEJEGIRO);

        % COORDENADAS DEFINIDAS A TRAVÉS DE PATHS
        \path (centrobase) -- +(right:\ANCHOELIPSE) coordinate (endxperp);
        \path (centrobase) -- +(left:\ANCHOELIPSE) coordinate (endmenosxperp);
        
        %%% PATHS
        % Path elipse (name=elipse)
        \path[name path=elipse]
        (centrobase) ellipse (\ANCHOELIPSE cm and \ALTOELIPSE cm);
        % Path altura cono (name=generatriz)
        \path[save path=\generatriz, name path=generatriz] (vertice) -- (centrobase);
        % Path Línea x perpendicular (name=xperp)
        \path[save path=\xperp, name path=xperp] (centrobase) -- +(right:\ANCHOELIPSE);
        % Path Línea producto vectorial
        \path[rotate=-125,name path=linexprod] (centrobase) -- +(right:\ANCHOELIPSE);
        % Path Línea (name=xprimaperp)
        \path[rotate=-30,name path=linexprimaperp] (centrobase) -- +(right:\ANCHOELIPSE);

        %%% INTERSECCIONES DE PATHS
        \path[name intersections={of=generatriz and elipse}]
        (intersection-1) coordinate (ptogeneratriz);
        \path[name intersections={of=elipse and linexprod}]
        (intersection-1) coordinate (endxprod);
        \path[name intersections={of=elipse and linexprimaperp}]
        (intersection-1) coordinate (endxprimaperp);

        %%% OTROS PATHS
        % Eje de giro oculto
        \path[save path=\generatrizoculta] (vertice) -- (ptogeneratriz);
        % generatriz visible
        \path[save path=\generatrizvisible] (ptogeneratriz) -- (centrobase);
        % Vector eje de giro
        \path[save path=\ejedegiro] (ptogeneratriz) -- (extremoejegiro);
        % Segmentos en la base del cono
        \path[save path=\xperp] (centrobase) -- (endxperp);
        \path[save path=\xprimaperp] (centrobase) -- (endxprimaperp);
        \path[save path=\xprod, name path=xprod] (centrobase) -- (endxprod);
        % Vector original y girado
        \path[save path=\x] (vertice) -- (endxperp);
        \path[save path=\xprima] (vertice) -- (endxprimaperp);
        % Versor n del eje de giro
        \path[save path=\versorn] (vertice) -- (extremoversorn);
      \end{scope}
    
      %%% DIBUJO
      % ZONA OCULTA POR EL CONO
      % Parte de la generatriz semioculta por el cono
      \draw[generatrizoculta] [use path=\generatrizoculta];

      % Como el cono será translúcido, hay que situar los elementos que queramos
      % que se aprecien semiocultos, antes de esta línea.
      
      % ZONA CONO Y ELIPSE
      % Cono (en realidad es un triángulo, donde el lado opuesto al vértice será
      % tapado por la elipse).
      \shade[cono] (endmenosxperp) -- (vertice) -- (endxperp) -- cycle;
      % Elipse
      \shade[elipse] (centrobase) ellipse (\ANCHOELIPSE cm and \ALTOELIPSE cm);

      % ZONA VISIBLE O VISIBILIZADA (como el vector unitario del eje de giro)
      
      % Parte de la generatriz visible
      \draw[vectorcomponente] [use path=\generatrizvisible];
      \path (vertice) -- node[green!50!black,pos=0.70,left]
      {\small $\vvv{x}_{\scriptstyle\parallel}$} (centrobase);

      % Vectores componentes en la base del cono
      % Vector xperp
      \draw[vectorcomponente] [use path=\xperp];
      \path (centrobase) -- node[green!50!black,pos=0.6,above]
      {\small $\vvv{x}_{\scriptstyle\perp}$} (endxperp);
      % Vector producto
      \draw[vectorcomponente] [use path=\xprod];
      \path (centrobase) -- node[green!50!black,pos=-0.1,left=8pt]
      {\small$\vvv{x}_{\scriptstyle\prodvec}$} (endxprod);
   
      % Ángulo beta
      \path (endxperp) -- (vertice) -- (centrobase)
      pic[draw=black!50,
      fill=black!30,"$\scriptstyle\beta$",angle radius=6mm,
      angle eccentricity=1.4,scale=1,bend left=120]
      {angle=endxperp--vertice--centrobase};

      % Versor n del eje de giro (está oculto, pero le damos algo más de visibilidad)
      \draw[vectorcomponente, black!50, line width=1pt] [use path=\versorn];
      \node[black!60,above left=-2pt and -1pt of extremoversorn] {\small $\xhat{n}$};

      % Vector x original
      \draw[vector] [use path=\x];
      \path (vertice) -- node[pos=0.5,right=4pt,green!50!black]
      {\small $\vvv{x}$} (endxperp);

      % Centro de la base y vértice
      \fill (centrobase) circle[radius=1pt];
      \fill (vertice) circle[radius=0.5pt];
      % Fondo amarillo
      \path (current bounding box.south west) ++(\ANGII:\MOD) coordinate (SW);
      \path (current bounding box.north east) ++(\ANGSD:\MOD) coordinate (NE);
      \begin{scope}[on background layer]
        \draw[background] (SW) rectangle (NE);
      \end{scope}
    \end{tikzpicture}
    \caption{Vectores de referencia para obtener el resultado de un giro pasivo de un
      vector $\vvv{x}$ un ángulo $\alpha$.}
    \label{fig:so3-giro_pasivo_vectores_referencia}
  \end{minipage}
\end{figure}

% #########################################################
% SEGUNDA PAREJA DE FIGURAS -CONOS-
% #########################################################
\begin{figure}[ht]
  \centering
  \begin{minipage}{0.44\linewidth}
    \pgfmathsetmacro{\ALTURAEJEGIRO}{5.5}
    \pgfmathsetmacro{\GENERATRIZ}{4.0}
    \pgfmathsetmacro{\ANCHOELIPSE}{1.8}
    \pgfmathsetmacro{\ALTOELIPSE}{0.6}
    % Factor de reducción de la elipse para representar el ángulo de giro alfa
    \pgfmathsetmacro{\ARCOELIPSE}{0.4}
    \pgfmathsetmacro{\ANCHOARCO}{\ARCOELIPSE * \ANCHOELIPSE}
    \pgfmathsetmacro{\ALTOARCO}{\ARCOELIPSE * \ALTOELIPSE}
    \pgfmathsetmacro{\VERSORN}{1.6}
    % Fondo
    \pgfmathsetmacro{\HORZ}{0.78}
    \pgfmathsetmacro{\VERT}{0.30}
    \pgfmathsetmacro{\MOD}{sqrt(\HORZ^2 + \VERT^2)}
    \pgfmathsetmacro{\ANGSD}{atan(\VERT / \HORZ)}
    \pgfmathsetmacro{\ANGII}{\ANGSD + 180.0}    
    \begin{tikzpicture}[%
      scale=1.0,
      baseline,
      anguloarea/.style={fill=orange!60},
      anguloarco/.style={%
        %draw=black!80,line width{-Latex[length=0.6pt,width=0pt]},shorten <=-1pt},
        draw=black!80,line width=0.8pt,-{Latex[width'=0.8pt 0.30]},shorten >= -2pt},
      elipse/.style={%
        draw=black!25,left color=black!15, middle color=white, right color=white,
        rotate=-30},
      cono/.style={%
        draw=black!30, left color=black!10, middle color=black!20,right color=black!40,
        opacity=0.85, line width=0.1pt, shading angle=60},
      vector/.style={green!50!black,-{Latex},line width=1.2pt},
      vectorgirado/.style={red!80!black,-{Latex},line width=1.2pt},
      generatrizoculta/.style={green!50!black, line width=0.9pt},
      vectorcomponente/.style={%
        generatrizoculta,-{Latex[width'=0pt 0.6]}},
      vectorcomponentegirado/.style={%
        red!80!black,-{Latex[width'=0pt 0.6]},line width=0.9pt},  
      background/.style={
        line width=\bgborderwidth,
        draw=\bgbordercolor,
        fill=\bgcolor,
      },
      ]
      %%% COORDENADAS
      % Vértice del cono
      \coordinate (vertice) at (0,0);
      \begin{scope}[rotate around={-30:(vertice)}]      
        % Centro de la base del cono
        \coordinate (centrobase) at (0,\GENERATRIZ);
        % Extremo del versor n
        \coordinate (extremoversorn) at (0,\VERSORN);
        % Extremo del eje de giro
        \coordinate (extremoejegiro) at (0,\ALTURAEJEGIRO);

        % COORDENADAS DEFINIDAS A TRAVÉS DE PATHS
        \path (centrobase) -- +(right:\ANCHOELIPSE) coordinate (endxperp);
        \path (centrobase) -- +(left:\ANCHOELIPSE) coordinate (endmenosxperp);
        
        %%% PATHS
        % Path elipse (name=elipse)
        \path[name path=elipse]
        (centrobase) ellipse (\ANCHOELIPSE cm and \ALTOELIPSE cm);
        % Path altura cono (name=generatriz)
        \path[save path=\generatriz, name path=generatriz] (vertice) -- (centrobase);
        % Path Línea x perpendicular (name=xperp)
        \path[save path=\xperp, name path=xperp] (centrobase) -- +(right:\ANCHOELIPSE);
        % Path vector producto vectorial
        \path[rotate=-125,name path=linexprod] (centrobase) -- +(right:\ANCHOELIPSE);
        % Path Línea (name=xprimaperp)
        \path[rotate=-30,name path=linexprimaperp] (centrobase) -- +(right:\ANCHOELIPSE);

        %%% INTERSECCIONES DE PATHS
        \path[name intersections={of=generatriz and elipse}]
        (intersection-1) coordinate (ptogeneratriz);
        \path[name intersections={of=elipse and linexprod}]
        (intersection-1) coordinate (endxprod);
        \path[name intersections={of=elipse and linexprimaperp}]
        (intersection-1) coordinate (endxprimaperp);

        %%% OTROS PATHS
        % Eje de giro oculto
        \path[save path=\generatrizoculta] (vertice) -- (ptogeneratriz);
        % generatriz visible
        \path[save path=\generatrizvisible] (ptogeneratriz) -- (centrobase);
        % Vector eje de giro
        \path[save path=\ejedegiro] (ptogeneratriz) -- (extremoejegiro);
        % Segmentos en la base del cono
        \path[save path=\xperp] (centrobase) -- (endxperp);
        \path[save path=\xprimaperp] (centrobase) -- (endxprimaperp);
        \path[save path=\xprod, name path=xprod] (centrobase) -- (endxprod);
        % Vector original y girado
        \path[save path=\x] (vertice) -- (endxperp);
        \path[save path=\xprima] (vertice) -- (endxprimaperp);
        % Versor n del eje de giro
        \path[save path=\versorn] (vertice) -- (extremoversorn);
        % Ángulo de giro
        % -Área del ángulo
        \coordinate (anguloini) at ($(centrobase) + (0:\ANCHOARCO)$);
        \path[save path=\anguloarea] (anguloini)
        arc (0:-58:\ANCHOARCO cm and \ALTOARCO cm) -- (centrobase) -- cycle;
        % -Arco del ángulo
        \path[save path=\anguloarco] ($(centrobase) + (0:\ANCHOARCO)$)
        arc (0:-58:\ANCHOARCO cm and \ALTOARCO cm);
        \coordinate (textoangulo) at ($(centrobase) + (-13:0.8)$);
      \end{scope}
    
      %%% DIBUJO
      % ZONA OCULTA POR EL CONO
      % Parte de la generatriz semioculta por el cono
      \draw[generatrizoculta] [use path=\generatrizoculta];

      % Como el cono será translúcido, hay que situar los elementos que queramos
      % que se aprecien semiocultos, antes de esta línea.
      
      % ZONA CONO Y ELIPSE
      % Cono (en realidad es un triángulo, donde el lado opuesto al vértice será
      % tapado por la elipse).
      \shade[cono] (endmenosxperp) -- (vertice) -- (endxperp) -- cycle;
      % Elipse
      \shade[elipse] (centrobase) ellipse (\ANCHOELIPSE cm and \ALTOELIPSE cm);

      % ZONA VISIBLE O VISIBILIZADA (como el vector unitario del eje de giro)

      % Ángulo de giro
      % Área del ángulo
      \fill[anguloarea] [use path=\anguloarea];
      % Arco del ángulo
      \draw[anguloarco] [use path=\anguloarco];
      \node at (textoangulo) {$\scriptstyle\alpha$};
      
      % Parte de la generatriz visible
      \draw[vectorcomponente] [use path=\generatrizvisible];
      \path (vertice) -- node[green!50!black,pos=0.70,left]
      {\scriptsize $\vvv{x}_{\scriptstyle\parallel}$} (centrobase);

      % Vectores componentes en la base del cono
      % Vector xperp
      \draw[vectorcomponente] [use path=\xperp];
      \path (centrobase) -- node[green!50!black,pos=0.6,above]
      {\scriptsize $\vvv{x}_{\scriptstyle\perp}$} (endxperp);
      % Vector xprimaperp
      \draw[vectorcomponentegirado] [use path=\xprimaperp];
      \path (centrobase) -- node[red!90!black,pos=0.6,left]
      {\scriptsize $\vvv{x}'_{\scriptstyle\perp}$} (endxprimaperp);

      % Vector producto
      \draw[vectorcomponente] [use path=\xprod];
      \path (centrobase) -- node[green!50!black,pos=-0.1,left=8pt]
      {\scriptsize$\vvv{x}_{\scriptstyle\prodvec}$} (endxprod);
   
      % Versor n del eje de giro (está oculto, pero le damos algo más de visibilidad)
      \draw[vectorcomponente, black!50, line width=1pt] [use path=\versorn];
      \node[black!60,above left=-2pt and -1pt of extremoversorn]
      {\scriptsize $\xhat{n}$};

      % Vector x original
      \draw[vector] [use path=\x];
      \path (vertice) -- node[pos=0.5,right=4pt,green!50!black]
      {\small $\vvv{x}$} (endxperp);

      % Vector x girado
      \draw[vectorgirado] [use path=\xprima];
      \path (vertice) -- node[pos=0.6,above=2pt,red!80!black]
      {\small $\vvv{x'}$} (endxprimaperp);

      % Centro de la base y vértice
      \fill (centrobase) circle[radius=1pt];
      \fill (vertice) circle[radius=0.5pt];
      % Fondo amarillo
      \path (current bounding box.south west) ++(\ANGII:\MOD) coordinate (SW);
      \path (current bounding box.north east) ++(\ANGSD:\MOD) coordinate (NE);
      \begin{scope}[on background layer]
        \draw[background] (SW) rectangle (NE);
      \end{scope}
    \end{tikzpicture}
    \caption{El vector transformado $\vvv{x}'$ es la suma de
      $\vvv{x}'_{\scriptstyle\perp}$ y
      $\vvv{x}'_{\scriptstyle\,\parallel} =
      \vvv{x}_{\scriptstyle\,\parallel}$.}
    \label{fig:so3-vector_x_transformado}
  \end{minipage}
  \hspace{1em}
  \begin{minipage}{0.46\linewidth}
    \pgfmathsetmacro{\RADIO}{2.0}
    \pgfmathsetmacro{\ALTURAEJEGIRO}{5.5}
    \pgfmathsetmacro{\ALTURACONO}{4.0}
    \pgfmathsetmacro{\ANCHOELIPSE}{1.8}
    \pgfmathsetmacro{\ALTOELIPSE}{0.6}
    \pgfmathsetmacro{\VERSORN}{1.6}
    \pgfmathsetmacro{\MARGENIZDO}{1.0}
    \pgfmathsetmacro{\MARGENDCHO}{1.0}
    % Fondo
    \pgfmathsetmacro{\HORZ}{0.30}
    \pgfmathsetmacro{\VERT}{0.46}
    \pgfmathsetmacro{\MOD}{sqrt(\HORZ^2 + \VERT^2)}
    \pgfmathsetmacro{\ANGSD}{atan(\VERT / \HORZ)}
    \pgfmathsetmacro{\ANGII}{\ANGSD + 180.0}             
    \begin{tikzpicture}[scale=1.0,
      baseline,
      circunferencia/.style={%
        fill=black!1.5,draw=black!25,rotate=-30,line width=0.1pt},
      background/.style={
        line width=\bgborderwidth,
        draw=\bgbordercolor,
        fill=\bgcolor,
      },      
      ]
      %%% Coordenadas
      \coordinate (origen) at (0,0);
      \coordinate (endx) at (0:\RADIO cm);
      \coordinate (endprod) at (-90:\RADIO cm);
      \coordinate (endxp) at (-40:\RADIO cm);
      
      %%% ---Dibujo---
      %%% Circunferencia
      \fill[circunferencia] (origen) circle[radius=\RADIO cm];
      
      %%% Ángulo girado
      \path (endx) -- (origen) -- (endxp)
      pic[draw=black!80,{Latex[length=4.5pt,width=3pt]}-,shorten
      <=-1pt, fill=orange!50,"$\scriptstyle\alpha$",angle
      radius=6mm,angle eccentricity=1.3,scale=1,bend left=70]
      {angle=endxp--origen--endx};
      
      %%% Ejes
      % x perp
      \draw[green!50!black,-{Latex[width'=0pt 0.6]},line width=0.9pt]
      (origen) -- (endx) node[right] {\small $\vvv{x}_{\scriptstyle\perp}$};
      % x prod
      \draw[green!50!black,-{Latex[width'=0pt 0.6]},line width=0.9pt]
      (origen) -- (endprod) node[below]
      {\small $\vvv{x}_{\scriptstyle\prodvec}$};
      
      %%% Vector x' perp
      \draw[red!80!black,-{Latex[width'=0pt 0.5]},line width=1.2pt]
      (origen) -- (endxp);
      % Texto
      \node[red!80!black,below right] at (endxp)
      {\small $\vvv{x}'_{\scriptstyle\perp}$};
      % Componente horizontal
      % \draw[red!90!black,-{Latex[width'=0pt 0.5]},line width=1.2pt]
      % (origen) --
      %%% Línea proyección sobre eje horizontal
      \draw[black!20,line width=0.1pt] ($(origen)!(endxp)!(endx)$)
      coordinate (linhoriz) -- (endxp);
      %%% Línea proyección sobre eje vertical
      \draw[black!20,line width=0.1pt] ($(origen)!(endxp)!(endprod)$)
      coordinate (linvert) -- (endxp);
      %%% Componente horizontal de x prime
      \draw[red!80!black,-{Latex[width'=0pt 0.5]},shorten >=-2pt,line
      width=1.2pt] (origen) -- (linhoriz);
      % Texto
      \path (origen) -- node[red!80!black,above]
      {\small $R\cos\alpha\,\xhat{u}_{\scriptstyle\perp}$} (linhoriz);

      %%% Componente verical de x prime
      \draw[red!80!black,-{Latex[width'=0pt 0.5]},shorten >=-2pt,line
      width=1.2pt] (origen) -- (linvert);
      % Texto
      \path (origen) -- node[red!80!black,rotate=90,above]
      {\small $R\sin\alpha\,\xhat{u}_{\scriptstyle\prodvec}$} (linvert);
      %%% Centro
      \fill[fill=black,draw=black] (origen) circle[radius=1pt];
      % Fondo amarillo
      \path (current bounding box.south west) ++(\ANGII:\MOD) coordinate (SW);
      \path (current bounding box.north east) ++(\ANGSD:\MOD) coordinate (NE);
      % \filldraw[draw=black, fill=red] (NE) circle[radius=1pt];
      \begin{scope}[on background layer]
        \draw[background] (SW) rectangle (NE);
      \end{scope}
    \end{tikzpicture}
    \caption{Circunferencia de giro y descomposición de
      $\vvv{x}'_{\scriptstyle\perp}$ en función de
      $\vvv{x}_{\scriptstyle\perp}$ y
      $\vvv{x}_{\scriptstyle\,\prodvec}$.}
    \label{fig:so3-descomposicion_xprod}
  \end{minipage}
\end{figure}

Nuestro objetivo es el vector $\vvv{x'}$ (figura \ref{fig:so3-vector_x_transformado}),
que es también una suma de dos vectores, uno paralelo al versor que define al eje de giro
y otro perpendicular
\begin{equation}
  \label{eq:so3-xprima_perpendicular}
  \vvv{x'} = \vvv{x'}_{\scriptstyle\,\parallel} +
  \vvv{x'}_{\scriptstyle\perp} = \vvv{x}_{\scriptstyle\,\parallel} +
  \vvv{x'}_{\scriptstyle\perp}
\end{equation}

Ayudándonos de la figura \ref{fig:so3-descomposicion_xprod}, descompondremos
$\vvv{x}'_{\scriptstyle\perp}$ en función de los vectores, perpendiculares entre sí,
$\vvv{x}_{\scriptstyle\perp}$ y $\vvv{x}_{\scriptstyle\prodvec} = \vvv{x}\times\xhat{n}$
\footnotetext{Obsérvese que
  $\vvv{x}_{\scriptstyle\prodvec} = \vvv{x}_{\scriptstyle\perp}\times\xhat{n}
  = \vvv{x}\times\xhat{n}$, pero nos interesa la igualdad que contiene expresamente
  a $\vvv{x}$.}
\begin{align*}
  \vvv{x'}_{\scriptstyle\perp}
  &=
    R\cos\alpha\,\xhat{u}_{\scriptstyle\perp}
    + R\sin\alpha\,\xhat{u}_{\scriptstyle\prodvec}
    =
    R\cos\alpha\,\frac{\vvv{x}_{\scriptstyle\perp}}{|\vvv{x}_{\scriptstyle\perp}|}
    + R\sin\alpha\,\frac{\vvv{x}\prodvec\xhat{n}}{|\vvv{x}\times\xhat{n}|}\\
    &=
    \cancelout{R}\cos\alpha\,\frac{\vvv{x}_{\scriptstyle\perp}}{\cancelout{R}}
      + \cancelout{R}\sin\alpha\,\frac{\vvv{x}\prodvec\xhat{n}}{\cancelout{R}}
    = \cos\alpha\,\vvv{x}_{\scriptstyle\,\perp} + \sin\alpha\,(\vvv{x}\times\xhat{n})
\end{align*}

Sustituimos el resultado anterior en \ref{eq:so3-xprima_perpendicular}
\[
  \vvv{x}'
  =
    \vvv{x}_{\scriptstyle\,\parallel}
    + \cos\alpha\,\vvv{x}_{\scriptstyle\,\perp} + \sin\alpha\, (\vvv{x}\times\xhat{n})
\]

Ahora utilizamos la ecuación~(\ref{eq:so3-x_perpendicular}) y desarrollamos la expresión
\begin{align*}
  \vvv{x}'
  &=
    \vvv{x}_{\scriptstyle\,\parallel}
    +\cos\alpha\,(\vvv{x} - \vvv{x}_{\scriptstyle\,\parallel})
    +\sin\alpha\,(\vvv{x}\prodvec\xhat{n})\\
  &=
    \vvv{x}_{\scriptstyle\,\parallel}
    +\cos\alpha\,\vvv{x}
    - \cos\alpha\,\vvv{x}_{\scriptstyle\,\parallel}
    +\sin\alpha\,(\vvv{x}\prodvec\xhat{n})\\
  &=
    (1-\cos\alpha)\,\vvv{x}_{\scriptstyle\,\parallel}
    +\cos\alpha\,\vvv{x}
    +\sin\alpha\,(\vvv{x}\prodvec\xhat{n})
\end{align*}

Por último, sustituimos la ecuación~(\ref{eq:so3-x_paralela}) y obtenemos la ecuación que
nos da las nuevas coordenadas del vector $\vvv{x}$ al girar los ejes de coordenadas un
ángulo $\alpha$, alrededor de un eje definido por el versor $\xhat{n}$
\begin{equation}
  \label{eq:so3-giro3dgeom}
  \vvv{x}'
  =
  (1-\cos\alpha)\,(\vvv{x}\cdot \xhat{n})\,\xhat{n}
  +\cos\alpha\,\vvv{x}
  +\sin\alpha\,(\vvv{x}\prodvec\xhat{n})
\end{equation}


\subsubsection{Expresión matricial de una rotación general alrededor de un eje}
Vamos a los vectores $\xhat{x}'$, $\xhat{n}$, $\vvv{x}$ y $\vvv{x}\times\xhat{n}$ en
función de sus componentes cartesianas
\begin{align}
  &\vvv{x}' = x'\xhat{i} + y'\xhat{j} + z'\xhat{k}\\
  &\xhat{n} = n_x\xhat{i} + n_y\xhat{j} + n_z\xhat{k}\\
  &\vvv{x} = x\xhat{i} + y\xhat{j} + z\xhat{k}\\
  &\vvv{x}\times\xhat{n}
    = (yn_z-zn_y)\,\xhat{i} + (zn_x-xn_z)\,\xhat{j} + (xn_y-yn_x)\,\xhat{k}
\end{align}

Sustituimos los vectores anteriores en la ecuación \eqref{eq:so3-giro3dgeom}
\begin{align*}
  \vvv{x}'
  &=
    (1-\cos\alpha)\,(\vvv{x}\cdot \xhat{n})\,
    \begin{pmatrix}
      n_x\\
      n_y\\
      n_z
    \end{pmatrix}
  +\cos\alpha\,
  \begin{pmatrix}
    x\\
    y\\
    z
  \end{pmatrix}
  +\sin\alpha\,
  \begin{pmatrix}
    yn_z-zn_y\\
    zn_x-xn_z\\
    xn_y-yn_x
  \end{pmatrix}    
\end{align*}
%}

Y expresamos matricialmente el resultado de la rotación
anterior~(\ref{eq:so3-giro3dgeom})
{\small
   \begin{equation}\label{eq:so3-giro3dmatricial}
     \begin{pmatrix}
       x' \\ y' \\ z'
     \end{pmatrix}
     =
     \mmm{R}(\xhat{n}, \alpha)
     \begin{pmatrix}
       x \\ y \\ z
     \end{pmatrix}
     =
     \begin{pmatrix}
       (1-\cos\alpha)(\vvv{x}\cdot\xhat{n})\,n_x
       + \cos\alpha \,x + \sin\alpha \,n_zy - \sin\alpha \,n_yz\\
       (1-\cos\alpha)(\vvv{x}\cdot\xhat{n})\,n_y + \cos\alpha \,y
       + \sin\alpha \,n_xz - \sin\alpha \,n_zx\\
       (1-\cos\alpha)(\vvv{x}\cdot\xhat{n})\,n_z + \cos\alpha \,z +
       \sin\alpha \,n_yx - \sin\alpha \,n_xy
     \end{pmatrix}
   \end{equation}
}

A partir de \ref{eq:so3-giro3dmatricial} deducimos la matriz de rotación general,
ecuación (\ref{eq:so3-matrizrotaciongeneral}).
Dejamos como ejercicio esta, desarrollando el término $\vvv{x}\cdot\xhat{n}$ mediante las
componentes cartesianas $xn_x + yn_y + zn_z$, sacando factor común los términos en $x$,
en $y$ y en $z$, comparándose con los términos del producto
$\mmm{R}(\xhat{n},\alpha)\vvv{x}$. El cálculo detallado se realiza en el apéndice.
{\small
  \begin{equation}\label{eq:so3-matrizrotaciongeneral}
    \mmm{R}(\xhat{n}, \alpha)
    =
    \begin{pmatrix}
      (1-\cos\alpha)n_x^2 + \cos\alpha & (1-\cos\alpha)n_yn_x
      +n_z\sin\alpha & (1-\cos\alpha)n_zn_x-n_y\sin\alpha\\[0.3ex]
      (1-\cos\alpha)n_xn_y-n_z\sin\alpha & (1-\cos\alpha)n_y^2
      +\cos\alpha & (1-\cos\alpha)n_zn_y+n_x\sin\alpha\\[0.3ex]
      (1-\cos\alpha)n_xn_z+n_Y\sin\alpha &
      (1-\cos\alpha)n_yn_z-n_x\sin\alpha &
      (1-\cos\alpha)n_z^2+\cos\alpha
    \end{pmatrix}
  \end{equation}
}

Si conociéramos el versor de giro en coordenadas esféricas, sus coordenadas cartesianas
rectangulares serían
\begin{align*}
  n_x &= \sin\theta \cos\phi\\
  n_y &= \sin\theta \sin\phi\\
  n_z &= \cos\theta
\end{align*}

\subsubsection{¿Qué es el grupo SO(3)?}
Es el conjunto de matrices $3\times 3$, ortogonales y singulares
\[
  SO(3) = \left\{\mmm{R}_{3\times 3} \,/\, \mmm{R}^\trasp \mmm{R} =
    \mmm{I}, \det\mmm{R} = 1\right\}
\]

El número $3$ viene de las tres direcciones del espacio.
Este grupo tiene tres parámetros independientes o grados de libertad (o bien tres
generadores independientes, como vimos anteriormente).
Por un lado, tenemos los cuatro parámetros, $\alpha$, $n_x$, $n_y$ y $n_z$, pero estos
últimos tienen que estar normalizados, por lo que hay una ecuación que los relaciona,
$n_x^2+n_y^2+n_z^2=1$; así quedan, $4-1=3$, tres parámetros independientes.
Los tres parámetros independientes podrían ser también, el ángulo girado, $\alpha$, y los
dos ángulos del vector de giro en coordenadas esféricas, $\theta$ y $\phi$. 

El grupo SO(4) sería el grupo de rotaciones en un espacio euclídeo de cuatro dimensiones,
pero tendría seis parámetros, que se pueden razonar hallando el número de generadores del
grupo, tal como hicimos anteriormente para SO(3), en la sección \ref{subsec:num_gen_so3}.






%%% Local Variables:
%%% mode: latex
%%% TeX-engine: luatex
%%% TeX-master: "../gruposlie.tex"
%%% End:
