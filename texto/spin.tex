% spin.tex
%
% Copyright (C) 2022--2025 José A. Navarro Ramón <janr.devel@gmail.com>
% Licencia del código GPLv2
% Licencia Creative Commons Recognition Non-Commercial Share-alike.
% (CC-BY-NC-SA)

\chapter{El spin}
En este capítulo distinguiremos un observable como el spin $S$, de su representación
mecanocuántica como operador hermítico $\xhat{S}$, que aunque lleva el acento
circunflejo, no representa un vector unitario, sino un operador hermítico.


\section{Definición}
El spin $S$ es un observable que \emph{vive} en nuestro mundo $\symbb{R}^{3}$ y tiene
por tanto tres componentes
\[
  \vvv{S} = (S_x, S_y, S_z)
\]

Como todo observable, el spin se corresponde con un operador hermítico $\xhat{S}$, ---matricial en este caso---, que se comporta como un vector bajo rotaciones; además es
un momento angular porque sus componentes cumplen las relaciones de conmutación
\[
  [\mmm{\hat{S}}_i, \mmm{\xhat{S}}_j] = i\hbar\epsilon_{ijk} \mmm{\xhat{S}}_k
\]

Este operador también se puede definir como la representación del momento angular en un
espacio de Hilbert $\symbb{C}^N$, donde $N$ es un entero positivo.
Cada componente $\mmm{\hat{S}}_i$ es una matriz cuadrada $N\times N$.
Nuestra misión es:
\begin{quote}
  ``Encontrar tres matrices $\mmm{\hat{S}}_x$, $\mmm{\hat{S}}_y$ y $\mmm{\hat{S}}_z$
  cuadradas $N\times N$, que se comporten como un vector bajo rotaciones, que cumplan las
  propiedades del momento angular y con las leyes de la mecánica cuántica (en particular
  que sean hermíticas).''
\end{quote}

Al final veríamos qué relación tiene todo esto con la realidad, sobre todo con las
partículas y en concreto con el electrón.


\section{\mathinhead{S^2}{s2ujns} es un escalar bajo rotaciones}
Normalmente, entendemos por escalares los números, tanto reales como complejos.
Vamos a extender el concepto a \emph{escalar bajo rotaciones}, que es un objeto que no
cambia cuando se somete a rotaciones.

Calculamos $S^2$, que es la norma o longitud del \emph{vector} $\vvv{S}$
\begin{equation}\label{eq:spn-S2}
  S^2 = S_x^2 + S_y^2 + S_z^2
\end{equation}

Recordemos que las rotaciones no modifican la longitud de los vectores, por lo que esta
magnitud es un \emph{escalar bajo rotaciones}.

\subsection{Ejemplo}
Pongamos un ejemplo para explicar el concepto. Imaginemos que tenemos un barión, por ejemplo la partícula delta neutra $\Delta^{\mathlarge\circ}$.
Su spin es $3/2$ aunque, por ahora, no sabemos lo que significa.
Lo demostraremos en su momento pero este spin implica que el espacio de Hilbert es de
dimensión cuatro, $\symbb{C}^{4}$, de donde deducimos que las tres componentes del spin
serán matrices $4\times 4$.

El estado cuántico del barión viene dado por una función de las coordenadas
$\ket{\phi (x,y,z)}$ y otra $\ket{\psi}$, que nos da el estado del spin.
Con esta función de onda se podría calcular la probabilidad de encontrar el barión en
una cierta región del espacio y con un spin determinado
\[
  \ket{\phi (x,y,z)}\, \otimes\,\ket{\psi}  
\]

La parte izquierda de la figura~\ref{fig:spn-bob-alice} representa un barión
$\Delta^\circ$ desde el punto de vista del observador Roberto.
Según éste, la máxima probabilidad de encontrarlo está en la posición $(1,0,0)$, mientras que su spin se encuentra alineado en el sentido positivo del eje $z$.

A su vez, en la parte derecha de la figura, Alicia observa la misma partícula desde su
sistema de referencia. Según ella, la máxima probabilidad de encontrarla está en la
posición $(0,-1,0)$ y su spin está en el sentido negativo del eje $x$.

Nótese que la transformación entre las dos observaciones es pasiva, esto es, no se ha
interaccionado con la partícula; en cierta medida se ha producido una cierta rotación
del espacio
\begin{figure}[ht]
\def\scl{0.8}
\def\longeje{3}
%
\tdplotsetmaincoords{60}{110}
% 
\pgfmathsetmacro{\rvec}{\longeje}
\pgfmathsetmacro{\thetavec}{90}
\pgfmathsetmacro{\phivec}{90}
%
%\pgfmathsetmacro{\xvec}{\xcoord{\rvec}{\thetavec}{\phivec}}
%\pgfmathsetmacro{\yvec}{\ycoord{1.0}{30}{50}}
%\pgfmathsetmacro{\zvec}{\zcoord{1.0}{30}{50}}
\pgfmathsetmacro{\xvec}{0}
\pgfmathsetmacro{\yvec}{\longeje}
\pgfmathsetmacro{\zvec}{0}
% Fondo
\pgfmathsetmacro{\HORZ}{0.25}
\pgfmathsetmacro{\VERT}{0.25}
%
\tikzfading[name=fade out, inner color=transparent!0, outer
color=transparent!100]
%
\centering
\begin{tikzpicture}[%
  scale=\scl,
  tdplot_main_coords,
  background/.style={
    line width=\bgborderwidth,
    draw=\bgbordercolor,
    fill=\bgcolor,
  },
  backgroundonly/.style={
    line width=\bgborderwidth,
    fill=\bgcolor,
  },
  ]
  \coordinate (dx) at (2.2,0,0);
  \coordinate (dy) at (0,2.2,0);
  \coordinate (dz) at (0,0,2.2);
  % Ejes de Roberto
  \draw[thick,->] (0,0,0) -- (\longeje,0,0)
  node[anchor=north east]{$x$} coordinate (x);
  \draw[thick,->] (0,0,0) -- (0,\longeje,0) node[anchor=north west]{$y$};
  \draw[thick,->] (0,0,0) -- (0,0,\longeje) node[anchor=south]{$z$};

  % Estado del spin
  \draw[-{Latex[round]},red] (1.5,0,0) -- (1.5,0,1.5)
  node[left] {\footnotesize $\vvv{S}$};
  % Nube de probabilidad (estado espacial)
  \fill [red,path fading=fade out] (1.5,0,0) circle [radius=10pt]
  node[below right=-1pt and 4pt] {$\Delta^\circ$};

  % Roberto
  % BUG: Hay un bug en tkzpeople cuando se utiliza 'businessman' que se
  % refleja en el archivo 'gruposlie.log'
  % con mensajes 'Missing character:'  '*1.25412pt'
  \node[name=b,dave,hair=brown!60!black,shirt=green!40!black,
  top color=green!70,bottom color=green!40!black,shading angle=45,
  tie=black,mirrored,minimum size=4mm] at (0,1.1,3) {};
    % Texto Nombre
  \path[tdplot_main_coords] (0,.2,2.5) --
  node[below,sloped,green!40!black] {\scriptsize ROBERTO} (0,2,2.5);
  % Bocadillo "Soy Roberto"
  %\node[ellipse callout, draw,xshift=.5cm,yshift= 2.75cm,
  %callout absolute pointer={(b.mouth)},font=\tiny] {Soy Roberto};

  \begin{scope}[xshift=7cm]
    \tdplotsetrotatedcoords{90}{90}{0}
    % Líneas débiles ejes
    \draw[ultra thin,black!20] (0,0,0) -- (\longeje,0,0);
    \draw[ultra thin,black!20] (0,0,0) -- (0,-\longeje,0);
    \draw[ultra thin,black!20] (0,0,0) -- (0,0,\longeje);
    % Ejes de Alicia 
    \draw[thick,color=black,tdplot_rotated_coords,->] (0,0,0) -- (\longeje,0,0)
    node[anchor=north]{$x’$};
    \draw[thick,color=black,tdplot_rotated_coords,->] (0,0,0) -- (0,\longeje,0)
    node[anchor=south west]{$y’$};
    \draw[thick,color=black,tdplot_rotated_coords,->] (0,0,0) -- (0,0,\longeje)
    node[anchor=west]{$z’$};
    % Estado del spin
    \draw[-{Latex[round]},red] (1.5,0,0) -- (1.5,0,1.5)
    node[left] {\footnotesize $\vvv{S}$};
    % Nube de probabilidad (estado espacial)
    \fill [red,path fading=fade out,tdplot_rotated_coords]
    (0,-1.5,0) circle [radius=10pt]
    node[below left=-1pt and 4pt] {$\Delta^\circ$};
    % Alicia
    \node[name=a,alice,minimum size=4mm,skin=pink,shirt=red,
    tdplot_rotated_coords,rotate=-110] at (-1.0,0,2.7) {};
    \node[red!50!black,tdplot_rotated_coords,rotate=-100]
    at (-1.0,0,2.0) {\scriptsize\textsf{ALICIA}};
    %\path[tdplot_rotated_coords] (-1.5,0,2.3) --
    %node[below,sloped,red!40!black,rotated=90] {\scriptsize ALICIA} (-.5,0,2.3);
    
    %\node[ellipse callout, draw,xshift=.5cm,yshift= 2.75cm,
    %callout absolute pointer={(b.mouth)},font=\tiny] {Soy Alicia};
  \end{scope}

  % Fondo amarillo
  \coordinate (SW) at ($(current bounding box.south west) + (-\HORZ cm,-\VERT cm)$);
  \coordinate (NE) at ($(current bounding box.north east) + (\HORZ cm,\VERT cm)$);    
  \begin{scope}[on background layer]
    \draw[background] (SW) rectangle (NE);
  \end{scope}

\end{tikzpicture}
\caption{En la figura de la izquierda Roberto observa el bosón $\Delta^\circ$,
  desde su sistema de referencia.
  A la derecha, Alicia lo observa desde otro sistema de referencia.}
\label{fig:spn-bob-alice}
\end{figure}

Pero la rotación pasiva es equivalente a su contrapartida activa, por ejemplo,
manteniendo todo el tiempo el punto de vista de Alicia y rotando la partícula
---interaccionando con ella---.
Las rotaciones que se deben realizar sobre la partícula se representan en la
figura~\ref{fig:spn-alice}
\begin{figure}[ht]
\def\scl{0.69}
\def\longeje{3}
%
\tdplotsetmaincoords{60}{110}
% 
\pgfmathsetmacro{\rvec}{\longeje}
\pgfmathsetmacro{\thetavec}{90}
\pgfmathsetmacro{\phivec}{90}
%
%\pgfmathsetmacro{\xvec}{\xcoord{\rvec}{\thetavec}{\phivec}}
%\pgfmathsetmacro{\yvec}{\ycoord{1.0}{30}{50}}
%\pgfmathsetmacro{\zvec}{\zcoord{1.0}{30}{50}}
\pgfmathsetmacro{\xvec}{0}
\pgfmathsetmacro{\yvec}{\longeje}
\pgfmathsetmacro{\zvec}{0}
% Fondo
\pgfmathsetmacro{\HORZ}{0.25}
\pgfmathsetmacro{\VERT}{0.25}
%
\tikzfading[name=fade out, inner color=transparent!0, outer
color=transparent!100]

\centering
\begin{tikzpicture}[
  scale=\scl,
  tdplot_main_coords,
  background/.style={
    line width=\bgborderwidth,
    draw=\bgbordercolor,
    fill=\bgcolor,
  },
  backgroundonly/.style={
    line width=\bgborderwidth,
    fill=\bgcolor,
  },  
  ]
  \coordinate (dx) at (2.2,0,0);
  \coordinate (dy) at (0,2.2,0);
  \coordinate (dz) at (0,0,2.2);
  % Líneas débiles ejes
  \draw[ultra thin,black!20] (0,0,0) -- (-\longeje,0,0);
  \draw[ultra thin,black!20] (0,0,0) -- (0,-\longeje +1,0);
  \draw[ultra thin,black!20] (0,0,0) -- (0,0,-2*\longeje /3);
  % Ejes de Alicia
  \draw[thick,->] (0,0,0) -- (\longeje,0,0) node[anchor=north]{$x$};
  \draw[thick,->] (0,0,0) -- (0,\longeje,0) node[anchor=north]{$y$};
  \draw[thick,->] (0,0,0) -- (0,0,\longeje) node[anchor=south]{$z$};

  % Giro eje z
  %BUG: 'Missing character: There is no ;' warning in 'gruposlie.log'
  \tdplotdrawarc[line width=1pt,-{Latex[round,length=6pt,width=4pt]},
  color=black!50] {(0,0,1.9)}{0.3}{-25}{-360}{}{};
  % Reponer parte superior del eje z
%  \draw[ultra thick,->] (0,0,1.9) -- (0,0,\longeje);

  % Estado del spin
  \draw[-{Latex[round]},red] (1.5,0,0) -- (1.5,0,1.5)
  node[left] {\footnotesize $\vvv{S}$};
  % Nube de probabilidad (estado espacial)
  \fill [red,path fading=fade out] (1.5,0,0) circle [radius=10pt]
  node[above left=2pt and 2pt] {$\scriptstyle\Delta^\circ$};

  % Alicia
  \node[alice,minimum size=1.5mm,skin=pink,shirt=red,mirrored] at (0,1.1,3) {};
  % Texto Nombre
  \path[tdplot_main_coords] (0,.2,2.5) --
  node[below,sloped,red!50!black] {\tiny ALICIA} (0,2,2.5);
  % Bocadillo "Soy Alicia"
  % \node[ellipse callout, draw,xshift=.5cm,yshift= 2.75cm,
  % callout absolute pointer={(b.mouth)},font=\tiny] {Soy Alicia};

  % Draw the arcs on each theta plane
  % The first position is obvious since we are in the x-y plane and rotating
  % around the z-axis.
  % The anchor already went crazy, north is pointing downwards...
  %\tdplotdrawarc[-{Latex[round]},color=black!50]
  %{(0,0,1.9)}{0.3}{0}{-350}{anchor=south west,color=black}{}
  % We move to the z-x axis
  %\tdplotsetthetaplanecoords{0}
  % Notice you have to tell tiks-3dplot you are now in rotated coords
  % Since tikz-3dplot swaps the planes in tdplotsetthetaplanecoords,
  % the former y axis is now the z axis.
  %\tdplotdrawarc[tdplot_rotated_coords,-{Latex[round]},color=red!80!black]
  %{(0,0,1.9)}{0.3}{340}{0}{anchor=south west,color=black}{}
  %%%% \tdplotsetthetaplanecoords{-90}
  % Once again we swaps the planes. I don't know why it's working like this
  % but we turn backwards so the arrow turns in the positive direction.
  %%%% \tdplotdrawarc[tdplot_rotated_coords,->,color=black]
  % {(0,0,0.7)}{0.1}{120}{470}{anchor=south west,color=black}{roll}
  % If you turn the theta plane of 90 degrees position and rotation are inverted.
  % \tdplotsetthetaplanecoords{90}
  % \tdplotdrawarc[tdplot_rotated_coords,->,color=black]
  % {(0,0,-0.7)}{0.1}{470}{120}{anchor=south east,color=black}{roll}

  % Giro eje z
  \tdplotdrawarc[-{Latex[round,length=5pt,width=3pt]},color=black!50]
  {(0,0,1.9)}{0.3}{-25}{-365}{anchor=south west,color=black}{};
  % Reponer parte superior del eje z
  \draw[thick,->] (0,0,1.9) -- (0,0,\longeje);

  \begin{scope}[xshift=6cm]
    % Líneas débiles ejes
    \draw[ultra thin,black!20] (0,0,0) -- (-\longeje,0,0);
    \draw[ultra thin,black!20] (0,0,0) -- (0,-\longeje +.5,0);
    \draw[ultra thin,black!20] (0,0,0) -- (0,0,-2*\longeje /3);
    % Ejes de Alicia
    \draw[thick,->] (0,0,0) -- (\longeje,0,0) node[anchor=north]{$x$};
    \draw[thick,->] (0,0,0) -- (0,\longeje,0) node[anchor=north]{$y$};
    \draw[thick,->] (0,0,0) -- (0,0,\longeje) node[anchor=south]{$z$};

    % Giro eje y
    \tdplotsetthetaplanecoords{0}
    % Notice you have to tell tiks-3dplot you are now in rotated coords
    % Since tikz-3dplot swaps the planes in tdplotsetthetaplanecoords,
    % the former y axis is now the z axis.
    \tdplotdrawarc[tdplot_rotated_coords,line width=1pt,
    -{Latex[round,length=6pt,width=4pt]},color=black!50]
    {(0,0,1.9)}{0.3}{380}{40}{anchor=south west,color=black}{}
    % Reponer parte derecha del eje y
    \draw[thick,->] (0,1.9,0) -- (0,\longeje,0);

    % Estado del spin
    \draw[-{Latex[round]},red] (0,-1.5,0) -- (0,-1.5,1.5) node[left]
    {\footnotesize $\vvv{S}$};
    % Nube de probabilidad (estado espacial)
    \fill [red,path fading=fade out] (0,-1.5,0) circle [radius=10pt]
    node[below left=-1pt and 3pt] {$\scriptstyle\Delta^\circ$};

    % Alicia
    \node[alice,minimum size=1.5mm,skin=pink,shirt=red,mirrored] at (0,1.1,3) {};
    % Texto Nombre
    \path[tdplot_main_coords] (0,.2,2.5) --
    node[below,sloped,red!50!black] {\tiny ALICIA} (0,2,2.5);
    % Bocadillo "Soy Alicia"
    % \node[ellipse callout, draw,xshift=.5cm,yshift= 2.75cm,
    % callout absolute pointer={(b.mouth)},font=\tiny] {Soy Alicia};
  \end{scope}

  \begin{scope}[xshift=12cm]
    % Líneas débiles ejes
    \draw[ultra thin,black!20] (0,0,0) -- (-\longeje,0,0);
    \draw[ultra thin,black!20] (0,0,0) -- (0,-\longeje +.5,0);
    \draw[ultra thin,black!20] (0,0,0) -- (0,0,-2*\longeje /3);
    % Ejes de Alicia
    \draw[thick,->] (0,0,0) -- (\longeje,0,0) node[anchor=north]{$x$};
    \draw[thick,->] (0,0,0) -- (0,\longeje,0) node[anchor=north]{$y$};
    \draw[thick,->] (0,0,0) -- (0,0,\longeje) node[anchor=south]{$z$};

    % Estado del spin
    \draw[-{Latex[round]},red] (0,-1.5,0) -- (-1.5,-1.5,0) node[left]
    {\footnotesize $\vvv{S}$};
    % Nube de probabilidad (estado espacial)
    \fill [red,path fading=fade out] (0,-1.5,0) circle [radius=10pt]
    node[below left=-1pt and 3pt] {$\scriptstyle\Delta^\circ$};

    % Alicia
    \node[alice,minimum size=1.5mm,skin=pink,shirt=red,mirrored] at (0,1.1,3) {};
    % Texto Nombre
    \path[tdplot_main_coords] (0,.2,2.5) --
    node[below,sloped,red!50!black] {\tiny ALICIA} (0,2,2.5);
    % Bocadillo "Soy Alicia"
    % \node[ellipse callout, draw,xshift=.5cm,yshift= 2.75cm,
    % callout absolute pointer={(b.mouth)},font=\tiny] {Soy Alicia};
  \end{scope}

  % Fondo amarillo
  \coordinate (SW) at ($(current bounding box.south west) + (-\HORZ cm,-\VERT cm)$);
  \coordinate (NE) at ($(current bounding box.north east) + (\HORZ cm,\VERT cm)$);    
  \begin{scope}[on background layer]
    \draw[background] (SW) rectangle (NE);
  \end{scope}

%  \begin{scope}[on background layer]
%    \filldraw[background]
%    rectangle (-2.1cm, -2.2cm) rectangle (3.4cm + 12cm, 3.4cm);     
%    \fill[backgroundonly]
%    rectangle (-2.1cm, -2.2cm) rectangle (3.4cm + 12cm, 3.4cm);     
%  \end{scope}
  
\end{tikzpicture}
\caption{A la izquierda de la figura el bosón aparece tal y como lo veía Roberto. Lo
  rotamos \ang{-90} alrededor del eje $z$, después de lo cual queda como se aprecia en
  el centro de la figura. Finalmente, lo rotamos otros \ang{-90} en torno al eje $y$ y
  queda como se observa a la derecha, tal y como lo veía Alicia. Además, obsérvese que
  la longitud $|\vvv{S}|$ es invariante bajo rotaciones.}
\label{fig:spn-alice}
\end{figure}

Desde el punto de vista de Alicia, para pasar el bosón desde lo observado por Roberto
hasta el suyo, debe realizar primero una rotación de \ang{-90} respecto del eje $z$ y
luego, otra rotación de \ang{-90} alrededor el eje $y$.

Rotar implica utilizar el momento angular correspondiente al mundo de que se trate. Así,
para rotar la función de estado $\phi(x,y,z) \otimes \ket{\psi}$, utilizaremos
$\xhat{L}$ para la función $\phi$ y $\xhat{S}$ para girar el vector de estado de spin
$\ket{\psi}$.

La rotación activa representada en la figura~\ref{fig:spn-alice} se efectúa aplicando
los operadores
\[
  \left(
    e^{-i\frac{-\pi/2}{\hbar}\,\xhat{L}_y}\,e^{-i\frac{-\pi/2}{\hbar}\,\xhat{L}_z}\,\phi
  \right)
  \otimes
  \left(
    e^{-i\frac{-\pi/2}{\hbar}\,\xhat{S}_y}\,e^{-i\frac{-\pi/2}{\hbar}\,\xhat{S}_z}\,\ket{\psi}
  \right)
\]

De la ecuación anterior sólo nos interesa aquí la parte del spin.
Observamos que $S^2$ es un escalar porque es invariante bajo rotaciones.
Por ejemplo, al rotarlo alrededor del eje $x$
\begin{equation}\label{eq:spn-S2-invariante}
  e^{-i\frac{\theta}{\hbar}\,\xhat{S}_x}\,S^2\,e^{i\frac{\theta}{\hbar}\,\xhat{S}_x}
  = S^2\,e^{-i\frac{\theta}{\hbar}\,\xhat{S}_x}\,e^{i\frac{\theta}{\hbar}\,\xhat{S}_x}
  = S^2\, e^0
  = S^2
\end{equation}

Para que esto suceda, $S^2$ debe ser independiente de rotaciones alrededor del eje $x$,
o lo que es lo mismo, $\xhat{S}^2$ y $\xhat{S}_x$ deben conmutar. Además, ocurrirá lo
mismo con los demás ejes
\[
  [\xhat{S}^2,\xhat{S}_x]
  = [\xhat{S}^2,\xhat{S}_y]
  = [\xhat{S}^2,\xhat{S}_z]
  = 0
\]

Podemos demostrar la afirmación anterior mediante la fórmula de Hadamard
\[
  e^{-\theta\mmm{A}}\,\mmm{B}\,e^{\theta\mmm{A}}
  =
  \mmm{B} + [\mmm{B},\mmm{A}]\theta + \cdots
\]

que aplicada a nuestra transformación alrededor del eje $x$ por ejemplo
\[
  e^{-i\frac{\theta}{\hbar}\mmm{S}_x}\,\xhat{S}^2\,e^{i\frac{\theta}{\hbar}\mmm{S}_x}
  =
  \xhat{S}^2 + \left[\xhat{S}^2,i\frac{1}{\hbar}\mmm{S}_x\right]\theta
  + \cdots
\]

Para que se cumpla~(\ref{eq:spn-S2-invariante}), debe ocurrir que
$[\xhat{S}^2,\mmm{S}_x] = 0$.

\section{Representación matricial de un operador}
Para simplificar la explicación nos centraremos en un espacio de Hilbert $\symbb{C}^2$,
aunque las conclusiones valdrán para cualquier dimensión.

Sea $\xhat{A}$ un operador mecanocuántico de un espacio de Hilbert, y sea
$B = \set{\ket{v_1},\ket{v_2}}$ una base ortonormal que no tiene por qué ser la base
propia de $\xhat{A}$.
Afirmamos que la siguiente matriz es la representación del operador $\xhat{A}$ en la
base considerada
\begin{equation}\label{eq:spn-representacion-matricial-operador}
  \xhat{A}
  =
  \begin{pmatrix}
    \braket{v_1|\xhat{A}|v_1} & \braket{v_1|\xhat{A}|v_2} \\
    \braket{v_2|\xhat{A}|v_1} & \braket{v_2|\xhat{A}|v_2}
  \end{pmatrix}
\end{equation}

Supongamos que al operar $\xhat{A}$ sobre los vectores de la base, obtenemos los
siguientes resultados
\begin{align*}
  \xhat{A}\ket{v_1} &= 3\ket{v_1} + 2\ket{v_2}\\
  \xhat{A}\ket{v_2} &= 5\ket{v_1} - \ket{v_2}  
\end{align*}

Realizamos los cálculos pertinentes, teniendo en cuenta que la base es ortonormal, por
lo que se cumple $\braket{v_i|v_j} = \delta_{ij}$
\begin{align*}
  \braket{v_1|\xhat{A}|v_1}
  &=
    3\braket{v_1|v_1} + 2\braket{v_1|v_2} = 3\\
  \braket{v_1|\xhat{A}|v_2}
  &=
    5\braket{v_1|v_1} - \braket{v_1|v_2} = 5\\
  \braket{v_2|\xhat{A}|v_1}
  &=
    3\braket{v_2|v_1} + 2\braket{v_2|v_1} = 2\\
  \braket{v_2|\xhat{A}|v_2}
  &=
    5\braket{v_2|v_1} - \braket{v_2|v_2} = -1\\
\end{align*}

La representación del operador $\xhat{A}$ en la base $B = \set{\ket{v_1}, \ket{v_2}}$ es,
según~(\ref{eq:spn-representacion-matricial-operador})
\[
  \xhat{A}
  =
  \begin{pmatrix}
    3 & 5\\
    2 & -1
  \end{pmatrix}
\]

En el resultado anterior se aprecia que la representación matricial de un operador está
formada por las componentes de $\xhat{A} \ket{v_i}$ dispuestas en la columna $i$. 


\section{Representación de
  \mathinhead{\xhat{S}_z}{matrsz} y
  \mathinhead{\xhat{S}^2}{matrs2} en su base propia}
Aclaramos que se podría haber elegido cualquiera de los operadores
$\xhat{S}_x$, $\xhat{S}_y$ o $\xhat{S}_z$, pero escogemos el último porque tiene una
representación matemática más sencilla.

La ecuación de autovalores y autovectores de $\xhat{S}_z$ es
\[
  \xhat{S}_z \ket{m} = \hbar m \ket{m}
\]
donde $\ket{m}$ es una función propia genérica, el operador tiene $N$ funciones propias
y $\hbar m$ es su valor propio. La constante $\hbar$ aparece porque los valores propios
deben tener unidades de momento angular\footnotemark{}.
\footnotetext{Se podría haber supuesto un valor propio $m$ y en el desarrollo posterior
  se hubiera deducido que $m$ contiene $\hbar$.}

Dado que los operadores $\xhat{S}_z$ y $\xhat{S}^2$ conmutan, hay un teorema que dice
que existe una base propia común de ambos operadores
\[
  \xhat{S}^2 \ket{m} = \hbar^2 c \ket{m}
\]
donde $c$ es un número no negativo porque $\xhat{S}^2$ es el cuadrado de la longitud o
norma del operador y $\hbar^2$ lo ponemos porque sus unidades son el cuadrado del
momento angular.
Además, como $\xhat{S}^2$ es invariante frente a rotaciones alrededor de los ejes de
coordenadas, debe ser proporcional a la matriz unidad ---$\xhat{S}^2$ es un invariante
Casimir---
\begin{equation}\label{eq:spn-S2casimir}
  \xhat{S}^2 = \hbar^2 c\,\mmm{I}
\end{equation}

Para escribir la representación matricial de $\xhat{S}_z$ realizamos el cálculo
\begin{equation}\label{eq:spn-elem-matriz-Sz-delta}
  \braket{m_i|\xhat{S}_z|m_j}
  =
  \braket{m_i|\hbar m_j|m_j}
  =
  \hbar m_j \braket{m_i|m_j}
  =
  \hbar m_j \,\delta_{ij}
\end{equation}
donde hemos tenido en cuenta que la base propia es ortonormal.

Los elementos no diagonales serán cero y los diagonales tendrán la forma
\[
  \xhat{S}_{z\, ii} = \hbar m_i
\]

La matriz será
\[
  \xhat{S}_z
  = \hbar\,
  \begin{pNiceMatrix}
    m_1   & 0      & \Cdots  & 0      \\
    0      & \Ddots &  \Ddots & \Vdots \\
    \Vdots & \Ddots &         & 0      \\
    0      & \Cdots & 0       & m_{N}
  \end{pNiceMatrix}
\]
donde las $m_i$ están ordenadas de mayor a menor
\[
  m_1 \ge m_2 \ge \cdots \ge m_{N}
\]

Por otra parte $\xhat{S}^2$, según~(\ref{eq:spn-S2casimir})
\[
  \xhat{S}^2
  = \hbar^2 c\,
  \begin{pNiceMatrix}
    1      & 0      & \Cdots  & 0      \\
    0      & \Ddots &  \Ddots & \Vdots \\
    \Vdots & \Ddots &         & 0      \\
    0      & \Cdots & 0       & 1    
  \end{pNiceMatrix}
\]

\section{Operadores escalera del momento angular de spin}
Por el momento no conocemos los valores propios ni las funciones propias del momento
angular. Utilizaremos una técnica llamada de los \emph{operadores escalera} para
calcularlos.

Construiremos unos operadores, $\xhat{S}_{+}$ y $\xhat{S}_{-}$, que aplicados sobre una
función propia del momento angular, consigan elevar o disminuir su valor propio.
Al aplicar sucesivamente estos operadores escalera a estas funciones propias, llegaremos
a un límite inferior y a otro superior.
De esta forma demostraremos que los valores propios están acotados tanto superior como
inferiormente y esto nos dará la clave para hallarlos.

De la ecuación~(\ref{eq:spn-S2}) obtenemos
\begin{equation}\label{eq:spn-Sx2Sy2}
  \xhat{S}_x^2 + \xhat{S}_y^2
  =
  \xhat{S}^2 - \xhat{S}_z^2
\end{equation}

\subsection{Definición y propiedades útiles}
\subsubsection{Definición de operadores escalera}
Definimos los operadores escalera ascendente y descendente
\begin{align}\label{eq:spn-S-}
  \xhat{S}_{-} &= \xhat{S}_x - i\xhat{S}_y\\
  \label{eq:spn-S+}
  \xhat{S}_{+} &= \xhat{S}_x + i\xhat{S}_y
\end{align}

\subsubsection{Cálculo de \mathinhead{\xhat{S}_{+}\xhat{S}_{-}}{s+s-}
  y \mathinhead{\xhat{S}_{-}\xhat{S}_{+}}{s-s+}}
Calculamos $\xhat{S}_{+} \xhat{S}_{-}$ porque nos será útil posteriormente
\begin{align*}
  \xhat{S}_{+} \xhat{S}_{-}
  &=
  (\xhat{S}_x + i\xhat{S}_y)
  (\xhat{S}_x - i\xhat{S}_y)
  =
  \xhat{S}_x^2 + \xhat{S}_y^2
  -i\xhat{S}_x\xhat{S}_y +i\xhat{S}_y\xhat{S}_x\\
  &=
  \xhat{S}_x^2 + \xhat{S}_y^2 + i[\xhat{S}_y,\xhat{S}_x]
    = \xhat{S}_x^2 + \xhat{S}_y^2 + i (-i\hbar\xhat{S}_z)
    = \xhat{S}_x^2 + \xhat{S}_y^2 + \hbar\xhat{S}_z
\end{align*}

Utilizando la ecuación~(\ref{eq:spn-Sx2Sy2})
\begin{equation}\label{eq:spn-S+S-}
  \xhat{S}_{+} \xhat{S}_{-}
  =
  \xhat{S}^2 - \xhat{S}_z^2 + \hbar\xhat{S}_z
\end{equation}

De forma similar obtendríamos
\begin{equation}\label{eq:spn-S-S+}
  \xhat{S}_{-} \xhat{S}_{+}
  =
  \xhat{S}^2 - \xhat{S}_z^2 - \hbar\xhat{S}_z
\end{equation}

\subsubsection{Norma de \mathinhead{\xhat{S}_{-}\ket{m}}{s-m}
  y \mathinhead{\xhat{S}_{+}\ket{m}}{s+m}}
Las funciones propias $\ket{m}$ están normalizadas, pero la normalización se pierde
cuando los operadores escalera se aplican a éstas.
Necesitamos pues, hallar la norma de $\xhat{S}_{-} \ket{m}$ y de $\xhat{S}_{+}\ket{m}$,
pues ya no valen la unidad%, siendo $\ket{m}$ una función propia genérica
\begin{align*}
  \Vert\xhat{S}_{-}\ket{m}\Vert
  &=
    \sqrt{(\xhat{S}_{-}\ket{m})^\dagger (\xhat{S}_{-} \ket{m})}
    = \sqrt{\braket{m|(\xhat{S}_{-})^\dagger\xhat{S}_{-}|m}}
    = \sqrt{\braket{m|\xhat{S}_{+}\xhat{S}_{-}|m}}\\
  &=
    \sqrt{\braket{m|\xhat{S}^2-\xhat{S}_z^2+\hbar\xhat{S}_z|m}}
    = \sqrt{\hbar^2c -\hbar^2m^2+\hbar^2m}
    = \hbar \sqrt{c-m^2+m}
\end{align*}
\begin{align*}
  \Vert\xhat{S}_{+}\ket{m}\Vert
  &=
    \sqrt{(\xhat{S}_{+}\ket{m})^\dagger (\xhat{S}_{+} \ket{m})}
    = \sqrt{\braket{m|(\xhat{S}_{+})^\dagger\xhat{S}_{+}|m}}
    = \sqrt{\braket{m|\xhat{S}_{-}\xhat{S}_{+}|m}}\\
  &=
    \sqrt{\braket{m|\xhat{S}^2-\xhat{S}_z^2-\hbar\xhat{S}_z|m}}
    = \sqrt{\hbar^2c -\hbar^2m^2-\hbar^2m}
    = \hbar \sqrt{c-m^2-m}
\end{align*}

Resumimos los resultados anteriores
\begin{align}\label{eq:spn-norma-S-z}
  \Vert\xhat{S}_{-}\ket{m}\Vert &= \hbar \sqrt{c-m^2+m}\\
  \label{eq:spn-norma-S+z}
  \Vert\xhat{S}_{+}\ket{m}\Vert &= \hbar \sqrt{c-m^2-m}  
\end{align}

\subsubsection{Cálculo de \mathinhead{\xhat{S}_{-}\xhat{S}_z}{s-sz}
  y \mathinhead{\xhat{S}_{+}\xhat{S}_z}{s+sz}}
Primero hallamos los conmutadores de $\xhat{S}_z$ con los operadores escalera
\begin{align*}
  [\xhat{S}_z,\xhat{S}_{-}]
  &=
  [\xhat{S}_z,\xhat{S}_x-i\xhat{S}_y]
  =
  [\xhat{S}_z,\xhat{S}_x] -i [\xhat{S}_z,\xhat{S}_y]
  =
  i\hbar\xhat{S}_y - i (-i\hbar\xhat{S}_x)\\
  &=
    -\hbar\xhat{S}_x + i\hbar\xhat{S}_y
    = -\hbar (\xhat{S}_x - i\xhat{S}_y)
    = -\hbar\xhat{S}_{-}
\end{align*}
\begin{align*}
  [\xhat{S}_z,\xhat{S}_{+}]
  &=
  [\xhat{S}_z,\xhat{S}_x+i\xhat{S}_y]
  =
  [\xhat{S}_z,\xhat{S}_x] +i [\xhat{S}_z,\xhat{S}_y]
  =
  i\hbar\xhat{S}_y + i (-i\hbar\xhat{S}_x)\\
  &=
    -\hbar\xhat{S}_x + i\hbar\xhat{S}_y
    = \hbar (\xhat{S}_x + i\xhat{S}_y)
    = \hbar\xhat{S}_{+}
\end{align*}

Y después, utilizamos los cálculos anteriores para calcular
$\xhat{S}_{-}\xhat{S}_z$ y $\xhat{S}_{+}\xhat{S}_z$
\begin{align}\label{eq:spn-S-z}
  \xhat{S}_{-}\xhat{S}_z
  &= \xhat{S}_z\xhat{S}_{-} + \hbar\xhat{S}_{-}\\
  \label{eq:spn-S+z}
  \xhat{S}_{+}\xhat{S}_z
  &= \xhat{S}_z\xhat{S}_{+} - \hbar\xhat{S}_{+}  
\end{align}

\subsection{Aplicación de los operadores escalera sobre las funciones \mathinhead{\ket{m}}{aplescm}}
Comenzamos aplicando los operadores escalera a la ecuación de autovalores y autovectores
de $\xhat{S}_z$
\[
  \xhat{S}_z\ket{m} = \hbar m\ket{m}
\]

\subsubsection{Aplicación de \mathinhead{\xhat{S}_{-}}{apls-} sobre
  \mathinhead{\ket{m}}{apls-m}}
Primero aplicamos el operador descendente
\[
  \xhat{S}_{-}\xhat{S}_z\ket{m} = \xhat{S}_{-}\hbar m\ket{m}
\]

Utilizamos el resultado de $\xhat{S}_{-}\xhat{S}_z$ en~(\ref{eq:spn-S-z})
\[
  (\xhat{S}_z\xhat{S}_{-}+\hbar\xhat{S}_{-})\ket{m}
  =
  \hbar m\xhat{S}_{-}\ket{m}
\]

Desarrollamos la expresión anterior
\[
  \xhat{S}_z\xhat{S}_{-}\ket{m} + \hbar\xhat{S}_{-}\ket{m}
  =
  \hbar m\xhat{S}_{-}\ket{m}
\]
\[
  \xhat{S}_z\xhat{S}_{-}\ket{m}
  =
  \hbar m\xhat{S}_{-}\ket{m}-\hbar\xhat{S}_{-}\ket{m}
\]
\begin{equation}\label{eq:spn-S-1}
  \xhat{S}_z\xhat{S}_{-}\ket{m}
  =
  \hbar (m-1)\xhat{S}_{-}\ket{m}
\end{equation}

Llegamos a la conclusión de que $\xhat{S}_{-}\ket{m}$ es una función propia de
$\xhat{S}_z$, bajando el valor propio en $\hbar$ unidades.
Por esta razón $\xhat{S}_{-}$ se denomina operador \emph{descendente}.

\subsubsection{Aplicación de \mathinhead{\xhat{S}_{+}}{apls+} sobre
  \mathinhead{\ket{m}}{apls+m}}
Después aplicamos el operador ascendente
$\xhat{S}_{+}$ a $\xhat{S}_z\ket{m} = \hbar m\ket{m}$.
Utilizando el resultado de $\xhat{S}_{+}\xhat{S}_z$ en~(\ref{eq:spn-S+z}), llegaríamos a
obtener
\begin{equation}\label{eq:spn-S+1}
  \xhat{S}_z\xhat{S}_{+}\ket{m}
  =
  \hbar (m+1)\xhat{S}_{+}\ket{m}
\end{equation}
de donde concluimos que $\xhat{S}_{+}\ket{m}$ es otra función propia de $\xhat{S}_z$,
que eleva el valor propio en $\hbar$ unidades; de ahí que $\xhat{S}_{+}$ se llame
operador \emph{ascendente}.

\subsubsection{¿Qué son realmente
  \mathinhead{\xhat{S}_{-}\ket{z}}{apls-z} y
  \mathinhead{\xhat{S}_{+}\ket{z}}{apls+z}?}
Ya sabíamos que eran funciones propias de $\xhat{S}_z$ con valores propios
$\hbar\,(m-1)$ y $\hbar\,(m+1)$, respectivamente.
\begin{align}\label{eq:spn-S-1-v2}
  \xhat{S}_z\left(\xhat{S}_{-}\ket{m}\right)
  &=
    \hbar (m-1)\left(\xhat{S}_{-}\ket{m}\right)\\
  \label{eq:spn-S+1-v2}
  \xhat{S}_z\left(\xhat{S}_{+}\ket{m}\right)
  &=
    \hbar (m+1)\left(\xhat{S}_{+}\ket{m}\right)
\end{align}

Ahora vamos a profundizar en su naturaleza.
Sabemos que $\ket{m-1}$ y $\ket{m+1}$ son funciones propias de $\xhat{S}_z$ con los
mismos valores propios
\begin{align}\label{eq:spn-Szm-1}
  \xhat{S}_z \ket{m-1}
  &=
    \hbar (m-1) \ket{m-1}\\
  \label{eq:spn-Szm+1}
  \xhat{S}_z \ket{m+1}
  &=
    \hbar (m+1) \ket{m+1}
\end{align}

Pero no son iguales
\begin{align*}
  \xhat{S}_{-}\ket{m} &\ne \ket{m-1}\\
  \xhat{S}_{+}\ket{m} &\ne \ket{m+1}
\end{align*}
porque al aplicar un operador escalera sobre una función propia del momento angular se
pierde la normalización. Recordemos que la base formada por las $\ket{m}$ es
ortononormal, de manera que la norma de cada vector de la base vale la unidad. Pues
bien, la norma de $\xhat{S}_{-}\ket{m}$ o $\xhat{S}_{+}\ket{m}$ no vale uno.

Así, las funciones son proporcionales y la constante de proporcionalidad es la norma de
$\xhat{S}_{-}\ket{m}$ o $\xhat{S}_{+}\ket{m}$
\begin{align}\label{eq:spn-proporcionalidad1}
  \xhat{S}_{-}\ket{m} &= \Vert\xhat{S}_{-}\ket{m}\Vert\, \ket{m-1}\\
  \label{eq:spn-proporcionalidad2}
  \xhat{S}_{+}\ket{m} &= \Vert\xhat{S}_{+}\ket{m}\Vert\, \ket{m+1}
\end{align}

La norma de las funciones se calcularon en~(\ref{eq:spn-norma-S-z})
y en~(\ref{eq:spn-norma-S+z})
\begin{align}\label{eq:spn-S-m-1}
  \xhat{S}_{-}\ket{m}
  &=
    \hbar\sqrt{c-m^2+m}\,\,\ket{m-1}\\
    \label{eq:spn-S+m+1}
  \xhat{S}_{+}\ket{m}
  &=
    \hbar\sqrt{c-m^2-m}\,\,\ket{m+1}
\end{align}

\subsection{Representación de \mathinhead{\xhat{S}_x}{sx}
  y \mathinhead{\xhat{S}_y}{sy}}
Ahora podremos calcular $\xhat{S}_x\ket{m}$ y $\xhat{S}_y\ket{m}$
reescribiendo~(\ref{eq:spn-S-m-1})
y~(\ref{eq:spn-S+m+1})
\begin{align}\label{eq:spn-S-m-1-v2}
  \xhat{S}_x\ket{m} - i\xhat{S}_y\ket{m}
  &=
    \hbar\sqrt{c-m^2+m}\,\,\ket{m-1}\\
  \label{eq:spn-S+m+1-v2}
  \xhat{S}_x\ket{m} + i\xhat{S}_y\ket{m}
  &=
    \hbar\sqrt{c-m^2-m}\,\,\ket{m+1}
\end{align}
de manera que tenemos dos ecuaciones con dos incógnitas $\xhat{S}_x$ y $\xhat{S}_y$.

Sumando y restando las ecuaciones, obtenemos las incógnitas
\begin{align}\label{eq:spn-Sxm}
  \xhat{S}_x\ket{m}
  &=
  \frac{\hbar}{2}
  \left(
    \sqrt{c-m^2+m}\,\,\ket{m-1} + \sqrt{c-m^2-m}\,\,\ket{m+1}
    \right)\\
  \label{eq:spn-Sym}
  \xhat{S}_y\ket{m}
  &= \frac{i\hbar}{2}
  \left(
  \sqrt{c-m^2+m}\,\,\ket{m-1} - \sqrt{c-m^2-m}\,\,\ket{m+1}
  \right)
\end{align}

Tenemos el valor de $\xhat{S}_x\ket{m}$ y $\xhat{S}_y\ket{m}$, pero no el de los
operadores $\xhat{S}_x$, $\xhat{S}_y$.
Necesitamos obtener su representación matricial y para ello calculamos los elementos de
las matrices
\begin{align}\label{eq:spn-elem-matriz-Sx}
  \braket{m_i|\xhat{S}_x|m_j}
  &=
  \frac{\hbar}{2}
  \left(
    \sqrt{c-m_j^2+m_j}\,\,\braket{m_i|m_{j-1}}
    + \sqrt{c-m_j^2-m_j}\,\,\braket{m_i|m_{j+1}}
    \right)\\
  \label{eq:spn-elem-matriz-Sy}
  \braket{m_i|\xhat{S}_y|m_j}
  &=
  \frac{i\hbar}{2}
  \left(
    \sqrt{c-m_j^2+m_j}\,\,\braket{m_i|m_{j-1}}
    - \sqrt{c-m_j^2-m_j}\,\,\braket{m_i|m_{j+1}}
  \right)    
\end{align}
donde las $\braket{m_i|m_{j\pm 1}}$ valen cero o uno porque la base de vectores $\ket{m}$
es ortonormal.

Como la base de funciones propias de $\xhat{S}_z$ es ortonormal, dejamos los elementos
matriz anteriores en función de $\delta_{ij}$
\begin{align}\label{eq:spn-elem-matriz-Sx-delta}
  \braket{m_i|\xhat{S}_x|m_j}
  &=
  \frac{\hbar}{2}
  \left(
    \sqrt{c-m_j^2+m_j}\,\,\delta_{m_i|m_j-1}
    + \sqrt{c-m_j^2-m_j}\,\,\delta_{m_i|m_j+1}
    \right)\\
  \label{eq:spn-elem-matriz-Sy-delta}
  \braket{m_i|\xhat{S}_y|m_j}
  &=
  \frac{i\hbar}{2}
  \left(
    \sqrt{c-m_j^2+m_j}\,\,\delta_{m_i|m_j-1}
    - \sqrt{c-m_j^2-m_j}\,\,\delta_{m_i|m_j+1}
  \right)  
\end{align}

\subsection{Cálculo de los valores propios \mathinhead{m}{m}
  y \mathinhead{c}{c}}
Recordemos que el operador $\xhat{S}_z$ tiene el valor propio $\hbar m$ y función propia
$\ket{m}$, pero al aplicar el operador descendente/ascendente sobre la función
$\ket{m}$, el resultado es otra función propia con un valor propio $\hbar$ unidades
inferior/superior.

Si no hubiera más limitaciones, $\xhat{S}_z$ tendría infinitos vectores y valores
propios, pero resulta que para el spin debe haber exactamente $N$
(la dimensión del espacio de Hilbert).
Por tanto el conjunto de valores y vectores propios del spin debe estar acotado, tanto
superior como inferiormente.

Si llamamos $\hbar m_{\text{inf}}$ al menor valor propio y $\hbar m_{\text{sup}}$ al
mayor, y además la dimensión del espacio de Hilbert fuera, por ejemplo $N=4$, tendríamos
una situación como la representada en la figura~\ref{fig:spn-msteps}
\begin{figure}[ht]
  \def\scl{2}
  % Fondo
  \pgfmathsetmacro{\HORZ}{0.25}
  \pgfmathsetmacro{\VERT}{0.25}
  % 
  \centering
  \begin{tikzpicture}[%
    scale=\scl,
    background/.style={
      line width=\bgborderwidth,
      draw=\bgbordercolor,
      fill=\bgcolor,
    },
    backgroundonly/.style={
      line width=\bgborderwidth,
      fill=\bgcolor,
    },
    ]
    \pgfmathsetmacro{\N}{5};
    \coordinate (O) at (0,0);
    \draw[-{Latex}] (O) -- +(right:\N) node[below right] {m};

    \foreach \i [count=\ii] in {\N,...,2}{
      \filldraw[fill=red,draw=black] (\i-1,0) circle [radius=1.5pt];
      \ifnum \ii<4
        \draw[-{Latex[length=7pt,width=6pt]},shorten >=4pt,shorten <=4pt,
             out=90,in=90] (\i-1,0) to node[above] {\small $\ii$} (\i-2,0);
      \fi
    };

    \node[below right=3pt and -7pt] at (1,0) {$m\kern1pt_{\text{inf}}$};
    \node[below right=3pt and -7pt] at (\N-1,0) {$m\kern1pt_\text{sup}$};
    % Fondo amarillo
    \coordinate (SW) at ($(current bounding box.south west) + (-\HORZ cm,-\VERT cm)$);
    \coordinate (NE) at ($(current bounding box.north east) + (\HORZ cm,\VERT cm)$);    
    \begin{scope}[on background layer]
      \draw[background] (SW) rectangle (NE);
    \end{scope}
    
  \end{tikzpicture}
  \caption{Los cuatro puntos representan los valores propios de
    $\xhat{S}_z$ en un espacio de Hilbert de dimensión $N=4$. El
    símbolo $m_{\text{inf}}$ representa el menor valor y
    $m_{\text{sup}}$ el mayor.  Las flechas indican las veces que se
    puede aplicar sucesivamente el operador descendente $\xhat{S}_{-}$
    a la función propia $\ket{m_{\text{sup}}}$ hasta llegar al menor
    valor propio posible.  Nótese que el valor propio se obtiene
    multiplicando la $m$ por $\hbar$.}
  \label{fig:spn-msteps}
\end{figure}

En la figura observamos que $\xhat{S}_z$ contiene $N=4$ vectores y valores propios y que
no se pueden dar más de $N-1=3$ \emph{saltos} para pasar de un valor propio extremo al
otro.
La relación entre $m_{\text{sup}}$, $m_{\text{inf}}$ y $N$ es
\begin{equation}\label{eq:relacion-mMN}
  m_{\text{sup}} - m_{\text{inf}} = N - 1
\end{equation}

La función propia $\ket{m_{\text{inf}}}$ de $\xhat{S}_z$ tiene el valor propio más bajo
posible $\hbar m_{\text{inf}}$.
Si aplicamos el operador descendente a esta función $\ket{m_{\text{inf}}}$, se obtendría
según~(\ref{eq:spn-S-m-1})
\[
  \xhat{S}_{-} \ket{m_{\text{inf}}}
  = \hbar \sqrt{c-m_{\text{inf}}^2+m_{\text{inf}}}\,\,\ket{m_{\text{inf}}-1}
\]

Pero este vector debe tener norma cero porque al aplicarle $\xhat{S}_z$ no puede
producir un valor propio menor que $m_{\text{inf}}$
\[
  c-m_{\text{inf}}^2+m_{\text{inf}} = 0
\]
\begin{equation}\label{eq:spn-cm}
  c = m_{\text{inf}}^2-m_{\text{inf}}
\end{equation}

En cambio, si tenemos la función propia $\ket{m_{\text{sup}}}$ de $\xhat{S}_z$ que
produce el máximo valor propio $\hbar m_{\text{inf}}$ y le aplicamos el operador
ascendente, nos pasaríamos del valor máximo.
Según~(\ref{eq:spn-S+m+1})
\[
  \xhat{S}_{+} \ket{m_{\text{sup}}}
  = \hbar \sqrt{c-m_{\text{sup}}^2-m_{\text{sup}}}\,\,\ket{m_{\text{sup}}+1}
\]

El módulo tiene que ser cero
\begin{equation}\label{eq:spn-cM}
  c= m_{\text{sup}}^2 + m_{\text{sup}}
\end{equation}

Restando las ecuaciones~(\ref{eq:spn-cm}) y (\ref{eq:spn-cM})
\[
  m_{\text{sup}}^2-m_{\text{inf}}^2+m_{\text{sup}}+m_{\text{inf}} = 0
\]
\begin{equation}\label{eq:spn-resta}
  (m_{\text{sup}}+m_{\text{inf}})\, (m_{\text{sup}}-m_{\text{inf}})+m_{\text{sup}}+m_{\text{inf}}
  = 0
\end{equation}

Sustituyendo la relación entre $m_{\text{inf}}$, $m_{\text{sup}}$ y $N$,
ecuación~(\ref{eq:relacion-mMN}) en~(\ref{eq:spn-resta})
\[
  (m_{\text{sup}}+m_{\text{inf}})\, (N-1) + m_{\text{sup}} + m_{\text{inf}} = 0
\]
\[
  (m_{\text{sup}}+m_{\text{inf}})\, N = 0
\]

Pero la dimensión del espacio de Hilbert $N$ no puede ser cero, de manera que
$m_{\text{sup}}+m_{\text{inf}} = 0$, de donde se deduce que
\begin{equation}\label{eq:spn-m-M}
  m_{\text{inf}} = -m_{\text{sup}}
\end{equation}

Si hacemos $m_{\text{sup}} = s$, podemos deducir que $m_{\text{inf}}=-2$
\begin{align}
  \label{eq:spn-msup}
  m_{\text{sup}} = s\\
  \label{eq:spn-minf}
  m_{\text{inf}} = -s
\end{align}
Y como consecuencia, los valores de $\xhat{S}_z$ van desde $-\hbar s$ hasta
$+\hbar s$, en saltos de $\hbar$ en $\hbar$.

El valor de $c$ lo calculamos con la ecuación~(\ref{eq:spn-cM})
\[
  c = m_{\text{sup}}^2 + m_{\text{sup}}
  = s^2 + s
\]
\begin{equation}\label{eq:spn-c}
  c = s\,(s+1)
\end{equation}

Así, llamando $s$ al spin, las ecuaciones de autovalores y autovectores de $\xhat{S}_z$
y $\xhat{S}^2$ son
\begin{align}\label{eq:spn-ecuaciones-Sz-S2}
  &\xhat{S}_z \ket{m} = \hbar m \ket{m}
  ;\hspace{1em} (m=-s,-s+1,\cdots,s-1,s)\\
  &\xhat{S}^2 \ket{m} = \hbar s\,(s+1) \ket{m}
\end{align}


\subsection{Relación entre el spin y la dimensión del espacio de Hilbert}
Podemos relacionar $m_{\text{sup}} = s$ con $N$
\[
  m_{\text{sup}}-m_{\text{inf}} = N-1
\]
\[
  s - (-s) = N - 1
\]
\[
  2s = N - 1
\]
\begin{equation}\label{eq:spn-relacion-s-N}
  s = \frac{N-1}{2}
  \hspace{1em}
  \text{o}
  \hspace{1em}
  N = 2s + 1
\end{equation}

Analizaremos algunos casos particulares aplicando la ecuación anterior:
\begin{itemize}
\item Si $N=1$, el spin vale cero, $s=0$. Una partícula sin spin es el bosón de Higgs.
\item Cuando $N=2$, el spin es $s=1/2$. El electrón es una partícula típica con
  este spin.
\item Si $N=3$, el spin vale uno, $s=1$. Una partícula con este spin es el fotón.
\item Si $N=4$, el spin es $3/2$ como la de la partícula $\Delta^\circ$.
\item Si $N=5$, el spin es $s=2$. Una partícula hipotética con este spin sería
  el gravitón.
\end{itemize}


\section{Operadores para distintos valores del spin}
Actualizamos la representación matricial de $\xhat{S}_x$ y $\xhat{S}_y$,
sustituyendo los valores calculados anteriormente
en~(\ref{eq:spn-elem-matriz-Sx-delta}) y en~(\ref{eq:spn-elem-matriz-Sy-delta})
con los valores propios descubiertos
{\small
\begin{align}\label{eq:spn-elem-matriz-Sx-delta-form}
  \braket{m_i|\xhat{S}_x|m_j}
  &=
  \frac{\hbar}{2}
  \left(
    \sqrt{s\,(s+1)-m_j^2+m_j}\,\,\delta_{m_i|m_j-1}
    + \sqrt{s\,(s+1)-m_j^2-m_j}\,\,\delta_{m_i|m_j+1}
    \right)\\
  \label{eq:spn-elem-matriz-Sy-delta-form}
  \braket{m_i|\xhat{S}_y|m_j}
  &=
  \frac{i\hbar}{2}
  \left(
    \sqrt{s(s+1)-m_j^2+m_j}\,\,\delta_{m_i|m_j-1}
    - \sqrt{s(s+1)-m_j^2-m_j}\,\,\delta_{m_i|m_j+1}
  \right)
\end{align}
}

La representación matricial de $\xhat{S}_z$ se escribió
en~(\ref{eq:spn-elem-matriz-Sy-delta-form})
\begin{equation}
  \braket{m_i|\xhat{S}_z|m_j}
  =
  \hbar m_j \delta_{m_i|m_j}
\end{equation}

Y la de $\xhat{S}^2$ se encuentra en~(\ref{eq:spn-S2casimir})
\begin{equation}\label{eq:spn-S2casimir-v2}
  \xhat{S}^2
  =
  \hbar^2 s\,(s+1)\mmm{I}
\end{equation}

\subsection{Matrices para spin \mathinhead{1/2}{spin1/2}}
El spin $1/2$ está relacionado con el espacio de Hilbert $\symbb{C}^2$.

\subsubsection{Operador \mathinhead{\xhat{S}_z}{spin1/2Sz}}
Empezamos con $\xhat{S}_z$. Es tradicional ordenar los valores propios
de mayor a menor $m_1=1/2$ y $m_2=-1/2$; esto nos podría traer problemas
al hallar las representaciones matriciales de $\xhat{S}_x$ y $\xhat{S}_y$.
Se avisará en su momento cuando aparezcan estos problemas.

La matriz es diagonal, con elementos $\hbar m$
\begin{equation}\label{eq:spn-spin-1/2-Sz}
  \xhat{S}_z
  =
  \begin{pmatrix}
    \hbar m_1 & 0\\
    0 & \hbar m_2
  \end{pmatrix}
    =
  \begin{pmatrix}
    \hbar/2 & 0\\
    0       & -\hbar/2
  \end{pmatrix}
    = \frac{\hbar}{2}
  \begin{pmatrix}
    1 & 0\\
    0   & -1
  \end{pmatrix}
  =
  \frac{\hbar}{2}\,\mmmg{\sigma}_z
\end{equation}

\subsubsection{Operador \mathinhead{\xhat{S}_x}{spin1/2Sx}}
Seguimos con $\xhat{S}_x$. Calculamos $s\,(s+1)$
\begin{equation}\label{eq:ssplus1Spin1/2}
  s(s+1) = \frac{1}{2}\left(\frac{1}{2}+1\right)
  = \frac{1}{2}\cdot\frac{3}{2}
  = \frac{3}{4}
\end{equation}

Recordamos por haber cambiado el orden de los valores propios,
empezando por el máximo
\[
  m_1=1/2 \hspace{.5em}>\hspace{.5em} m_2=-1/2
\]
no podemos suponer, por ejemplo, que $m_1+1 = m_2$; lo que trae como
consecuencia que para calcular las delta de Kronecker debamos bien,
sustituir las $m_i$ por su valor $1/2$ o $-1/2$, o bien cambiar el
signo $m_i\pm 1$ por $m_i\mp 1$ en los productos escalares.
Seguiremos el primer camino
\begin{itemize}
\item Los elementos diagonales $\braket{m_i|\xhat{S}_x|m_i}$ valen cero
  porque $m_i \ne m_i -1$ y $m_i \ne m_i +1$ y los productos escalares
  $\braket{m_i,\,m_i-1} = 0$ y $\braket{m_i,\,m_i+1} = 0$
  {\footnotesize
  \begin{align*}
    \braket{m_i|\xhat{S}_x|m_i}
    &= \frac{\hbar}{2}\left(
      \sqrt{s\,(s+1)-m_i^2+m_i}\,\,\cancelout{\braket{m_i|m_i-1}}
      \,+\,
      \sqrt{s\,(s+1)-m_i^2-m_i}\,\,\cancelout{\braket{m_i|m_2+1}}\right)\\
    &=
      0
  \end{align*}
  }
\item $\braket{m_1|\xhat{S}_x|m_2}$:
  Calculamos las $\braket{m_1|m_2\pm 1}$. El valor $-3/2$ en rojo indica que
  nos hemos pasado del valor propio inferior, $-1/2$
  {\small
  \begin{align*}
    \braket{m_1|m_2 - 1}
    &= \braket{+1/2|-1/2-1}
      = \braket{+1/2|\textcolor{red!80!black}{-3/2}} = 0\\
    \braket{m_1|m_2 + 1}
    &= \braket{+1/2|-1/2+1}
      = \braket{+1/2|+1/2} = 1
  \end{align*}
  }
  {\footnotesize
  \begin{align*}
    \braket{m_1|\xhat{S}_x|m_2}
    &= \frac{\hbar}{2}\left(
      \sqrt{s\,(s+1)-m_2^2+m_2}\,\,\cancelout{\braket{m_1|m_2-1}}
      \,+\,
      \sqrt{s\,(s+1)-m_2^2-m_2}\,\,\braket{m_1|m_2+1}\right)\\
    &=
      \frac{\hbar}{2}
      \sqrt{\frac{3}{4}-\left(-\frac{1}{2}\right)^2-\left(-\frac{1}{2}\right)}
      = \frac{\hbar}{2} \sqrt{\frac{3}{4} - \frac{1}{4} + \frac{2}{4}}
      = \frac{\hbar}{2}
  \end{align*}
  }
\item $\braket{m_2|\xhat{S}_x|m_1}$:
  Calculamos las $\braket{m_2|m_1\pm 1}$. El valor $3/2$ en rojo indica que
  hemos sobrepasado el valor propio superior $1/2$
  {\small
  \begin{align*}
    \braket{m_2|m_1 - 1}
    &= \braket{-1/2|1/2-1}
      = \braket{-1/2|-1/2} = 1\\
    \braket{m_2|m_1 + 1}
    &= \braket{-1/2|1/2+1}
    = \braket{+1/2|\textcolor{red!80!black}{3/2}} = 0
  \end{align*}
  }
  {\footnotesize
  \begin{align*}
    \braket{m_2|\xhat{S}_x|m_1}
    &= \frac{\hbar}{2}\left(
      \sqrt{s\,(s+1)-m_1^2+m_1}\,\,\braket{m_2|m_1-1}
      \,+\,
      \sqrt{s\,(s+1)-m_1^2-m_1}\,\,\cancelout{\braket{m_2|m_1+1}}\right)\\
    &=
      \frac{\hbar}{2}
      \sqrt{\frac{3}{4}-\left(\frac{1}{2}\right)^2+\left(\frac{1}{2}\right)}
      = \frac{\hbar}{2} \sqrt{\frac{3}{4} - \frac{1}{4} + \frac{2}{4}}
      = \frac{\hbar}{2}
  \end{align*}
  }

  La matriz queda
  \begin{equation}\label{eq:spn-spin-1/2-Sx}
    \xhat{S}_x
    =
    \begin{pmatrix}
      0 & \hbar/2 \\
      \hbar/2 & 0
    \end{pmatrix}
    = \frac{\hbar}{2}
    \begin{pmatrix}
      0 & 1\\
      1 & 0
    \end{pmatrix}
    = \frac{\hbar}{2}
    \mmm{\sigma}_x
  \end{equation}
\end{itemize}

\subsubsection{Operador \mathinhead{\xhat{S}_y}{spin1/2Sy}}
Ahora le toca el turno a $\xhat{S}_y$. Sabemos por~(\ref{eq:ssplus1Spin1/2})
que $s(s+1) = 3/4$. Recordamos que hemos cambiado el orden de los valores
propios, empezando por el máximo, $m_1=1/2$ y $m_2=-1/2$.
Debido a esto no podemos suponer, por ejemplo, que $m_1+1 = m_2$;
lo que trae como consecuencia que para calcular los productos escalares
de las funciones propias de $\xhat{S}_z$ debamos sustituir las $m_i$
por su valor $1/2$ o $-1/2$
\begin{itemize}
\item Los elementos diagonales $\braket{m_i|\xhat{S}_x|m_i}$ valen cero
  porque $m_i \ne m_i -1$ y $m_i \ne m_i +1$ y los productos escalares
  $\braket{m_i,\,m_i-1} = 0$ y $\braket{m_i,\,m_i+1} = 0$
  {\footnotesize
  \begin{align*}
    \braket{m_i|\xhat{S}_x|m_i}
    &= \frac{i\hbar}{2}\left(
      \sqrt{s\,(s+1)-m_i^2+m_i}\,\,\cancelout{\braket{m_i|m_i-1}}
      \,-\,
      \sqrt{s\,(s+1)-m_i^2-m_i}\,\,\cancelout{\braket{m_i|m_2+1}}\right)\\
    &=
      0
  \end{align*}
  }
\item $\braket{m_1|\xhat{S}_y|m_2}$:
  Calculamos las $\braket{m_1|m_2\pm 1}$. El valor $-3/2$ en rojo indica que
  nos hemos pasado del valor propio inferior, $-1/2$
  {\small
  \begin{align*}
    \braket{m_1|m_2 - 1}
    &= \braket{+1/2|-1/2-1}
    = \braket{+1/2|\textcolor{red!80!black}{-3/2}} = 0\\
    \braket{m_1|m_2 + 1}
    &= \braket{+1/2|-1/2+1}
    = \braket{+1/2|+1/2} = 1
  \end{align*}
  }
  {\footnotesize
  \begin{align*}
    \braket{m_1|\xhat{S}_y|m_2}
    &= \frac{i\hbar}{2}\left(
      \sqrt{s\,(s+1)-m_2^2+m_2}\,\,\cancelout{\braket{m_1|m_2-1}}
      \,-\,
      \sqrt{s\,(s+1)-m_2^2-m_2}\,\,\braket{m_1|m_2+1}\right)\\
    &=
      -\frac{i\hbar}{2}
      \sqrt{\frac{3}{4}-\left(-\frac{1}{2}\right)^2-\left(-\frac{1}{2}\right)}
      = -\frac{i\hbar}{2} \sqrt{\frac{3}{4} - \frac{1}{4} + \frac{2}{4}}
      = -\frac{i\hbar}{2}
  \end{align*}
  }
\item $\braket{m_2|\xhat{S}_y|m_1}$:
  Calculamos las $\braket{m_2|m_1\pm 1}$. El valor $3/2$ en rojo indica que
  hemos sobrepasado el valor propio superior, $1/2$
  {\small
  \begin{align*}
    \braket{m_2|m_1 - 1}
    &= \braket{-1/2|1/2-1}
    = \braket{-1/2|-1/2} = 1\\
    \braket{m_2|m_1 + 1}
    &= \braket{-1/2|1/2+1}
    = \braket{+1/2|\textcolor{red!80!black}{3/2}} = 0
  \end{align*}
  }
  {\footnotesize
  \begin{align*}
    \braket{m_2|\xhat{S}_y|m_1}
    &= \frac{i\hbar}{2}\left(
      \sqrt{s\,(s+1)-m_1^2+m_1}\,\,\braket{m_2|m_1-1}
      \,-\,
      \sqrt{s\,(s+1)-m_1^2-m_1}\,\,\cancelout{\braket{m_2|m_1+1}}\right)\\
    &=
      \frac{i\hbar}{2}
      \sqrt{\frac{3}{4}-\left(\frac{1}{2}\right)^2+\left(\frac{1}{2}\right)}
      = \frac{i\hbar}{2} \sqrt{\frac{3}{4} - \frac{1}{4} + \frac{2}{4}}
      = \frac{i\hbar}{2}
  \end{align*}
  }

  La matriz queda
  \begin{equation}\label{eq:spn-spin-1/2-Sy}
    \xhat{S}_y
    =
    \begin{pmatrix}
      0 & -i\hbar/2 \\
      i\hbar/2 & 0
    \end{pmatrix}
    = \frac{\hbar}{2}
    \begin{pmatrix}
      0 & -i\\
      i & 0
    \end{pmatrix}
    = \frac{\hbar}{2}
    \mmm{\sigma}_y
  \end{equation}
\end{itemize}

\subsubsection{Operador \mathinhead{\xhat{S}^2}{spin1/2S2}}
Terminamos con $\xhat{S}^2$. Según~(\ref{eq:spn-S2casimir-v2})
\begin{equation}\label{eq:spn-spin-1/2-S2}
  \xhat{S}^2
  =
  \hbar^2 s\,(s+1)\mmm{I}
  =
  \frac{3\hbar^2}{4}
  \begin{pmatrix}
    1 & 0 \\
    0 & 1 
  \end{pmatrix}
  = \frac{3\hbar^2}{4} \mmm{I}
\end{equation}

\subsection{Matrices para spin \mathinhead{1}{spin1}}
El spin $1$ está relacionado con el espacio de Hilbert $\symbb{C}^{3}$.

\subsubsection{Operador \mathinhead{\xhat{S}_z}{spin1Sz}}
Empezamos con $\xhat{S}_z$. Es tradicional ordenar los valores propios
de mayor a menor $m_1=1$, $m_2=0$ y $m_3=-1$; esto nos podría traer problemas
con las representaciones matriciales de $\xhat{S}_x$ y $\xhat{S}_y$,
de los que se avisará en su momento.

La matriz es diagonal, con elementos $\hbar m$
\begin{equation}\label{eq:spn-spin-1-Sz}
  \xhat{S}_z
  =
  \begin{pmatrix}
    \hbar m_1 & 0 & 0\\
    0 & \hbar m_2 & 0\\
    0 & 0 & \hbar m_3
  \end{pmatrix}
    =
  \begin{pmatrix}
    \hbar & 0  & 0     \\
    0     & 0  & 0     \\
    0     & 0  & -\hbar
  \end{pmatrix}
    = \hbar\,
  \begin{pmatrix}
    1 & 0 & 0\\
    0 & 0 & 0\\
    0 & 0 & -1
  \end{pmatrix}
\end{equation}

\subsubsection{Operador \mathinhead{\xhat{S}_x}{spin1Sx}}
Seguimos con $\xhat{S}_x$. Calculamos $s\,(s+1)$
\begin{equation}\label{eq:ssplus1Spin1}
  s(s+1) = 1(1+1) = 2
\end{equation}

Además, recordamos que hemos cambiado el orden de los valores
propios, empezando por el máximo, $m_1=1$, $m_2=0$ y $m_3=-1$.
Por esto no podemos suponer, por ejemplo, que $m_1+1 = m_2$;
lo que trae como consecuencia que para calcular los productos escalares
de funciones propias de $\xhat{S}_z$ debamos, bien sustituir las $m_i$
por su valor $1$, $0$ o $-1$, o bien cambiar el signo $m_i\pm 1$ por
$m_i\mp 1$ en los productos escalares. Seguiremos el primer criterio
\begin{itemize}
\item Los elementos diagonales $\braket{m_i|\xhat{S}_x|m_i}$ valen cero
  porque $m_i \ne m_i -1$ y $m_i \ne m_i +1$ y los productos escalares
  $\braket{m_i,\,m_i-1} = 0$ y $\braket{m_i,\,m_i+1} = 0$
  {\footnotesize
  \begin{align*}
    \braket{m_i|\xhat{S}_x|m_i}
    &= \frac{\hbar}{2}\left(
      \sqrt{s\,(s+1)-m_i^2+m_i}\,\,\cancelout{\braket{m_i|m_i-1}}
      \,+\,
      \sqrt{s\,(s+1)-m_i^2-m_i}\,\,\cancelout{\braket{m_i|m_2+1}}\right)\\
    &=
      0
  \end{align*}
  }
\item $\braket{m_1|\xhat{S}_x|m_2}$:
  Calculamos las $\braket{m_1|m_2\pm 1}$
  {\small
  \begin{align*}
    \braket{m_1|m_2 - 1}
    &= \braket{1|0-1}
    = \braket{1|-1} = 0\\
     \braket{m_1|m_2 + 1}
    &= \braket{1|0+1}
    = \braket{1|1} = 1
  \end{align*}
  }
  {\footnotesize
  \begin{align*}
    \braket{m_1|\xhat{S}_x|m_2}
    &= \frac{\hbar}{2}\left(
      \sqrt{s\,(s+1)-m_2^2+m_2}\,\,\cancelout{\braket{m_1|m_2-1}}
      \,+\,
      \sqrt{s\,(s+1)-m_2^2-m_2}\,\,\braket{m_1|m_2+1}\right)\\
    &=
      \frac{\hbar}{2}
      \sqrt{2-0^2-0}
      = \frac{\hbar}{\sqrt{2}}
  \end{align*}
  }
\item $\braket{m_1|\xhat{S}_x|m_3}$:
  Calculamos las $\braket{m_1|m_3\pm 1}$. El valor $-2$ en rojo indica que
  hemos sobrepasado el valor propio inferior, $-1$
  {\small
  \begin{align*}
    \braket{m_1|m_3 - 1}
    &= \braket{1|-1-1}
    = \braket{1|\textcolor{red!80!black}{-2}} = 0\\
    \braket{1|-1 + 1}
    &= \braket{1|0} = 0
  \end{align*}
  }
  {\footnotesize
  \begin{align*}
    \braket{m_1|\xhat{S}_x|m_3}
    &= \frac{\hbar}{2}\left(
      \sqrt{s\,(s+1)-m_3^2+m_3}\,\,\cancelout{\braket{m_1|m_3-1}}
      \,+\,
      \sqrt{s\,(s+1)-m_3^2-m_3}\,\,\cancelout{\braket{m_1|m_3+1}}\right)\\
    &= 0
  \end{align*}
  }
\item $\braket{m_2|\xhat{S}_x|m_1}$:
  Calculamos las $\braket{m_2|m_1\pm 1}$
  {\small
  \begin{align*}
    \braket{m_2|m_1 - 1}
    &= \braket{0|1-1}
    = \braket{0|0} = 1\\
    \braket{m_2|m_1 + 1}
    &= \braket{0|0+1}
    = \braket{0|1} = 0
  \end{align*}
  }
  {\footnotesize
  \begin{align*}
    \braket{m_2|\xhat{S}_x|m_1}
    &= \frac{\hbar}{2}\left(
      \sqrt{s\,(s+1)-m_1^2+m_1}\,\,\braket{m_2|m_1-1}
      \,+\,
      \sqrt{s\,(s+1)-m_1^2-m_1}\,\,\cancelout{\braket{m_2|m_1+1}}\right)\\
    &=
      \frac{\hbar}{2}
      \sqrt{2-1^2+1}
      = \frac{\hbar}{\sqrt{2}}
  \end{align*}
  }
\item $\braket{m_2|\xhat{S}_x|m_3}$:
  Calculamos las $\braket{m_2|m_3\pm 1}$. El valor $-2$ en rojo indica que
  hemos sobrepasado el valor propio inferior, $-1$
  {\small
  \begin{align*}
    \braket{m_2|m_3 - 1}
    &= \braket{0|-1-1}
    = \braket{0|\textcolor{red!80!black}{-2}} = 0\\
    \braket{m_2|m_3 + 1}
    &= \braket{0|-1+1}
    = \braket{0|0} = 1
  \end{align*}
  }
  {\footnotesize
  \begin{align*}
    \braket{m_2|\xhat{S}_x|m_3}
    &= \frac{\hbar}{2}\left(
      \sqrt{s\,(s+1)-m_3^2+m_3}\,\,\cancelout{\braket{m_2|m_3-1}}
      \,+\,
      \sqrt{s\,(s+1)-m_3^2-m_3}\,\,\braket{m_2|m_3+1}\right)\\
    &=
      \frac{\hbar}{2}
      \sqrt{2-(-1)^2-(-1)}
      = \frac{\hbar}{\sqrt{2}}
  \end{align*}
  }
\item $\braket{m_3|\xhat{S}_x|m_1}$:
  Calculamos las $\braket{m_3|m_1\pm 1}$. El valor $2$ en rojo indica que
  hemos sobrepasado el valor propio superior, $1$
  {\small
  \begin{align*}
    \braket{m_3|m_1 - 1}
    &= \braket{-1|1-1}
    = \braket{-1|-0} = 0\\
    \braket{m_3|m_1 + 1}
    &= \braket{-1|1+1}
    = \braket{0|\textcolor{red!80!black}{2}} = 0
  \end{align*}
  }
  {\footnotesize
  \begin{align*}
    \braket{m_3|\xhat{S}_x|m_1}
    &= \frac{\hbar}{2}\left(
      \sqrt{s\,(s+1)-m_1^2+m_1}\,\,\cancelout{\braket{m_3|m_1-1}}
      \,+\,
      \sqrt{s\,(s+1)-m_1^2-m_1}\,\,\cancelout{\braket{m_3|m_1+1}}\right)\\
    &= 0
  \end{align*}
  }
\item $\braket{m_3|\xhat{S}_x|m_2}$:
  Calculamos las $\braket{m_3|m_2\pm 1}$
  {\small
  \begin{align*}
    \braket{m_3|m_2 - 1}
    &= \braket{-1|0-1}
    = \braket{-1|-1} = 1\\
    \braket{m_3|m_2 + 1}
    &= \braket{-1|0+1}
    = \braket{-1|1} = 0
  \end{align*}
  }
  {\footnotesize
  \begin{align*}
    \braket{m_3|\xhat{S}_x|m_2}
    &= \frac{\hbar}{2}\left(
      \sqrt{s\,(s+1)-m_2^2+m_2}\,\,\braket{m_3|m_2-1}
      \,+\,
      \sqrt{s\,(s+1)-m_2^2-m_2}\,\,\cancelout{\braket{m_3|m_2+1}}\right)\\
    &=
      \frac{\hbar}{2}
      \sqrt{2-0^2+0}
      = \frac{\hbar}{\sqrt{2}}
  \end{align*}
  }
  
  La matriz queda
  \begin{equation}\label{eq:spn-spin-1-Sx}
    \xhat{S}_x
    =
    \begin{pmatrix}
      0 & \hbar/\sqrt{2} & 0\\
      \hbar/\sqrt{2} & 0 & \hbar/\sqrt{2}\\
      0 & \hbar/\sqrt{2} & 0
    \end{pmatrix}
    = \frac{\hbar}{\sqrt{2}}
    \begin{pmatrix}
      0 & 1 & 0\\
      1 & 0 & 1\\
      0 & 1 & 0
    \end{pmatrix}
  \end{equation}
\end{itemize}

\subsubsection{Operador \mathinhead{\xhat{S}_y}{spin1Sy}}
Ahora le toca el turno a $\xhat{S}_y$. Sabemos por~(\ref{eq:ssplus1Spin1})
que $s(s+1) = 2$.
Además, recordamos que hemos cambiado el orden de los valores
propios, empezando por el máximo, $m_1=1/$, $m_2=0$ y $m_2=-1$.
Debido a esto no podemos suponer, por ejemplo, que $m_1+1 = m_2$;
lo que trae como consecuencia que para calcular los productos escalares
de funciones propias de $\xhat{S}_z$ debamos, bien sustituir las $m_i$
por su valor $1$, $0$ o $-1$, o bien cambiar el signo $m_i\pm 1$ por
$m_i\mp 1$ en los productos escalares. Seguiremos el primer camino
\begin{itemize}
\item Los elementos diagonales $\braket{m_i|\xhat{S}_y|m_i}$ valen cero
  porque $m_i \ne m_i -1$ y $m_i \ne m_i +1$ y las funciones delta
  $\braket{m_i,\,m_i-1} = 0$ y $\braket{m_i,\,m_i+1} = 0$
  \[
    \braket{m_i|\xhat{S}_y|m_i}
    = 0
  \]
\item $\braket{m_1|\xhat{S}_y|m_2}$:
  Calculamos las $\braket{m_1|m_2\pm 1}$
  {\small
  \begin{align*}
    \braket{m_1|m_2 - 1}
    &= \braket{1|0-1}
    = \braket{1|-1} = 0\\
    \braket{m_1|m_2 + 1}
    &= \braket{1|0+1}
    = \braket{1|1} = 1
  \end{align*}
  }
  {\footnotesize
  \begin{align*}
    \braket{m_1|\xhat{S}_y|m_2}
    &= \frac{i\hbar}{2}\left(
      \sqrt{s\,(s+1)-m_2^2+m_2}\,\,\cancelout{\braket{m_1|m_2-1}}
      \,-\,
      \sqrt{s\,(s+1)-m_2^2-m_2}\,\,\braket{m_1|m_2+1}\right)\\
    &=
      -\frac{i\hbar}{2}
      \sqrt{2-0^2-0}
      = -\frac{i\hbar}{\sqrt{2}}
  \end{align*}
  }
\item $\braket{m_1|\xhat{S}_y|m_3}$:
  Calculamos las $\braket{m_1|m_3\pm 1}$. El valor $-2$ en rojo indica que
  hemos sobrepasado el valor propio inferior, $-1$
  {\small
  \begin{align*}
    \braket{m_1|m_3 - 1}
    &= \braket{1|-1-1}
    = \braket{1|\textcolor{red!80!black}{-2}} = 0\\
    \braket{1|-1 + 1}
    &= \braket{1|0} = 0
  \end{align*}
  }
  {\footnotesize
  \begin{align*}
    \braket{m_1|\xhat{S}_y|m_3}
    &= \frac{i\hbar}{2}\left(
      \sqrt{s\,(s+1)-m_3^2+m_3}\,\,\cancelout{\braket{m_1|m_3-1}}
      \,-\,
      \sqrt{s\,(s+1)-m_3^2-m_3}\,\,\cancelout{\braket{m_1|m_3+1}}\right)\\
    &= 0
  \end{align*}
  }

\item $\braket{m_2|\xhat{S}_y|m_1}$:
  Calculamos las $\braket{m_2|m_1\pm 1}$
  {\small
  \begin{align*}
    \braket{m_2|m_1 - 1}
    &= \braket{0|1-1}
    = \braket{0|0} = 1\\
    \braket{m_2|m_1 + 1}
    &= \braket{0|0+1}
    = \braket{0|1} = 0
  \end{align*}
  }
  {\footnotesize
  \begin{align*}
    \braket{m_2|\xhat{S}_y|m_1}
    &= \frac{i\hbar}{2}\left(
      \sqrt{s\,(s+1)-m_1^2+m_1}\,\,\braket{m_2|m_1-1}
      \,-\,
      \sqrt{s\,(s+1)-m_1^2-m_1}\,\,\cancelout{\braket{m_2|m_1+1}}\right)\\
    &=
      \frac{i\hbar}{2}
      \sqrt{2-1^2+1}
      = \frac{i\hbar}{\sqrt{2}}
  \end{align*}
  }
 \item $\braket{m_2|\xhat{S}_y|m_3}$:
  Calculamos las $\braket{m_2|m_3\pm 1}$. El valor $-2$ en rojo indica que
  hemos sobrepasado el valor propio inferior, $-1$
  {\small
  \begin{align*}
    \braket{m_2|m_3 - 1}
    &= \braket{0|-1-1}
    = \braket{0|\textcolor{red!80!black}{-2}} = 0\\
    \braket{m_2|m_3 + 1}
    &= \braket{0|-1+1}
    = \braket{0|0} = 1
  \end{align*}
  }
  {\footnotesize
  \begin{align*}
    \braket{m_2|\xhat{S}_y|m_3}
    &= \frac{i\hbar}{2}\left(
      \sqrt{s\,(s+1)-m_3^2+m_3}\,\,\cancelout{\braket{m_2|m_3-1}}
      \,-\,
      \sqrt{s\,(s+1)-m_3^2-m_3}\,\,\braket{m_2|m_3+1}\right)\\
    &=
      -\frac{i\hbar}{2}
      \sqrt{2-(-1)^2-(-1)}
      = -\frac{i\hbar}{\sqrt{2}}
  \end{align*}
  }
\item $\braket{m_3|\xhat{S}_y|m_1}$:
  Calculamos las $\braket{m_3|m_1\pm 1}$. El valor $2$ en rojo indica que
  hemos sobrepasado el valor propio superior, $1$
  {\small
  \begin{align*}
    \braket{m_3|m_1 - 1}
    &= \braket{-1|1-1}
    = \braket{-1|-0} = 0\\
    \braket{m_3|m_1 + 1}
    &= \braket{-1|1+1}
    = \braket{0|\textcolor{red!80!black}{2}} = 0
  \end{align*}
  }
  {\footnotesize
  \begin{align*}
    \braket{m_3|\xhat{S}_y|m_1}
    &= \frac{i\hbar}{2}\left(
    \sqrt{s\,(s+1)-m_1^2+m_1}\,\,\cancelout{\braket{m_3|m_1-1}}
      \,-\,
      \sqrt{s\,(s+1)-m_1^2-m_1}\,\,\cancelout{\braket{m_3|m_1+1}}\right)\\
    &= 0
  \end{align*}
  }
\item $\braket{m_3|\xhat{S}_y|m_2}$:
  Calculamos las $\braket{m_3|m_2\pm 1}$
  {\small
  \begin{align*}
    \braket{m_3|m_2 - 1}
    &= \braket{-1|0-1}
    = \braket{-1|-1} = 1\\
    \braket{m_3|m_2 + 1}
    &= \braket{-1|0+1}
    = \braket{-1|1} = 0
  \end{align*}
  }
  {\footnotesize
  \begin{align*}
    \braket{m_3|\xhat{S}_y|m_2}
    &= \frac{i\hbar}{2}\left(
    \sqrt{s\,(s+1)-m_2^2+m_2}\,\,\braket{m_3|m_2-1}
      \,-\,
      \sqrt{s\,(s+1)-m_2^2-m_2}\,\,\cancelout{\braket{m_3|m_2+1}}\right)\\
    &=
      \frac{i\hbar}{2}
      \sqrt{2-0^2+0}
      = \frac{i\hbar}{\sqrt{2}}
  \end{align*}
  }
  
  La matriz queda
  \begin{equation}\label{eq:spn-spin-1-Sy}
    \xhat{S}_y
    =
    \begin{pmatrix}
      0 & -i\hbar/\sqrt{2} & 0\\
      i\hbar/\sqrt{2} & 0 & -i\hbar/\sqrt{2}\\
      0 & i\hbar/\sqrt{2} & 0
    \end{pmatrix}
    = \frac{i\hbar}{\sqrt{2}}
    \begin{pmatrix}
      0 & -1 & 0\\
      1 & 0 & -1\\
      0 & 1 & 0
    \end{pmatrix}
  \end{equation}
\end{itemize}

\subsubsection{Operador \mathinhead{\xhat{S}^2}{spin1S2}}
Terminamos con $\xhat{S}^2$. Según~(\ref{eq:spn-S2casimir-v2})
\begin{equation}\label{eq:spn-spin-1-S2}
  \xhat{S}^2
  =
  \hbar^2 s\,(s+1)\mmm{I}
  =
  2\hbar^2
  \begin{pmatrix}
    1 & 0 & 0\\
    0 & 1 & 0\\
    0 & 0 & 1
  \end{pmatrix}
  = 2\hbar^2\mmm{I}
\end{equation}


\subsection{Matrices para spin \mathinhead{3/2}{spin3/2}}
El spin $3/2$ está relacionado con el espacio de Hilbert $\symbb{C}^{4}$.

\subsubsection{Operador \mathinhead{\xhat{S}_z}{spin3/2Sz}}
Empezamos con $\xhat{S}_z$. Es tradicional ordenar los valores propios
de mayor a menor $m_1=3/2$, $m_2=1/2$, $m_3=-1/2$ y $m_4=-3/2$;
esto nos podría traer problemas con las representaciones matriciales de
$\xhat{S}_x$ y $\xhat{S}_y$, que se avisarán en su momento.

La matriz es diagonal, con elementos $\hbar m$
{\small
\begin{equation}\label{eq:spn-spin-3/2-Sz}
  \xhat{S}_z
  =
  \begin{pmatrix}
    \hbar m_1 & 0 & 0 & 0\\
    0 & \hbar m_2 & 0 & 0\\
    0 & 0 & \hbar m_3 & 0\\
    0 & 0 & 0 & \hbar m_4
  \end{pmatrix}
%    =
%    \begin{pmatrix}
%      3\hbar/2 & 0 & 0 & 0\\
%      0 & \hbar/2 & 0 & 0\\
%      0 & 0 & -\hbar/2 & 0\\
%      0 & 0 & 0 & -3\hbar/2
%    \end{pmatrix}
    = \frac{\hbar}{2}\,
    \begin{pmatrix}
      3 & 0 & 0 & 0\\
      0 & 1 & 0 & 0\\
      0 & 0 & -1 & 0\\
      0 & 0 & 0 & -3
    \end{pmatrix}
\end{equation}
}

\subsubsection{Operador \mathinhead{\xhat{S}_x}{spin3/2Sx}}
Seguimos con $\xhat{S}_x$. Calculamos $s\,(s+1)$
\begin{equation}\label{eq:ssplus1Spin3/2}
  s(s+1)
  = \frac{3}{2}\left(\frac{3}{2}+1\right)
  = \frac{3}{2}\cdot\frac{5}{2}
  = \frac{15}{4}
\end{equation}

Además, recordamos que hemos cambiado el orden de los valores
propios, empezando por el máximo, $m_1=3/2$, $m_2=1/2$, $m_3=-1/2$
y $m_4=-3/2$.
Por esto no podemos suponer, por ejemplo, que $m_1+1 = m_2$;
lo que trae como consecuencia que para calcular los productos escalares
de funciones propias de $\xhat{S}_z$ debamos, bien sustituir las $m_i$
por su valor $3/2$, $1/2$, $-1/2$ o $-3/2$, o bien cambiar el signo
$m_i\pm 1$ por $m_i\mp 1$ en los productos escalares.
Seguiremos el primer criterio
\begin{itemize}
\item Los elementos diagonales $\braket{m_i|\xhat{S}_x|m_i}$ valen cero
  porque $m_i \ne m_i -1$ y $m_i \ne m_i +1$ y los productos escalares
  se anulan, $\braket{m_i,\,m_i-1} = 0$ y $\braket{m_i,\,m_i+1} = 0$
  {\footnotesize
  \begin{align*}
    \braket{m_i|\xhat{S}_x|m_i}
    &= \frac{\hbar}{2}\left(
      \sqrt{s\,(s+1)-m_i^2+m_i}\,\,\cancelout{\braket{m_i|m_i-1}}
      \,+\,
      \sqrt{s\,(s+1)-m_i^2-m_i}\,\,\cancelout{\braket{m_i|m_2+1}}\right)\\
    &=
      0
  \end{align*}
  }
\item $\braket{m_1|\xhat{S}_x|m_2}$:
  Calculamos las $\braket{m_1|m_2\pm 1}$
  {\small
  \begin{align*}
    \braket{m_1|m_2 - 1}
    &= \braket{3/2|1/2-1}
    = \braket{3/2|-1/2} = 0\\
     \braket{m_1|m_2 + 1}
    &= \braket{3/2|1/2+1}
    = \braket{3/2|3/2} = 1
  \end{align*}
  }
  {\footnotesize
  \begin{align*}
    \braket{m_1|\xhat{S}_x|m_2}
    &= \frac{\hbar}{2}\left(
      \sqrt{s\,(s+1)-m_2^2+m_2}\,\,\cancelout{\braket{m_1|m_2-1}}
      \,+\,
      \sqrt{s\,(s+1)-m_2^2-m_2}\,\,\braket{m_1|m_2+1}\right)\\
    &=
      \frac{\hbar}{2}
      \sqrt{\frac{15}{4}-\left(\frac{1}{2}\right)^2-\frac{1}{2}}
      =\frac{\hbar}{2}\sqrt{\frac{15}{4}-\frac{1}{4}-\frac{2}{4}}
      = \frac{\sqrt{3}\hbar}{2}
  \end{align*}
  }
\item $\braket{m_1|\xhat{S}_x|m_3}$:
  Calculamos las $\braket{m_1|m_3\pm 1}$
  {\small
  \begin{align*}
    \braket{m_1|m_3 - 1}
    &= \braket{3/2|-1/2-1}
    = \braket{3/2|-3/2} = 0\\
     \braket{m_1|m_3 + 1}
    &= \braket{3/2|-1/2+1}
    = \braket{3/2|+1/2} = 0
  \end{align*}
  }
  {\footnotesize
  \begin{align*}
    \braket{m_1|\xhat{S}_x|m_3}
    &= \frac{\hbar}{2}\left(
      \sqrt{s\,(s+1)-m_3^2+m_3}\,\,\cancelout{\braket{m_1|m_3-1}}
      \,+\,
      \sqrt{s\,(s+1)-m_3^2-m_3}\,\,\cancelout{\braket{m_1|m_3+1}}\right)\\
    &=
      0
  \end{align*}
  }
\item $\braket{m_1|\xhat{S}_x|m_4}$:
  Calculamos las $\braket{m_1|m_4\pm 1}$. El valor $-5/2$ en rojo indica que
  hemos sobrepasado el valor propio inferior, $-3/2$
  {\small
  \begin{align*}
    \braket{m_1|m_4 - 1}
    &= \braket{3/2|-3/2-1}
    = \braket{3/2|\textcolor{red!80!black}{-5/2}} = 0\\
     \braket{m_1|m_4 + 1}
    &= \braket{3/2|-3/2+1}
    = \braket{3/2|-1/2} = 0
  \end{align*}
  }
  {\footnotesize
  \begin{align*}
    \braket{m_1|\xhat{S}_x|m_4}
    &= \frac{\hbar}{2}\left(
      \sqrt{s\,(s+1)-m_4^2+m_4}\,\,\cancelout{\braket{m_1|m_4-1}}
      \,+\,
      \sqrt{s\,(s+1)-m_4^2-m_4}\,\,\cancelout{\braket{m_1|m_4+1}}\right)\\
    &=
      0
  \end{align*}
  }
\item $\braket{m_2|\xhat{S}_x|m_1}$:
  Calculamos las $\braket{m_2|m_1\pm 1}$. El valor $5/2$ en rojo indica que
  hemos sobrepasado el valor propio superior, $3/2$
  {\small
  \begin{align*}
    \braket{m_2|m_1 - 1}
    &= \braket{+1/2|3/2-1}
    = \braket{+1/2|+1/2} = 1\\
     \braket{m_2|m_1 + 1}
    &= \braket{+1/2|3/2+1}
    = \braket{3/2|\textcolor{red!80}{5/2}} = 0
  \end{align*}
  }
  {\footnotesize
  \begin{align*}
    \braket{m_2|\xhat{S}_x|m_1}
    &= \frac{\hbar}{2}\left(
      \sqrt{s\,(s+1)-m_1^2+m_1}\,\,\braket{m_2|m_1-1}
      \,+\,
      \sqrt{s\,(s+1)-m_1^2-m_1}\,\,\cancelout{\braket{m_2|m_1+1}}\right)\\
    &=
      \frac{\hbar}{2}
      \sqrt{\frac{15}{4}-\left(\frac{3}{2}\right)^2+\frac{3}{2}}
      =\frac{\hbar}{2}\sqrt{\frac{15}{4}-\frac{9}{4}+\frac{6}{4}}
      = \frac{\sqrt{3}\hbar}{2}
  \end{align*}
  }
\item $\braket{m_2|\xhat{S}_x|m_3}$:
  Calculamos las $\braket{m_2|m_3\pm 1}$
  {\small
  \begin{align*}
    \braket{m_2|m_3 - 1}
    &= \braket{+1/2|-1/2-1}
    = \braket{+1/2|-3/2} = 0\\
     \braket{m_2|m_3 + 1}
    &= \braket{+1/2|-1/2+1}
    = \braket{+1/2|+1/2} = 1
  \end{align*}
  }
  {\footnotesize
  \begin{align*}
    \braket{m_2|\xhat{S}_x|m_3}
    &= \frac{\hbar}{2}\left(
      \sqrt{s\,(s+1)-m_3^2+m_3}\,\,\cancelout{\braket{m_2|m_3-1}}
      \,+\,
      \sqrt{s\,(s+1)-m_3^2-m_3}\,\,\braket{m_2|m_3+1}\right)\\
    &=
      \frac{\hbar}{2}
      \sqrt{\frac{15}{4}-\left(-\frac{1}{2}\right)^2-\left(-\frac{1}{2}\right)}
      =\frac{\hbar}{2}\sqrt{\frac{15}{4}-\frac{1}{4}+\frac{2}{4}}
      =\frac{\hbar}{2}\cdot 2
      = \frac{2\hbar}{2}
  \end{align*}
  }
\item $\braket{m_2|\xhat{S}_x|m_4}$:
  Calculamos las $\braket{m_2|m_4\pm 1}$. El valor $-5/2$ en rojo indica que
  hemos sobrepasado el valor propio superior, $-3/2$
  {\small
  \begin{align*}
    \braket{m_2|m_4 - 1}
    &= \braket{+1/2|-3/2-1}
    = \braket{+1/2|\textcolor{red!80!black}{-5/2}} = 0\\
     \braket{m_2|m_4 + 1}
    &= \braket{+1/2|-3/2+1}
    = \braket{+1/2|-1/2} = 0
  \end{align*}
  }
  {\footnotesize
  \begin{align*}
    \braket{m_2|\xhat{S}_x|m_4}
    &= \frac{\hbar}{2}\left(
      \sqrt{s\,(s+1)-m_4^2+m_4}\,\,\cancelout{\braket{m_2|m_4-1}}
      \,+\,
      \sqrt{s\,(s+1)-m_4^2-m_4}\,\,\cancelout{\braket{m_2|m_4+1}}\right)\\
    &=
      0
  \end{align*}
  }
\item $\braket{m_3|\xhat{S}_x|m_1}$:
  Calculamos las $\braket{m_3|m_1\pm 1}$. El valor $5/2$ en rojo indica que
  hemos sobrepasado el valor propio superior, $3/2$
  {\small
  \begin{align*}
    \braket{m_3|m_1 - 1}
    &= \braket{-1/2|3/2-1}
    = \braket{-1/2|+1/2} = 0\\
     \braket{m_3|m_1 + 1}
    &= \braket{-1/2|3/2+1}
    = \braket{-1/2|\textcolor{red!80}{5/2}} = 0
  \end{align*}
  }
  {\footnotesize
  \begin{align*}
    \braket{m_3|\xhat{S}_x|m_1}
    &= \frac{\hbar}{2}\left(
      \sqrt{s\,(s+1)-m_1^2+m_1}\,\,\braket{m_3|m_1-1}
      \,+\,
      \sqrt{s\,(s+1)-m_1^2-m_1}\,\,\cancelout{\braket{m_3|m_1+1}}\right)\\
    &=
      0
  \end{align*}
  }
\item $\braket{m_3|\xhat{S}_x|m_2}$:
  Calculamos las $\braket{m_3|m_2\pm 1}$
  {\small
  \begin{align*}
    \braket{m_3|m_2 - 1}
    &= \braket{-1/2|1/2-1}
    = \braket{-1/2|-1/2} = 1\\
     \braket{m_3|m_2 + 1}
    &= \braket{-1/2|1/2+1}
    = \braket{-1/2|3/2} = 0
  \end{align*}
  }
  {\footnotesize
  \begin{align*}
    \braket{m_3|\xhat{S}_x|m_2}
    &= \frac{\hbar}{2}\left(
      \sqrt{s\,(s+1)-m_2^2+m_2}\,\,\braket{m_3|m_2-1}
      \,+\,
      \sqrt{s\,(s+1)-m_2^2-m_2}\,\,\cancelout{\braket{m_3|m_2+1}}\right)\\
    &=
      \frac{\hbar}{2}
      \sqrt{\frac{15}{4}-\left(\frac{1}{2}\right)^2+\frac{1}{2}}
      =\frac{\hbar}{2}\sqrt{\frac{15}{4}-\frac{1}{4}+\frac{2}{4}}
      = \frac{\hbar}{2}\cdot 2
      = \frac{2\hbar}{2}
  \end{align*}
  }
\item $\braket{m_3|\xhat{S}_x|m_4}$:
  Calculamos las $\braket{m_3|m_4\pm 1}$. El valor $-5/2$ en rojo indica que
  hemos sobrepasado el valor propio superior, $-3/2$
  {\small
  \begin{align*}
    \braket{m_3|m_4 - 1}
    &= \braket{-1/2|-3/2-1}
    = \braket{-1/2|\textcolor{red!80!black}{-5/2}} = 0\\
     \braket{m_3|m_4 + 1}
    &= \braket{-1/2|-3/2+1}
    = \braket{-1/2|-1/2} = 1
  \end{align*}
  }
  {\footnotesize
  \begin{align*}
    \braket{m_3|\xhat{S}_x|m_4}
    &= \frac{\hbar}{2}\left(
      \sqrt{s\,(s+1)-m_4^2+m_4}\,\,\cancelout{\braket{m_3|m_4-1}}
      \,+\,
      \sqrt{s\,(s+1)-m_4^2-m_4}\,\,\braket{m_3|m_4+1}\right)\\
    &=
      \frac{\hbar}{2}
      \sqrt{\frac{15}{4}-\left(-\frac{3}{2}\right)^2-\left(-\frac{3}{2}\right)}
      =\frac{\hbar}{2}\sqrt{\frac{15}{4}-\frac{9}{4}+\frac{6}{4}}
      = \frac{\sqrt{3}\hbar}{2}
  \end{align*}
  }
\item $\braket{m_4|\xhat{S}_x|m_1}$:
  Calculamos las $\braket{m_4|m_1\pm 1}$. El valor $5/2$ en rojo indica que
  hemos sobrepasado el valor propio superior, $3/2$
  {\small
  \begin{align*}
    \braket{m_4|m_1 - 1}
    &= \braket{-3/2|3/2-1}
    = \braket{-3/2|+1/2} = 0\\
     \braket{m_4|m_1 + 1}
    &= \braket{-3/2|3/2+1}
    = \braket{-3/2|\textcolor{red!80}{5/2}} = 0
  \end{align*}
  }
  {\footnotesize
  \begin{align*}
    \braket{m_4|\xhat{S}_x|m_1}
    &= \frac{\hbar}{2}\left(
      \sqrt{s\,(s+1)-m_1^2+m_1}\,\,\braket{m_4|m_1-1}
      \,+\,
      \sqrt{s\,(s+1)-m_1^2-m_1}\,\,\cancelout{\braket{m_4|m_1+1}}\right)\\
    &=
      0
  \end{align*}
  }
\item $\braket{m_4|\xhat{S}_x|m_2}$:
  Calculamos las $\braket{m_4|m_2\pm 1}$
  {\small
  \begin{align*}
    \braket{m_4|m_2 - 1}
    &= \braket{-3/2|1/2-1}
    = \braket{-3/2|-1/2} = 0\\
     \braket{m_4|m_2 + 1}
    &= \braket{-3/2|1/2+1}
    = \braket{-3/2|3/2} = 0
  \end{align*}
  }
  {\footnotesize
  \begin{align*}
    \braket{m_4|\xhat{S}_x|m_2}
    &= \frac{\hbar}{2}\left(
      \sqrt{s\,(s+1)-m_2^2+m_2}\,\,\cancelout{\braket{m_4|m_2-1}}
      \,+\,
      \sqrt{s\,(s+1)-m_2^2-m_2}\,\,\cancelout{\braket{m_4|m_2+1}}\right)\\
    &=
      0
  \end{align*}
  }
\item $\braket{m_4|\xhat{S}_x|m_3}$:
  Calculamos las $\braket{m_4|m_3\pm 1}$
  {\small
  \begin{align*}
    \braket{m_4|m_3 - 1}
    &= \braket{-3/2|-1/2-1}
    = \braket{-3/2|-3/2} = 1\\
     \braket{m_4|m_3 + 1}
    &= \braket{-3/2|-1/2+1}
    = \braket{-3/2|+1/2} = 0
  \end{align*}
  }
  {\footnotesize
  \begin{align*}
    \braket{m_4|\xhat{S}_x|m_3}
    &= \frac{\hbar}{2}\left(
      \sqrt{s\,(s+1)-m_3^2+m_3}\,\,\braket{m_4|m_3-1}
      \,+\,
      \sqrt{s\,(s+1)-m_3^2-m_3}\,\,\cancelout{\braket{m_4|m_3+1}}\right)\\
    &=
      \frac{\hbar}{2}
      \sqrt{\frac{15}{4}-\left(-\frac{1}{2}\right)^2+\left(-\frac{1}{2}\right)}
      =\frac{\hbar}{2}\sqrt{\frac{15}{4}-\frac{1}{4}-\frac{2}{4}}
      = \frac{\sqrt{3}\hbar}{2}
  \end{align*}
  }

  La matriz queda
  \begin{equation}\label{eq:spn-spin-3/2-Sx}
    \xhat{S}_x
%    =
%    \begin{pmatrix}
%      0 & \sqrt{3}\hbar/2 & 0 & 0\\
%      \sqrt{3}\hbar/2 & 0 & 2\hbar/2 & 0\\
%      0 & 2\hbar/2 & 0 & 0\\
%      0 & 0 & \sqrt{3}\hbar/2 & 0
%    \end{pmatrix}
    = \frac{\hbar}{2}
    \begin{pmatrix}
      0 & \sqrt{3} & 0 & 0\\
      \sqrt{3} & 0 & 2 & 0\\
      0 & 2 & 0 & \sqrt{3}\\
      0 & 0 & \sqrt{3} & 0
    \end{pmatrix}
  \end{equation}
\end{itemize}

\subsubsection{Operador \mathinhead{\xhat{S}_y}{spin3/2Sy}}
Ahora le toca el turno a $\xhat{S}_y$. Sabemos por~(\ref{eq:ssplus1Spin3/2})
que $s(s+1) = 15/4$.
Además, recordamos que hemos cambiado el orden de los valores
propios, empezando por el máximo, $m_1=3/2$, $m_2=1/2$, $m_3=-1/2$
y $m_4=-3/2$.
Por esto no podemos suponer, por ejemplo, que $m_1+1 = m_2$;
lo que trae como consecuencia que para calcular los productos escalares
de funciones propias de $\xhat{S}_z$ debamos, bien sustituir las $m_i$
por su valor $3/2$, $1/2$, $-1/2$ o $-3/2$, o bien cambiar el signo
$m_i\pm 1$ por $m_i\mp 1$ en los productos escalares.
Seguiremos el primer criterio
\begin{itemize}
\item Los elementos diagonales $\braket{m_i|\xhat{S}_y|m_i}$ valen cero
  porque $m_i \ne m_i -1$ y $m_i \ne m_i +1$ y los productos escalares
  se anulan, $\braket{m_i,\,m_i-1} = 0$ y $\braket{m_i,\,m_i+1} = 0$
  {\footnotesize
  \begin{align*}
    \braket{m_i|\xhat{S}_y|m_i}
    &= \frac{i\hbar}{2}\left(
      \sqrt{s\,(s+1)-m_i^2+m_i}\,\,\cancelout{\braket{m_i|m_i-1}}
      \,-\,
      \sqrt{s\,(s+1)-m_i^2-m_i}\,\,\cancelout{\braket{m_i|m_2+1}}\right)\\
    &=
      0
  \end{align*}
  }
\item $\braket{m_1|\xhat{S}_y|m_2}$:
  Calculamos las $\braket{m_1|m_2\pm 1}$
  {\small
  \begin{align*}
    \braket{m_1|m_2 - 1}
    &= \braket{3/2|1/2-1}
    = \braket{3/2|-1/2} = 0\\
     \braket{m_1|m_2 + 1}
    &= \braket{3/2|1/2+1}
    = \braket{3/2|3/2} = 1
  \end{align*}
  }
  {\footnotesize
  \begin{align*}
    \braket{m_1|\xhat{S}_y|m_2}
    &= \frac{i\hbar}{2}\left(
      \sqrt{s\,(s+1)-m_2^2+m_2}\,\,\cancelout{\braket{m_1|m_2-1}}
      \,-\,
      \sqrt{s\,(s+1)-m_2^2-m_2}\,\,\braket{m_1|m_2+1}\right)\\
    &=
      -\frac{i\hbar}{2}
      \sqrt{\frac{15}{4}-\left(\frac{1}{2}\right)^2-\frac{1}{2}}
      =-\frac{i\hbar}{2}\sqrt{\frac{15}{4}-\frac{1}{4}-\frac{2}{4}}
      = -\frac{i\sqrt{3}\hbar}{2}
  \end{align*}
  }
\item $\braket{m_1|\xhat{S}_y|m_3}$:
  Calculamos las $\braket{m_1|m_3\pm 1}$
  {\small
  \begin{align*}
    \braket{m_1|m_3 - 1}
    &= \braket{3/2|-1/2-1}
    = \braket{3/2|-3/2} = 0\\
     \braket{m_1|m_3 + 1}
    &= \braket{3/2|-1/2+1}
    = \braket{3/2|+1/2} = 0
  \end{align*}
  }
  {\footnotesize
  \begin{align*}
    \braket{m_1|\xhat{S}_y|m_3}
    &= \frac{i\hbar}{2}\left(
      \sqrt{s\,(s+1)-m_3^2+m_3}\,\,\cancelout{\braket{m_1|m_3-1}}
      \,-\,
      \sqrt{s\,(s+1)-m_3^2-m_3}\,\,\cancelout{\braket{m_1|m_3+1}}\right)\\
    &=
      0
  \end{align*}
  }
\item $\braket{m_1|\xhat{S}_y|m_4}$:
  Calculamos las $\braket{m_1|m_4\pm 1}$. El valor $-5/2$ en rojo indica que
  hemos sobrepasado el valor propio inferior, $-3/2$
  {\small
  \begin{align*}
    \braket{m_1|m_4 - 1}
    &= \braket{3/2|-3/2-1}
    = \braket{3/2|\textcolor{red!80!black}{-5/2}} = 0\\
     \braket{m_1|m_4 + 1}
    &= \braket{3/2|-3/2+1}
    = \braket{3/2|-1/2} = 0
  \end{align*}
  }
  {\footnotesize
  \begin{align*}
    \braket{m_1|\xhat{S}_y|m_4}
    &= \frac{i\hbar}{2}\left(
      \sqrt{s\,(s+1)-m_4^2+m_4}\,\,\cancelout{\braket{m_1|m_4-1}}
      \,-\,
      \sqrt{s\,(s+1)-m_4^2-m_4}\,\,\cancelout{\braket{m_1|m_4+1}}\right)\\
    &=
      0
  \end{align*}
  }
\item $\braket{m_2|\xhat{S}_y|m_1}$:
  Calculamos las $\braket{m_2|m_1\pm 1}$. El valor $5/2$ en rojo indica que
  hemos sobrepasado el valor propio superior, $3/2$
  {\small
  \begin{align*}
    \braket{m_2|m_1 - 1}
    &= \braket{+1/2|3/2-1}
    = \braket{+1/2|+1/2} = 1\\
     \braket{m_2|m_1 + 1}
    &= \braket{+1/2|3/2+1}
    = \braket{3/2|\textcolor{red!80}{5/2}} = 0
  \end{align*}
  }
  {\footnotesize
  \begin{align*}
    \braket{m_2|\xhat{S}_y|m_1}
    &= \frac{i\hbar}{2}\left(
      \sqrt{s\,(s+1)-m_1^2+m_1}\,\,\braket{m_2|m_1-1}
      \,-\,
      \sqrt{s\,(s+1)-m_1^2-m_1}\,\,\cancelout{\braket{m_2|m_1+1}}\right)\\
    &=
      \frac{i\hbar}{2}
      \sqrt{\frac{15}{4}-\left(\frac{3}{2}\right)^2+\frac{3}{2}}
      =\frac{i\hbar}{2}\sqrt{\frac{15}{4}-\frac{9}{4}+\frac{6}{4}}
      = \frac{i\sqrt{3}\hbar}{2}
  \end{align*}
  }
\item $\braket{m_2|\xhat{S}_y|m_3}$:
  Calculamos las $\braket{m_2|m_3\pm 1}$
  {\small
  \begin{align*}
    \braket{m_2|m_3 - 1}
    &= \braket{+1/2|-1/2-1}
    = \braket{+1/2|-3/2} = 0\\
     \braket{m_2|m_3 + 1}
    &= \braket{+1/2|-1/2+1}
    = \braket{+1/2|+1/2} = 1
  \end{align*}
  }
  {\footnotesize
  \begin{align*}
    \braket{m_2|\xhat{S}_y|m_3}
    &= \frac{i\hbar}{2}\left(
      \sqrt{s\,(s+1)-m_3^2+m_3}\,\,\cancelout{\braket{m_2|m_3-1}}
      \,-\,
      \sqrt{s\,(s+1)-m_3^2-m_3}\,\,\braket{m_2|m_3+1}\right)\\
    &=
      -\frac{i\hbar}{2}
      \sqrt{\frac{15}{4}-\left(-\frac{1}{2}\right)^2-\left(-\frac{1}{2}\right)}
      =-\frac{i\hbar}{2}\sqrt{\frac{15}{4}-\frac{1}{4}+\frac{2}{4}}
      =-\frac{i\hbar}{2}\cdot 2
      = -\frac{2\hbar}{2}
  \end{align*}
  }
\item $\braket{m_2|\xhat{S}_y|m_4}$:
  Calculamos las $\braket{m_2|m_4\pm 1}$. El valor $-5/2$ en rojo indica que
  hemos sobrepasado el valor propio superior, $-3/2$
  {\small
  \begin{align*}
    \braket{m_2|m_4 - 1}
    &= \braket{+1/2|-3/2-1}
    = \braket{+1/2|\textcolor{red!80!black}{-5/2}} = 0\\
     \braket{m_2|m_4 + 1}
    &= \braket{+1/2|-3/2+1}
    = \braket{+1/2|-1/2} = 0
  \end{align*}
  }
  {\footnotesize
  \begin{align*}
    \braket{m_2|\xhat{S}_y|m_4}
    &= \frac{i\hbar}{2}\left(
      \sqrt{s\,(s+1)-m_4^2+m_4}\,\,\cancelout{\braket{m_2|m_4-1}}
      \,-\,
      \sqrt{s\,(s+1)-m_4^2-m_4}\,\,\cancelout{\braket{m_2|m_4+1}}\right)\\
    &=
      0
  \end{align*}
  }
\item $\braket{m_3|\xhat{S}_y|m_1}$:
  Calculamos las $\braket{m_3|m_1\pm 1}$. El valor $5/2$ en rojo indica que
  hemos sobrepasado el valor propio superior, $3/2$
  {\small
  \begin{align*}
    \braket{m_3|m_1 - 1}
    &= \braket{-1/2|3/2-1}
    = \braket{-1/2|+1/2} = 0\\
     \braket{m_3|m_1 + 1}
    &= \braket{-1/2|3/2+1}
    = \braket{-1/2|\textcolor{red!80}{5/2}} = 0
  \end{align*}
  }
  {\footnotesize
  \begin{align*}
    \braket{m_3|\xhat{S}_y|m_1}
    &= \frac{i\hbar}{2}\left(
      \sqrt{s\,(s+1)-m_1^2+m_1}\,\,\braket{m_3|m_1-1}
      \,-\,
      \sqrt{s\,(s+1)-m_1^2-m_1}\,\,\cancelout{\braket{m_3|m_1+1}}\right)\\
    &=
      0
  \end{align*}
  }
\item $\braket{m_3|\xhat{S}_y|m_2}$:
  Calculamos las $\braket{m_3|m_2\pm 1}$
  {\small
  \begin{align*}
    \braket{m_3|m_2 - 1}
    &= \braket{-1/2|1/2-1}
    = \braket{-1/2|-1/2} = 1\\
     \braket{m_3|m_2 + 1}
    &= \braket{-1/2|1/2+1}
    = \braket{-1/2|3/2} = 0
  \end{align*}
  }
  {\footnotesize
  \begin{align*}
    \braket{m_3|\xhat{S}_y|m_2}
    &= \frac{i\hbar}{2}\left(
      \sqrt{s\,(s+1)-m_2^2+m_2}\,\,\braket{m_3|m_2-1}
      \,-\,
      \sqrt{s\,(s+1)-m_2^2-m_2}\,\,\cancelout{\braket{m_3|m_2+1}}\right)\\
    &=
      \frac{i\hbar}{2}
      \sqrt{\frac{15}{4}-\left(\frac{1}{2}\right)^2+\frac{1}{2}}
      =\frac{i\hbar}{2}\sqrt{\frac{15}{4}-\frac{1}{4}+\frac{2}{4}}
      = \frac{i\hbar}{2}\cdot 2
      = \frac{i2\hbar}{2}
  \end{align*}
  }
\item $\braket{m_3|\xhat{S}_y|m_4}$:
  Calculamos las $\braket{m_3|m_4\pm 1}$. El valor $-5/2$ en rojo indica que
  hemos sobrepasado el valor propio superior, $-3/2$
  {\small
  \begin{align*}
    \braket{m_3|m_4 - 1}
    &= \braket{-1/2|-3/2-1}
    = \braket{-1/2|\textcolor{red!80!black}{-5/2}} = 0\\
     \braket{m_3|m_4 + 1}
    &= \braket{-1/2|-3/2+1}
    = \braket{-1/2|-1/2} = 1
  \end{align*}
  }
  {\footnotesize
  \begin{align*}
    \braket{m_3|\xhat{S}_y|m_4}
    &= \frac{i\hbar}{2}\left(
      \sqrt{s\,(s+1)-m_4^2+m_4}\,\,\cancelout{\braket{m_3|m_4-1}}
      \,-\,
      \sqrt{s\,(s+1)-m_4^2-m_4}\,\,\braket{m_3|m_4+1}\right)\\
    &=
      \frac{i\hbar}{2}
      \sqrt{\frac{15}{4}-\left(-\frac{3}{2}\right)^2-\left(-\frac{3}{2}\right)}
      =-\frac{i\hbar}{2}\sqrt{\frac{15}{4}-\frac{9}{4}+\frac{6}{4}}
      = -\frac{i\sqrt{3}\hbar}{2}
  \end{align*}
  }
\item $\braket{m_4|\xhat{S}_y|m_1}$:
  Calculamos las $\braket{m_4|m_1\pm 1}$. El valor $5/2$ en rojo indica que
  hemos sobrepasado el valor propio superior, $3/2$
  {\small
  \begin{align*}
    \braket{m_4|m_1 - 1}
    &= \braket{-3/2|3/2-1}
    = \braket{-3/2|+1/2} = 0\\
     \braket{m_4|m_1 + 1}
    &= \braket{-3/2|3/2+1}
    = \braket{-3/2|\textcolor{red!80}{5/2}} = 0
  \end{align*}
  }
  {\footnotesize
  \begin{align*}
    \braket{m_4|\xhat{S}_y|m_1}
    &= \frac{i\hbar}{2}\left(
      \sqrt{s\,(s+1)-m_1^2+m_1}\,\,\braket{m_4|m_1-1}
      \,-\,
      \sqrt{s\,(s+1)-m_1^2-m_1}\,\,\cancelout{\braket{m_4|m_1+1}}\right)\\
    &=
      0
  \end{align*}
  }
\item $\braket{m_4|\xhat{S}_y|m_2}$:
  Calculamos las $\braket{m_4|m_2\pm 1}$
  {\small
  \begin{align*}
    \braket{m_4|m_2 - 1}
    &= \braket{-3/2|1/2-1}
    = \braket{-3/2|-1/2} = 0\\
     \braket{m_4|m_2 + 1}
    &= \braket{-3/2|1/2+1}
    = \braket{-3/2|3/2} = 0
  \end{align*}
  }
  {\footnotesize
  \begin{align*}
    \braket{m_4|\xhat{S}_y|m_2}
    &= \frac{i\hbar}{2}\left(
      \sqrt{s\,(s+1)-m_2^2+m_2}\,\,\cancelout{\braket{m_4|m_2-1}}
      \,-\,
      \sqrt{s\,(s+1)-m_2^2-m_2}\,\,\cancelout{\braket{m_4|m_2+1}}\right)\\
    &=
      0
  \end{align*}
  }
\item $\braket{m_4|\xhat{S}_y|m_3}$:
  Calculamos las $\braket{m_4|m_3\pm 1}$
  {\small
  \begin{align*}
    \braket{m_4|m_3 - 1}
    &= \braket{-3/2|-1/2-1}
    = \braket{-3/2|-3/2} = 1\\
     \braket{m_4|m_3 + 1}
    &= \braket{-3/2|-1/2+1}
    = \braket{-3/2|+1/2} = 0
  \end{align*}
  }
  {\footnotesize
  \begin{align*}
    \braket{m_4|\xhat{S}_y|m_3}
    &= \frac{i\hbar}{2}\left(
      \sqrt{s\,(s+1)-m_3^2+m_3}\,\,\braket{m_4|m_3-1}
      \,-\,
      \sqrt{s\,(s+1)-m_3^2-m_3}\,\,\cancelout{\braket{m_4|m_3+1}}\right)\\
    &=
      \frac{i\hbar}{2}
      \sqrt{\frac{15}{4}-\left(-\frac{1}{2}\right)^2+\left(-\frac{1}{2}\right)}
      =\frac{i\hbar}{2}\sqrt{\frac{15}{4}-\frac{1}{4}-\frac{2}{4}}
      = \frac{i\sqrt{3}\hbar}{2}
  \end{align*}
  }

  La matriz queda
  \begin{equation}\label{eq:spn-spin-3/2-Sy}
    \xhat{S}_y
%    =
%    \begin{pmatrix}
%      0 & -i\sqrt{3}\hbar/2 & 0 & 0\\
%      i\sqrt{3}\hbar/2 & 0 & -i2\hbar/2 & 0\\
%      0 & i2\hbar/2 & 0 & -i\sqrt{3}\hbar/2\\
%      0 & 0 & i\sqrt{3}\hbar/2 & 0
%    \end{pmatrix}
    = \frac{i\hbar}{2}
    \begin{pmatrix}
      0 & -\sqrt{3} & 0 & 0\\
      \sqrt{3} & 0 & -2 & 0\\
      0 & 2 & 0 & -\sqrt{3}\\
      0 & 0 & \sqrt{3} & 0
    \end{pmatrix}
  \end{equation}
\end{itemize}


\subsubsection{Operador \mathinhead{\xhat{S}^2}{spin3/2S2}}
Terminamos con $\xhat{S}^2$. Según~(\ref{eq:spn-S2casimir-v2})
\begin{equation}\label{eq:spn-spin-3/2-S2}
  \xhat{S}^2
  =
  \hbar^2 s\,(s+1)\,\mmm{I}
  =
  \frac{15\hbar^2}{4}
  \begin{pmatrix}
    1 & 0 & 0 & 0\\
    0 & 1 & 0 & 0\\
    0 & 0 & 1 & 0\\
    0 & 0 & 0 & 1
  \end{pmatrix}
  = \frac{15\hbar^2}{4}\,\mmm{I}
\end{equation}


\subsection{Resumen de las matrices de spin 1/2, 1 y 3/2}
\begin{itemize}
\item Matrices de spin $1/2$
  {\footnotesize
  \begin{align}\label{eq:spn-matrices-SxSySz-spin-1/2}
      \xhat{S}_x
      &= \frac{\hbar}{2}
      \begin{pmatrix}
        0 & 1\\
        1 & 0
      \end{pmatrix}
      = \frac{\hbar}{2}\mmm{\sigma}_x
      ;\hspace{1em}
      \xhat{S}_y
      = \frac{\hbar}{2}
      \begin{pmatrix}
        0 & -i\\
        i & 0
      \end{pmatrix}
      = \frac{\hbar}{2}\mmm{\sigma}_y
      ;\hspace{1em}
      \xhat{S}_z
      = \frac{\hbar}{2}
      \begin{pmatrix}
        1 & 0\\
        0   & -1
      \end{pmatrix}
      = \frac{\hbar}{2}\,\mmmg{\sigma}_z\\
      \xhat{S}^2
      &=
      \frac{3\hbar^2}{4}
      \begin{pmatrix}
        1 & 0 \\
        0 & 1 
      \end{pmatrix}
      = \frac{3\hbar^2}{4} \mmm{I}
  \end{align}
  }
  
\item Matrices de spin $1$
  {\footnotesize
    \begin{align}\label{eq:spn-matrices-SxSySz-spin-1}
      \xhat{S}_x
      &=
        \frac{\hbar}{\sqrt{2}}
        \begin{pmatrix}
          0 & 1 & 0\\
          1 & 0 & 1\\
          0 & 1 & 0
        \end{pmatrix}
      ;\hspace{1em}
      \xhat{S}_y
      = \frac{i\hbar}{\sqrt{2}}
        \begin{pmatrix}
          0 & -1 & 0\\
          1 & 0 & -1\\
          0 & 1 & 0
        \end{pmatrix}
      ;\hspace{1em}
      \xhat{S}_z
      = \hbar\,\begin{pmatrix}
        1 & 0 & 0\\
        0 & 0 & 0\\
        0 & 0 & -1
      \end{pmatrix}\\
      \xhat{S}^2
      &=
        2\hbar^2
        \begin{pmatrix}
          1 & 0 & 0\\
          0 & 1 & 0\\
          0 & 0 & 1
        \end{pmatrix}
        = 2\hbar^2\mmm{I}
    \end{align}
  }

\item Matrices de spin $3/2$
  {\scriptsize
    \begin{align}\label{eq:spn-matrices-SxSySz-spin-3/2}
      \xhat{S}_x
      &= \frac{\hbar}{2}
      \begin{pmatrix}
        0 & \sqrt{3} & 0 & 0\\
        \sqrt{3} & 0 & 2 & 0\\
        0 & 2 & 0 & \sqrt{3}\\
        0 & 0 & \sqrt{3} & 0
      \end{pmatrix}
      ;\hspace{.8em}
      \xhat{S}_y
      = \frac{i\hbar}{2}
      \begin{pmatrix}
        0 & -\sqrt{3} & 0 & 0\\
        \sqrt{3} & 0 & -2 & 0\\
        0 & 2 & 0 & -\sqrt{3}\\
        0 & 0 & \sqrt{3} & 0
      \end{pmatrix}
      ;\hspace{.8em}
      \xhat{S}_z
      = \frac{\hbar}{2}\,
     \begin{pmatrix}
       3 & 0 & 0 & 0\\
       0 & 1 & 0 & 0\\
       0 & 0 & -1 & 0\\
       0 & 0 & 0 & -3
     \end{pmatrix}\\
  \xhat{S}^2
  &=
  \frac{15\hbar^2}{4}
  \begin{pmatrix}
    1 & 0 & 0 & 0\\
    0 & 1 & 0 & 0\\
    0 & 0 & 1 & 0\\
    0 & 0 & 0 & 1
  \end{pmatrix}
  = \frac{15\hbar^2}{4}\,\mmm{I}
    \end{align}
  }
\end{itemize}


\section{Jugando con el spin 1/2}

\subsection{Valores propios y vectores propios de
  \mathinhead{\xhat{S}_z}{vpfpSz}}
Queremos calcular los valores propios y funciones propias de $\xhat{S}_z$
\[
  \xhat{S}_z
  = \begin{pmatrix}
    \hbar/2 & 0\\
    0 & -\hbar/2
  \end{pmatrix}
\]

El método más directo en este caso consiste en observar que la matriz
$\xhat{S}_z$ es diagonal. Los valores propios son los elementos de la diagonal
\begin{equation}\label{eq:spn-valprop-Sz}
  \lambda_1 = \hbar/2
  ;\hspace{1em}
  \lambda_2 = -\hbar/2
\end{equation}

Además, cualquier matriz no singular $\mmm{A}$ se puede diagonalizar mediante
\[
  \mmm{D} = \mmm{M}^\trasp \mmm{A} \mmm{M}
\]
donde $\mmm{D}$ es la matriz diagonalizada y $\mmm{M}$ es la matriz ortogonal
formada por los vectores propios de $\mmm{A}$ dispuestos en columnas.
\[
  \mmm{M} =
  \begin{tikzpicture}[baseline={(m.center)}]
    \matrix (m) [
    matrix of math nodes,
    left delimiter=(,
    right delimiter=),
    minimum width=width(-1),
    ]
    {
      \phantom{1} & \phantom{1} \\
      \phantom{1} & \phantom{1} \\
    };
    \draw (m-1-1.north west) rectangle (m-2-1.south east);
    \draw (m-1-2.north west) rectangle (m-2-2.south east);
    \draw (m-1-1.center) node[below] {$\vvv{v}_1$};
    \draw (m-1-2.south) node {$\vvv{v}_2$};
    
  \end{tikzpicture}
\]

Pero nuestra matriz matriz $\xhat{S}_z$ ya está diagonalizada, por lo que
\[
  \xhat{S}_z = \mmm{M}^\trasp\xhat{S}_z\mmm{M}
\]

Concluimos que la matriz formada por los vectores propios $\mmm{M}$, debe
ser la matriz identidad
\[
  \mmm{M}
  =
  \begin{tikzpicture}[baseline={(m.center)}]
    \matrix (m) [
    matrix of math nodes,
    left delimiter=(,
    right delimiter=),
    minimum width=width(-1),
    ]
    {
      \phantom{1} & \phantom{1}\\
      \phantom{1} & \phantom{1}\\
    };
    \draw (m-1-1.north west) rectangle (m-2-1.south east);
    \draw (m-1-2.north west) rectangle (m-2-2.south east);
    \draw (m-1-1.center) node[below] {$\vvv{v}_1$};
    \draw (m-1-2.south) node {$\vvv{v}_2$};
    
  \end{tikzpicture}
  =
  \begin{tikzpicture}[baseline={(m.center)}]
    \matrix (m) [
    matrix of math nodes,
    left delimiter=(,
    right delimiter=),
    minimum width=width(-1),
    nodes={align=right},
    ]
    {
      1 & 0\\
      0 & 1\\
    };
    \draw (m-1-1.north west) rectangle (m-2-1.south east);
    \draw (m-1-2.north west) rectangle (m-2-2.south east);
  \end{tikzpicture}
  =
  \mmm{I}
\]

Lo que nos permite determinar los vectores propios
\begin{equation}\label{eq:spn-vecprop-Sz}
  \vvv{v}_1
  =
  \ket{z+}
  = \begin{pmatrix}
    1 \\ 0
  \end{pmatrix}
  ;\hspace{1em}
  \vvv{v}_2
  =
  \ket{z-}
  = \begin{pmatrix}
    0 \\ 1
  \end{pmatrix}
\end{equation}
donde hemos llamado $\ket{z+}$ a la función propia de $\xhat{S}_z$ con
valor propio $\hbar/2$ y $\ket{z-}$ a la que tiene valor propio $-\hbar/2$.

\subsection{Valores propios y vectores propios de
  \mathinhead{\xhat{S}_x}{vpfpSx}}
Queremos calcular los valores y funciones propias del operador $\xhat{S}_x$
\[
  \xhat{S}_x
  = \begin{pmatrix}
    0 & \hbar/2\\
    \hbar/2 & 0
    \end{pmatrix}
\]

Como no es diagonal tenemos que hallar los valores propios mediante la
ecuación de valores y vectores propios
\[
  \begin{pmatrix}
    0 & \hbar/2\\
    \hbar/2 & 0    
  \end{pmatrix}
  \begin{pmatrix}
    x \\ y
  \end{pmatrix}
  =
  \lambda
  \begin{pmatrix}
    x \\ y
  \end{pmatrix}
\]
\[
  \begin{pmatrix}
    \frac{\hbar}{2}\,y \\ \frac{\hbar}{2}\,x
  \end{pmatrix}
  =
  \begin{pmatrix}
    \lambda x \\ \lambda y
  \end{pmatrix}
\]
\[
  \begin{pmatrix}
    -\lambda x +\frac{\hbar}{2}\,y \\ \frac{\hbar}{2}\,x -\lambda y
  \end{pmatrix}
  =
  \begin{pmatrix}
    0 \\ 0
  \end{pmatrix}
\]

Esto nos deja con dos ecuaciones con tres incógnitas, $\lambda$, $x$ e $y$
\begin{equation}\label{eq:spn-sist-ecuaciones-Sx}
  \begin{cases}
    \begin{array}{rlrll}
      -\lambda x & + &\frac{\hbar}{2} y &= &0\\
      \frac{\hbar}{2} x& - &\lambda y &= &0
    \end{array}
  \end{cases}
\end{equation}

Para que el sistema sea indeterminado, el determinante de los coeficientes
debe ser nulo
\[
  \begin{vmatrix}
    -\lambda & \hbar/2\\
    \hbar/2 & -\lambda
  \end{vmatrix}
  = 0
\]
\[
  \lambda^2 - \frac{\hbar^2}{4} = 0
\]

\[
  \lambda = \pm \frac{\hbar}{2}
\]

Los dos valores propios de $\xhat{S}_x$ son los mismos que los del
operador $\xhat{S}_z$
\begin{equation}\label{eq:spn-valprop-Sx}
  \lambda_1 = \hbar/2
  ;\hspace{1em}
  \lambda_2 = -\hbar/2
\end{equation}

Ahora buscamos los vectores propios.
\begin{itemize}
\item Si tomamos el valor propio $\lambda_1 = \hbar/2$ y sustituimos en
  el sistema de ecuaciones~(\ref{eq:spn-sist-ecuaciones-Sx})
  \[
  \begin{cases}
    \begin{array}{rlrll}
      -\frac{\hbar}{2} x & + &\frac{\hbar}{2} y &= &0\\
      \frac{\hbar}{2} x& - &\frac{\hbar}{2} y &= &0
    \end{array}
  \end{cases}
\]

Obtenemos $y=x$. Así, el vector propio de $\xhat{S}_x$ con valor propio
$\hbar/2$ es
\[
  \ket{x+}
  = N_1
  \begin{pmatrix}
    1 \\ 1
  \end{pmatrix}
\]

\item Ahora elegimos el valor propio $\lambda_2 = -\hbar/2$ y sustituimos en
  el sistema de ecuaciones~(\ref{eq:spn-sist-ecuaciones-Sx})
  \[
    \begin{cases}
      \begin{array}{rlrll}
        \frac{\hbar}{2} x & + &\frac{\hbar}{2} y &= &0\\
        \frac{\hbar}{2} x& + &\frac{\hbar}{2} y &= &0
      \end{array}
    \end{cases}
  \]

Obtenemos $y=-x$. Así, el vector propio de $\xhat{S}_x$ con valor propio
$\hbar/2$ es\footnotemark{}
\[
  \ket{x-}
  = N_2
  \begin{pmatrix}
    -1 \\ 1
  \end{pmatrix}
\]
\footnotetext{No se ha elegido el vector propio
  $\ket{x-} = \begin{pmatrix} 1 \\ -1\end{pmatrix}$ porque queremos que la
  matriz formada por los vectores propios
  $\mmm{M} = |N|^2 \begin{vmatrix} 1 & -1\\ 1 & 1\end{vmatrix}$
  tenga un determinante positivo y sea una rotación en un espacio
  abstracto.
}
\end{itemize}

Ahora normalizamos los vectores propios eligiendo $\theta=0$.
El primero tiene una constante de normalización $N=1/\sqrt{2}$
\[
  \braket{x+|x+}
  =
  N^{*}
  \begin{pmatrix}
    1 & 1
  \end{pmatrix}
  N
  \begin{pmatrix}
    1 \\ 1
  \end{pmatrix}
  =
  |N|^2\, 2
  =
  1
\]
\[
  N = \frac{1}{\sqrt{2}}
\]

El segundo vector propio tiene la misma constante de normalización
\[
  \braket{x-|x-}
  =
  N^{*}
  \begin{pmatrix}
    -1 & 1
  \end{pmatrix}
  N
  \begin{pmatrix}
    -1 \\ 1
  \end{pmatrix}
  =
  |N|^2\, 2
  =
  1
\]
\[
  N = \frac{1}{\sqrt{2}}
\]

Los vectores propios de $\xhat{S}_x$ quedan
\begin{equation}\label{eq:spn-vec-prop-Sx}
  \ket{x+} = \frac{1}{\sqrt{2}}\begin{pmatrix} 1 \\ 1\end{pmatrix}
  ;\hspace{1em}
  \ket{x-} = \frac{1}{\sqrt{2}}\begin{pmatrix} -1 \\ 1\end{pmatrix}
\end{equation}


\subsection{Valores propios y vectores propios de
  \mathinhead{\xhat{S}_y}{vpfpSy}}
Queremos calcular los valores y funciones propias del operador $\xhat{S}_x$
\[
  \xhat{S}_y
  = \begin{pmatrix}
    0 & -i\hbar/2\\
    i\hbar/2 & 0
    \end{pmatrix}
\]

Como no es diagonal tenemos que hallar los valores propios mediante la
ecuación de valores y vectores propios
\[
  \begin{pmatrix}
    0 & -i\hbar/2\\
    i\hbar/2 & 0    
  \end{pmatrix}
  \begin{pmatrix}
    x \\ y
  \end{pmatrix}
  =
  \lambda
  \begin{pmatrix}
    x \\ y
  \end{pmatrix}
\]
\[
  \begin{pmatrix}
    \frac{-i\hbar}{2}\,y \\ \frac{i\hbar}{2}\,x
  \end{pmatrix}
  =
  \begin{pmatrix}
    \lambda x \\ \lambda y
  \end{pmatrix}
\]
\[
  \begin{pmatrix}
    -\lambda x -i\frac{\hbar}{2}\,y \\ i\frac{\hbar}{2}\,x -\lambda y
  \end{pmatrix}
  =
  \begin{pmatrix}
    0 \\ 0
  \end{pmatrix}
\]

Esto nos deja con dos ecuaciones con tres incógnitas, $\lambda$, $x$ e $y$
\begin{equation}\label{eq:spn-sist-ecuaciones-Sy}
  \begin{cases}
    \begin{array}{rlrll}
      -\lambda x & - &\frac{i\hbar}{2} y &= &0\\
      \frac{i\hbar}{2} x& - &\lambda y &= &0
    \end{array}
  \end{cases}
\end{equation}

Para que el sistema sea indeterminado, el determinante de los coeficientes
debe ser nulo
\[
  \begin{vmatrix}
    -\lambda & -i\hbar/2\\
    i\hbar/2 & -\lambda
  \end{vmatrix}
  = 0
\]
\[
  \lambda^2 - \frac{\hbar^2}{4} = 0
\]

\[
  \lambda = \pm \frac{\hbar}{2}
\]

Los dos valores propios de $\xhat{S}_y$ son los mismos que los de los
operadores $\xhat{S}_y$ y $\xhat{S}_z$
\begin{equation}\label{eq:spn-valprop-Sy}
  \lambda_1 = \hbar/2
  ;\hspace{1em}
  \lambda_2 = -\hbar/2
\end{equation}

Ahora buscamos los vectores propios.
\begin{itemize}
\item Si tomamos el valor propio $\lambda_1 = \hbar/2$ y sustituimos en
  el sistema de ecuaciones~(\ref{eq:spn-sist-ecuaciones-Sy})
  \[
  \begin{cases}
    \begin{array}{rlrll}
      -\frac{\hbar}{2} x & - &\frac{i\hbar}{2} y &= &0\\
      \frac{i\hbar}{2} x& - &\frac{\hbar}{2} y &= &0
    \end{array}
  \end{cases}
\]

Obtenemos $y=ix$. Así, el vector propio de $\xhat{S}_x$ con valor propio
$\hbar/2$ es
\[
  \ket{y+}
  = N_1
  \begin{pmatrix}
    1 \\ i
  \end{pmatrix}
\]

\item Ahora elegimos el valor propio $\lambda_2 = -\hbar/2$ y sustituimos en
  el sistema de ecuaciones~(\ref{eq:spn-sist-ecuaciones-Sy})
  \[
    \begin{cases}
      \begin{array}{rlrll}
        \frac{\hbar}{2} x & - &\frac{i\hbar}{2} y &= &0\\
        \frac{i\hbar}{2} x& + &\frac{\hbar}{2} y &= &0
      \end{array}
    \end{cases}
  \]

Obtenemos $y=-ix$. Así, el vector propio de $\xhat{S}_x$ con valor propio
$\hbar/2$ es%\footnotemark{}
\[
  \ket{y-}
  = N_2
  \begin{pmatrix}
    i \\ 1
  \end{pmatrix}
\]
donde, de nuevo se ha preferido que el determinante de los vectores
propios sea positivo para que represente una rotación en $\symbb{C}^2$.
\end{itemize}

Ahora normalizamos los vectores propios eligiendo $\theta=0$.
El primero tiene una constante de normalización $N=1/\sqrt{2}$
\[
  \braket{y+|y+}
  =
  N^{*}
  \begin{pmatrix}
    1 & -i
  \end{pmatrix}
  N
  \begin{pmatrix}
    1 \\ i
  \end{pmatrix}
  =
  |N|^2\, 2
  =
  1
\]
\[
  N = \frac{1}{\sqrt{2}}
\]

El segundo vector propio tiene la misma constante de normalización
\[
  \braket{y-|y-}
  =
  N^{*}
  \begin{pmatrix}
    -i & 1
  \end{pmatrix}
  N
  \begin{pmatrix}
    i \\ 1
  \end{pmatrix}
  =
  |N|^2\, 2
  =
  1
\]
\[
  N = \frac{1}{\sqrt{2}}
\]

Los vectores propios de $\xhat{S}_y$ quedan
\begin{equation}\label{eq:spn-vec-prop-Sy}
  \ket{y+} = \frac{1}{\sqrt{2}}\begin{pmatrix} 1 \\ i\end{pmatrix}
  ;\hspace{1em}
  \ket{y-} = \frac{1}{\sqrt{2}}\begin{pmatrix} i \\ 1\end{pmatrix}
\end{equation}


\subsection{La medida del spin 1/2}
Nos centraremos en el spin 1/2 debido a su importancia --el electrón
es una partícula con este spin-- y a su sencillez. La mayoría de las
técnicas que aplicaremos sirven también para otros valores del spin.

La descripción física de una partícula con spin se especifica completamente
mediante su función de onda
\[
  \Psi = \phi(x,y,z) \otimes \ket{\varphi}
\]

El cuadrado de $\phi(x,y,z)$ es la densidad de probabilidad de encontrar a la
partícula en una cierta región del espacio. En cambio, $\ket{\varphi}$ 
está relacionada con la probabilidad de que el spin tenga un valor determinado
en una cierta dirección.

Nos olvidamos conscientemente de la parte espacial para centrarnos en el spin.
Nos proponemos calcular la probabilidad de que al medir la componente del spin
según una cierta dirección nos dé un determinado valor.

La dimensión del espacio de Hilbert de este tipo de spin es $2$
\[
  N = 2s + 1 = 2\,\frac{1}{2} + 1 = 2
\]

\subsubsection{Medida de \mathinhead{\xhat{S}_z}{mdSz} para
una \mathinhead{\ket{\varphi}}{mpuphi} de ejemplo}
Elegimos una base para este espacio $\symbb{C}^2$, puede ser cualquiera
pero por sencillez se suele elegir la base propia de $\xhat{S}_z$
\[
  B
  = \set{\ket{z+},\ket{z-}}
  = \Set{
  \begin{pmatrix}
    1 \\ 0
  \end{pmatrix}
  , \begin{pmatrix}
    0 \\ 1
    \end{pmatrix}
  }
\]

Supongamos ahora que la función de onda de spin fuera, por ejemplo
\[
  \ket{\varphi}
  = N
  \begin{pmatrix}
    3 \\ i
  \end{pmatrix}
\]

Calcularemos la constante de normalización $N$ obligando a que
$\braket{\varphi|\varphi} = 1$
\[
  \braket{\varphi|\varphi}
  =
  N^{*}
  \begin{pmatrix}
    3 & -i
  \end{pmatrix}
  N
  \begin{pmatrix}
    3 \\ i
  \end{pmatrix}
  =
  |N|^2\,
  (9 + 1)
  =
  10\, |N|^2
  =
  1
\]

Despejamos el módulo de la constante de normalización
\[
  |N| = \frac{1}{\sqrt{10}}
\]

Pero $N$ es un un número complejo y la fase puede ser cualquiera.
En general
\[
  N = \frac{1}{\sqrt{10}}\,e^{i\theta}
\]

Elegimos la fase más sencilla con $\theta = 0$
\[
  N = \frac{1}{\sqrt{10}}
\]

La función de onda queda, como combinación lineal de los vectores propios
de $\xhat{S}_z$
\begin{equation}\label{eq:spn-phioriginal-z+-}
  \ket{\varphi}
  =
  \frac{1}{\sqrt{10}}\,
  \begin{pmatrix}
    3 \\ i
  \end{pmatrix}
  =
  \frac{3}{\sqrt{10}}\,\begin{pmatrix} 1 \\ 0 \end{pmatrix}
  + \frac{i}{\sqrt{10}}\,\begin{pmatrix} 0 \\ 1 \end{pmatrix}
  =
  \frac{3}{\sqrt{10}}\,\ket{z+} + \frac{i}{\sqrt{10}}\,\ket{z-}
\end{equation}

Los coeficientes del desarrollo $3/\sqrt{10}$ e $i\sqrt{10}$ son las amplitudes
de probabilidad de que al medir el spin de la partícula en el eje $z$ se
obtenga, $\hbar/2$ o $-\hbar/2$, respectivamente. Las probabilidades son
el cuadrado de estas amplitudes.

Así, las probabilidades de obtener los distintos resultados de la medida son
\begin{align*}
  P_{+1/2}
  &=
    \left(\frac{3}{\sqrt{10}}\cdot\frac{3}{\sqrt{10}}\right)
    = \frac{9}{10} = 0,9 = \SI{90}{\percent}\\
  P_{-1/2}
  &=
    \left(\frac{-i}{\sqrt{10}}\cdot\frac{i}{\sqrt{10}}\right)
    = \frac{1}{10} = 0,1 = \SI{10}{\percent}
\end{align*}

Con nuestra función de onda del ejemplo es mucho más probable obtener
un spin $+1/2$ cuando se orienta un imán en el eje $z$, aunque
en una medida puntual se podría obtener el spin $-1/2$, aunque sería
mucho menos frecuente.

Podemos decir que la partícula está en una superposición de
estados de la componente $z$ del spin $+1/2$ y $-1/2$ aunque no con igual
probabilidad.
El valor que se mide es completamente aleatorio;
no hay nada en la partícula que nos anticipe la componente $z$ del spin
que mediremos.


\subsubsection{Medida de \mathinhead{\xhat{S}_x}{mdSx} para
una \mathinhead{\ket{\varphi}}{mpuphi} de ejemplo}
También podríamos preguntarnos que dada la función de estado
$\ket{\varphi}$ anterior, cuál sería la probabilidad de obtener
un spin $+\hbar/2$ o $-\hbar/2$ en otra dirección, por ejemplo en el eje $x$.
Necesitamos desarrollar la función de onda en términos de los vectores
propios de $\xhat{S}_x$.
Podemos conseguir este objetivo de dos formas equivalentes:
\begin{itemize}
\item Escribimos los vectores propios de $\xhat{S}_x$ en función de
  las de $\xhat{S}_z$
  \begin{align*}
    \ket{x+}
    &=
      \frac{1}{\sqrt{2}}\,\begin{pmatrix} 1 \\ 1 \end{pmatrix}
    = \frac{1}{\sqrt{2}}\begin{pmatrix} 1 \\ 0 \end{pmatrix}
    +
    \frac{1}{\sqrt{2}}\begin{pmatrix} 0 \\ 1 \end{pmatrix}
    = \frac{1}{\sqrt{2}} \ket{z+} + \frac{1}{\sqrt{2}} \ket{z-}\\
    \ket{x-}
    &=
      \frac{1}{\sqrt{2}}\,\begin{pmatrix} -1 \\ 1 \end{pmatrix}
    = \frac{-1}{\sqrt{2}}\begin{pmatrix} 1 \\ 0 \end{pmatrix}
    +
    \frac{1}{\sqrt{2}}\begin{pmatrix} 0 \\ 1 \end{pmatrix}
    = -\frac{1}{\sqrt{2}} \ket{z+} + \frac{1}{\sqrt{2}} \ket{z-}\\
  \end{align*}

  Si restamos, por un lado, y sumamos por otro las dos ecuaciones, obtenemos
  \begin{align*}
    \ket{x+} - \ket{x-}
    &=
      \sqrt{2} \ket{z+}\\
    \ket{x+} + \ket{x-}
    &=
      \sqrt{2} \ket{z-}
  \end{align*}

  Despejamos $\ket{z+}$ y $\ket{z-}$
  \begin{align}\label{eq:spn-z+x+x-}
    \ket{z+}
    &= \frac{1}{\sqrt{2}}\, (\ket{x+} - \ket{x-})\\
    \label{eq:spn-z-x+x-}
    \ket{z-}
    &= \frac{1}{\sqrt{2}}\, (\ket{x+} + \ket{x-})
  \end{align}

  Sustituimos estos desarrollos de $\ket{z+}$ y $\ket{z-}$ en la expresión
  de la función de onda~(\ref{eq:spn-phioriginal-z+-})
  \[
    \ket{\varphi}
    = \frac{3}{\sqrt{10}}\,\frac{1}{\sqrt{2}} (\ket{x+} - \ket{x-})
    + \frac{i}{\sqrt{10}}\,\frac{1}{\sqrt{2}} (\ket{x+} + \ket{x-})\\
  \]
  
  El vector de estado queda, en términos de los vectores propios de
  $\xhat{S}_x$ 
  \[
    \ket{\varphi}
    = c_1 \ket{x+} + c_2 \ket{x-}   
    = \frac{3+i}{\sqrt{20}} \ket{x+} + \frac{-3+i}{\sqrt{20}} \ket{x-}
  \]

\item También podríamos haber expresado el vector de estado del spin
  en función de los vectores propios de $\xhat{S}_x$ de forma directa
  \[
    \ket{\varphi}
    =
    \frac{1}{\sqrt{10}} \begin{pmatrix} 3 \\ i \end{pmatrix}
    =
    c_1 \ket{x+} + c_2 \ket{x-}
    = c_1 \frac{1}{\sqrt{2}} \begin{pmatrix} 1 \\ 1 \end{pmatrix}
    + c_2 \frac{1}{\sqrt{2}} \begin{pmatrix} -1 \\ 1 \end{pmatrix}
  \]

  Esto nos lleva a un sistema de ecuaciones
  \[
    \begin{cases}
      \begin{array}{rlrll}
        \frac{1}{\sqrt{2}}\,c_1 & - & \frac{1}{\sqrt{2}}\,c_2 &=
        & \frac{3}{\sqrt{10}}\\
        \frac{1}{\sqrt{2}}\,c_1 & + & \frac{1}{\sqrt{2}}\,c_2 &=
        & \frac{i}{\sqrt{10}}
      \end{array}
    \end{cases}
  \]

  Resolviendo el sistema obtenemos las amplitudes de probabilidad
  \[
    c_1 = \frac{3+i}{\sqrt{20}}
    ;\hspace{1em}
    c_2 = \frac{-3+i}{\sqrt{20}}
  \]  
\end{itemize}

Las probabilidades de que el resultado de la medida del spin en la
dirección $x$ sea $+\hbar/2$ o $-\hbar/2$ son
\begin{align*}
  P_{+1/2}
  &=
    \left|\frac{3+i}{\sqrt{20}}\right|^2
    = \left(\frac{3-i}{\sqrt{20}}\cdot\frac{3+i}{\sqrt{20}}\right)
    = \frac{10}{20} = 0,5 = \SI{50}{\percent}\\
  P_{-1/2}
  &=
    \left|\frac{-3+i}{\sqrt{20}}\right|^2      
    = \left(\frac{-3-i}{\sqrt{20}}\cdot\frac{-3+i}{\sqrt{20}}\right)
    = \frac{10}{20} = 0,5 = \SI{50}{\percent}
\end{align*}

\subsubsection{Medida de \mathinhead{\xhat{S}_y}{mdSy} para
una \mathinhead{\ket{\varphi}}{mpuphi} de ejemplo}
También podríamos preguntarnos, dada la función de estado
$\ket{\varphi}$ anterior, cuál sería la probabilidad de obtener
un spin $+\hbar/2$ o $-\hbar/2$ en el eje $y$.
Necesitamos desarrollar la función de onda en términos de los vectores
propios de $\xhat{S}_y$.
Podemos conseguir este objetivo de dos formas equivalentes:
\begin{itemize}
\item Escribimos los vectores propios de $\xhat{S}_y$ en función de
  las de $\xhat{S}_z$
  \begin{align*}
    \ket{y+}
    &=
      \frac{1}{\sqrt{2}}\,\begin{pmatrix} 1 \\ i \end{pmatrix}
    = \frac{i}{\sqrt{2}}\begin{pmatrix} 1 \\ 0 \end{pmatrix}
    +
    \frac{1}{\sqrt{2}}\begin{pmatrix} 0 \\ 1 \end{pmatrix}
    = \frac{1}{\sqrt{2}} \ket{z+} + \frac{i}{\sqrt{2}} \ket{z-}\\
    \ket{y-}
    &=
      \frac{1}{\sqrt{2}}\,\begin{pmatrix} i \\ 1 \end{pmatrix}
    = \frac{i}{\sqrt{2}}\begin{pmatrix} 1 \\ 0 \end{pmatrix}
    +
    \frac{1}{\sqrt{2}}\begin{pmatrix} 0 \\ 1 \end{pmatrix}
    = \frac{i}{\sqrt{2}} \ket{z+} + \frac{1}{\sqrt{2}} \ket{z-}\\
  \end{align*}

  Sumamos la primera multiplicada por $i$ más la segunda, por un lado,
  y la primera más la segunda multiplicada por $i$, obteniendo
  \begin{align*}
    i\ket{y+} + \ket{y-}
    &=
      i\sqrt{2} \ket{z+}\\
    \ket{y+} + i\ket{y-}
    &=
      i\sqrt{2} \ket{z-}
  \end{align*}

  Despejamos $\ket{z+}$ y $\ket{z-}$
  \begin{align}\label{eq:spn-z+y+y-}
    \ket{z+}
    &= \frac{1}{\sqrt{2}}\, (\ket{y+} - i\ket{y-})\\
    \label{eq:spn-z-y+y-}
    \ket{z-}
    &= \frac{1}{\sqrt{2}}\, (-i\ket{y+} + \ket{y-})
  \end{align}

  Sustituimos estos desarrollos de $\ket{z+}$ y $\ket{z-}$ en la expresión
  de la función de onda~(\ref{eq:spn-phioriginal-z+-})
  \[
    \ket{\varphi}
    = \frac{3}{\sqrt{10}}\,\frac{1}{\sqrt{2}} (\ket{y+} - i\ket{y-})
    + \frac{i}{\sqrt{10}}\,\frac{1}{\sqrt{2}} (-i\ket{y+} + \ket{y-})\\
  \]
  
  El vector de estado queda, en términos de los vectores propios de
  $\xhat{S}_y$ 
  \[
    \ket{\varphi}
    = c_1 \ket{y+} + c_2 \ket{y-}   
    = \frac{2}{\sqrt{5}} \ket{y+} - \frac{i}{\sqrt{5}} \ket{y-}
  \]

\item También podríamos haber expresado el vector de estado del spin
  en función de los vectores propios de $\xhat{S}_y$ de forma directa
  \[
    \ket{\varphi}
    =
    \frac{1}{\sqrt{10}} \begin{pmatrix} 3 \\ i \end{pmatrix}
    =
    c_1 \ket{y+} + c_2 \ket{y-}
    = c_1 \frac{1}{\sqrt{2}} \begin{pmatrix} 1 \\ i \end{pmatrix}
    + c_2 \frac{1}{\sqrt{2}} \begin{pmatrix} i \\ 1 \end{pmatrix}
  \]

  Esto nos lleva a un sistema de ecuaciones
  \[
    \begin{cases}
      \begin{array}{rlrll}
        \frac{1}{\sqrt{2}}\,c_1 & + & \frac{i}{\sqrt{2}}\,c_2 &=
        & \frac{3}{\sqrt{10}}\\
        \frac{i}{\sqrt{2}}\,c_1 & + & \frac{1}{\sqrt{2}}\,c_2 &=
        & \frac{i}{\sqrt{10}}
      \end{array}
    \end{cases}
  \]

  Resolviendo el sistema obtenemos las amplitudes de probabilidad
  \[
    c_1 = \frac{2}{\sqrt{5}}
    ;\hspace{1em}
    c_2 = \frac{-i}{\sqrt{5}}
  \]
\end{itemize}

  Las probabilidades de que el resultado de la medida del spin en la
  dirección $y$ sea $+\hbar/2$ o $-\hbar/2$ son
  \begin{align*}
    P_{+1/2}
    &=
      \left|\frac{2}{\sqrt{5}}\right|^2
      = \frac{4}{5} = 0,8 = \SI{80}{\percent}\\
    P_{-1/2}
    &=
      \left|\frac{-i}{\sqrt{5}}\right|^2      
      = \frac{1}{5} = 0,2 = \SI{20}{\percent}
  \end{align*}


\subsection{Colapso de la función de onda}
Vamos a ver algo extraño: supongamos que medimos la componente $z$ del
spin de la partícula y encontramos, por ejemplo, que obtenemos un
spin $+1/2$; entonces la función de onda \emph{colapsa} y quedaría
\[
  \ket{\varphi}
  =
  \ket{z+}
\]

Ahora la amplitud de probabilidad del spin $+1/2$ es uno y la del spin
$-1/2$ es cero. Si volviéramos a medir el spin en el eje $z$ siempre
obtendríamos $+1/2$, sin importar el número de medidas del spin en esta
dirección. Ya no hay superposición de estados.

Pero ¿qué ocurriría si midiéramos ahora la componente del spin según un
eje diferente, por ejemplo el eje $x$?

Ahora queremos representar la función de onda colapsada en la base propia
de $\xhat{S}_x$, porque vamos a medir la componente $x$ del spin. Para
ello calculamos los coeficientes del desarrollo de la función de onda
\[
  \ket{\varphi}
  =
  \begin{pmatrix}
    1 \\ 0
  \end{pmatrix}
  =
  c_1
  \frac{1}{\sqrt{2}}
  \begin{pmatrix}
    1 \\ 1
  \end{pmatrix}
  +
  c_2
  \frac{1}{\sqrt{2}}
  \begin{pmatrix}
    -1 \\ 1
  \end{pmatrix}
\]
\[
  \begin{cases}
    \begin{array}{rlrll}
      \frac{1}{\sqrt{2}}\,c_1 & - & \frac{1}{\sqrt{2}}\,c_2 &=  & 1\\
      \frac{1}{\sqrt{2}}\,c_1 & + & \frac{1}{\sqrt{2}}\,c_2 &=  & 0
    \end{array}
  \end{cases}
\]

Los coeficientes son
\[
  c_1 = \frac{1}{\sqrt{2}}
  ;\hspace{1em}
  c_2 = -\frac{1}{\sqrt{2}}
\]

Por tanto, la función de onda en términos de la base propia de $\xhat{S}_x$
queda
\[
  \ket{\varphi}
  =
  \frac{1}{\sqrt{2}}\,\begin{pmatrix} 1 \\ 1 \end{pmatrix}
  - \frac{1}{\sqrt{2}}\,\begin{pmatrix} -1 \\ 1 \end{pmatrix}
\]

Así, las probabilidades de obtener el spin $+1/2$ es
\begin{align*}
  P_{+1/2}
  &=
    |c_1|^2
    =
    \frac{1}{\sqrt{2}}\cdot\frac{1}{\sqrt{2}}
    =
    \frac{1}{2}
    =
    0,5
    =
    \SI{50}{\percent}
\end{align*}

Y para obtener el spin $-1/2$
\begin{align*}
  P_{-1/2}
  &=
    |c_2|^2
    =
    \frac{1}{\sqrt{2}}\cdot\frac{1}{\sqrt{2}}
    =
    \frac{1}{2}
    =
    0,5
    =
    \SI{50}{\percent}
\end{align*}

Ahora los dos estados del spin en el eje $x$ se encuentran en superposición
de estados al \SI{50}{\percent}.


\subsection{Valor promedio}
Para hallar experimentalmente el valor promedio de un observable medimos
muchas veces el sistema --en el mismo estado-- y se divide entre el número
de medidas. Por supuesto cuanto mayor sea el número de medidas realizadas,
más exacto será el valor.

En mecánica cuántica se puede calcular el valor promedio de un observable
mediante
\begin{equation}\label{eq:spn-valor-promedio-mc}
  \braket{\xhat{A}} = \braket{\varphi|\xhat{A}|\varphi}
\end{equation}


\subsection{Valor promedio del spin para nuestra
  \mathinhead{\ket{\varphi}}{sferses} de ejemplo}
Podríamos calcular el valor promedio de forma teórica si conociéramos
la probabilidad de cada posible resultado de la medida. Aprovechamos que
conocemos las probabilidades de que se obtengan los únicos valores posibles
de la componente del spin en los ejes $x$, $y$ y $z$ para una partícula 
en el estado
\[
  \ket{\varphi}
  = \frac{1}{\sqrt{10}}
  \begin{pmatrix}
    3 \\ i
  \end{pmatrix}
\]

Por ejemplo, para calcular los valores promedio de $\xhat{S}_x$, $\xhat{S}_y$
y $\xhat{S}_z$, hacemos
\begin{align*}
  \braket{\xhat{S}_x}
  &=
    0,5\cdot\left(\frac{\hbar}{2}\right) + 0,5\cdot\left(-\frac{\hbar}{2}\right)
    = 0\\
  \braket{\xhat{S}_y}
  &=
    0,8\cdot\left(\frac{\hbar}{2}\right) + 0,2\cdot\left(-\frac{\hbar}{2}\right)
    = 0,3\,\hbar = \frac{3}{10}\,\hbar\\
  \braket{\xhat{S}_z}
  &=
    0,9\cdot\left(\frac{\hbar}{2}\right) + 0,1\cdot\left(-\frac{\hbar}{2}\right)
    = 0,4\,\hbar = \frac{2}{5}\,\hbar\\
\end{align*}

Ahora calcularemos el valor promedio de los operadores del spin mediante la
ecuación~(\ref{eq:spn-valor-promedio-mc})
\begin{align*}
  \braket{\xhat{S}_x}
  &=
    \braket{\varphi|\xhat{S}_x|\varphi}
    = \frac{1}{\sqrt{10}} \begin{pmatrix} 3 & -i \end{pmatrix}
      \frac{\hbar}{2} \begin{pmatrix} 0 & 1\\ 1 & 0\end{pmatrix}
      \frac{1}{\sqrt{10}} \begin{pmatrix} 3 \\ i \end{pmatrix}\\
  &= \frac{\hbar}{20} \begin{pmatrix} 3 & -i \end{pmatrix}
      \begin{pmatrix} i \\ 3 \end{pmatrix}
  = \frac{\hbar}{20}\,(3i-3i) = 0
\end{align*}
\begin{align*}
  \braket{\xhat{S}_y}
  &=
    \braket{\varphi|\xhat{S}_y|\varphi}
    = \frac{1}{\sqrt{10}} \begin{pmatrix} 3 & -i \end{pmatrix}
      \frac{\hbar}{2} \begin{pmatrix} 0 & -i\\ i & 0\end{pmatrix}
      \frac{1}{\sqrt{10}} \begin{pmatrix} 3 \\ i \end{pmatrix}\\
  &= \frac{\hbar}{20} \begin{pmatrix} 3 & -i \end{pmatrix}
      \begin{pmatrix} 1 \\ 3i \end{pmatrix}
  = \frac{\hbar}{20}\,(3+3) = \frac{3}{10}\,\hbar
\end{align*}
\begin{align*}
  \braket{\xhat{S}_z}
  &=
    \braket{\varphi|\xhat{S}_z|\varphi}
    = \frac{1}{\sqrt{10}} \begin{pmatrix} 3 & -i \end{pmatrix}
      \frac{\hbar}{2} \begin{pmatrix} 1 & 0\\ 0 & -1\end{pmatrix}
      \frac{1}{\sqrt{10}} \begin{pmatrix} 3 \\ i \end{pmatrix}\\
  &= \frac{\hbar}{20} \begin{pmatrix} 3 & -i \end{pmatrix}
      \begin{pmatrix} 3 \\ -i \end{pmatrix}
  = \frac{\hbar}{20}\,(9-1) = \frac{2}{5}\,\hbar
\end{align*}
\begin{align*}
  \braket{\xhat{S}^2}
  &=
    \braket{\varphi|\xhat{S}^2|\varphi}
    = \frac{1}{\sqrt{10}} \begin{pmatrix} 3 & -i \end{pmatrix}
      \frac{3\hbar^2}{4} \begin{pmatrix} 1 & 0\\ 0 & 1\end{pmatrix}
      \frac{1}{\sqrt{10}} \begin{pmatrix} 3 \\ i \end{pmatrix}\\
  &= \frac{3\hbar^2}{40} \begin{pmatrix} 3 & -i \end{pmatrix}
      \begin{pmatrix} 3 \\ i \end{pmatrix}
  = \frac{3\hbar^2}{40}\,(9+1) = \frac{3}{4}\,\hbar^2
\end{align*}


\subsection{Incertidumbre en la medida de un observable}
El hecho de que frecuentemente no se conozca con precisión absoluta
el resultado de una medida, genera una incertidumbre o
indeterminacion\footnotemark{}.
\footnotetext{Esta incertidumbre no se basa en el desconocimiento de
  algún parámetro. Se trata de una indeterminación \emph{intrínseca} que no
  podemos evitar.}

\subsubsection{Desviación estándar}
Supongamos que estamos interesados en medir el resultado de la medida de
una magnitud $x$, podemos realizar y registrar muchas medida --cuantas
más mejor--.
Supongamos, además, que siempre obtuviéramos el mismo valor
\[
  x = \set{3,3,3,3,\cdots}
\]

El valor promedio sería $\braket{x} = 3$. En estadística hay varias formas
de medir la dispersión en la medida; una de ellas se llama desviación
estándar $\sigma$. En nuestro ejemplo la desviación estándar valdría cero,
$\sigma = 0$, porque no hay ninguna dispersión en las medidas.
Cuanto mayor fuera la desviación estándar, habría menos uniformidad en
los resultados.

En mecánica cuántica la desviación estándar es una medida de la incertidumbre
que se tiene en el valor de un observable físico.
Su expresión para el observable $x$
\[
  \sigma
  = \sqrt{\braket{x^2} - \braket{x}^2}
\]


\subsubsection{Incertidumbre en la medida del spin
  \mathinhead{\xhat{S}}{imss} para nuestra
  \mathinhead{\ket{\varphi}}{ielmds} de ejemplo}
Queremos calcular
\[
  \sigma
  = \sqrt{\braket{\xhat{S}^2} - \braket{\xhat{S}}^2}
\]

Ya se calculó anteriormente $\braket{\xhat{S}^2}$
\[
  \braket{\xhat{S}^2} = \frac{3}{4}\,\hbar^2
\]

\begin{align*}
  &\braket{\xhat{S}}
  = \braket{\varphi|\xhat{S}|\varphi}
  =\left(\braket{\xhat{S}_x},\braket{\xhat{S}_y},\braket{\xhat{S}_z}\right)
    = \frac{\hbar}{10}\,(0,3,4)\\
  &\braket{\xhat{S}}^2
  = \braket{\xhat{S}_x}^2 + \braket{\xhat{S}_y}^2 + \braket{\xhat{S}_z}^2
  = \frac{\hbar^2}{100}\,\left(0^2 + 3^2 + 4^2\right)
  = \frac{\hbar^2}{4}
\end{align*}

La incertidumbre en la medida del spin nos delata que hay una incertidumbre
intrínseca en la longitud del spin
\[
  \sigma
  = \sqrt{\braket{\xhat{S}^2} - \braket{\xhat{S}}^2}
  = \sqrt{\frac{3\hbar^2}{4} - \frac{\hbar^2}{4}}
  = \sqrt{\frac{\hbar^2}{2}}
  = \frac{\hbar}{\sqrt{2}}
\]

Vemos que en nuestro estado $\ket{\varphi}$ el spin no tiene componente $x$
\[
  \braket{\xhat{S}}
  =
  \hbar\,(0, 3/10, 4/10)
\]

\begin{figure}[ht]
  \begin{minipage}{.40\linewidth}
\def\scl{1}
\def\longeje{3}
%
\tdplotsetmaincoords{60}{110}
% 
\pgfmathsetmacro{\rvec}{2.7}
\pgfmathsetmacro{\thetavec}{36.87}
\pgfmathsetmacro{\phivec}{90}
% Fondo
\pgfmathsetmacro{\HORZ}{0.25}
\pgfmathsetmacro{\VERT}{0.25}
%
\tikzfading[name=fade out, inner color=transparent!0,
outer color=transparent!100]

\begin{tikzpicture}[
  scale=\scl,tdplot_main_coords,
    background/.style={
      line width=\bgborderwidth,
      draw=\bgbordercolor,
      fill=\bgcolor,
    },
    backgroundonly/.style={
      line width=\bgborderwidth,
      fill=\bgcolor,
    },  
  ]
  \coordinate (dx) at (2.2,0,0);
  \coordinate (dy) at (0,2.2,0);
  \coordinate (dz) at (0,0,2.2);
  % Ejes de Alicia
  \draw[ultra thick,->] (0,0,0) -- (\longeje,0,0) node[anchor=north]{$x$};
  \draw[ultra thick,->] (0,0,0) -- (0,\longeje,0) node[anchor=north]{$y$};
  \draw[ultra thick,->] (0,0,0) -- (0,0,\longeje) node[anchor=south]{$z$};

%  % Giro eje x
%  \tdplotsetthetaplanecoords{-90}
%  % Notice you have to tell tiks-3dplot you are now in rotated coords
%  % Since tikz-3dplot swaps the planes in tdplotsetthetaplanecoords,
%  % the former y axis is now the z axis.
%  \tdplotdrawarc[tdplot_rotated_coords,line width=2pt,
%  -{Latex[length=11pt,width=7pt,flex=1]},color=black!50]
%  {(0,0,1.9)}{0.3}{380}{40}{anchor=south west,color=black}{}
%  % Reponer parte derecha del eje y
%  \draw[ultra thick,->] (1.9,0,0) -- (\longeje,0,0);
%%  \tdplotdrawarc[tdplot_rotated_coords,line width=2pt,
%%  -{Latex[round,length=6pt,width=4pt]},color=black!50]
%%  {(0,0,1.9)}{0.3}{380}{40}{anchor=south west,color=black}{}
%%  % Reponer parte derecha del eje y
%%  \draw[ultra thick,->] (1.9,0,0) -- (\longeje,0,0);



  \tdplotsetcoord{P}{\rvec}{\thetavec}{\phivec}
  % draw a vector from origin to point (P)
  \draw[-{Stealth[width=7pt]},color=red!80!black,ultra thick] (O) -- (P)
  node[above right=0pt and 0pt] {\footnotesize $\braket{\vec{S}}$};
  
  % draw projection on xy plane, and a connecting line
  \draw[dashed, color=red] (P) -- (Pz);
  \draw[dashed, color=red] (P) -- (Pxy);
  \tdplotsetthetaplanecoords{\thetavec}

  % draw the angle \phi, and label it
  % syntax:
  % \tdplotdrawarc[coordinate frame, draw options]
  % {center point}{r}{angle}{label options}{label}
  \tdplotdrawarc{(O)}{0.8}{0}{\phivec}{anchor=north}{\footnotesize $\pi/2$}

  % set the rotated coordinate system so the x'-y' plane lies within the
  % "theta plane" of the main coordinate system
  % syntax: \tdplotsetthetaplanecoords{\phi}
  \tdplotsetthetaplanecoords{\phivec}

  % draw theta arc and label, using rotated coordinate system
  \tdplotdrawarc[tdplot_rotated_coords]
  {(0,0,0)}{0.8}{0}{\thetavec}{above right=1pt and 1pt}{\footnotesize $\theta$}
  % Fondo amarillo
  \coordinate (SW) at ($(current bounding box.south west) + (-\HORZ cm,-\VERT cm)$);
  \coordinate (NE) at ($(current bounding box.north east) + (\HORZ cm,\VERT cm)$);    
  \begin{scope}[on background layer]
    \draw[background] (SW) rectangle (NE);
  \end{scope}

\end{tikzpicture}
\caption{El spin para la $\ket{\varphi}$ del ejemplo no tiene componente $x$.}
\label{fig:spn-vector-S}

\end{minipage}
\hspace{1em}
\begin{minipage}{.55\linewidth}
  \vspace{-14ex}

Calculamos el ángulo $\theta$
\[
  \tan\theta
  =
  \frac{\braket{\xhat{S}_y}}{\braket{\xhat{S}_z}}
  =
  \frac{3\cancelout{\hbar}/\cancelout{10}}{4\cancelout{\hbar}/\cancelout{10}}
  =
  \frac{3}{4}
\]

El ángulo es aproximadamente igual a $\ang{38,7}$ o en radianes
\[
  \theta \approx 0,205\pi\,\si{\radian}
\]

En coordenadas esféricas
\[
  \braket{\xhat{S}}
  =
  (\hbar/2, 0.205\pi, \pi/2)
\]
\end{minipage}
\end{figure}

Otra forma de expresar el spin sería
\[
  \cos\theta
  = \frac{\braket{\xhat{S}_z}}{|\braket{\xhat{S}}|}
  = \frac{2\cancelout{\hbar}/5}{\cancelout{\hbar}/2}
  = \frac{4}{5}
  ;\hspace{1em}
  \sin\theta
  = \frac{\braket{\xhat{S}_y}}{|\braket{\xhat{S}}|}
  = \frac{3\cancelout{\hbar}/10}{\cancelout{\hbar}/2}
  = \frac{3}{5}
\]
\begin{align*}
  \braket{\xhat{S}}
  &=
  (0, |\braket{\xhat{S}}| \sin\theta, |\braket{\xhat{S}}| \cos\theta)
  =
  |\braket{\xhat{S}}|\,(0,\sin\theta,\cos\theta)\\
  &=
  \frac{\hbar}{2}\,(0,3/5,4/5)
\end{align*}


\subsubsection{Giro de \mathinhead{\braket{\xhat{S}}}{gds} para
  la \mathinhead{\ket{\varphi}}{gdspl}}
Como divertimento vamos a girar el vector $\braket{\xhat{S}}$ alrededor del
eje $x$ un ángulo aproximado de $\ang{36,87}$ en sentido positivo y deberíamos
comprobar que el spin se alinea con el eje $z$ y resultará
\[
  \braket{\xhat{S}'} = \frac{\hbar}{2}\,(0,0,1)
\]

\begin{figure}[ht]
\def\scl{1}
\def\longeje{3}
%
\tdplotsetmaincoords{60}{110}
% 
\pgfmathsetmacro{\rvec}{2.7}
\pgfmathsetmacro{\thetavec}{36.87}
\pgfmathsetmacro{\phivec}{90}
% Fondo
\pgfmathsetmacro{\HORZ}{0.25}
\pgfmathsetmacro{\VERT}{0.25}
%
\tikzfading[name=fade out, inner color=transparent!0,
outer color=transparent!100]

\begin{tikzpicture}[
  scale=\scl,tdplot_main_coords,
  background/.style={
    line width=\bgborderwidth,
    draw=\bgbordercolor,
    fill=\bgcolor,
  },
  backgroundonly/.style={
    line width=\bgborderwidth,
    fill=\bgcolor,
  },  
  ]
  % SISTEMA ORIGINAL (A LA IZQUIERDA)
  % Para 'tikz-3dplot' son los ejes principales
  \draw[ultra thick,->] (0,0,0) -- (\longeje,0,0) node[anchor=north]{$x$};
  \draw[ultra thick,->] (0,0,0) -- (0,\longeje,0) node[anchor=north]{$y$};
  \draw[ultra thick,->] (0,0,0) -- (0,0,\longeje) node[anchor=south]{$z$};

  % REPRESENTACIÓN DEL GIRO ALREDEDOR DEL EJE X
  \tdplotsetthetaplanecoords{-90}
  % Notice you have to tell tiks-3dplot you are now in rotated coords
  % Since tikz-3dplot swaps the planes in tdplotsetthetaplanecoords,
  % the former y axis is now the z axis.
  \tdplotdrawarc[tdplot_rotated_coords,line width=2pt,
  -{Latex[round,length=12pt,width=7pt,bend]},color=black!30]
  {(0,0,1.9)}{0.3}{10}{360}{anchor=south west,color=black}{}
  % Reponer parte final del eje x para simular perspectiva
  \draw[ultra thick,->] (1.9,0,0) -- (\longeje,0,0);

  % VECTOR DE SPIN ORIGINAL
  % Define el extremo del vector (P) y sus proyecciones (Px), (Py), (Pxy), etc.
  \tdplotsetcoord{P}{\rvec}{\thetavec}{\phivec}
  % Dibuja el vector desde el origen hasta el punto (P)
  \draw[-{Latex[width=7pt]},color=red!90!black,line width=2pt] (O) -- (P)
  node[above right=-1pt and -1pt] {\footnotesize $\braket{\vec{S}}$};
  
  % PROYECCIONES
  % Proyección del spin en línea roja discontínua en el plano xy y en el eje z
  \draw[dashed, color=red] (P) -- (Pz);
  \draw[dashed, color=red] (P) -- (Pxy);
  \tdplotsetthetaplanecoords{\thetavec}

  % ÁNGULO PHI
  \tdplotsetthetaplanecoords{\thetavec}
  % Dibuja el ángulo \phi y etiquétalo
  % sintaxis:
  % \tdplotdrawarc[coordinate frame, draw options]
  % {center point}{r}{angle}{label options}{label}
  \tdplotdrawarc{(O)}{0.8}{0}{\phivec}{anchor=north}{\footnotesize $\pi/2$}

  % ÁNGULO THETA
  % set the rotated coordinate system so the x'-y' plane lies within the
  % "theta plane" of the main coordinate system
  % sintaxis: \tdplotsetthetaplanecoords{\phi}
  \tdplotsetthetaplanecoords{\phivec}
  % draw theta arc and label, using rotated coordinate system
  \tdplotdrawarc[tdplot_rotated_coords,{Latex}-,shorten <=1pt]
  {(0,0,0)}{0.8}{0}{\thetavec}{above right=1pt and 1pt}{\footnotesize $\theta$}

  % FLECHA QUE APUNTA AL SISTEMA ROTADO (CENTRO)
  % (dibujo auxiliar fuera del sistema 3D)
  \begin{scope}[xshift=3.25cm]
    \draw[-{Latex}] (0,0,0) --
    node[above] {\small Giro activo}
    node[below] {\small $\theta$\,\,\si{\radian} (eje $x$)}
    +(right:2.0cm);
  \end{scope}
  
  % El sistema rotado tiene la misma orientación que el principal
  \tdplotsetrotatedcoords{0}{0}{0}
  \begin{scope}[tdplot_screen_coords]
    % Desplazamiento del origen a 7.5cm a la derecha de (O)
    \coordinate (despl) at (7.5,0);
    % Punto extremo del vector (S) antes de rotar
    \path (P)+(despl) circle [radius=2pt] coordinate (Pp);
  \end{scope}
  % Desplaza el sistema rotado 
  \tdplotsetrotatedcoordsorigin{(despl)}

  % Ejes trasladados
  \draw[ultra thick,->,tdplot_rotated_coords]
  (0,0,0) -- (\longeje,0,0) node[anchor=north]{$x$};
  \draw[ultra thick,->,tdplot_rotated_coords]
  (0,0,0) -- (0,\longeje,0) node[anchor=north]{$y$};
  \draw[ultra thick,->,tdplot_rotated_coords]
  (0,0,0) -- (0,0,\longeje) node[anchor=south]{$z$};

  % Vector de spin antes de rotar en gris claro
  \draw[-{Latex[width=7pt]},black!20,line width=2pt,tdplot_screen_coords]
  (despl) -- (Pp)
  node[above right=-1pt and -1pt] {\footnotesize $\braket{\vec{S}}$};
  
  % Vector de spin rotado en rojo
  \draw[-{Latex[width=7pt]},red!90!black,line width=2pt,tdplot_rotated_coords]
  (0,0,0) -- (0,0,\rvec) node[left=3pt] {\footnotesize $\braket{\vec{S}}$};
  
  \tdplotsetrotatedthetaplanecoords{\phivec}
  % Dibuja el ángulo theta y etiquétalo, usando el sistema rotado.
  \tdplotdrawarc[tdplot_rotated_coords,
  {Latex[round,length=5pt,width=4pt,bend]}-,shorten <=1pt,color=black!40]
  {(0,0,0)}{0.8}{0}{\thetavec}{above right=1pt and 1pt}{\footnotesize $\theta$}  

  % Giro eje x
  \tdplotsetrotatedthetaplanecoords{180}
  % Notice you have to tell tiks-3dplot you are now in rotated coords
  % Since tikz-3dplot swaps the planes in tdplotsetthetaplanecoords,
  % the former y axis is now the z axis.
  \tdplotdrawarc[tdplot_rotated_coords,line width=2pt,
  -{Latex[round,length=12pt,width=7pt,bend]},color=black!30]
  {(0,0,1.9)}{0.3}{10}{360}{anchor=south west,color=black}{}
  % Reponer parte derecha del eje x
  \draw[ultra thick,->,tdplot_rotated_coords] (0,0,1.9) -- (0,0,\longeje);

  % Fondo amarillo
  \coordinate (SW) at ($(current bounding box.south west) + (-\HORZ cm,-\VERT cm)$);
  \coordinate (NE) at ($(current bounding box.north east) + (\HORZ cm,\VERT cm)$);    
  \begin{scope}[on background layer]
    \draw[background] (SW) rectangle (NE);
  \end{scope}
  
\end{tikzpicture}
\caption{Al girar un ángulo $\theta\approx\ang{+36,87}$ el vector de
  estado $\ket{\varphi}$ de ejemplo, esperamos que gire el vector
  de spin.}
\label{fig:spn-giro-vector-S}
\end{figure}

Vamos a rotar nuestro vector de estado de ejemplo
\begin{equation}\label{eq:spn-giroxvectorestado}
  \ket{\tilde{\varphi}} =  e^{-i\frac{\theta}{\hbar}\,\xhat{S}_x}  \ket{\varphi}
\end{equation}

Sustituimos la matriz que representa al operador $\xhat{S}_x$ y
usamos la identidad de Euler
\begin{align*}
  e^{-i\frac{\theta}{\hbar}\,\xhat{S}_x}
  &=
  e^{-i\frac{\theta}{\hbar}\,
  {\scriptscriptstyle\begin{pmatrix}0 & 1\\1 & 0\end{pmatrix}}}
  = \begin{pmatrix}1 & 0\\ 0 & 1\end{pmatrix}\cos\left(-\frac{\theta}{2}\right)
  +i\begin{pmatrix}0 & 1\\1 & 0\end{pmatrix}\sin\left(-\frac{\theta}{2}\right)\\
   &=
     \begin{pmatrix}1 & 0\\ 0 & 1\end{pmatrix}\cos\left(\frac{\theta}{2}\right)
    -i\begin{pmatrix}0 & 1\\1 & 0\end{pmatrix}\sin\left(\frac{\theta}{2}\right)\\
  &=
    \begin{pmatrix}
      \cos(\theta/2) & -i\sin(\theta/2)\\
      -i\sin(\theta/2) & \cos(\theta/2)
    \end{pmatrix}
\end{align*}

Calculamos las funciones trigonométricas
\begin{align*}
  \cos\left(\frac{\theta}{2}\right)
  &= \sqrt{\frac{1+\cos\theta}{2}}
    = \sqrt{\frac{1+4/5}{2}}
    =\frac{3}{\sqrt{10}}\\
  \sin\left(\frac{\theta}{2}\right)
  &= \sqrt{\frac{1-\cos\theta}{2}}
    = \sqrt{\frac{1-4/5}{2}}
    =\frac{1}{\sqrt{10}}
\end{align*}

La exponencial queda
\[
  e^{-i\frac{\theta}{\hbar}\,\xhat{S}_x}
  = \begin{pmatrix}
    3/\sqrt{10} & -i/\sqrt{10}\\
    -i/\sqrt{10} & 3/\sqrt{10}
  \end{pmatrix}
  = \frac{1}{\sqrt{10}}
  \begin{pmatrix}
    3 & -i\\
    -i & 3
  \end{pmatrix}
\]

Ahora podemos girar el vector de estado según~(\ref{eq:spn-giroxvectorestado})
\[
  \ket{\tilde{\varphi}}
  = \frac{1}{\sqrt{10}}
  \begin{pmatrix}
    3 & -i \\
    -i & 3
  \end{pmatrix}
  \frac{1}{\sqrt{10}}
  \begin{pmatrix}
    3 \\ i
  \end{pmatrix}
  =
  \frac{1}{10}
  \begin{pmatrix}
    9+1 \\ -3i+3i
  \end{pmatrix}
  =
  \begin{pmatrix}
    1 \\ 0
  \end{pmatrix}
  =
  \ket{z+}
\]
y vemos que el vector de estado rotado se encuentra sobre el eje $z$.
Al medir ahora $\xhat{S}_z$ nos daría $+\hbar/2$ con un \SI{100}{\percent}
de seguridad.

Aún así vamos a comprobarlo
\begin{align*}
  \braket{\tilde{\varphi}|\xhat{S}_x|\tilde{\varphi}}
  &=
    \begin{pmatrix}
      1 & 0
    \end{pmatrix}
    \frac{\hbar}{2}
    \begin{pmatrix}
      0 & 1\\
      1 & 0
    \end{pmatrix}
    \begin{pmatrix}
      1 \\ 0
    \end{pmatrix}
  = 0\\
  \braket{\tilde{\varphi}|\xhat{S}_y|\tilde{\varphi}}
  &=
    \begin{pmatrix}
      1 & 0
    \end{pmatrix}
    \frac{\hbar}{2}
    \begin{pmatrix}
      0 & i\\
      i & 0
    \end{pmatrix}
    \begin{pmatrix}
      1 \\ 0
    \end{pmatrix}
  = 0\\
  \braket{\tilde{\varphi}|\xhat{S}_x|\tilde{\varphi}}
  &=
    \begin{pmatrix}
      1 & 0
    \end{pmatrix}
    \frac{\hbar}{2}
    \begin{pmatrix}
      1 & 0\\
      0 & -1
    \end{pmatrix}
    \begin{pmatrix}
      1 \\ 0
    \end{pmatrix}
  = \frac{\hbar}{2}
\end{align*}

El giro del vector de estado provoca un giro del spin
\[
  \braket{\tilde{\vvv{S}}}
  = \braket{\tilde{\varphi}|\vvv{S}|\tilde{\varphi}}
\]

Con esto hemos demostrado que
\[
  e^{-i\frac{\theta}{\hbar}\,\left(\xhat{n}\cdot\vvv{S}\right)}
\]
gira los elementos del espacio de Hilbert $\symbb{C}^2$.





  
%%% Local Variables:
%%% mode: latex
%%% TeX-engine: luatex
%%% TeX-master: "../gruposlie.tex"
%%% End:
  
%%% Local Variables:
%%% mode: latex
%%% TeX-engine: luatex
%%% TeX-master: "../gruposlie.tex"
%%% End:
